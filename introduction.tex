\chapter{Introduction}

% In this chapter, we introduce the fundamentals of genetics and DNA sequencing. Furthermore, we introduce the important research problem in the field of genetics, the diploid genome assembly (haplotyping). 
% We will see how we can mathematically formualate the diploid assembly problem as a computer scientist.
% Then we provide a high level description of the main methods used in this problem.
% Thereafter, we describe the limits and challenges faced currently in this field. We finish this chapter by an outline of the thesis.


Genetics studies the phenomenon of \textit{life} at its most basic level, thus the science of genetics is important and fascinating.
% In this section we describe the basics about genetics and haplotyping. We provide motivation about why understanding genetics for different organisms is so important.
% https://www.gen.cam.ac.uk/undergraduate/whygenetics
% http://www.helsinki.fi/biosciences/genetics/
An important field in genetics is haplotyping.
Haplotyping is the process of determining the sequences of both copies of homologous chromosomes, which are inherited from each parent in diploid organisms.
Haplotyping has applications in different fields such as evolutionary studies, clinical diagnosis, precision medicine, and biotechnology. 
Current sequencing technologies allow for reading off the genome sequence, and thus the reconstruction of haplotypes is
possible, in principle. Unfortunately, sequencing is prone to errors and the use of advanced algorithms and models is essential to correct for errors, in order to reconstruct accurate haplotypes.
However, the process of correcting these errors poses various computational challenges.
In this thesis, we present our novel algorithms to addressing various computational challenges in this field.

\section{Genetics, DNA sequencing and Haplotyping}\label{sec:dna_seq}
Genetics is the study of genes, genetic variation, and heredity in living organisms. 
Genetics controls what an organism looks like and how it functions.
% The DNA is the carrier of hereditary information and thus understanding DNA in detail has been the central aim for many researchers.
% Classically the genetic variants (mutations) present in the living organisms have been used to study and investigate the cause of biological diseases and, additionally helps to make deductions about the way cells and organisms worked. 
% Genetics is the study of genes, genetic variation, and heredity in living organisms.
Specifically, there are two sides to the science of genetics.
On the one hand, the availability of different types of molecular information, such as sequence information and gene expression levels, paired with gene editing techniques with which we can perturb the genome in a controlled fashion and observe its biological effects, 
provides powerful explanation about the functions of the genes.
On the other hand, genetics provides a fundamental understanding of how organisms, populations, and species evolve. 
In the last few years, one of the most exciting new developments is the way in which these two sides have begun to converge \citep{casillas2017molecular}.
This convergence is achieved through the development of sequencing technologies that provide datasets from the genomic level to epigenomic, transcriptomic, proteomics or metabolomic level.
% The integrative analyses of these datasets promise to provide a systematic view of the causes and consequences of evolution, development and functions of organisms.

% these vary during the lifetime of the individual, therefore, have the potential to provide insights  effort of using molecular techniques to understand the processes of development, evolution, and speciation.
% % Modern computational biology tools (analysis of genomic sequences and bioinformatics) use these integrative principles 
% % to answer various difficult biological questions ranging from the mechanisms of evolution to the development of complex diseases.
% Thus, genetics has a central role in modern biology and its influence is ever increasing.

% s\todo{citations are missing in this para or even figures.}
% from italy person thesis.
\textit{What actually is genetic information?} 
The discovery of spatial structure of double helix deoxyribonucleic acid (DNA) by Watson and Crick in 1953, laid the main foundation for the understanding of genetics.
In most living organisms, genetic information is encoded in the form of DNA molecules.
A DNA molecule is a chain in which many bases are ordered in a linear sequence, the bases --- A, T, G, and C --- are the letters of the genetic alphabet.
The whole information within the DNA molecule of an organism is called its genome. The genome is further divided into chromosomes.
Genomes have single (haploid), double (diploid) or higher ploidy with more than two homologous chromosomes (polyploids). 
In this study, we focus on \textit{diploid} organisms. For example, humans are diploids, consisting of two copies of each chromosome called homologous chromosomes or \textit{haplotypes} --- one inherited from the mother and the other from the father.
There are differences between these two copies of each chromosome known as genetic variation. 
% \todo{this para explained in more detail in biological background.}

In early 2000s, a historic breakthrough happened with the sequencing of the human genome \citep{collins2003human}.
\textit{Sequencing} is the operation that comprises in determining the base sequence of a DNA molecule.
For sequencing genomes, there exist several kinds of sequencing technologies which share the following common properties:
\begin{itemize}
 \item They yield genomes in fragments called ``reads''.
 \item The position of reads along the genome sequence is unknown and also, mostly, the strand from which the read was sampled is unknown.
 \item The reads contain errors.
\end{itemize}

\bigskip
Huge amounts of sequencing data are produced by next generation sequencing technologies (NGS) routinely. 
The major challenge with these sequencing datasets is that the genomic information is partial and erroneous. Therefore, no sequencing technology delivers data from which we can assemble the complete genome.

The sequencing technologies differ in terms of error rates and lengths of the produced reads. We define the error rate of a read as the ratio of the number of incorrectly sequenced bases to the length of the read.
Broadly, the sequencing technologies can be categorized into four classes:
\begin{itemize}
 \item Short read sequencing: It includes Illumina/Solexa sequencing \citep{bentley2008accurate}, which has been the most widespread technology. 
 This technology produces short reads (hundreds of bases) with an error rate ($\le 1$\%). 
  \item Long read sequencing: This approach includes Single Molecule Real Time sequencing (PacBio) \citep{eid2009real} and Oxford Nanopore sequencing (ONT) \citep{laszlo2014decoding}. These technologies produce very long sequences, up to hundreds of kilo-bases. 
  The downside is that PacBio and ONT exhibit very high error rates of up to 15\% and 38\% respectively. 
  \item Synthetic long read sequencing: This technology includes 10x Genomics technology \citep{eisenstein2015startups}, which adds a unique barcode to every short read (produced from Illumina platform) generated from an individual molecule. 
  The barcode information allows to link short reads together at a long-range and the linked-read length is ~50 kb.
 \item Single-cell sequencing: This technology includes Strand specific sequencing (Strand-Seq), which produces Illumina reads, along with information on the directionality of DNA. Each single strand of a DNA molecule is labeled regarding its 5'–3' orientation \citep{falconer2012dna}.
\end{itemize}

Furthermore, each sequencing technology has some bias due to the protocols employed.
The high G/C content regions are sequenced less frequently by the short read technologies than the rest of the genome and the downstream end of reads exhibit high error rate \citep{aird2011analyzing, dohm2008substantial}.

The sequencing data is routinely used to reconstruct the underlying genome. There are inherent benefits and challenges when utilizing sequencing reads from different sequencing technologies. 
Specifically, upcoming long-read technologies deliver reads that span multiple variants and therefore, provide power to reconstruct end-to-end whole genome.
The challenge is high sequencing error rates. To overcome this challenge, the long noisy sequencing datasets are often supplemented with short accurate datasets.
Thus, the potential way to generating the complete genomes could be to combining multiple sequencing technologies, one that produces long reads and others that produce low error rates and provide long-range information.

\textit{What are the common obstacles across all technologies in producing a diploid genome?}
For all the existing sequencing technologies, the first challenge arises from the lack of information of the location of reads.
For diploid organisms, the next challenge is the lack of the \textit{haplotypic identity} of reads, meaning the haplotype that the read comes from.
Knowing the phase of reads is essential for reconstructing both copies of each chromosome, which helps to understand the true biological characteristics of diploid organisms.
% Thus, from the biological point of view, the single individual haplotyping (SIH) problem consists in the reassignment of each read to the original haplotype.
% Once we know this identity, it becomes easier to assemble the reads from each haplotype separately to further reconstruct the two genome sequences of diploid organisms.
% The process of reconstructing the diploid assemblies from sequencing reads is known as \textit{diploid genome assembly} or \textit{haplotyping}.

The haplotypes are required in order to correctly understand allele-specific expression and compound heterozygosity. Compound heterozygosity is a phenomenon of having two heterozgous recessive alleles at a marker site, which could be the cause for genetic diseases.
Haplotypes also help in investigating the genetics of common diseases, and to performing population-genetic analyses of admixture, migration and selection \citep{tewhey2011importance, Glusman2014}. 
Furthermore, the haplotype sequences are used in relating genotypes to phenotypes and for understanding how the arrangement of cis- and trans-acting variants across the two homologous copies of a genomic region affects phenotypic expression.

\begin{figure}[t!]\centering
\includegraphics[width=\columnwidth]{{ex1-intro1}.pdf}
\caption{Seven variants covered by reads (horizontal bars) in a single individual.
The alleles that a read supports are printed in white. The middle panel shows the phased reads in colors and haplotypes at the bottom over the seven variants.}
\label{fig:ex1_intro}
\end{figure}

Broadly, the approaches to obtain haplotypes are classified into two categories: Reference-based and \textit{de novo} assembly.

% This absence of context and the small size and error rates of the sequences obtained, relatively to the genome size, makes it difficult to use reads as such.
% Ideally, we would need the access to the underlying genomes in their entirety.
% Since the beginning of sequencing of DNA molecules, genomes are produced by structuring and ordering reads information. 
% Then these reconstructed genomes can be used
% as references. 
Reference genomes provide reliable information on the genomic of reads. 
Over the years, a lot of efforts have been devoted to generate good quality reference genomes.
The reference genomes play an important role in gene regulation of living beings \citep{encode2004encode}. 
They also reveal the inner organization of the genome,
including the relative positions of genes or chromosomes structure. 
They have been used by biologists for other tasks, for instance, finding about the known genes
positions and functions to annotate the genome \citep{harrow2012gencode}.

Therefore, the reference genome is a good backbone for solving the problem of identifying the location or origin of each read.
To identifying the origin of reads, we align them to the reference genome. 
Once the reads are aligned to the reference genome, the aligned reads are then partitioned into two sets to generate haplotypes. This process is called \textit{reference-based} haplotyping (phasing).

In reference-based haplotyping, there is a reference bias. The reference genome does not capture the genome diversity of population or complete information of a target individual.
The reads that are unique to the target genome, are not aligned or wrongly aligned to the reference genome.
The usage of reference genome for read alignment generates a bias in the target genome. This reference bias causes problems for further downstream analyses. 
To overcome these problems, the approach that uses reference genome is generalized to perform \textit{haplotype-aware} de novo assembly.
In this approach, the reference genome is not used, but the haplotypes are instead constructed directly from the reads.
% In a reference-based approach, the reads are aligned to the reference genome and then the aligned reads are partitioned into two sets to generate the haplotypes.
% This process is 
% To determine the assignment of each read to one of the homologous copies of chromosome (haplotype) is an important question. 
% Thus, from the biological point of view, the haplotyping (SIH) problem consists in the reassignment of each read to the original haplotype.
% Once we know this identity, it becomes easier to assemble the reads from each haplotype separately to further reconstruct the two genome sequences of diploid organisms.
% The process of reconstructing the diploid assemblies from sequencing reads is known as \textit{diploid genome assembly} or \textit{haplotyping}.

These approaches for haplotyping are discussed below in detail (Sections \ref{sec:ref} and \ref{sec:dip}). 
First, we present formulation and methods to solving reference-based haplotyping (single individual haplotyping) and then explain the generalized approach for performing \textit{haplotype-aware} de novo assembly.
\section{Reference-based Haplotyping}\label{sec:ref}
The \textit{reference-based} phasing method is applied when a reference genome is available for the species of the target genome.
We expect the target genome to be very close to the reference, more specifically, given the reads of the target genome, we expect the reference to be from the same species as the reads.
% We are interested to phase the differences of target genome to its reference.

In reference-based phasing method, the standard pipeline consists of the following steps: 
\begin{itemize}
 \item Align the reads to the reference genome.
 \item Detect variants.
 \item Phase the variants based on how aligned reads connect the alleles over them, to generate two haplotypes.
\end{itemize}
% The main focus of this study is the third step in the pipeline.
% For illustration, the toy example is given in Figure~\ref{fig:ex1_intro}.

\begin{example}
 To illustrate the haplotyping problem for a single genome, we consider a small example in Figure~\ref{fig:ex1_intro}. The example shows seven variants. Also shown are the sequencing reads aligned to the reference genome.
 The alleles that the reads support are printed in white. 
 Erroneous alleles in the reads are shown in red. In practice, we don't know the alleles that contain these sequencing errors. 
 The goal of \textit{reference-based} haplotyping problem is the re-assignment of phases to the reads, meaning assigning haplotype specific colors (green or purple) to each read. 
 In the middle panel, the colored bars represent the assignment of each read to either the green or purple haplotype.
 Finally, the reads from each haplotype are separately assembled together to output two haplotypes shown at the bottom in purple and green.
\end{example}
% In this study, we identify the limitations of the existing methods in phasing a single genome. To overcome these limitations, we propose a novel efficient algorithm.

\begin{figure}[t!]\centering
\includegraphics[width=\columnwidth]{{ex1-intro}.pdf}
\caption{Example shows the SNP matrix for the example shown in Fig.~\ref{fig:ex1_intro}. Seven variants covered by reads (horizontal bars) in a single individual.
 The allele in read is encoded as 1 if it matches the allele in the reference position at that position and 0 otherwise.
 The middle panel shows the phased reads in colors and haplotypes at the bottom over these seven variants.}
\label{fig:ex1_intro1}
\end{figure}

\bigskip
% you may refer also this: http://homolog.us/Tutorials/index.php?p=1.4&s=1

\subsection{Haplotyping as Combinatorial Optimization problem}
For diploid genomes, large volumes of sequencing data, which contain errors, are generated every day. Extracting useful information from these datasets to understand the biology of diploid genomes is a challenging problem. 

The objective of extracting useful information from noisy data in order to achieve the best possible value of goal, or objective, 
requires formulating an \textit{optimization} problem, that is defined as: given an object $\math{o}$, find a solution such that an optimization criterion $\math{f}$ is minimized or maximized.

% Reference genome reconstruction is therefore crucial in various domains where raw,
% out of context reads are unusable. The task of reordering the reads to reconstruct the
% sequenced genome for diploid organisms is called diploid genome assembly or haplotyping.
% Over the genome, there are difference between two copes for each chromosome, known as variants \todo{explained in biological background}. 
% Traditionally, there are two ways to solve it, one is based on the ordered reads to the reference genome and the other one is \textit{denovo assembly}.
% As it will be detailed further, diploid genome assembly is especially complex as the bases distribution is far
% from being uniform. Genomes present specific patterns such as large repeated sequences
% (repeats), regions with very specific distributions of nucleotides or extremely repeated
% sequences of nucleotides. Such patterns make genomes different from a uniformly distributed sequence of nucleotides. \todo{[13]} shows that a human genome is largely constituted
% of repeated sequences of significant lengths.


% \fbox{\begin{minipage}{30em}
%  In summary, the core messages of DNA, sequencing and haplotyping:
%  \begin{itemize}
%   \item The molecular information of the genomes can be obtained using different sequencing technologies.
%   \item The sequencing data is big, erroneous and not complete.
%   \item The task of recovering the genome sequences for diploid organisms is called as diploid genome assembly or haplotyping.
%  \end{itemize}
% \end{minipage}}
We will see how we can formulate the haplotyping problem as combinatorial optimization problem.

For the haplotyping problem, which consists of determining the \textit{haplotypic identity} of each read, we consider the reads aligned to the reference genome.
A read aligner maps the reads to the reference genome, ideally to positions with a high similarity score for the read. 
The number of read alignments that cover a position is known as the coverage at that position.
Furthermore, we have structural variants (SNVs) detected using different variant calling algorithms.
In case of bi-allelic variants, that is, those for which two different alleles are known on two copies of chromosome, three genotypes are possible.
One typically denotes the reference allele as 0 and the alternative one as 1. Using this
notation, the two chromosomal copies either both carry the reference allele (genotype
0/0), (or alternative allele 1/1) or one of them contains the reference while
the other one carries the alternative allele (genotype 0/1). If both chromosomal copies
carry the same allele (i.e. genotype 0/0 or 1/1), the genotype is called homozygous, while
genotype 0/1 is referred to as heterozygous.

Given the variants and the alignments, the goal here is to phase the variants and generate the haplotypes.
The variants over the genome can be phased by using reads aligned to the reference genome. This process is known as \textit{read-based phasing} or \textit{single individual haplotyping}.

Mathematically, the aligned reads over the variants are encoded in the form of a SNP matrix.
The SNP matrix for the example given in Figure~\ref{fig:ex1_intro} is illustrated in Figure~\ref{fig:ex1_intro1}.
The SNP matrix is $\mathcal{F}\in\{0,1,-\}^{R\times M}$, where $R$ is the number of reads and $M$ is the number of variants along a chromosome.
Each matrix entry $\mathcal{F}(j,k)$ is $0$ (indicating that the read matches the reference allele) or $1$, indicating that the read matches the alternative allele if the read covers that position and ``$-$'' otherwise.
Note that the ``$-$'' character can also be used to encode the unsequenced ``internal segment'' of a paired-end read.
The goal of haplotype assembly problem is to generate two haplotypes $h^0,h^1\in\{0,1\}^M$ under some conditions explained later.

The presence of sequencing and mapping errors makes the haplotype assembly problem a challenging task. 
Computationally, this problem has been generally modeled as an optimization problem to correct the sequencing and mapping errors.
In literature, different combinatorial formulations of the problem have been
proposed \citep{lippert2002algorithmic}. Among them, Minimum Error Correction (MEC) \citep{lippert2002algorithmic} has
been proven particularly successful in the reconstruction of accurate haplotypes for
diploid species \citep{martin2016whatshap, he2010optimal, CDW13_exact, Glusman2014}. It aims at correcting the input data with the minimum
number of corrections to the SNP values, such that the resulting reads can be unambiguously partitioned into two sets, each one identifying a haplotype. 
To mathematically formulate the minimum number of corrections (MEC) as an optimization problem, we require a few definitions.

The quality of a solution relies on the measure $d(r_1,r_2)$ based on the Hamming distance between any two rows $r_1,r_2\in\{0,1, -\}^M$ in $\mathcal{F}$.
\begin{definition}[Distance] 
 Formally, the distance is given by,
\[d(r_1,r_2):= \big|\big\{k\,\big|\,r_1(k)\neq -\ \wedge\ r_2(k)\neq -\ \wedge\ r_1(k)\neq r_2(k)\big\}\big|.\]
\label{eq:distance}
\end{definition}

\begin{definition}[Feasibility]
A SNP matrix $\mathcal{F}\in\{0,1,-\}^{R\times M}$ is called \emph{feasible} if there exists a bi-partition of rows (i.\,e., reads) into two sets such that all pairwise distances of two rows within the same set are zero.
\label{def:feasible-mec}
\end{definition}
Feasibility of a matrix $\mathcal{F}$ is equivalent to the existence of two haplotypes $h^0,h^1\in\{0,1\}^M$ such that every read $r$ in the matrix has a distance of zero to $h^0$ or to $h^1$ (or both).
The MEC problem can now simply be stated in terms of flipping bits in $\mathcal{F}$, where entries that are $0$ or $1$ can be flipped and ``$-$'' entries remain unchanged.

\begin{problem}[MEC]
Given a matrix $\mathcal{F}\in\{0,1,-\}^{R\times M}$, flip a minimum number of entries in $\mathcal{F}$ to obtain a feasible matrix.
\label{prob:mec}
\end{problem}


\begin{definition}[MEC cost]
The MEC cost for a feasible solution $h^0,h^1\in\{0,1\}^M$ is given by: \\
    \[\cost_{\mathcal{F}}(h^0,h^1) := \sum_{i=1}^{n} \min\{\dist(r_i,h^0), \dist(r_i,h^1)\}\]\,.

where $r_i \in \{0,1, -\}^M$ is the $i$-th row of a SNP matrix $\mathcal{F}$.
\label{eq:cost}
\end{definition}

In theory, the MEC problem is NP-hard \citep{Cilibrasi2007}.

In practice, the following different types of MEC instances are generated from different sequencing datasets:
\begin{itemize}
 \item $\MEC$: Instances in which entries in each of the $n$ rows of $\mathcal{F}$ are from $\{0,1,-\}$. There is no restriction on the placement of entries $0, 1$ and $-$. These instances are generated from Illumina, 10x Genomics and Strand-Seq sequencing technologies.
 \item $\GMEC$: A \MEC instance is called \emph{gapless} if the entries in each of the $n$ rows of $\mathcal{F}$ follows a regular expression \texttt{$-^*\{0,1\}^*-^*$}. 
 These instances are generated from PacBio, ONT and single-end Illumina like technologies.
 \item $\BMEC$: Instances in which entries in each of the $n$ rows of $\mathcal{F}$ are from $\{0,1\}$ with no gaps.
\end{itemize}

We now illustrate these \MEC instances for a single genome from different sequencing technologies through an example.
\begin{example}
In Figure~\ref{fig:ex_MECs}, the example shows a mathematical representation of reads from different sequencing technologies, that cover seven variants. 
 The top panel shows a general \BMEC instance consisting of binary values with no gaps, the middle panel shows a \GMEC instance with binary values in between and gaps at its two ends and, the bottom is 
a \MEC instance which consists of binary values and gaps, that are randomly distributed in the rows.
\end{example}



\begin{figure}[t!]\centering
\includegraphics[width=\columnwidth]{{ex-MECs}.pdf}
\caption{Seven variants covered by reads (horizontal bars) in a single individual are represented as \MEC instances. At the top is a general \MEC instance with arbitrary gaps, the middle is a \GMEC instance with gaps only at its two ends and the bottom is 
a \BMEC instance which consists of only binary values.}
\label{fig:ex_MECs}
\end{figure}

Additionally, we consider a weighted version of the MEC (wMEC), in which a cost is associated to every matrix entry.
This is useful in practice since each nucleotide in a sequencing read usually comes with a ``phred-scaled'' base quality $Q$ that corresponds to an estimated probability of $10^{-Q/10}$ that this base has been wrongly sequenced.
These phred scores can hence serve as costs of flipping a letter, allowing less confident base calls to be corrected at a lower cost compared to high confidence ones.

\begin{problem}[wMEC]
Given a matrix $\mathcal{F}\in\{0,1,-\}^{R\times M}$ and a weight matrix $\mathcal{W}\in\N^{R\times M}$, flip entries in $\mathcal{F}$ to obtain a feasible matrix, while minimizing the sum of incurred costs, where flipping entry $\mathcal{F}(j,k)$ incurs a cost of $\mathcal{W}(j,k)$.
\label{prob:wmec}
\end{problem}

Similar to MEC formulation and its versions, there are other objective functions useful for solving haplotype assembly problem.

\begin{figure}[t!]\centering
\includegraphics[width=\columnwidth]{{integrative_datasets}.pdf}
\caption{Variants covered by reads in a single individual are represented as \MEC instances from different sequencing technologies. The weights are shown in red. Figure from a paper by \cite{klau2017guided}.}
\label{fig:ex_all_datas}
\end{figure}
% \todo{maybe also define heterozgous and homogygous variants, explain genotypes too.}
% \todo{define read-length and coverage here.}

\subsubsection{Other problem formulations}
The other problem formulations to solving haplotype assembly problem are as follows:
\begin{itemize}
 \item Minimum fragment removal (MFR) and its weighted version (WMFR): These objective functions derive the haplotype assembly by removing rows of the matrix $\mathcal{F}$.
 \item Minimum SNP removal (MSR) and its weighted version (WMSR): These objective functions derive the haplotype assembly by removing the columns of the matrix $\mathcal{F}$.
 \item Minimum fragment cut (MFC): This function involves partitioning of the rows into two segments representing haplotypes.
 \item Other objective functions such as Graph and Satisfiability (SAT) formulations.
\end{itemize}

However, this study does not focus on these formulations. Specifically, the focus of the study is on the Minimum Error Correction (MEC) formulation to solving haplotype assembly problem.
\subsection{Main approaches}
We present here the algorithmic approaches, mainly focused on the Minimum Error Correction formulation to solving the haplotype assembly problem, both in theory and practice.
\subsubsection{Theoretical approaches}
In this section, we discuss theoretical approaches for the \MEC and its versions for solving haplotype assembly.
For the simplest version \BMEC, it is easy to see that finding its optimal solution is equivalent to solving the hypercube 2-segmentation problem (H2S).
The H2S problem was introduced by \cite{KPR98_segmentation,KPR04_segmentation} and is known to be $\NP$-hard \citep{Fei14_np,KPR04_segmentation}.
The optimization version of \BMEC  differs from H2S, in the former we minimize the number of mismatches instead of maximizing the number of matches.
In particular, the $\NP$-hardness of H2S directly implies the $\NP$-hardness of \BMEC, \GMEC, and \MEC.
\GMEC and \MEC were shown to be $\NP$-hard by \cite{Cilibrasi2007}.\footnote{Their result predates the hardness result of \cite{Fei14_np} for H2S. The proof of the claimed $\NP$-hardness of H2S by ~\cite{KPR98_segmentation} was never published.}

To guarantee approximation status, \cite{OR02_polynomial} obtained a polynomial time approximation scheme (PTAS) (see Definition~\ref{def:ptas}) for \BMEC based on random embeddings.
Building on the work of \cite{LMW02_finding}, \cite{JXL04_k} presented a deterministic PTAS for \BMEC.
\GMEC is a generalization of \BMEC, with logarithmic factor approximation as the best known approximation algorithm so far.
Furthermore, the most generalized version is the \MEC that contains arbitrary gaps and binary values. The \MEC is known to be APX-hard. 
Therefore, there is a gap on the approximation status of \GMEC,  whether it is as easy as \BMEC or as hard as \MEC, is unknown.

% Additionally, they showed that allowing a single gap in each strings renders the problem $\APX$-hard.
% More recently, Bonizzoni et al.~\citep{BDK+16_minimum} showed that it is unique games hard to approximate \MEC with constant performance guarantee, whereas it is approximable within a logarithmic factor in the size of the input. 
% To our knowledge, previous to our result their logarithmic factor approximation was also the best known approximation algorithm for \GMEC.

\begin{gaps}
 The approximation status of \GMEC is an open problem; \GMEC instances are very important and are often produced by single-ended PacBio or ONT technologies.
 Deriving polynomial time approximation algorithms (PTAS) for \GMEC instances provide evidence that the problem can be solved in polynomial time.
 \label{gap:gap1}
\end{gaps}

\subsubsection{Approaches in practice}
Here, we discuss exact, as well as heuristic approaches, which are generally used in practice for solving haplotype assembly problem.
\paragraph{Exact approaches} The exact approaches, which solve the problem optimally, include integer linear programming \citep{Fouilhoux2012,CDW13_exact}, and fixed-parameter tractable (FPT) algorithms \citep{he2010optimal,Patterson2015,Pirola2015}.

\textit{Integer Linear Programming (ILP)} consists of two parts, constraints or conditions and objective function. Additionally, the objective function and the constraints are linear.
An ILP in standard form is expressed as
\[{\begin{aligned}&{\text{maximize}}&&\mathbf {c} ^{\mathrm {T} }\mathbf {x} \\&{\text{subject to}}&&A\mathbf {x} +\mathbf {s} =\mathbf {b} ,\\&&&\mathbf {s} \geq \mathbf {0} ,\\&{\text{and}}&&\mathbf {x} \in \mathbb {Z} ^{n},\end{aligned}}\]
where $\displaystyle \mathbf {c} ,\mathbf {b} $, $\mathbf {x}$ and $\mathbf {s}$ are vectors, $\displaystyle A $ is a matrix and entries in $\mathbf {x}$ are integers.

\cite{CDW13_exact} focused on MEC formulation for haplotype assembly problem and proposed an ILP-based approach to solving this problem.
Basically, they consider the binary variables for each row and column and their corresponding values are supposed to be dependent on haplotypes.
By using some auxiliary variables, they relax the integer program to linear, which can be solved in polynomial time.
% \todo{do I need to mention the actual ILP program?}

\textit{Branch-and-Bound algorithm.}
A branch-and-bound algorithm consists of enumerating the candidate solutions by using a rooted tree.
The algorithm explores the branches of this tree, which represent subsets of the solution set.
Before enumerating the candidate solutions of a branch, the branch is compared to the upper and lower estimated bounds on the optimal solution, and is ignored if it cannot produce a better solution than the best one found so far by the algorithm.
\cite{wang2005haplotype} also focused on MEC formulation and applied a branch
and bound algorithm to finding haplotypes as optimal path in a binary tree. This approach solves the problem in an exact way, but does not scale well for large datasets.

\textit{Parameterized algorithms.} 
The parameterized algorithms consist of choosing fixed parameters and the sophisticated algorithms that allow to solving some problems in time exponential only in the size of a fixed parameter, but polynomial in the input size. 
Such an algorithm is called a fixed-parameter tractable (fpt-) algorithm, because the problem can be solved efficiently for small values of the fixed parameter.

Based on NGS data analysis, there are several parameters, such as read length, coverage and sequencing errors, that can help in solving genomics problems efficiently. 
Choosing a parameter, that is small enough to work in practice, is an art.
In the works by \cite{he2010optimal,Patterson2015,Pirola2015}, different parameters are proposed to solve the MEC formulation of haplotyping problem using NGS datasets.
It is shown that the parameterized algorithms using different types of NGS datasets independently, work well in practice for solving haplotyping \citep{martin2016whatshap, klau2017guided}. 

\begin{example}
 The corresponding \MEC instances for a single genome from different technologies such as Illumina, PacBio and Strand-Seq technologies are shown in Figure~\ref{fig:ex_all_datas}. 
 Shown is the toy example of \MEC instances for a human genome that consists of heterozygous variants, one in every 1000 bp. Over these variants, we observe that the matrix from Illumina data contains more gaps compared to binary values in every row because of the short read nature.
 The matrix from long-read technologies such as PacBio and ONT contain more binary values in every row. The matrix from Strand-Seq data is very sparse with more gaps, and arbitrary positioning of gaps and binary values. 
 The reads from any technology independently do not fill all the columns in the matrix with binary values and therefore, cannot generate end-to-end haplotypes.
 The joint matrix from different combinations of technologies contains binary values in all the columns and thus can produce end-to-end haplotypes.
\end{example}

As described in Section~\ref{sec:dna_seq}, the NGS datasets have their inherent challenges and benefits and, no technology independently enables producing the complete haplotypes.
For instance, long-read technologies have high sequencing error rates and short-read technologies produce short reads, which are not sufficient to producing the whole genome.
This necessitates to optimally combine the advantages of all technologies available at hand, in a joint framework, in order to facilitate producing the whole genome sequences.
However, an efficient algorithm to solve haplotyping, by combining all sequencing datasets in an integrative framework, is unknown.
Therefore developing a parameterized algorithm for this integrative framework and deciding parameters that work well in practice is very important. 

\begin{gaps}
Finding a parameterized algorithm for a version of MEC that uses multiple sequencing technologies in an integrative framework, is an open problem.
\label{gap:gap2}
\end{gaps}

\begin{figure}[t!]\centering
\includegraphics[width=\columnwidth]{{pedigree}.pdf}
\caption{Seven SNP loci covered by reads (horizontal bars) in three individuals. Unphased genotypes are indicated by labels 0/0, 0/1 and 1/1. The alleles that a read supports are printed in white.}
\label{fig:ex_pedigree}
\end{figure}
\paragraph{Heuristic approaches}
A heuristic algorithm is one that is designed to solve a problem in a faster and more efficient fashion in terms of speed and memory requirements, at the cost of sacrificing optimality.
Heuristic algorithms are most often employed when approximate solutions are sufficient and exact solutions are necessarily computationally expensive.

There is a lot of literature that relies on heuristic approaches for haplotyping.
HASH (haplotype assembly for single human) used a Markov chain Monte Carlo (MCMC) algorithm and graph partitioning approach to assemble haplotypes,
given a list of heterozygous variants and a set of shotgun sequence reads mapped to a reference genome assembly \citep{bansal2008mcmc}. 
In the work by \cite{wang2007clustering}, a clustering algorithm is deployed into split the rows of $\mathcal{F}$ in two sets based on MEC formulation.
HapCut \citep{Bansal2008} utilized the overlapping structure of the fragment matrix and max-cut computations to find the minimum error correction (MEC) solution for haplotype assembly. 
\cite{Duitama2010} followed a heuristic approach for max-cut to find haplotypes efficiently. 
MixSIH \citep{matsumoto2013mixsih} utilized a probabilistic mixture model to solve haplotyping.
H-BOP \citep{xie2012fast} followed a heuristic algorithm for optimizing a combination of the MEC and Maximum Fragments Cut models.
ProbHap \citep{Kuleshov2014b} fundamentally employed a similar approach to WhatsHap \citep{Patterson2015}, but uses the Viterbi algorithm to solve the maximum likelihood function specified by a probabilistic graphical model. 
% % https://www.ncbi.nlm.nih.gov/pmc/articles/PMC2935405/
% \textit{Clustering algorithms.} In the work by \citep{wang2007clustering}, a clustering algorithm is used to split the rows of $\mathcal{F}$ in two sets. 
% The main contribution consists in the combination of the two distance functions used by the clustering algorithm. 
% The first distance is the Hamming distance as defined in Equation~\ref{eq:distance}. This distance takes into account only the number of mismatches between two fragments. 
% The second distance $D'$ also takes into account the number of matches between the two fragments.
% This means that given a certain fixed number of mismatches between two fragments, the more they overlap the closer they are.
% Using the above distance functions, a simple iterative clustering procedure is given as follows.
% \begin{enumerate}
%  \item for each possible pair of fragments in the SNP matrix the generalized Hamming distance is computed. Let $r_1$ and $r_2$ be the two furthest fragments according to Hamming distance, the two sets are initialized as $C1 = r_1$ and $C2 = r_2$.
%   \item Let H1 and H2 be the two consensus strings derived from C1 and C2: all the fragments are compared with H1 and H2 and assigned to the corresponding closer set. If a fragment is equidistant from the two consensus strings, the distance $D'$ is used to decide to which set assign the fragment.
%   \item Once all fragments are assigned, the consensus strings H1 and H2 are updated and the algorithm restarts from (2). The procedure loops until a stable haplotype pair is found (i.e. when the consensus haplotypes are the same before and after the update).
% \end{enumerate}
% 
% \textit{Max-Cut based algorithm.}
% HapCUT~\citep{Bansal2008} approaches the haplotype assembly as a MAX-CUT problem.
% Given a certain haplotype pair $H$, a graph $G(H)$ is constructed such that there is a vertex for each column of the matrix $\mathcal{F}$ and 
% there is an edge between two vertices of $G(H)$ if the corresponding columns in $\mathcal{F}$ are linked by at least one fragment. 
% Consider the fragment $\mathcal{F}(j)$ such that it covers both positions k1 and k2. Let $\mathcal{F}(j)[k1,k2]$ and H[k1, k2] represent the restriction of  $\mathcal{F}(j)$ and H to loci k1 and k2. 
% There are two cases: $\mathcal{F}(j)[k1,k2]$ matches one of the two haplotype strings of H[k1, k2], or $\mathcal{F}(j)[k1,k2]$ does not match any. 
% The weight w (k1, k2) associated with the edge between node k1 and k2 in the graph $G(H)$ is given by the number of fragments such that $\mathcal{F}(j)[k1,k2]$ does not match any string in 
% H[k1, k2] minus the number of fragments such that the match exists. 
% The higher w(k1, k2), the weaker is the correlation between the haplotype pair H and the SNP matrix restricted to columns k1 and k2. 
% Let (S, $\mathcal{F}$ - S) be a cut of G, the weight of the cut is defined as follows:
% \[ w(S) = \sum_{j\in S,k\in \mathcal{F}- S} w(j,k)\]
% 
% Consider the haplotype pair $H_S$ derived from $H$ by flipping all the elements involved in S. It is shown that the problem of finding a haplotype pair minimizing the \MEC score
% is reduced to the problem of finding a max-cut in $G(H)$. To solve max-cut problem, HapCut initializes the random haplotype pair, and then iteratively attempts to refine the haplotype pair to reduce \MEC score
% till it is no longer possible to further reduce it.

In addition, there are some pedigree based approaches for performing haplotype assembly for single individuals.

\subsection{Pedigree of genomes}
Another approach to haplotyping takes into account the sequencing datasets from families of genomes.
Specifically, such an approach takes advantage of two things: one is the sequencing data itself of each individual and the other is the principles of the Mendelian segregation of alleles in pedigrees. 
There are alleles that are specific to a single founding chromosome within a pedigree, which are referred to as lineage-specific alleles. 
These are highly informative for identifying haplotypes that are identical-by-descent (IBD) between individuals within a pedigree.
At the simplest level of a family trio (both parents and one child), very simple rules indicate which alleles in the child were inherited from each parent, thus largely separating the two haplotypes in the child.  
Nevertheless, genetic analysis cannot phase positions in which all family members are heterozygous. 
In cases where genetic analysis information is missing, using sequencing dataset is an excellent choice.
The genetic analysis largely supplement with the sequencing datasets for complete haplotype assembly.
% Furthermore, it is not always feasible to recruit the required participants for family-based studies. 
% In the absence of a family context, molecular haplotyping is an excellent choice because it does not require DNA samples from other family members. 
% The sequencing based haplotyping largely supplement the need for genetic analysis.

The Haploscribe method \citep{Roach2011} phased whole-genome data based on genetic analysis. Haploscribe followed a parsimony approach to generate
meiosis-indicator (inheritance state) vectors and obtained haplotypes by modeling haplotyping problem using a hidden Markov model (HMM). 
Other tools by \cite{abecasis2002merlin, williams2010rapid} are also based on genetic analysis information. 

All these previous approaches lack the joint usage of both information sources based on IBD and sequencing datasets of individuals in the pedigree. 
Finding an efficient method that uses both sequencing data and genetic inheritance principles in an integrative fashion, for performing phasing, is very important to generating complete haplotypes.


\begin{gaps}
 Combining both principles of genetic inheritance and sequencing reads into one framework is an open problem. Furthermore, to come up with an efficient algorithm that works well in practice is an important question.
 \label{gap:gap3}
\end{gaps}
\begin{example}
 To illustrate the motivation to combine genetic and read-based haplotyping, the corresponding \MEC instance for a single genome is shown in Figure~\ref{fig:ex_pedigree}. 
There are seven SNP positions covered by reads in three related individuals. 
It illustrates how the ideas of genetic and read-based haplotyping complement each other. 
All genotypes at SNP 3 are heterozygous. 
Thus, its phasing cannot be inferred by genetic phasing, that is, using only the given genotypes and not the reads. SNP 4, 
in contrast, is not covered by any read in the child. When only using reads in the child (corresponding to single-individual read-based phasing), 
no inference can be made about the phase of SNP 4 and neither about the phase between SNP 3 and SNP 5. 
The phases of all SNPs for child can be easily inferred based on the observation that all seven child genotypes are compatible with the combination of brown and green haplotypes from the parents. 
This example demonstrates that using pedigree information, genotypes and sequencing reads jointly are very powerful for establishing phase information.
\end{example}
\begin{figure}[t!]\centering
\includegraphics[width=\columnwidth]{{assembly_graphs}.pdf}
\caption{Figure shows the reads and reconstructed haplotypes using two graph approaches: (a) de Bruijn graph and (b) overlap graph.}
\label{fig:assembly_graphs}
\end{figure}

In summary, we have presented an overview of computational approaches for performing haplotype assembly by using reference genome as a backbone. 
The main point is the reference genome used, which may hinder the correct downstream analyses in some cases.
First, we cannot phase variants that are de novo or rare in the target genome.
Second, we make the prior hypothesis that the target genome is very close to the reference, which may not always be true in reality.
Third, the method is obviously not self-sufficient since a prior reference needs to be
constructed. For these reasons, we additionally consider \textit{haplotype-aware de novo} (without reference) assembly, which is also known as \textit{diploid} assembly.

\section{Diploid assembly}\label{sec:dip}
In diploid assembly, the reads from the diploid genome are directly used to assemble the haplotype sequences. The obtained haplotype sequences are known as diploid assemblies.
The diploid assembly process involves partitioning the input read set into two sets and then gluing together the reads from each set in proper order to produce diploid assemblies.
Therefore, it is necessary to know the \textit{ordering} and \textit{haplotypic} identity of these reads. 
The diploid assembly process is challenging due to short read lengths, incomplete data, sequencing errors and repetitive regions on the genome.
% The other challenges occur are, reads are very short compared to the genome size and are not long enough to span the whole repeats.

How to formulate the diploid assembly problem to finding the ordering and haplotypic identity of reads for producing diploid assemblies, while avoiding misassemblies in complex repetitive regions, is discussed below.

\subsection{Diploid assembly as graph problem}
The diploid assembly from sequencing reads is modeled as the assembly graph problem.
The assembly graph restores reliable information about the \textit{ordering} of reads.
Assembly graphs can be categorized into two families: overlap graphs and de Bruijn graphs.
\paragraph{Overlap graph}
Given a set of reads, the overlap graph consists of nodes that represent reads and the edges represent the overlap between read sequences.
The weight on the edges represents the maximal overlap length between two sequences. As illustrated in Figure~\ref{fig:assembly_graphs}b, 
given four reads, the goal is to reconstruct the haplotypes sequences H0 and H1.
In overlap graph, there are four nodes R1, R2, R3 and R4 for these reads and there are edges between these nodes based on the overlap, for example, there is a link between R1 and R2 with a weight of 3.

The overlap graphs can be simplified to string graphs by the transitive reduction of edges.
Also, the contained edges are removed, which occur when one read is a substring of other reads.

Most of the assemblers generate only one sequence and the algorithm to construct this sequence (instead of two) can be outlined as follows.
\begin{itemize}
 \item Overlap: calculate pairwise overlaps between reads
 \item  Layout: look for a parsimonious solution (as a generalized Hamiltonian path visiting each node at least once while minimizing the total string length)
 \item  Consensus: merge reads, using redundancy to correct sequencing errors
\end{itemize}

The first OLC assembler was Celera \citep{myers2000whole}, which was designed to handle sequencing data from Sanger technology.
Celera uses a BLAST-like approach to performing all-vs-all read alignment. It then compacts the overlaps with no ambiguity and applies some heuristics on the complex regions involving repeats. The final sequences
are generated by removing sequencing errors. The complex repetitive regions are hard to resolve and, this results in fragmented assemblies.  We call these fragmented sequences
``contigs'' for contiguous consensus sequences. Furthermore, a series of contigs are connected using long-range read information to generate long assemblies called ``scaffolds''.

This paradigm was used with long Sanger sequences and for relatively small
genomes. Currently, the OLC-based algorithms are also used with PacBio datasets. 
Because of the cost of the pairwise overlaps computation, the OLC is too time
consuming for large NGS datasets. Thus, other solutions are required to deal with the amount of reads to assemble large genomes.
\bigskip

\begin{figure}[t!]\centering
\includegraphics[width=\columnwidth]{ex_sv.pdf}
\caption{Given the input reads (middle) from the two sequences (top), we show a corresponding assembly graph at the bottom.
The bubbles in the sequence graph (bottom) show three different heterozygous variations; the first one is an SNV, the second one is an SV, and the third one is an indel. }
\label{fig:ex_sv}
\end{figure}

\paragraph{De Bruijn graphs}
% https://genome.sph.umich.edu/w/images/b/b4/666.2012.01.pdf
In this type of assembly graph, each read is broken into a sequence of overlapping k-mers. The distinct
k-mers are added as vertices to the graph, and k-mers that originate from adjacent positions in
a read are linked by an edge.
In Figure~\ref{fig:assembly_graphs}a, the reads are divided into words of fixed length $k$, where $k=4$. Here,
each node in the graph is a word and the connection between the nodes is based on the overlap between nodes.
Basically, the de Bruijn graph is a directed graph representing overlaps
between sequences of symbols, named after Nicolass Govert de Bruijn \citep{todd1933combinatorial}. Given an
alphabet $\sigma$ of $n$ symbols, a $k$ dimensional de Bruijn graph has the following properties.
\begin{enumerate}
 \item $n^k$ vertices produced by all words of length $k$ from alphabet $\sigma$
 \item Two vertices X and Y are connected if and only if the $k$ - 1
suffix of X is equal to the $k$ - 1 prefix of Y.
\end{enumerate}

The first application of the de Bruijn graph in genome assembly was introduced into
the EULER assembler \citep{pevzner2001eulerian} in order to tackle assembly complexity. 
The assembly problem can then be formulated as finding a walk
through the graph that visits each edge in the graph once --- an Eulerian path problem.
Due to the repeats, it is hard to find the Euler path. 
In most instances, the assembler attempts to construct
contigs consisting of the unambiguous, unbranching regions of the graph.

\paragraph{De Bruijn graph and overlap graph}
The de Bruijn graph theoretically achieves
the same tasks that the overlap graph does, but in an efficient manner.
The de Bruijn graph became widely used when the short reads from NGS appeared. The OLC
approach did not scale well on the high number of sequences generated by NGS. The
use of the de Bruijn graph is very interesting for short read assembly for its ability to
deal with the high redundancy of such sequencing in a very efficient way. Indeed a k-mer
present dozens of times in the sequencing dataset appears only once in the graph. This
makes the de Bruijn graph structure not very sensible to high coverage, unlike
OLC. The de Bruijn graph was first proposed as an alternative structure \citep{pevzner2001eulerian} because it
was less sensible to repeats. Repeats that were problematic in the OLC, creating very
complex and edges heavy zones, are collapsed in the de Bruijn graph.

From the above, it follows that the assembly graphs contain the information about ordering of reads. 
We now discuss the special structures in assembly graphs, that can occur due to repeats and heterozygous regions.

\paragraph{Graph structures} There are mainly two types of structures (bubbles and repeats) that occur in the assembly graphs for diploid genomes.

\textit{Bubbles.}
Bubbles are defined as a set of disjoint paths that share the same start and end nodes.
Bubbles in the graph represent heterozygosity for the diploid organism.
Bubbles can contain simple SNVs with only one allele difference, or even large complex structural variations in the order of a few kilo-bases.  
Figure~\ref{fig:ex_sv} illustrates how the bubbles in an assembly graph can contain both small variants (SNPs and indels up to several dozen base-pairs in length) and large structural variants.
\begin{figure}[t!]\centering
\includegraphics[width=\columnwidth]{repeats.pdf}
\caption{At the top, shown are the heterozygosity (in vertical bars) and repetitive regions (in red) over the genome. At the bottom, shown is the graph with nodes as heterozygous or repetitive region, and connections are based on the successive read overlap.
The graph has cycles because of repetitive region shown by R, which also causes two branches.}
\label{fig:repeats}
\end{figure}

\textit{Repeats.} The repeats over the genomes cause branches or cycles in the assembly graphs and, therefore, make a graph more complex and break its linear chain properties.
This is illustrated in Figure~\ref{fig:repeats}. In this example, the repeat $R$ causes cycles and branches in the graph.
Assemblers generally handle these repeats by making a guess as to which branch to follow.
Incorrect guesses create false joins (chimeric contigs) and erroneous copy numbers. 
If the assembler is more conservative, it will break the assembly at these branch points, leading to an accurate but fragmented assembly with fairly small contigs.
The maximal repeat resolution  depends on the read-length. If there is a read that is long enough to span the repeat region, then the repeat is resolvable.
Therefore, upcoming long-read sequencing technologies have the power to obtain maximal repeat-resolved diploid assemblies.

We have explored the assembly graphs and their special structures. We now provide the main approaches followed for reconstructing the genome from NGS data. 
\begin{figure}[t!]\centering
\includegraphics[width=\columnwidth]{ex_graph_approach.pdf}
\caption{Input: an assembly graph (top) (consisting of four SNVs and two SVs) and the PacBio reads $r_1, r_2, r_3, r_4, r_5, r_6$ (gray). 
Output: the phased reads (colored in blue and red) and haplotigs (bottom) using Falcon Unzip and our graph-based approach. Our graph-based phases central region, contrarily, Falcon Unzip does not.  }
\label{fig:ex_graph_approach}
\end{figure}

\subsection{Main approaches for diploid assembly}
Over the last decade, the development of various NGS technologies has impacted the assembly problem.
In theory, the problem of \textit{de novo assembly}---computing the consensus of two or more sequences---is NP-hard, when the problem is modeled either as string graphs or de Bruijn graphs \citep{medvedev2007computability}. 
There are several heuristic approaches to approximate the optimal de novo haploid assembly based on NGS datasets \citep{idury1995new, myers1995toward, myers2005fragment, pevzner2001eulerian, nagarajan2009parametric, nagarajan2013sequence, sovic2013approaches}.

However, even with Sanger (reads of the order of 800-1000 base pairs) and Illumina sequencing, which deliver short reads with low error rates, de novo assembly of heterozygous diploid genomes has been a difficult problem \citep{vinson2005assembly, levy2007diploid}.
In practice, there are several short-read assemblers based on Illumina data for heterozygous genomes \citep{kajitani2014efficient, pryszcz2016redundans, simpson2012efficient, bankevich2012spades, li2015fermikit}.
The assemblies that they produce are accurate, but contain gaps and are composed of relatively short contigs and scaffolds. 
Third generation sequencing technologies such as methods available from Pacific Biosciences (PacBio) and Oxford Nanopore Technologies (ONT) deliver much longer reads, but with high error rates.
There are now several long-read assemblers \citep{koren2017canu, vaser2017fast, xiao2016mecat, berlin2015assembling, chin2013nonhybrid, hunt2015circlator, lin2016assembly} that use these long-read data for de novo assembly.
The assemblies that are delivered from these assemblers are more contiguous, with longer contigs and scaffolds.
Finally, there are hybrid assemblers that take advantage of long-read data (with its high error rate) and short-read data (with its low error rate) \citep{bashir2012hybrid, antipov2015hybridspades, zimin2017hybrid} and attempt to combine the best aspects of both.
These hybrid assemblers have the power to deliver highly accurate, repeat-resolved assemblies.

The main drawback of above state-of-the-art assemblers is that they generate only one consensus sequence even for diploid organisms. To date, there is only one assembler that can produce diploid assemblies for diploid genomes.

\textit{Diploid assembly.} A recent and currently only available diploid assembly method --- Falcon Unzip \citep{chin2016phased} --- is purely PacBio based diploid assembler.
Falcon Unzip generates haplotype contigs or ``haplotigs'' that represent the diploid genome with correctly phased homologous chromosomes.
Falcon Unzip involves constructing a string graph from long PacBio reads, and generating haplotigs in a greedy manner using local conservative approach.


For generating haplotigs, Falcon Unzip first identifies phase for each read based on the condition that the read covers at-least one SNV for phasing.
In the regions over the genome where the SNVs are at a long distance to each other, the Falcon Unzip can not phase those regions, resulting in incomplete assemblies.
Additionally, the Falcon Unzip cannot phase all large structural variants and regions with high heterozygosity. 

% The pipeline is given in Figure~\ref{fig:falcon_unzip}.
% Falcon Unzip begins by using reads to construct a string graph that contains sets of ``haplotype-fused contigs'' , also called as ``primary contigs'', as well as bubbles representing divergent regions between homologous sequences (Fig.~\ref{fig:falcon_unzip}a). 
% Next, Falcon-Unzip identifies read haplotypes using phasing information from heterozygous positions that it identifies (Fig.~\ref{fig:falcon_unzip}b). 
% Phased reads are then used to assemble haplotigs and primary contigs (backbone contigs for both haplotypes) (Fig.~\ref{fig:falcon_unzip}c) 
% that form the final diploid assembly with phased single-nucleotide polymorphisms (SNPs) and structural variants (SVs).
% 
% \textit{Phasing using primary contigs.}
% In Falcon Unzip, the reads are aligned to primary contigs and heterozygous SNPs (het-SNPs) are called by analyzing the base frequency of the detailed sequence alignments.
% A simple phasing algorithm was developed to identify phased SNPs. 
% Along each contig, the algorithm assigns phasing blocks where ``chained phased SNPs'' can be identified. 
% Within each block, if a raw read contains a sufficient number of het-SNPs, it assigns a haplotype phase for the read unambiguously. 
% Combined with the block and the haplotype phase information, it assigns a ``block-phase'' tag for each phased read in each phasing block.
% Some reads might not have enough phasing information. For example, if there are not enough het-SNP sites covered by a read, it assigns a special 'un-phased tag' for each un-phased read.
% The initial assembly graph is fused using phased reads and the haplotigs are generated in a greedy manner using local conservative approach.
% % \todo{maybe add example how haplotigs from haplotype fused assembly graph works?}

% \begin{figure}[t!]\centering
% \includegraphics[width=\columnwidth]{{cropped_falcon-unzip}.pdf}
% \caption{(a) An initial assembly is computed by FALCON, which error corrects the raw reads (not shown) and then assembles them using a string graph of the read overlaps. 
% The assembled contigs are further refined by FALCON-Unzip into a final set of contigs and haplotigs. 
% (b) Phase heterozygous SNPs and group reads by haplotype. (c) The phased reads are used to open up the haplotype-fused path and generate as output a set of primary contigs and associated haplotigs.}
% \label{fig:falcon_unzip}
% \end{figure}

There is no known algorithm that works at different levels of heterozygosity, phases all types of structural variants and generates complete diploid assemblies.
A potential step to achieve this task of complete diploid assemblies is to phasing directly from the assembly graph.
Moreover, it becomes easier to detect large structural variants, such as translocations and other rearrangements, in an assembly graph.
Thus, working in the space of assembly graphs provides deep insights to detect all types of structural variation, which further helps in phasing whole genomes.

Additionally, the Falcon Unzip is purely PacBio based assembler, which uses very noisy data and, therefore, the Falcon Unzip requires high coverage data to producing accurate assemblies.
In contrast, hybrid approaches that combine accurate Illumina and long read PacBio data, conceptually have the potential to producing good quality assemblies even at low coverages.
However, there is no known algorithm, that combines mutiple sequencing datasets such as accurate Illumina and long read PacBio data, for producing good quality haplotigs.
\begin{gaps}
 Phasing bubbles directly from the assembly graph is an open problem. Additionally, the MEC formulation for phasing that works on graphs, by combining datasets from multiple sequencing technologies, is unknown. 
 \label{gap:gap4}
\end{gaps}

\begin{example}
Figure~\ref{fig:ex_graph_approach} demonstrates the conceptual advantages of our graph-based approach over the Falcon Unzip method.
Consider four SNVs separated by two large SVs and there are four reads spanning these variants.
Falcon Unzip can not phase the central region because the reads $r_3$ and $r_4$ do not cover any SNVs, resulting in incomplete haplotigs.
In contrast, the graph-based approaches attempt to detect all types of SVs and phase all of them.
\end{example}
Based on the above example, we observe that it is possible to deliver complete and contiguous haplotigs using assembly graph-based approach.

% \begin{figure}[t!]\centering
% \includegraphics[width=\columnwidth]{ex_sv.pdf}
% \caption{Based on reads (middle) from the two sequences (top), the bubbles in the graph (bottom) show three different heterozygous structural variations; first is a SNV, second is a SV and third is an indel. }
% \label{fig:ex_sv}
% \end{figure}
\section{Outline of our contributions}
In the above section, I highlighted four ``open problems'' in the arena of haplotyping using NGS data.

\begin{itemize}
 \item In Chapter 2, I provide a general background on the different types of algorithms. I outline the motivation on how these algorithms are used in solving small daily examples fast. 
 I highlight the advantages and disadvantages of these algorithms in context of large problems.
 \item Solving MEC instances using NGS data are NP-hard. In Chapter 3, I describe the dynamic programming based algorithm to solve these instances approximately.
 I discuss the approximation guarantee, which provides hints that these instances can be solved in practice in polynomial time. (Problem~\ref{gap:gap1})
  \item In Chapter 4, I discuss different types of NGS datasets, with their advantages and disadvantages. I describe the integrative phasing framework obtained by combining NGS datasets. I discuss the parameterized algorithm that solves these instances efficiently in practice. 
  I demonstrate the effectiveness of this algorithm on real genomic datasets. (Problem~\ref{gap:gap2})
 \item In Chapter 5, I describe the generalized parameterized approach to incorporate information from pedigrees. I show the experiments on real datasets and highlight that pedigree data has an additional advantage in delivering better quality haplotypes. (Problem~\ref{gap:gap3})
 \item In Chapter 6, I describe the generalized approach --- haplotype-aware diploid assembly --- in a graph framework, that has the ability to handle all levels of heterozygosity and structural variations to produce accurate and complete haplotype assemblies.
 I present this approach as a hybrid of different types of NGS datasets and show its effectiveness on the pseudo-diploid genome. (Problem~\ref{gap:gap4})
 \item Finally, in Chapter 7, I provide a summary of the results presented in this thesis, along with the future outlook and perspectives.
\end{itemize}

\section{Relevant publications}
\begin{itemize}
 \item David Porubsky*, \underline{Shilpa Garg}*, Ashley D. Sanders*, V. Guryev, Peter M. Lansdorp, T. Marschall,
\textit{Dense And Accurate Whole-Chromosome Haplotyping Of Individual Genomes}, Nature Communications, 2017.

In this paper, my contribution was in developing an integrative pipeline and writing draft for WhatsHap section.
\item \underline{Shilpa Garg}, Marcel Martin and Tobias Marschall, \textit{Read-Based Phasing of Related Individuals},
Proceedings of ISMB 2016/Bioinformatics.
\item \underline{Shilpa Garg}, Mikko Rautiainen, Adam M Novak, Erik Garrison, Richard Durbin, Tobias Marschall, \textit{A graph-based approach to diploid genome assembly}, ISMB 2018 (to appear).
\item \underline{Shilpa Garg}, Tobias Moemke, \textit{A QPTAS for Gapless-MEC}, Submitted.
\item Preprint: M. Martin*, M. Patterson*, \underline{Shilpa Garg}, S. O. Fischer, N. Pisanti, G. W. Klau, A. Schnhuth, T.
Marschall, \textit{WhatsHap: fast and accurate read-based phasing}.

In this paper, my contribution was in developing some parts of the pipeline and making figures.
\end{itemize}


% \subsection{Issues we address}
% To address those problems, we will present new algorithmic approaches. 
% \begin{itemize}
% \item In the second chapter, we present the fundamental concepts of different type of algorithms.
%  \item In the third chapter, we present dynamic programming based algorithm to prove the near-polynomial approximation status of \GMEC.
%  \item In the forth chapter, we present a parameterized algorithm to solve \MEC instances integatively from different datasets.
%   \item In the fifth chapter, we present a integrative framework to solve sequencing-based and genetic haplotyping, helps to generate complete and accurate haplotypes.
%  \item In the sixth chapter, we introduce new way to represent the assembly graph and futher, finding long read paths in the graph based on different types of datasets, which futher helps in better phasing. 
% \end{itemize}


% 
% 
%  
% 
% \todo{cover these issues in approaches}
% \subsection{Diploid genome assembly hardness}
% \begin{itemize}
% \item Theoretical approximation gurantee on gapless-MEC. It is important because even for high coverages, we can solve it in polynomial time approximately.
%  \item integrating datasets to produce more accurarte and complete
%  \item non-reference denovo based, can not detect large SVs, directly from graph, hybrid
%  \item pedigree of genomes
% \end{itemize}
% 
% \subsection{Goals and achievements}
% \begin{enumerate}
%  \item DNA genomes ranging from small yeast like genome to larger ones like human.
%  \item end-to-end full genome sequences
%  \item efficient algorithms to generate optimal or near-optimal solution.
% \end{enumerate}
% 
% 
% 
% 
% \subsection{Outline of our contributions}
% \begin{enumerate}
%  \item Chpater 1 consists of ...
%   \item Chpater 2 consists of ...
%    \item Chpater 3 consists of ...
% \end{enumerate}






