%*******************************************************
% Abstract
%*******************************************************
%\pdfbookmark[1]{Abstract}{Abstract}
\chapter*{Abstract}

Genomics has paved a new way to comprehend life and its evolution, and also to investigate causes of diseases and their treatment.
One of the important problems in genomic analyses is haplotype assembly. Constructing complete and accurate haplotypes plays an essential role in understanding population genetics and how species evolve.
In this thesis, we focus on computational approaches to haplotype assembly from third generation sequencing technologies.
This involves huge amounts of sequencing data, and such data contain errors due to the single molecule sequencing protocols employed. 
Taking advantage of combinatorial formulations helps to correct for these errors to solve the haplotyping problem.
Various computational techniques such as dynamic programming, parameterized algorithms, and graph algorithms are used to solve this problem.

This thesis presents several contributions concerning the area of haplotyping. First, a novel algorithm based on dynamic programming is proposed to provide approximation guarantees for phasing a single individual.
Second, an integrative approach is introduced to combining multiple sequencing datasets to generating complete and accurate haplotypes.
The effectiveness of this integrative approach is demonstrated on a real human genome.
Third, we provide a novel efficient approach to phasing pedigrees and demonstrate its advantages in comparison to phasing a single individual.
Fourth, we present a generalized graph-based framework for performing haplotype-aware de novo assembly. 
Specifically, this generalized framework consists of a hybrid pipeline for generating accurate and complete haplotypes from data stemming from multiple sequencing technologies, one that provides accurate reads and other
that provides long reads.

% In summary, this thesis introduces novel computational approaches to generating haplotypes, which make a significant contribution to
% the field of genetics, further opening up avenues for developing life saving technologies.

\chapter*{Kurzfassung}

Die Genomik hat neue Wege eröffnet, die es ermöglichen, die Evolution lebendiger Organismen zu verstehen, sowie die Ursachen zahlreicher Krankheiten zu erforschen und neue Therapien zu entwickeln.
Ein wichtiges Problem ist die Assemblierung der Haplotypen eines Individuums. Diese Rekonstruktion von Haplotypen spielt eine zentrale Rolle für das Verständnis der Populationsgenetik und der Evolution einer Spezies.
In der vorliegenden Arbeit werden Algorithmen zur Assemblierung von Haplotypen vorgestellt, die auf Sequenzierdaten der dritten Generation basieren. Dies erfordert große Mengen an Daten, welche wiederum Fehler enthalten,
die die zugrunde liegenden Sequenzierprotokolle hervorbringen. Durch kombinatorische Formulierungen des Problems ist die Rekonstruktion von Haplotypen dennoch möglich, da Fehler erfolgreich korrigiert werden können.
Verschiedene informatische Methoden, wie dynamische Programmierung, parametrisierte Algorithmen und Graph Algorithmen können verwendet werden, um dieses Problem zu lösen.

Die vorliegende Arbeit stellt mehrere Lösungsansätze für die Rekonstruktion von Haplotypen vor.
Als erstes wird ein neuartiger Algorithmus vorgestellt, der basierend auf dem Prinzip der dynamischen Programmierung Approximationsgarantien für das Haplotyping eines einzelnen Individuums
liefert.
Als zweites wird ein integrativer Ansatz präsentiert, um mehrere Sequenzierdatensätze zu kombinieren und somit akkurate Haplotypen zu generieren.
Die Effektivität dieser Methode wird auf einem echten, menschlichen Datensatz demonstriert.
Als drittes wird ein neuer, effizienter Algorithmus beschrieben, um Haplotypen verwandter Individuen simultan zu konstruieren und die Vorteile gegenüber der Betrachtung einzelner Individuen aufgezeigt.
Als viertes präsentieren wir eine Graph-basierte Methode um mittels Haplotypinformation de-novo Assemblierung durchzuführen. Dieser Methode kombiniert Daten stammend von verschiedenen Sequenziertechnologien, welche entweder
genaue oder aber lange Sequenzierreads liefern.


 

%%% Local Variables:
%%% mode: latex
%%% TeX-master: "../main"
%%% End:
