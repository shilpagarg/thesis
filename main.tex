\documentclass[
  % Replace twoside with oneside if you are printing your thesis on a single side
  % of the paper, or for viewing on screen.
%  oneside,
  twoside,
  11pt, a4paper, openright
]{book}



\usepackage[left=1in,right=1in,top=1in,bottom=1in]{geometry}
\usepackage{microtype}
\usepackage{cleveref}
\usepackage{xcolor}
\usepackage[draft]{fixme}
\fxusetheme{colorsig}
\FXRegisterAuthor{tgm}{atgm}{TM}
\FXRegisterAuthor{sg}{asg}{SG}
\usepackage{amsmath}
\usepackage{amssymb}
\usepackage{xspace}
\usepackage{complexity}
\usepackage{times}
\usepackage{flafter}
\usepackage{paralist}
\usepackage{enumitem}
\usepackage{amsthm}
\fxusetheme{colorsig}
\usepackage{kbordermatrix}

\renewcommand{\kbldelim}{(}% Left delimiter
\renewcommand{\kbrdelim}{)}% Right delimiter

\theoremstyle{plain}
% \newtheorem{theorem}{Theorem}
% \newtheorem{lemma}{Lemma}
% \newtheorem{corollary}{Corollary}
% \newtheorem{observation}{Observation}

\theoremstyle{remark}
\newtheorem{remark}{Remark}
\newtheorem{case}{Case}
% \newtheorem{claim}{Claim}[lemma]
% 
% \theoremstyle{definition}
% \newtheorem{definition}{Definition}

\newenvironment{subproof}{%
  \renewcommand{\qedsymbol}{$\Diamond$}%
  \begin{proof}[Proof of Claim]%
}{%
  \end{proof}%
}

\usepackage{enumitem}
\setlist{nolistsep, noitemsep, topsep=0pt}

\setcounter{secnumdepth}{2}

\usepackage{tikz}
\usetikzlibrary{calc}
\usetikzlibrary{decorations.markings}
\usetikzlibrary{shapes.geometric,positioning}
\usetikzlibrary{arrows}
\usetikzlibrary{patterns}
\tikzset{node/.style={minimum size=1.8mm,circle,fill=black,draw,inner sep=0pt},
         decoration={markings,mark=at position .5 with {\arrow[black,thick]{stealth}}}}
\tikzset{req/.style={minimum size=1.8mm,circle,fill=white,draw,inner sep=0pt},
         decoration={markings,mark=at position .5 with {\arrow[black,thick]{stealth}}}}

\usepackage[noblocks]{authblk}



\renewcommand{\epsilon}{\varepsilon}
\newcommand{\ie}{i.e.}
\newcommand{\eg}{e.g.}
\newcommand{\WLOG}{w.l.o.g.\xspace}
\newcommand{\cost}{\ensuremath{\mathrm{cost}}\xspace}
\newcommand{\MEC}{\textsc{MEC}\xspace}
\newcommand{\GMEC}{\textsc{Gapless-MEC}\xspace}
\newcommand{\BMEC}{\textsc{Binary-MEC}\xspace}
\newcommand{\SEG}{\textsc{SEG}\xspace}
\newcommand{\GSEG}{\textsc{Gapless-SEG}\xspace}
\newcommand{\dist}{\ensuremath{\mathrm{dist}}\xspace}
\newcommand{\gain}{\ensuremath{\mathrm{gain}}\xspace}
\newcommand{\agree}{\ensuremath{\mathrm{agree}}\xspace}
\newcommand{\euler}{\textrm{e}}
\newcommand{\SWC}{\textsc{SWC}_{\epsilon^3}}
\renewcommand{\st}{\textrm{start}}
\newcommand{\en}{\textrm{end}}
\newcommand{\lev}{\mathcal{L}}

\renewcommand{\DP}[1]{\ensuremath{\mathrm{DP}(#1)}\xspace}
%The following name is used, since \dp already has a meaning in latex.
\newcommand{\CDP}[1]{\ensuremath{\mathrm{dp}(#1)}\xspace}
\newclass{\opt}{opt}
\newclass{\reference}{ref}
\newclass{\UG}{UG}
\usepackage{lipsum}
\usepackage{mathtools}
\usepackage{amsmath}
\usepackage{amsthm}
\usepackage{acronym}
\usepackage{listings}
\usepackage{fancyhdr}
\usepackage[utf8]{inputenc}
\usepackage{color}
\usepackage{balance}
\usepackage{graphicx}
\usepackage{array}
\usepackage{tikz}
\usetikzlibrary{arrows,calc,matrix,shapes,trees,positioning}
\usepackage[boxed,vlined,linesnumbered]{algorithm2e}
\usepackage{booktabs}
\usepackage{tabularx}
\usepackage{rotating}
\usepackage{paralist}
\usepackage{subfigure}
\usepackage{datetime}
\usepackage{multirow}
\usepackage{tipa}
\usepackage{pgfplots}
\usepackage{pgfplotstable}
\usepackage{soul}
\usepackage[T1]{fontenc}
\usepackage{lmodern}
\usepackage{url}
\usepackage[font={bf}]{caption}
\usepackage{amsmath}
\usepackage{amssymb}
\usepackage{listings}
\lstset{language=Java}
\usepackage{moreverb}
%\renewcommand*\ttdefault{lmvtt}
\SetKwComment{Comment}{$\triangleright$\ }{}
\usepackage[english]{babel}
\usepackage{setspace}
\usepackage[bookmarks]{hyperref}
\usepackage{xcolor}
\usepackage{mathtools}
%\usepackage{natbib}
\usepackage{enumitem}
\usepackage{pdflscape}
\usepackage{appendix}
% \usepackage[object=vectorian]{pgfornament}

\setlist[itemize]{leftmargin=*}
\DeclareMathOperator*{\argmin}{arg\,min}


%%Algorithm style
\SetAlFnt{\sffamily}
\setlength{\algoheightrule}{0.8pt} % thickness of the rules above and below
\setlength{\algotitleheightrule}{0.8pt} % thicknes of the rule below the title
%%


\usepackage{titlesec}
\titleformat{\chapter}[display]
{\normalfont\bfseries}
{{\LARGE {\it Chapter}~~}{\HUGE \thechapter}}
{1ex}
{\vspace{2ex}  \huge }
[\vspace{1ex}]

\usepackage{etoolbox}

\makeatletter
\patchcmd{\ttlh@hang}{\parindent\z@}{\parindent\z@\leavevmode}{}{}
\patchcmd{\ttlh@hang}{\noindent}{}{}{}
\makeatother

\hypersetup{
    colorlinks,
    linkcolor={red!50!black},
    citecolor={red!50!black},
    urlcolor={blue!80!black}
}

%\usepackage{unicode-math}
\usepackage{libertine}
\usepackage[italic]{mathastext}
\renewcommand{\ttdefault}{cmtt}
%\renewcommand{\sfdefault}{lmss}
%\usepackage{euler}


\usepackage{bbm}
\usepackage{amsthm}
\usepackage{amssymb}
\usepackage{stmaryrd}
\usepackage{url}



\newcommand{\todo}[1]{\textbf{[TODO: #1]}}
\newcommand{\N}{\mathbbm{N}}
\newcommand{\cupdot}{\dot{\cup}}
\newcommand{\iverl}{\llbracket}
\newcommand{\iverr}{\rrbracket}
\newcommand{\abs}[1]{{\left\lvert #1 \right\rvert}}
\usepackage{amsfonts}

%\setmathfont{Latin Modern Math}
%\setmathfont[range=\mathit/{latin,Latin,num,Greek,greek}]{Linux Biolinum O Italic}
%\setmathfont[range=\mathup/{latin,Latin,num,Greek,greek}]{Linux Biolinum O}
%\setmathfont[range=\mathbfup/{latin,Latin,num,Greek,greek}]{Linux Biolinum O Bold}
%\setmathfont[range={\mathrm,0048-0057}]{Linux Biolinum O}

 \newtheorem{theorem}{\normalfont{\scshape Theorem}}[chapter]
 \newtheorem{lemma}{\normalfont{\scshape Lemma}}[chapter]
 \newtheorem{claim}{\normalfont{\scshape Claim}}[chapter]
 \newtheorem{proposition}{\normalfont{\scshape Proposition}}[chapter]
 \newtheorem{corollary}{\normalfont{\scshape Corollary}}[chapter]
 \newtheorem{observation}{\normalfont{\scshape Observation}}[chapter]
 \newtheorem{definition}{\normalfont{\scshape  Definition}}[chapter]
 \newtheorem{example}{\normalfont{\scshape Example}}[chapter]
 \newtheorem{problem}{\normalfont{\scshape Problem}}[chapter]
 \newtheorem{weakness}{\normalfont{\scshape Weakness}}[chapter]
  \newtheorem{gaps}{\normalfont{\scshape Open Problem}}[chapter]
 
 

 \newcommand{\squishlist}{
 \begin{list}{$\bullet$}
  { \setlength{\itemsep}{0pt}
     \setlength{\parsep}{1pt}
     \setlength{\topsep}{1pt}
     \setlength{\partopsep}{0pt}
     \setlength{\leftmargin}{3em}
     \setlength{\labelwidth}{1em}
     \setlength{\labelsep}{0.5em} } }
 \newcommand{\squishend}{\end{list}}

 %%%% Tikz related libraries
\usetikzlibrary{positioning,shapes,shadows,arrows}
\usetikzlibrary{patterns}
\usepgfplotslibrary{groupplots}
\usetikzlibrary{trees}
\setlength{\textfloatsep}{5pt}
\setlength{\floatsep}{0pt}
\setlength{\intextsep}{0pt}
%%%%

%%%% Misc
%\newcommand{\verbatimfont}[1]{\def\verbatim@font{#1}}%
%\verbatimfont{\sffamily}

\def\CC{{C\nolinebreak[4]\hspace{-.05em}\raisebox{.4ex}{\tiny\bf ++}}}
\def\Giraphpp{{Giraph\nolinebreak[4]\hspace{-.05em}\raisebox{.4ex}{\tiny\bf ++}}}
\def\vgap{\vspace*{0.3cm}}
\def\sf{\sffamily}
%%%%


%%% bib
% \usepackage[backref=true]{biblatex}
% \usepackage{hyperref}
% \DefineBibliographyStrings{english}{%
%   backrefpage = {page},% originally "cited on page"
%   backrefpages = {pages},% originally "cited on pages"
% }

 \usepackage{natbib}%
% \hypersetup{urlcolor=blue, colorlinks=true}
% \usepackage{backref}

%%% 


\setlength{\headheight}{15.2pt}

%%% Header conf
\fancypagestyle{fancybook}{%
    \fancyhf{}%
    % Note the ## here. It's required because \fancypagestyle is making a macro (\ps@fancybook).
    % If we just wrote #1, TeX would think that it's the argument to \ps@fancybook, but
    % \ps@fancybook doesn't take any arguments, so TeX would complain with an error message.
    % You are not expected to understand this.
    \renewcommand*{\sectionmark}[1]{ \markright{\thesection\ ##1} }%
    \renewcommand*{\chaptermark}[1]{ \markboth{\chaptername\ \thechapter: ##1}{} }%
    % Increase the length of the header such that the folios
    % (typography jargon for page numbers) move into the margin
    \fancyhfoffset[LE]{6mm}% slightly less than 0.25in
    \fancyhfoffset[RO]{6mm}%
    % Put some space and a vertical bar between the folio and the rest of the header
    \fancyhead[LE]{\nouppercase{\sf \thepage~|~~{\slshape \leftmark}}}%
    %\fancyhead[LE]{\thepage\hfill{\leftmark}}%
    \fancyhead[RO]{\nouppercase{\sf {\slshape \rightmark}~~|~\thepage}}%
%    \fancyfoot[C]{{\thepage}}
}

%% (more) Tikz definitions
% Graphs
\tikzstyle{boundary}=[circle,fill=black!25,minimum size=15pt,inner sep=0pt]
\tikzstyle{vertex}=[circle,draw,minimum size=15pt,inner sep=0pt]

%% tikz
 \tikzset{
    line/.style = {draw},
    block/.style = {rounded rectangle,fill, text centered, minimum height=0.4em, inner sep=1pt},
    greyblock/.style = {fill=black!10, rounded rectangle,draw, text centered, minimum height=1em},
}
%%


% Query
\definecolor{query-color}{RGB}{240,240,235}
\tikzstyle{query}=[draw=query-color,fill=query-color,inner sep=6pt]

%Queries

\newcommand{\sparqlq}{

}



%%% Local Variables:
%%% mode: latex
%%% TeX-master: "main"
%%% End:

%%


\usepackage{afterpage}

\newcommand\blankpage{%
    \null
    \thispagestyle{empty}%
%    \addtocounter{page}{-1}%
    \newpage}
    
\newcommand{\restated}[2]{
    \begin{trivlist}
    \item[\hskip\labelsep {\bf #1} (Restated)] \emph{#2}
    \end{trivlist}}

\begin{document}



%\pagenumbering{gobble}
\pagenumbering{roman}

\pagestyle{empty}
{

\centering

\vspace*{2cm}

{
\setstretch{1.5}

% \begin{center}
% \pgfornament[width=2cm,symmetry=h]{88}
% %\pgfornament[width=5cm,symmetry=h]{49}
% %\pgfornament[width=10cm]{49}
% \end{center}


{\LARGE \textbf{Computational Haplotyping: theory and practice }}%of Large Labelled Graphs}}





\vspace*{2cm}
{\large \textbf {Shilpa Garg}}

%\text{Max-Planck Institute for Informatics}

}

\vspace*{4cm}


\text{Thesis for obtaining the title of Doctor of Engineering}

\text{of the Faculty of Mathematics and Computer Science}

\text{of Saarland University}

%\vfill
\vspace*{4cm}

{ \large Saarbr\"ucken \\ YYYY}

\vfill
}


%%% Local Variables:
%%% mode: latex
%%% TeX-master: "../main"
%%% End:

%\pagestyle{plain}
{

\vspace*{2cm}
{

{\bf Colloquium}
\vspace*{5mm}

\begin{tabular}{lrl}
Date &\hspace*{3.9cm:}&\\
Place &:&\\
Dean &:&
\end{tabular}

\vspace*{1cm}

{\bf Examination Board}

\vspace*{5mm}
\begin{tabular}{lrl}
Chairman&\hspace*{2cm:}&\\
Reviewer &:&\\
Reviewer &:&\\
Reviewer &:&\\
Scentific Assitant &:&
\end{tabular}


}

%%% Local Variables:
%%% mode: latex
%%% TeX-master: "../main"
%%% End:

\pagestyle{fancybook}
\include{FrontBackMatter/abstract}
%%*******************************************************
% Dedication
%*******************************************************
\thispagestyle{empty}
%\pdfbookmark[1]{Dedication}{Dedication}

\vspace*{3cm}

\begin{center}
    To my mother
\end{center}

%*******************************************************
% Acknowledgments
%*******************************************************
%\pdfbookmark[1]{Acknowledgements}{acknowledgements}
\chapter*{Acknowledgements}

I would like to express my deep gratitude to all the people who supported me during this work.

First and foremost, I am grateful to my supervisor Dr Tobias Marschall who introduced me to this haplotyping problem.
He helped me to develop problem solving skills for solving a research problem
and provided me his able guidance in completing this thesis work. 
Thanks a million to Prof Volkhard Helms for his consent to give time to review this thesis at a short notice.
I would like to offer special thanks to Prof Thomas Lengauer for providing me his valuable remarks on an earlier version of some of the chapters of this thesis.
He provided me financial support and guided me about ways to pursue my passion.
I am thankful to Prof George Church for his valuable comments on an earlier version of this thesis.

I am deeply grateful to Prof Richard Durbin whose lab I visited for my research internship.
His enthusiasm for genomics and guidance on future career prospects largely motivated me.
Dr Tobias Mömke has been great source of inspiration and advice for me.
I am continually amazed by his problem solving ability, which provided me motivation to solve ambitious problems.

I am immensely thankful to all collaborators and coauthors for their support and assistance.
It was great working with David Porubsky and Ashley Sanders, and learning the magic of Strand-Seq technology.
Thanks to Mikko Rautiainen and vgteam for great discussions on the graph world of genomics.

I appreciate the discussions with my office-mate Ali Ghaffaari on violin and classical music.
Special thanks to Jana for always providing a friendly support and guidance.
I would like to extend my thanks to my group mates, who organized different events at weekends. The events were enjoyable experiences.

Jana, Fabio, Lara, Prabhav and Adam proofread a few chapters of this thesis and I greatly appreciate their valuable remarks and feedback.

I appreciate the travel financial support from Michelle Carnell, technical help from Achim, George and MPII-IST and other administrative help from Susanne and Ruth.  

Finally, I would like to express the gratitude to my family and friends, who greatly helped me in this journey.





%%% Local Variables:
%%% mode: latex
%%% TeX-master: "../main"
%%% End:

%\include{FrontBackMatter/declaration}
\include{FrontBackMatter/contents}



\pagenumbering{arabic}
\setcounter{page}{0}
\chapter{Introduction}

% In this chapter, we introduce the fundamentals of genetics and DNA sequencing. Furthermore, we introduce the important research problem in the field of genetics, the diploid genome assembly (haplotyping). 
% We will see how we can mathematically formualate the diploid assembly problem as a computer scientist.
% Then we provide a high level description of the main methods used in this problem.
% Thereafter, we describe the limits and challenges faced currently in this field. We finish this chapter by an outline of the thesis.


Genetics studies the phenomenon of \textit{life} at its most basic level, thus the science of genetics is important and fascinating.
% In this section we describe the basics about genetics and haplotyping. We provide motivation about why understanding genetics for different organisms is so important.
% https://www.gen.cam.ac.uk/undergraduate/whygenetics
% http://www.helsinki.fi/biosciences/genetics/
An important field in genetics is haplotyping.
Haplotyping is the process of determining the sequences of both copies of homologous chromosomes, which are inherited from each parent in diploid organisms.
Haplotyping has applications in different fields such as evolutionary studies, clinical diagnosis, precision medicine, and biotechnology. 
Current sequencing technologies allow for reading off the genome sequence, and thus the reconstruction of haplotypes is
possible, in principle. Unfortunately, sequencing is prone to errors and the use of advanced algorithms and models is essential to correct for errors, in order to reconstruct accurate haplotypes.
However, the process of correcting these errors poses various computational challenges.
In this thesis, we present our novel algorithms to addressing various computational challenges in this field.

\section{Genetics, DNA sequencing and Haplotyping}\label{sec:dna_seq}
Genetics is the study of genes, genetic variation, and heredity in living organisms. 
Genetics controls what an organism looks like and how it functions.
% The DNA is the carrier of hereditary information and thus understanding DNA in detail has been the central aim for many researchers.
% Classically the genetic variants (mutations) present in the living organisms have been used to study and investigate the cause of biological diseases and, additionally helps to make deductions about the way cells and organisms worked. 
% Genetics is the study of genes, genetic variation, and heredity in living organisms.
Specifically, there are two sides to the science of genetics.
On the one hand, the availability of different types of molecular information, such as sequence information and gene expression levels, paired with gene editing techniques with which we can perturb the genome in a controlled fashion and observe its biological effects, 
provides powerful explanation about the functions of the genes.
On the other hand, genetics provides a fundamental understanding of how organisms, populations, and species evolve. 
In the last few years, one of the most exciting new developments is the way in which these two sides have begun to converge \citep{casillas2017molecular}.
This convergence is achieved through the development of sequencing technologies that provide datasets from the genomic level to epigenomic, transcriptomic, proteomics or metabolomic level.
% The integrative analyses of these datasets promise to provide a systematic view of the causes and consequences of evolution, development and functions of organisms.

% these vary during the lifetime of the individual, therefore, have the potential to provide insights  effort of using molecular techniques to understand the processes of development, evolution, and speciation.
% % Modern computational biology tools (analysis of genomic sequences and bioinformatics) use these integrative principles 
% % to answer various difficult biological questions ranging from the mechanisms of evolution to the development of complex diseases.
% Thus, genetics has a central role in modern biology and its influence is ever increasing.

% s\todo{citations are missing in this para or even figures.}
% from italy person thesis.
\textit{What actually is genetic information?} 
The discovery of spatial structure of double helix deoxyribonucleic acid (DNA) by Watson and Crick in 1953, laid the main foundation for the understanding of genetics.
In most living organisms, genetic information is encoded in the form of DNA molecules.
A DNA molecule is a chain in which many bases are ordered in a linear sequence, the bases --- A, T, G, and C --- are the letters of the genetic alphabet.
The whole information within the DNA molecule of an organism is called its genome. The genome is further divided into chromosomes.
Genomes have single (haploid), double (diploid) or higher ploidy with more than two homologous chromosomes (polyploids). 
In this study, we focus on \textit{diploid} organisms. For example, humans are diploids, consisting of two copies of each chromosome called homologous chromosomes or \textit{haplotypes} --- one inherited from the mother and the other from the father.
There are differences between these two copies of each chromosome known as genetic variation. 
% \todo{this para explained in more detail in biological background.}

In early 2000s, a historic breakthrough happened with the sequencing of the human genome \citep{collins2003human}.
\textit{Sequencing} is the operation that comprises in determining the base sequence of a DNA molecule.
For sequencing genomes, there exist several kinds of sequencing technologies which share the following common properties:
\begin{itemize}
 \item They yield genomes in fragments called ``reads''.
 \item The position of reads along the genome sequence is unknown and also, mostly, the strand from which the read was sampled is unknown.
 \item The reads contain errors.
\end{itemize}

\bigskip
Huge amounts of sequencing data are produced by next generation sequencing technologies (NGS) routinely. 
The major challenge with these sequencing datasets is that the genomic information is partial and erroneous. Therefore, no sequencing technology delivers data from which we can assemble the complete genome.

The sequencing technologies differ in terms of error rates and lengths of the produced reads. We define the error rate of a read as the ratio of the number of incorrectly sequenced bases to the length of the read.
Broadly, the sequencing technologies can be categorized into four classes:
\begin{itemize}
 \item Short read sequencing: It includes Illumina/Solexa sequencing \citep{bentley2008accurate}, which has been the most widespread technology. 
 This technology produces short reads (hundreds of bases) with an error rate ($\le 1$\%). 
  \item Long read sequencing: This approach includes Single Molecule Real Time sequencing (PacBio) \citep{eid2009real} and Oxford Nanopore sequencing (ONT) \citep{laszlo2014decoding}. These technologies produce very long sequences, up to hundreds of kilo-bases. 
  The downside is that PacBio and ONT exhibit very high error rates of up to 15\% and 38\% respectively. 
  \item Synthetic long read sequencing: This technology includes 10x Genomics technology \citep{eisenstein2015startups}, which adds a unique barcode to every short read (produced from Illumina platform) generated from an individual molecule. 
  The barcode information allows to link short reads together at a long-range and the linked-read length is ~50 kb.
 \item Single-cell sequencing: This technology includes Strand specific sequencing (Strand-Seq), which produces Illumina reads, along with information on the directionality of DNA. Each single strand of a DNA molecule is labeled regarding its 5'–3' orientation \citep{falconer2012dna}.
\end{itemize}

Furthermore, each sequencing technology has some bias due to the protocols employed.
The high G/C content regions are sequenced less frequently by the short read technologies than the rest of the genome and the downstream end of reads exhibit high error rate \citep{aird2011analyzing, dohm2008substantial}.

The sequencing data is routinely used to reconstruct the underlying genome. There are inherent benefits and challenges when utilizing sequencing reads from different sequencing technologies. 
Specifically, upcoming long-read technologies deliver reads that span multiple variants and therefore, provide power to reconstruct end-to-end whole genome.
The challenge is high sequencing error rates. To overcome this challenge, the long noisy sequencing datasets are often supplemented with short accurate datasets.
Thus, the potential way to generating the complete genomes could be to combining multiple sequencing technologies, one that produces long reads and others that produce low error rates and provide long-range information.

\textit{What are the common obstacles across all technologies in producing a diploid genome?}
For all the existing sequencing technologies, the first challenge arises from the lack of information of the location of reads.
For diploid organisms, the next challenge is the lack of the \textit{haplotypic identity} of reads, meaning the haplotype that the read comes from.
Knowing the phase of reads is essential for reconstructing both copies of each chromosome, which helps to understand the true biological characteristics of diploid organisms.
% Thus, from the biological point of view, the single individual haplotyping (SIH) problem consists in the reassignment of each read to the original haplotype.
% Once we know this identity, it becomes easier to assemble the reads from each haplotype separately to further reconstruct the two genome sequences of diploid organisms.
% The process of reconstructing the diploid assemblies from sequencing reads is known as \textit{diploid genome assembly} or \textit{haplotyping}.

The haplotypes are required in order to correctly understand allele-specific expression and compound heterozygosity. Compound heterozygosity is a phenomenon of having two heterozgous recessive alleles at a marker site, which could be the cause for genetic diseases.
Haplotypes also help in investigating the genetics of common diseases, and to performing population-genetic analyses of admixture, migration and selection \citep{tewhey2011importance, Glusman2014}. 
Furthermore, the haplotype sequences are used in relating genotypes to phenotypes and for understanding how the arrangement of cis- and trans-acting variants across the two homologous copies of a genomic region affects phenotypic expression.

\begin{figure}[t!]\centering
\includegraphics[width=\columnwidth]{{ex1-intro1}.pdf}
\caption{Seven variants covered by reads (horizontal bars) in a single individual.
The alleles that a read supports are printed in white. The middle panel shows the phased reads in colors and haplotypes at the bottom over the seven variants.}
\label{fig:ex1_intro}
\end{figure}

Broadly, the approaches to obtain haplotypes are classified into two categories: Reference-based and \textit{de novo} assembly.

% This absence of context and the small size and error rates of the sequences obtained, relatively to the genome size, makes it difficult to use reads as such.
% Ideally, we would need the access to the underlying genomes in their entirety.
% Since the beginning of sequencing of DNA molecules, genomes are produced by structuring and ordering reads information. 
% Then these reconstructed genomes can be used
% as references. 
Reference genomes provide reliable information on the genomic of reads. 
Over the years, a lot of efforts have been devoted to generate good quality reference genomes.
The reference genomes play an important role in gene regulation of living beings \citep{encode2004encode}. 
They also reveal the inner organization of the genome,
including the relative positions of genes or chromosomes structure. 
They have been used by biologists for other tasks, for instance, finding about the known genes
positions and functions to annotate the genome \citep{harrow2012gencode}.

Therefore, the reference genome is a good backbone for solving the problem of identifying the location or origin of each read.
To identifying the origin of reads, we align them to the reference genome. 
Once the reads are aligned to the reference genome, the aligned reads are then partitioned into two sets to generate haplotypes. This process is called \textit{reference-based} haplotyping (phasing).

In reference-based haplotyping, there is a reference bias. The reference genome does not capture the genome diversity of population or complete information of a target individual.
The reads that are unique to the target genome, are not aligned or wrongly aligned to the reference genome.
The usage of reference genome for read alignment generates a bias in the target genome. This reference bias causes problems for further downstream analyses. 
To overcome these problems, the approach that uses reference genome is generalized to perform \textit{haplotype-aware} de novo assembly.
In this approach, the reference genome is not used, but the haplotypes are instead constructed directly from the reads.
% In a reference-based approach, the reads are aligned to the reference genome and then the aligned reads are partitioned into two sets to generate the haplotypes.
% This process is 
% To determine the assignment of each read to one of the homologous copies of chromosome (haplotype) is an important question. 
% Thus, from the biological point of view, the haplotyping (SIH) problem consists in the reassignment of each read to the original haplotype.
% Once we know this identity, it becomes easier to assemble the reads from each haplotype separately to further reconstruct the two genome sequences of diploid organisms.
% The process of reconstructing the diploid assemblies from sequencing reads is known as \textit{diploid genome assembly} or \textit{haplotyping}.

These approaches for haplotyping are discussed below in detail (Sections \ref{sec:ref} and \ref{sec:dip}). 
First, we present formulation and methods to solving reference-based haplotyping (single individual haplotyping) and then explain the generalized approach for performing \textit{haplotype-aware} de novo assembly.
\section{Reference-based Haplotyping}\label{sec:ref}
The \textit{reference-based} phasing method is applied when a reference genome is available for the species of the target genome.
We expect the target genome to be very close to the reference, more specifically, given the reads of the target genome, we expect the reference to be from the same species as the reads.
% We are interested to phase the differences of target genome to its reference.

In reference-based phasing method, the standard pipeline consists of the following steps: 
\begin{itemize}
 \item Align the reads to the reference genome.
 \item Detect variants.
 \item Phase the variants based on how aligned reads connect the alleles over them, to generate two haplotypes.
\end{itemize}
% The main focus of this study is the third step in the pipeline.
% For illustration, the toy example is given in Figure~\ref{fig:ex1_intro}.

\begin{example}
 To illustrate the haplotyping problem for a single genome, we consider a small example in Figure~\ref{fig:ex1_intro}. The example shows seven variants. Also shown are the sequencing reads aligned to the reference genome.
 The alleles that the reads support are printed in white. 
 Erroneous alleles in the reads are shown in red. In practice, we don't know the alleles that contain these sequencing errors. 
 The goal of \textit{reference-based} haplotyping problem is the re-assignment of phases to the reads, meaning assigning haplotype specific colors (green or purple) to each read. 
 In the middle panel, the colored bars represent the assignment of each read to either the green or purple haplotype.
 Finally, the reads from each haplotype are separately assembled together to output two haplotypes shown at the bottom in purple and green.
\end{example}
% In this study, we identify the limitations of the existing methods in phasing a single genome. To overcome these limitations, we propose a novel efficient algorithm.

\begin{figure}[t!]\centering
\includegraphics[width=\columnwidth]{{ex1-intro}.pdf}
\caption{Example shows the SNP matrix for the example shown in Fig.~\ref{fig:ex1_intro}. Seven variants covered by reads (horizontal bars) in a single individual.
 The allele in read is encoded as 1 if it matches the allele in the reference position at that position and 0 otherwise.
 The middle panel shows the phased reads in colors and haplotypes at the bottom over these seven variants.}
\label{fig:ex1_intro1}
\end{figure}

\bigskip
% you may refer also this: http://homolog.us/Tutorials/index.php?p=1.4&s=1

\subsection{Haplotyping as Combinatorial Optimization problem}
For diploid genomes, large volumes of sequencing data, which contain errors, are generated every day. Extracting useful information from these datasets to understand the biology of diploid genomes is a challenging problem. 

The objective of extracting useful information from noisy data in order to achieve the best possible value of goal, or objective, 
requires formulating an \textit{optimization} problem, that is defined as: given an object $\math{o}$, find a solution such that an optimization criterion $\math{f}$ is minimized or maximized.

% Reference genome reconstruction is therefore crucial in various domains where raw,
% out of context reads are unusable. The task of reordering the reads to reconstruct the
% sequenced genome for diploid organisms is called diploid genome assembly or haplotyping.
% Over the genome, there are difference between two copes for each chromosome, known as variants \todo{explained in biological background}. 
% Traditionally, there are two ways to solve it, one is based on the ordered reads to the reference genome and the other one is \textit{denovo assembly}.
% As it will be detailed further, diploid genome assembly is especially complex as the bases distribution is far
% from being uniform. Genomes present specific patterns such as large repeated sequences
% (repeats), regions with very specific distributions of nucleotides or extremely repeated
% sequences of nucleotides. Such patterns make genomes different from a uniformly distributed sequence of nucleotides. \todo{[13]} shows that a human genome is largely constituted
% of repeated sequences of significant lengths.


% \fbox{\begin{minipage}{30em}
%  In summary, the core messages of DNA, sequencing and haplotyping:
%  \begin{itemize}
%   \item The molecular information of the genomes can be obtained using different sequencing technologies.
%   \item The sequencing data is big, erroneous and not complete.
%   \item The task of recovering the genome sequences for diploid organisms is called as diploid genome assembly or haplotyping.
%  \end{itemize}
% \end{minipage}}
We will see how we can formulate the haplotyping problem as combinatorial optimization problem.

For the haplotyping problem, which consists of determining the \textit{haplotypic identity} of each read, we consider the reads aligned to the reference genome.
A read aligner maps the reads to the reference genome, ideally to positions with a high similarity score for the read. 
The number of read alignments that cover a position is known as the coverage at that position.
Furthermore, we have structural variants (SNVs) detected using different variant calling algorithms.
In case of bi-allelic variants, that is, those for which two different alleles are known on two copies of chromosome, three genotypes are possible.
One typically denotes the reference allele as 0 and the alternative one as 1. Using this
notation, the two chromosomal copies either both carry the reference allele (genotype
0/0), (or alternative allele 1/1) or one of them contains the reference while
the other one carries the alternative allele (genotype 0/1). If both chromosomal copies
carry the same allele (i.e. genotype 0/0 or 1/1), the genotype is called homozygous, while
genotype 0/1 is referred to as heterozygous.

Given the variants and the alignments, the goal here is to phase the variants and generate the haplotypes.
The variants over the genome can be phased by using reads aligned to the reference genome. This process is known as \textit{read-based phasing} or \textit{single individual haplotyping}.

Mathematically, the aligned reads over the variants are encoded in the form of a SNP matrix.
The SNP matrix for the example given in Figure~\ref{fig:ex1_intro} is illustrated in Figure~\ref{fig:ex1_intro1}.
The SNP matrix is $\mathcal{F}\in\{0,1,-\}^{R\times M}$, where $R$ is the number of reads and $M$ is the number of variants along a chromosome.
Each matrix entry $\mathcal{F}(j,k)$ is $0$ (indicating that the read matches the reference allele) or $1$, indicating that the read matches the alternative allele if the read covers that position and ``$-$'' otherwise.
Note that the ``$-$'' character can also be used to encode the unsequenced ``internal segment'' of a paired-end read.
The goal of haplotype assembly problem is to generate two haplotypes $h^0,h^1\in\{0,1\}^M$ under some conditions explained later.

The presence of sequencing and mapping errors makes the haplotype assembly problem a challenging task. 
Computationally, this problem has been generally modeled as an optimization problem to correct the sequencing and mapping errors.
In literature, different combinatorial formulations of the problem have been
proposed \citep{lippert2002algorithmic}. Among them, Minimum Error Correction (MEC) \citep{lippert2002algorithmic} has
been proven particularly successful in the reconstruction of accurate haplotypes for
diploid species \citep{martin2016whatshap, he2010optimal, CDW13_exact, Glusman2014}. It aims at correcting the input data with the minimum
number of corrections to the SNP values, such that the resulting reads can be unambiguously partitioned into two sets, each one identifying a haplotype. 
To mathematically formulate the minimum number of corrections (MEC) as an optimization problem, we require a few definitions.

The quality of a solution relies on the measure $d(r_1,r_2)$ based on the Hamming distance between any two rows $r_1,r_2\in\{0,1, -\}^M$ in $\mathcal{F}$.
\begin{definition}[Distance] 
 Formally, the distance is given by,
\[d(r_1,r_2):= \big|\big\{k\,\big|\,r_1(k)\neq -\ \wedge\ r_2(k)\neq -\ \wedge\ r_1(k)\neq r_2(k)\big\}\big|.\]
\label{eq:distance}
\end{definition}

\begin{definition}[Feasibility]
A SNP matrix $\mathcal{F}\in\{0,1,-\}^{R\times M}$ is called \emph{feasible} if there exists a bi-partition of rows (i.\,e., reads) into two sets such that all pairwise distances of two rows within the same set are zero.
\label{def:feasible-mec}
\end{definition}
Feasibility of a matrix $\mathcal{F}$ is equivalent to the existence of two haplotypes $h^0,h^1\in\{0,1\}^M$ such that every read $r$ in the matrix has a distance of zero to $h^0$ or to $h^1$ (or both).
The MEC problem can now simply be stated in terms of flipping bits in $\mathcal{F}$, where entries that are $0$ or $1$ can be flipped and ``$-$'' entries remain unchanged.

\begin{problem}[MEC]
Given a matrix $\mathcal{F}\in\{0,1,-\}^{R\times M}$, flip a minimum number of entries in $\mathcal{F}$ to obtain a feasible matrix.
\label{prob:mec}
\end{problem}


\begin{definition}[MEC cost]
The MEC cost for a feasible solution $h^0,h^1\in\{0,1\}^M$ is given by: \\
    \[\cost_{\mathcal{F}}(h^0,h^1) := \sum_{i=1}^{n} \min\{\dist(r_i,h^0), \dist(r_i,h^1)\}\]\,.

where $r_i \in \{0,1, -\}^M$ is the $i$-th row of a SNP matrix $\mathcal{F}$.
\label{eq:cost}
\end{definition}

In theory, the MEC problem is NP-hard \citep{Cilibrasi2007}.

In practice, the following different types of MEC instances are generated from different sequencing datasets:
\begin{itemize}
 \item $\MEC$: Instances in which entries in each of the $n$ rows of $\mathcal{F}$ are from $\{0,1,-\}$. There is no restriction on the placement of entries $0, 1$ and $-$. These instances are generated from Illumina, 10x Genomics and Strand-Seq sequencing technologies.
 \item $\GMEC$: A \MEC instance is called \emph{gapless} if the entries in each of the $n$ rows of $\mathcal{F}$ follows a regular expression \texttt{$-^*\{0,1\}^*-^*$}. 
 These instances are generated from PacBio, ONT and single-end Illumina like technologies.
 \item $\BMEC$: Instances in which entries in each of the $n$ rows of $\mathcal{F}$ are from $\{0,1\}$ with no gaps.
\end{itemize}

We now illustrate these \MEC instances for a single genome from different sequencing technologies through an example.
\begin{example}
In Figure~\ref{fig:ex_MECs}, the example shows a mathematical representation of reads from different sequencing technologies, that cover seven variants. 
 The top panel shows a general \BMEC instance consisting of binary values with no gaps, the middle panel shows a \GMEC instance with binary values in between and gaps at its two ends and, the bottom is 
a \MEC instance which consists of binary values and gaps, that are randomly distributed in the rows.
\end{example}



\begin{figure}[t!]\centering
\includegraphics[width=\columnwidth]{{ex-MECs}.pdf}
\caption{Seven variants covered by reads (horizontal bars) in a single individual are represented as \MEC instances. At the top is a general \MEC instance with arbitrary gaps, the middle is a \GMEC instance with gaps only at its two ends and the bottom is 
a \BMEC instance which consists of only binary values.}
\label{fig:ex_MECs}
\end{figure}

Additionally, we consider a weighted version of the MEC (wMEC), in which a cost is associated to every matrix entry.
This is useful in practice since each nucleotide in a sequencing read usually comes with a ``phred-scaled'' base quality $Q$ that corresponds to an estimated probability of $10^{-Q/10}$ that this base has been wrongly sequenced.
These phred scores can hence serve as costs of flipping a letter, allowing less confident base calls to be corrected at a lower cost compared to high confidence ones.

\begin{problem}[wMEC]
Given a matrix $\mathcal{F}\in\{0,1,-\}^{R\times M}$ and a weight matrix $\mathcal{W}\in\N^{R\times M}$, flip entries in $\mathcal{F}$ to obtain a feasible matrix, while minimizing the sum of incurred costs, where flipping entry $\mathcal{F}(j,k)$ incurs a cost of $\mathcal{W}(j,k)$.
\label{prob:wmec}
\end{problem}

Similar to MEC formulation and its versions, there are other objective functions useful for solving haplotype assembly problem.

\begin{figure}[t!]\centering
\includegraphics[width=\columnwidth]{{integrative_datasets}.pdf}
\caption{Variants covered by reads in a single individual are represented as \MEC instances from different sequencing technologies. The weights are shown in red. Figure from a paper by \cite{klau2017guided}.}
\label{fig:ex_all_datas}
\end{figure}
% \todo{maybe also define heterozgous and homogygous variants, explain genotypes too.}
% \todo{define read-length and coverage here.}

\subsubsection{Other problem formulations}
The other problem formulations to solving haplotype assembly problem are as follows:
\begin{itemize}
 \item Minimum fragment removal (MFR) and its weighted version (WMFR): These objective functions derive the haplotype assembly by removing rows of the matrix $\mathcal{F}$.
 \item Minimum SNP removal (MSR) and its weighted version (WMSR): These objective functions derive the haplotype assembly by removing the columns of the matrix $\mathcal{F}$.
 \item Minimum fragment cut (MFC): This function involves partitioning of the rows into two segments representing haplotypes.
 \item Other objective functions such as Graph and Satisfiability (SAT) formulations.
\end{itemize}

However, this study does not focus on these formulations. Specifically, the focus of the study is on the Minimum Error Correction (MEC) formulation to solving haplotype assembly problem.
\subsection{Main approaches}
We present here the algorithmic approaches, mainly focused on the Minimum Error Correction formulation to solving the haplotype assembly problem, both in theory and practice.
\subsubsection{Theoretical approaches}
In this section, we discuss theoretical approaches for the \MEC and its versions for solving haplotype assembly.
For the simplest version \BMEC, it is easy to see that finding its optimal solution is equivalent to solving the hypercube 2-segmentation problem (H2S).
The H2S problem was introduced by \cite{KPR98_segmentation,KPR04_segmentation} and is known to be $\NP$-hard \citep{Fei14_np,KPR04_segmentation}.
The optimization version of \BMEC  differs from H2S, in the former we minimize the number of mismatches instead of maximizing the number of matches.
In particular, the $\NP$-hardness of H2S directly implies the $\NP$-hardness of \BMEC, \GMEC, and \MEC.
\GMEC and \MEC were shown to be $\NP$-hard by \cite{Cilibrasi2007}.\footnote{Their result predates the hardness result of \cite{Fei14_np} for H2S. The proof of the claimed $\NP$-hardness of H2S by ~\cite{KPR98_segmentation} was never published.}

To guarantee approximation status, \cite{OR02_polynomial} obtained a polynomial time approximation scheme (PTAS) (see Definition~\ref{def:ptas}) for \BMEC based on random embeddings.
Building on the work of \cite{LMW02_finding}, \cite{JXL04_k} presented a deterministic PTAS for \BMEC.
\GMEC is a generalization of \BMEC, with logarithmic factor approximation as the best known approximation algorithm so far.
Furthermore, the most generalized version is the \MEC that contains arbitrary gaps and binary values. The \MEC is known to be APX-hard. 
Therefore, there is a gap on the approximation status of \GMEC,  whether it is as easy as \BMEC or as hard as \MEC, is unknown.

% Additionally, they showed that allowing a single gap in each strings renders the problem $\APX$-hard.
% More recently, Bonizzoni et al.~\citep{BDK+16_minimum} showed that it is unique games hard to approximate \MEC with constant performance guarantee, whereas it is approximable within a logarithmic factor in the size of the input. 
% To our knowledge, previous to our result their logarithmic factor approximation was also the best known approximation algorithm for \GMEC.

\begin{gaps}
 The approximation status of \GMEC is an open problem; \GMEC instances are very important and are often produced by single-ended PacBio or ONT technologies.
 Deriving polynomial time approximation algorithms (PTAS) for \GMEC instances provide evidence that the problem can be solved in polynomial time.
 \label{gap:gap1}
\end{gaps}

\subsubsection{Approaches in practice}
Here, we discuss exact, as well as heuristic approaches, which are generally used in practice for solving haplotype assembly problem.
\paragraph{Exact approaches} The exact approaches, which solve the problem optimally, include integer linear programming \citep{Fouilhoux2012,CDW13_exact}, and fixed-parameter tractable (FPT) algorithms \citep{he2010optimal,Patterson2015,Pirola2015}.

\textit{Integer Linear Programming (ILP)} consists of two parts, constraints or conditions and objective function. Additionally, the objective function and the constraints are linear.
An ILP in standard form is expressed as
\[{\begin{aligned}&{\text{maximize}}&&\mathbf {c} ^{\mathrm {T} }\mathbf {x} \\&{\text{subject to}}&&A\mathbf {x} +\mathbf {s} =\mathbf {b} ,\\&&&\mathbf {s} \geq \mathbf {0} ,\\&{\text{and}}&&\mathbf {x} \in \mathbb {Z} ^{n},\end{aligned}}\]
where $\displaystyle \mathbf {c} ,\mathbf {b} $, $\mathbf {x}$ and $\mathbf {s}$ are vectors, $\displaystyle A $ is a matrix and entries in $\mathbf {x}$ are integers.

\cite{CDW13_exact} focused on MEC formulation for haplotype assembly problem and proposed an ILP-based approach to solving this problem.
Basically, they consider the binary variables for each row and column and their corresponding values are supposed to be dependent on haplotypes.
By using some auxiliary variables, they relax the integer program to linear, which can be solved in polynomial time.
% \todo{do I need to mention the actual ILP program?}

\textit{Branch-and-Bound algorithm.}
A branch-and-bound algorithm consists of enumerating the candidate solutions by using a rooted tree.
The algorithm explores the branches of this tree, which represent subsets of the solution set.
Before enumerating the candidate solutions of a branch, the branch is compared to the upper and lower estimated bounds on the optimal solution, and is ignored if it cannot produce a better solution than the best one found so far by the algorithm.
\cite{wang2005haplotype} also focused on MEC formulation and applied a branch
and bound algorithm to finding haplotypes as optimal path in a binary tree. This approach solves the problem in an exact way, but does not scale well for large datasets.

\textit{Parameterized algorithms.} 
The parameterized algorithms consist of choosing fixed parameters and the sophisticated algorithms that allow to solving some problems in time exponential only in the size of a fixed parameter, but polynomial in the input size. 
Such an algorithm is called a fixed-parameter tractable (fpt-) algorithm, because the problem can be solved efficiently for small values of the fixed parameter.

Based on NGS data analysis, there are several parameters, such as read length, coverage and sequencing errors, that can help in solving genomics problems efficiently. 
Choosing a parameter, that is small enough to work in practice, is an art.
In the works by \cite{he2010optimal,Patterson2015,Pirola2015}, different parameters are proposed to solve the MEC formulation of haplotyping problem using NGS datasets.
It is shown that the parameterized algorithms using different types of NGS datasets independently, work well in practice for solving haplotyping \citep{martin2016whatshap, klau2017guided}. 

\begin{example}
 The corresponding \MEC instances for a single genome from different technologies such as Illumina, PacBio and Strand-Seq technologies are shown in Figure~\ref{fig:ex_all_datas}. 
 Shown is the toy example of \MEC instances for a human genome that consists of heterozygous variants, one in every 1000 bp. Over these variants, we observe that the matrix from Illumina data contains more gaps compared to binary values in every row because of the short read nature.
 The matrix from long-read technologies such as PacBio and ONT contain more binary values in every row. The matrix from Strand-Seq data is very sparse with more gaps, and arbitrary positioning of gaps and binary values. 
 The reads from any technology independently do not fill all the columns in the matrix with binary values and therefore, cannot generate end-to-end haplotypes.
 The joint matrix from different combinations of technologies contains binary values in all the columns and thus can produce end-to-end haplotypes.
\end{example}

As described in Section~\ref{sec:dna_seq}, the NGS datasets have their inherent challenges and benefits and, no technology independently enables producing the complete haplotypes.
For instance, long-read technologies have high sequencing error rates and short-read technologies produce short reads, which are not sufficient to producing the whole genome.
This necessitates to optimally combine the advantages of all technologies available at hand, in a joint framework, in order to facilitate producing the whole genome sequences.
However, an efficient algorithm to solve haplotyping, by combining all sequencing datasets in an integrative framework, is unknown.
Therefore developing a parameterized algorithm for this integrative framework and deciding parameters that work well in practice is very important. 

\begin{gaps}
Finding a parameterized algorithm for a version of MEC that uses multiple sequencing technologies in an integrative framework, is an open problem.
\label{gap:gap2}
\end{gaps}

\begin{figure}[t!]\centering
\includegraphics[width=\columnwidth]{{pedigree}.pdf}
\caption{Seven SNP loci covered by reads (horizontal bars) in three individuals. Unphased genotypes are indicated by labels 0/0, 0/1 and 1/1. The alleles that a read supports are printed in white.}
\label{fig:ex_pedigree}
\end{figure}
\paragraph{Heuristic approaches}
A heuristic algorithm is one that is designed to solve a problem in a faster and more efficient fashion in terms of speed and memory requirements, at the cost of sacrificing optimality.
Heuristic algorithms are most often employed when approximate solutions are sufficient and exact solutions are necessarily computationally expensive.

There is a lot of literature that relies on heuristic approaches for haplotyping.
HASH (haplotype assembly for single human) used a Markov chain Monte Carlo (MCMC) algorithm and graph partitioning approach to assemble haplotypes,
given a list of heterozygous variants and a set of shotgun sequence reads mapped to a reference genome assembly \citep{bansal2008mcmc}. 
In the work by \cite{wang2007clustering}, a clustering algorithm is deployed into split the rows of $\mathcal{F}$ in two sets based on MEC formulation.
HapCut \citep{Bansal2008} utilized the overlapping structure of the fragment matrix and max-cut computations to find the minimum error correction (MEC) solution for haplotype assembly. 
\cite{Duitama2010} followed a heuristic approach for max-cut to find haplotypes efficiently. 
MixSIH \citep{matsumoto2013mixsih} utilized a probabilistic mixture model to solve haplotyping.
H-BOP \citep{xie2012fast} followed a heuristic algorithm for optimizing a combination of the MEC and Maximum Fragments Cut models.
ProbHap \citep{Kuleshov2014b} fundamentally employed a similar approach to WhatsHap \citep{Patterson2015}, but uses the Viterbi algorithm to solve the maximum likelihood function specified by a probabilistic graphical model. 
% % https://www.ncbi.nlm.nih.gov/pmc/articles/PMC2935405/
% \textit{Clustering algorithms.} In the work by \citep{wang2007clustering}, a clustering algorithm is used to split the rows of $\mathcal{F}$ in two sets. 
% The main contribution consists in the combination of the two distance functions used by the clustering algorithm. 
% The first distance is the Hamming distance as defined in Equation~\ref{eq:distance}. This distance takes into account only the number of mismatches between two fragments. 
% The second distance $D'$ also takes into account the number of matches between the two fragments.
% This means that given a certain fixed number of mismatches between two fragments, the more they overlap the closer they are.
% Using the above distance functions, a simple iterative clustering procedure is given as follows.
% \begin{enumerate}
%  \item for each possible pair of fragments in the SNP matrix the generalized Hamming distance is computed. Let $r_1$ and $r_2$ be the two furthest fragments according to Hamming distance, the two sets are initialized as $C1 = r_1$ and $C2 = r_2$.
%   \item Let H1 and H2 be the two consensus strings derived from C1 and C2: all the fragments are compared with H1 and H2 and assigned to the corresponding closer set. If a fragment is equidistant from the two consensus strings, the distance $D'$ is used to decide to which set assign the fragment.
%   \item Once all fragments are assigned, the consensus strings H1 and H2 are updated and the algorithm restarts from (2). The procedure loops until a stable haplotype pair is found (i.e. when the consensus haplotypes are the same before and after the update).
% \end{enumerate}
% 
% \textit{Max-Cut based algorithm.}
% HapCUT~\citep{Bansal2008} approaches the haplotype assembly as a MAX-CUT problem.
% Given a certain haplotype pair $H$, a graph $G(H)$ is constructed such that there is a vertex for each column of the matrix $\mathcal{F}$ and 
% there is an edge between two vertices of $G(H)$ if the corresponding columns in $\mathcal{F}$ are linked by at least one fragment. 
% Consider the fragment $\mathcal{F}(j)$ such that it covers both positions k1 and k2. Let $\mathcal{F}(j)[k1,k2]$ and H[k1, k2] represent the restriction of  $\mathcal{F}(j)$ and H to loci k1 and k2. 
% There are two cases: $\mathcal{F}(j)[k1,k2]$ matches one of the two haplotype strings of H[k1, k2], or $\mathcal{F}(j)[k1,k2]$ does not match any. 
% The weight w (k1, k2) associated with the edge between node k1 and k2 in the graph $G(H)$ is given by the number of fragments such that $\mathcal{F}(j)[k1,k2]$ does not match any string in 
% H[k1, k2] minus the number of fragments such that the match exists. 
% The higher w(k1, k2), the weaker is the correlation between the haplotype pair H and the SNP matrix restricted to columns k1 and k2. 
% Let (S, $\mathcal{F}$ - S) be a cut of G, the weight of the cut is defined as follows:
% \[ w(S) = \sum_{j\in S,k\in \mathcal{F}- S} w(j,k)\]
% 
% Consider the haplotype pair $H_S$ derived from $H$ by flipping all the elements involved in S. It is shown that the problem of finding a haplotype pair minimizing the \MEC score
% is reduced to the problem of finding a max-cut in $G(H)$. To solve max-cut problem, HapCut initializes the random haplotype pair, and then iteratively attempts to refine the haplotype pair to reduce \MEC score
% till it is no longer possible to further reduce it.

In addition, there are some pedigree based approaches for performing haplotype assembly for single individuals.

\subsection{Pedigree of genomes}
Another approach to haplotyping takes into account the sequencing datasets from families of genomes.
Specifically, such an approach takes advantage of two things: one is the sequencing data itself of each individual and the other is the principles of the Mendelian segregation of alleles in pedigrees. 
There are alleles that are specific to a single founding chromosome within a pedigree, which are referred to as lineage-specific alleles. 
These are highly informative for identifying haplotypes that are identical-by-descent (IBD) between individuals within a pedigree.
At the simplest level of a family trio (both parents and one child), very simple rules indicate which alleles in the child were inherited from each parent, thus largely separating the two haplotypes in the child.  
Nevertheless, genetic analysis cannot phase positions in which all family members are heterozygous. 
In cases where genetic analysis information is missing, using sequencing dataset is an excellent choice.
The genetic analysis largely supplement with the sequencing datasets for complete haplotype assembly.
% Furthermore, it is not always feasible to recruit the required participants for family-based studies. 
% In the absence of a family context, molecular haplotyping is an excellent choice because it does not require DNA samples from other family members. 
% The sequencing based haplotyping largely supplement the need for genetic analysis.

The Haploscribe method \citep{Roach2011} phased whole-genome data based on genetic analysis. Haploscribe followed a parsimony approach to generate
meiosis-indicator (inheritance state) vectors and obtained haplotypes by modeling haplotyping problem using a hidden Markov model (HMM). 
Other tools by \cite{abecasis2002merlin, williams2010rapid} are also based on genetic analysis information. 

All these previous approaches lack the joint usage of both information sources based on IBD and sequencing datasets of individuals in the pedigree. 
Finding an efficient method that uses both sequencing data and genetic inheritance principles in an integrative fashion, for performing phasing, is very important to generating complete haplotypes.


\begin{gaps}
 Combining both principles of genetic inheritance and sequencing reads into one framework is an open problem. Furthermore, to come up with an efficient algorithm that works well in practice is an important question.
 \label{gap:gap3}
\end{gaps}
\begin{example}
 To illustrate the motivation to combine genetic and read-based haplotyping, the corresponding \MEC instance for a single genome is shown in Figure~\ref{fig:ex_pedigree}. 
There are seven SNP positions covered by reads in three related individuals. 
It illustrates how the ideas of genetic and read-based haplotyping complement each other. 
All genotypes at SNP 3 are heterozygous. 
Thus, its phasing cannot be inferred by genetic phasing, that is, using only the given genotypes and not the reads. SNP 4, 
in contrast, is not covered by any read in the child. When only using reads in the child (corresponding to single-individual read-based phasing), 
no inference can be made about the phase of SNP 4 and neither about the phase between SNP 3 and SNP 5. 
The phases of all SNPs for child can be easily inferred based on the observation that all seven child genotypes are compatible with the combination of brown and green haplotypes from the parents. 
This example demonstrates that using pedigree information, genotypes and sequencing reads jointly are very powerful for establishing phase information.
\end{example}
\begin{figure}[t!]\centering
\includegraphics[width=\columnwidth]{{assembly_graphs}.pdf}
\caption{Figure shows the reads and reconstructed haplotypes using two graph approaches: (a) de Bruijn graph and (b) overlap graph.}
\label{fig:assembly_graphs}
\end{figure}

In summary, we have presented an overview of computational approaches for performing haplotype assembly by using reference genome as a backbone. 
The main point is the reference genome used, which may hinder the correct downstream analyses in some cases.
First, we cannot phase variants that are de novo or rare in the target genome.
Second, we make the prior hypothesis that the target genome is very close to the reference, which may not always be true in reality.
Third, the method is obviously not self-sufficient since a prior reference needs to be
constructed. For these reasons, we additionally consider \textit{haplotype-aware de novo} (without reference) assembly, which is also known as \textit{diploid} assembly.

\section{Diploid assembly}\label{sec:dip}
In diploid assembly, the reads from the diploid genome are directly used to assemble the haplotype sequences. The obtained haplotype sequences are known as diploid assemblies.
The diploid assembly process involves partitioning the input read set into two sets and then gluing together the reads from each set in proper order to produce diploid assemblies.
Therefore, it is necessary to know the \textit{ordering} and \textit{haplotypic} identity of these reads. 
The diploid assembly process is challenging due to short read lengths, incomplete data, sequencing errors and repetitive regions on the genome.
% The other challenges occur are, reads are very short compared to the genome size and are not long enough to span the whole repeats.

How to formulate the diploid assembly problem to finding the ordering and haplotypic identity of reads for producing diploid assemblies, while avoiding misassemblies in complex repetitive regions, is discussed below.

\subsection{Diploid assembly as graph problem}
The diploid assembly from sequencing reads is modeled as the assembly graph problem.
The assembly graph restores reliable information about the \textit{ordering} of reads.
Assembly graphs can be categorized into two families: overlap graphs and de Bruijn graphs.
\paragraph{Overlap graph}
Given a set of reads, the overlap graph consists of nodes that represent reads and the edges represent the overlap between read sequences.
The weight on the edges represents the maximal overlap length between two sequences. As illustrated in Figure~\ref{fig:assembly_graphs}b, 
given four reads, the goal is to reconstruct the haplotypes sequences H0 and H1.
In overlap graph, there are four nodes R1, R2, R3 and R4 for these reads and there are edges between these nodes based on the overlap, for example, there is a link between R1 and R2 with a weight of 3.

The overlap graphs can be simplified to string graphs by the transitive reduction of edges.
Also, the contained edges are removed, which occur when one read is a substring of other reads.

Most of the assemblers generate only one sequence and the algorithm to construct this sequence (instead of two) can be outlined as follows.
\begin{itemize}
 \item Overlap: calculate pairwise overlaps between reads
 \item  Layout: look for a parsimonious solution (as a generalized Hamiltonian path visiting each node at least once while minimizing the total string length)
 \item  Consensus: merge reads, using redundancy to correct sequencing errors
\end{itemize}

The first OLC assembler was Celera \citep{myers2000whole}, which was designed to handle sequencing data from Sanger technology.
Celera uses a BLAST-like approach to performing all-vs-all read alignment. It then compacts the overlaps with no ambiguity and applies some heuristics on the complex regions involving repeats. The final sequences
are generated by removing sequencing errors. The complex repetitive regions are hard to resolve and, this results in fragmented assemblies.  We call these fragmented sequences
``contigs'' for contiguous consensus sequences. Furthermore, a series of contigs are connected using long-range read information to generate long assemblies called ``scaffolds''.

This paradigm was used with long Sanger sequences and for relatively small
genomes. Currently, the OLC-based algorithms are also used with PacBio datasets. 
Because of the cost of the pairwise overlaps computation, the OLC is too time
consuming for large NGS datasets. Thus, other solutions are required to deal with the amount of reads to assemble large genomes.
\bigskip

\begin{figure}[t!]\centering
\includegraphics[width=\columnwidth]{ex_sv.pdf}
\caption{Given the input reads (middle) from the two sequences (top), we show a corresponding assembly graph at the bottom.
The bubbles in the sequence graph (bottom) show three different heterozygous variations; the first one is an SNV, the second one is an SV, and the third one is an indel. }
\label{fig:ex_sv}
\end{figure}

\paragraph{De Bruijn graphs}
% https://genome.sph.umich.edu/w/images/b/b4/666.2012.01.pdf
In this type of assembly graph, each read is broken into a sequence of overlapping k-mers. The distinct
k-mers are added as vertices to the graph, and k-mers that originate from adjacent positions in
a read are linked by an edge.
In Figure~\ref{fig:assembly_graphs}a, the reads are divided into words of fixed length $k$, where $k=4$. Here,
each node in the graph is a word and the connection between the nodes is based on the overlap between nodes.
Basically, the de Bruijn graph is a directed graph representing overlaps
between sequences of symbols, named after Nicolass Govert de Bruijn \citep{todd1933combinatorial}. Given an
alphabet $\sigma$ of $n$ symbols, a $k$ dimensional de Bruijn graph has the following properties.
\begin{enumerate}
 \item $n^k$ vertices produced by all words of length $k$ from alphabet $\sigma$
 \item Two vertices X and Y are connected if and only if the $k$ - 1
suffix of X is equal to the $k$ - 1 prefix of Y.
\end{enumerate}

The first application of the de Bruijn graph in genome assembly was introduced into
the EULER assembler \citep{pevzner2001eulerian} in order to tackle assembly complexity. 
The assembly problem can then be formulated as finding a walk
through the graph that visits each edge in the graph once --- an Eulerian path problem.
Due to the repeats, it is hard to find the Euler path. 
In most instances, the assembler attempts to construct
contigs consisting of the unambiguous, unbranching regions of the graph.

\paragraph{De Bruijn graph and overlap graph}
The de Bruijn graph theoretically achieves
the same tasks that the overlap graph does, but in an efficient manner.
The de Bruijn graph became widely used when the short reads from NGS appeared. The OLC
approach did not scale well on the high number of sequences generated by NGS. The
use of the de Bruijn graph is very interesting for short read assembly for its ability to
deal with the high redundancy of such sequencing in a very efficient way. Indeed a k-mer
present dozens of times in the sequencing dataset appears only once in the graph. This
makes the de Bruijn graph structure not very sensible to high coverage, unlike
OLC. The de Bruijn graph was first proposed as an alternative structure \citep{pevzner2001eulerian} because it
was less sensible to repeats. Repeats that were problematic in the OLC, creating very
complex and edges heavy zones, are collapsed in the de Bruijn graph.

From the above, it follows that the assembly graphs contain the information about ordering of reads. 
We now discuss the special structures in assembly graphs, that can occur due to repeats and heterozygous regions.

\paragraph{Graph structures} There are mainly two types of structures (bubbles and repeats) that occur in the assembly graphs for diploid genomes.

\textit{Bubbles.}
Bubbles are defined as a set of disjoint paths that share the same start and end nodes.
Bubbles in the graph represent heterozygosity for the diploid organism.
Bubbles can contain simple SNVs with only one allele difference, or even large complex structural variations in the order of a few kilo-bases.  
Figure~\ref{fig:ex_sv} illustrates how the bubbles in an assembly graph can contain both small variants (SNPs and indels up to several dozen base-pairs in length) and large structural variants.
\begin{figure}[t!]\centering
\includegraphics[width=\columnwidth]{repeats.pdf}
\caption{At the top, shown are the heterozygosity (in vertical bars) and repetitive regions (in red) over the genome. At the bottom, shown is the graph with nodes as heterozygous or repetitive region, and connections are based on the successive read overlap.
The graph has cycles because of repetitive region shown by R, which also causes two branches.}
\label{fig:repeats}
\end{figure}

\textit{Repeats.} The repeats over the genomes cause branches or cycles in the assembly graphs and, therefore, make a graph more complex and break its linear chain properties.
This is illustrated in Figure~\ref{fig:repeats}. In this example, the repeat $R$ causes cycles and branches in the graph.
Assemblers generally handle these repeats by making a guess as to which branch to follow.
Incorrect guesses create false joins (chimeric contigs) and erroneous copy numbers. 
If the assembler is more conservative, it will break the assembly at these branch points, leading to an accurate but fragmented assembly with fairly small contigs.
The maximal repeat resolution  depends on the read-length. If there is a read that is long enough to span the repeat region, then the repeat is resolvable.
Therefore, upcoming long-read sequencing technologies have the power to obtain maximal repeat-resolved diploid assemblies.

We have explored the assembly graphs and their special structures. We now provide the main approaches followed for reconstructing the genome from NGS data. 
\begin{figure}[t!]\centering
\includegraphics[width=\columnwidth]{ex_graph_approach.pdf}
\caption{Input: an assembly graph (top) (consisting of four SNVs and two SVs) and the PacBio reads $r_1, r_2, r_3, r_4, r_5, r_6$ (gray). 
Output: the phased reads (colored in blue and red) and haplotigs (bottom) using Falcon Unzip and our graph-based approach. Our graph-based phases central region, contrarily, Falcon Unzip does not.  }
\label{fig:ex_graph_approach}
\end{figure}

\subsection{Main approaches for diploid assembly}
Over the last decade, the development of various NGS technologies has impacted the assembly problem.
In theory, the problem of \textit{de novo assembly}---computing the consensus of two or more sequences---is NP-hard, when the problem is modeled either as string graphs or de Bruijn graphs \citep{medvedev2007computability}. 
There are several heuristic approaches to approximate the optimal de novo haploid assembly based on NGS datasets \citep{idury1995new, myers1995toward, myers2005fragment, pevzner2001eulerian, nagarajan2009parametric, nagarajan2013sequence, sovic2013approaches}.

However, even with Sanger (reads of the order of 800-1000 base pairs) and Illumina sequencing, which deliver short reads with low error rates, de novo assembly of heterozygous diploid genomes has been a difficult problem \citep{vinson2005assembly, levy2007diploid}.
In practice, there are several short-read assemblers based on Illumina data for heterozygous genomes \citep{kajitani2014efficient, pryszcz2016redundans, simpson2012efficient, bankevich2012spades, li2015fermikit}.
The assemblies that they produce are accurate, but contain gaps and are composed of relatively short contigs and scaffolds. 
Third generation sequencing technologies such as methods available from Pacific Biosciences (PacBio) and Oxford Nanopore Technologies (ONT) deliver much longer reads, but with high error rates.
There are now several long-read assemblers \citep{koren2017canu, vaser2017fast, xiao2016mecat, berlin2015assembling, chin2013nonhybrid, hunt2015circlator, lin2016assembly} that use these long-read data for de novo assembly.
The assemblies that are delivered from these assemblers are more contiguous, with longer contigs and scaffolds.
Finally, there are hybrid assemblers that take advantage of long-read data (with its high error rate) and short-read data (with its low error rate) \citep{bashir2012hybrid, antipov2015hybridspades, zimin2017hybrid} and attempt to combine the best aspects of both.
These hybrid assemblers have the power to deliver highly accurate, repeat-resolved assemblies.

The main drawback of above state-of-the-art assemblers is that they generate only one consensus sequence even for diploid organisms. To date, there is only one assembler that can produce diploid assemblies for diploid genomes.

\textit{Diploid assembly.} A recent and currently only available diploid assembly method --- Falcon Unzip \citep{chin2016phased} --- is purely PacBio based diploid assembler.
Falcon Unzip generates haplotype contigs or ``haplotigs'' that represent the diploid genome with correctly phased homologous chromosomes.
Falcon Unzip involves constructing a string graph from long PacBio reads, and generating haplotigs in a greedy manner using local conservative approach.


For generating haplotigs, Falcon Unzip first identifies phase for each read based on the condition that the read covers at-least one SNV for phasing.
In the regions over the genome where the SNVs are at a long distance to each other, the Falcon Unzip can not phase those regions, resulting in incomplete assemblies.
Additionally, the Falcon Unzip cannot phase all large structural variants and regions with high heterozygosity. 

% The pipeline is given in Figure~\ref{fig:falcon_unzip}.
% Falcon Unzip begins by using reads to construct a string graph that contains sets of ``haplotype-fused contigs'' , also called as ``primary contigs'', as well as bubbles representing divergent regions between homologous sequences (Fig.~\ref{fig:falcon_unzip}a). 
% Next, Falcon-Unzip identifies read haplotypes using phasing information from heterozygous positions that it identifies (Fig.~\ref{fig:falcon_unzip}b). 
% Phased reads are then used to assemble haplotigs and primary contigs (backbone contigs for both haplotypes) (Fig.~\ref{fig:falcon_unzip}c) 
% that form the final diploid assembly with phased single-nucleotide polymorphisms (SNPs) and structural variants (SVs).
% 
% \textit{Phasing using primary contigs.}
% In Falcon Unzip, the reads are aligned to primary contigs and heterozygous SNPs (het-SNPs) are called by analyzing the base frequency of the detailed sequence alignments.
% A simple phasing algorithm was developed to identify phased SNPs. 
% Along each contig, the algorithm assigns phasing blocks where ``chained phased SNPs'' can be identified. 
% Within each block, if a raw read contains a sufficient number of het-SNPs, it assigns a haplotype phase for the read unambiguously. 
% Combined with the block and the haplotype phase information, it assigns a ``block-phase'' tag for each phased read in each phasing block.
% Some reads might not have enough phasing information. For example, if there are not enough het-SNP sites covered by a read, it assigns a special 'un-phased tag' for each un-phased read.
% The initial assembly graph is fused using phased reads and the haplotigs are generated in a greedy manner using local conservative approach.
% % \todo{maybe add example how haplotigs from haplotype fused assembly graph works?}

% \begin{figure}[t!]\centering
% \includegraphics[width=\columnwidth]{{cropped_falcon-unzip}.pdf}
% \caption{(a) An initial assembly is computed by FALCON, which error corrects the raw reads (not shown) and then assembles them using a string graph of the read overlaps. 
% The assembled contigs are further refined by FALCON-Unzip into a final set of contigs and haplotigs. 
% (b) Phase heterozygous SNPs and group reads by haplotype. (c) The phased reads are used to open up the haplotype-fused path and generate as output a set of primary contigs and associated haplotigs.}
% \label{fig:falcon_unzip}
% \end{figure}

There is no known algorithm that works at different levels of heterozygosity, phases all types of structural variants and generates complete diploid assemblies.
A potential step to achieve this task of complete diploid assemblies is to phasing directly from the assembly graph.
Moreover, it becomes easier to detect large structural variants, such as translocations and other rearrangements, in an assembly graph.
Thus, working in the space of assembly graphs provides deep insights to detect all types of structural variation, which further helps in phasing whole genomes.

Additionally, the Falcon Unzip is purely PacBio based assembler, which uses very noisy data and, therefore, the Falcon Unzip requires high coverage data to producing accurate assemblies.
In contrast, hybrid approaches that combine accurate Illumina and long read PacBio data, conceptually have the potential to producing good quality assemblies even at low coverages.
However, there is no known algorithm, that combines mutiple sequencing datasets such as accurate Illumina and long read PacBio data, for producing good quality haplotigs.
\begin{gaps}
 Phasing bubbles directly from the assembly graph is an open problem. Additionally, the MEC formulation for phasing that works on graphs, by combining datasets from multiple sequencing technologies, is unknown. 
 \label{gap:gap4}
\end{gaps}

\begin{example}
Figure~\ref{fig:ex_graph_approach} demonstrates the conceptual advantages of our graph-based approach over the Falcon Unzip method.
Consider four SNVs separated by two large SVs and there are four reads spanning these variants.
Falcon Unzip can not phase the central region because the reads $r_3$ and $r_4$ do not cover any SNVs, resulting in incomplete haplotigs.
In contrast, the graph-based approaches attempt to detect all types of SVs and phase all of them.
\end{example}
Based on the above example, we observe that it is possible to deliver complete and contiguous haplotigs using assembly graph-based approach.

% \begin{figure}[t!]\centering
% \includegraphics[width=\columnwidth]{ex_sv.pdf}
% \caption{Based on reads (middle) from the two sequences (top), the bubbles in the graph (bottom) show three different heterozygous structural variations; first is a SNV, second is a SV and third is an indel. }
% \label{fig:ex_sv}
% \end{figure}
\section{Outline of our contributions}
In the above section, I highlighted four ``open problems'' in the arena of haplotyping using NGS data.

\begin{itemize}
 \item In Chapter 2, I provide a general background on the different types of algorithms. I outline the motivation on how these algorithms are used in solving small daily examples fast. 
 I highlight the advantages and disadvantages of these algorithms in context of large problems.
 \item Solving MEC instances using NGS data are NP-hard. In Chapter 3, I describe the dynamic programming based algorithm to solve these instances approximately.
 I discuss the approximation guarantee, which provides hints that these instances can be solved in practice in polynomial time. (Problem~\ref{gap:gap1})
  \item In Chapter 4, I discuss different types of NGS datasets, with their advantages and disadvantages. I describe the integrative phasing framework obtained by combining NGS datasets. I discuss the parameterized algorithm that solves these instances efficiently in practice. 
  I demonstrate the effectiveness of this algorithm on real genomic datasets. (Problem~\ref{gap:gap2})
 \item In Chapter 5, I describe the generalized parameterized approach to incorporate information from pedigrees. I show the experiments on real datasets and highlight that pedigree data has an additional advantage in delivering better quality haplotypes. (Problem~\ref{gap:gap3})
 \item In Chapter 6, I describe the generalized approach --- haplotype-aware diploid assembly --- in a graph framework, that has the ability to handle all levels of heterozygosity and structural variations to produce accurate and complete haplotype assemblies.
 I present this approach as a hybrid of different types of NGS datasets and show its effectiveness on the pseudo-diploid genome. (Problem~\ref{gap:gap4})
 \item Finally, in Chapter 7, I provide a summary of the results presented in this thesis, along with the future outlook and perspectives.
\end{itemize}

\section{Relevant publications}
\begin{itemize}
 \item David Porubsky*, \underline{Shilpa Garg}*, Ashley D. Sanders*, V. Guryev, Peter M. Lansdorp, T. Marschall,
\textit{Dense And Accurate Whole-Chromosome Haplotyping Of Individual Genomes}, Nature Communications, 2017.

In this paper, my contribution was in developing an integrative pipeline and writing draft for WhatsHap section.
\item \underline{Shilpa Garg}, Marcel Martin and Tobias Marschall, \textit{Read-Based Phasing of Related Individuals},
Proceedings of ISMB 2016/Bioinformatics.
\item \underline{Shilpa Garg}, Mikko Rautiainen, Adam M Novak, Erik Garrison, Richard Durbin, Tobias Marschall, \textit{A graph-based approach to diploid genome assembly}, ISMB 2018 (to appear).
\item \underline{Shilpa Garg}, Tobias Moemke, \textit{A QPTAS for Gapless-MEC}, Submitted.
\item Preprint: M. Martin*, M. Patterson*, \underline{Shilpa Garg}, S. O. Fischer, N. Pisanti, G. W. Klau, A. Schnhuth, T.
Marschall, \textit{WhatsHap: fast and accurate read-based phasing}.

In this paper, my contribution was in developing some parts of the pipeline and making figures.
\end{itemize}


% \subsection{Issues we address}
% To address those problems, we will present new algorithmic approaches. 
% \begin{itemize}
% \item In the second chapter, we present the fundamental concepts of different type of algorithms.
%  \item In the third chapter, we present dynamic programming based algorithm to prove the near-polynomial approximation status of \GMEC.
%  \item In the forth chapter, we present a parameterized algorithm to solve \MEC instances integatively from different datasets.
%   \item In the fifth chapter, we present a integrative framework to solve sequencing-based and genetic haplotyping, helps to generate complete and accurate haplotypes.
%  \item In the sixth chapter, we introduce new way to represent the assembly graph and futher, finding long read paths in the graph based on different types of datasets, which futher helps in better phasing. 
% \end{itemize}


% 
% 
%  
% 
% \todo{cover these issues in approaches}
% \subsection{Diploid genome assembly hardness}
% \begin{itemize}
% \item Theoretical approximation gurantee on gapless-MEC. It is important because even for high coverages, we can solve it in polynomial time approximately.
%  \item integrating datasets to produce more accurarte and complete
%  \item non-reference denovo based, can not detect large SVs, directly from graph, hybrid
%  \item pedigree of genomes
% \end{itemize}
% 
% \subsection{Goals and achievements}
% \begin{enumerate}
%  \item DNA genomes ranging from small yeast like genome to larger ones like human.
%  \item end-to-end full genome sequences
%  \item efficient algorithms to generate optimal or near-optimal solution.
% \end{enumerate}
% 
% 
% 
% 
% \subsection{Outline of our contributions}
% \begin{enumerate}
%  \item Chpater 1 consists of ...
%   \item Chpater 2 consists of ...
%    \item Chpater 3 consists of ...
% \end{enumerate}







\chapter{Biological and Algorithmic background}
% http://web.stanford.edu/group/hopes/cgi-bin/hopes_test/an-introduction-to-dna-and-chromosomes-text-and-audio/#what-is-dna-making-the-single-strand

In the first part of this chapter we provide explanation of some of the biological concepts we introduced in the earlier chapter. This explanation is required for understanding the material presented later in the thesis.
In the second part we provide types of algorithms used in computer science to solve biological problems that can be formulated as optimization problem.
% formulation for the diploid assembly problem and describe briefly various methods to perform haplotyping including graph-based approaches.
% Furthermore, we discuss about the challenges to solve the problem and provide the main contributions of this thesis.
\section{Biological Background}
\subsection{DNA structure}
As we learnt in the previous section, the genetic information of different organisms is encoded in DNA.
The DNA molecule is a chain on which many bases are ordered in a linear sequence, the bases --- A, T, G, and C --- as the letters of a genetic alphabet.
The DNA sequence length can be short, for example, ~12 million nucleotides long for yeast genomes, or long in the order of approximately 3 billion nucleotides for humans.
One can think of DNA as a “genetic database” for organisms.
% If we think of these four bases --- A, T, G, and C --- as the letters of a genetic alphabet, we have the building blocks necessary to encode lots of information within these relatively compact DNA molecules.

DNA stands for deoxyribonucleic acid. 
By breaking the name itself suggests the structure of the molecule, which consists of three components: 1) a sugar molecule, 2) a phosphate group, and 3) a nitrogenous base.
The nitrogenous bases are what make DNA variable. 
There are 4 different types of bases in DNA:adenine, guanine, thymine, and cytosine. 
Biologists commonly abbreviate these bases as the letters A, G, T, and C, respectively. 
Each one of the bases is chemically distinguishable from the others, it is the variability of these bases that constitutes the genetic code.

Futhermore, a double helix of DNA is composed of two spiraling, complementary strands of DNA. 
Each strand is composed of a sugar and phosphate backbone with varying nitrogenous bases sticking in towards the center. 
The two strands are joined together at the center by pairing bases lined up with one another. 
DNA is often described structurally as a \textit{twisting ladder}. 
In this ladder, the “rungs” are the pairs of bases linked together, and the “sides” are the two separate sugar and phosphate backbones.
The double helix is important because it preserves all of the information-carrying features of a single DNA strand 
while at the same time introducing elements that make it easier for living cells to make copies of their DNA. 
Because every base pair in the double helix must match its pairing partner (A with T, C with G), 
we can easily determine the sequence of an unknown strand of DNA if its matching strand is known. 
For example, if one strand of a double helix has the nucleotide sequence GATTCGTACG, then its complementary strand will be CTAAGCATGC.

\subsubsection{Chromosomes} 
DNA is divided into bundles known as chromosomes.
Chromosomes have several important features. 
First of all, the DNA packs so tightly that one can see it under a simple light microscope. 
Secondly, recall that because the cell is getting ready to divide in two, the DNA of a visible chromosome has already been duplicated, so that each successor cell will have its own copy. 
This means that, on close inspection, a cell that is ready to divide will have four strands of DNA, two helices of two strands each. 
Each of these double strands of DNA condenses into a single rod called a sister chromatid. 
The two chromatids are therefore exact replicas of one another, and the center of each is joined together prior to the division of the cell. 
As a result, most chromosomes take on the appearance of the letter X.

The human genome is composed of 23 kinds of chromosomes. 
However, because humans conceive through sexual reproduction, every child receives two sets of 23 chromosomes – one from mother and the other from father. 
As a result, every individual has 23 pairs of chromosomes, for a total of 46. Of these 23 pairs, one pair is responsible for determining sex. 
The chromosomes in this pair are therefore called sex chromosomes. The chromosomes in the remaining 22 pairs are called autosomes.
The two chromosomes in a pair of autosomes are called homologues, or a “homologous pair,” meaning that they contain corresponding sequences of DNA given in Figure~\ref{fig:background1}. 
These two chromosomes come from separate parents. The homologous chromosomes contain DNA sequences that are similar, but they are not identical copies of each other.

\begin{figure}[t!]\centering
\includegraphics[width=\columnwidth]{{background1}.pdf}
\caption{Homologous chromosomes, sister chromatids.}
\label{fig:background1}
\end{figure}

\subsection{Diploid nature of organisms}
The diploid nature of organisms occur due to the process of mitosis.
\paragraph{Concept of Mitosis}
Figure~\ref{fig:mitosis1} is a simplified diagram illustrating the overall process of mitosis and the detailed phases are in given in Figure~\ref{fig:mitosis}.
In the first step, called interphase, the DNA strand of a chromosome is copied (the DNA strand is replicated) and this copied strand is attached to the original strand at a spot called the centromere. 
This new structure is called a bivalent chromosome. 
A bivalent chromosome consists of two sister chromatids (DNA strands that are replicas of each other). 
When a chromosome exists as just one chromatid, just one DNA strand and its associated proteins, it is called a monovalent chromosome.
The second and third steps of mitosis organize the newly created bivalent chromosomes so that they they can be split in an orderly fashion.
In the second step, prophase, the bivalent chromosomes condense into tight packages. 
In the third step of mitosis, called metaphase, each chromosome lines up in a single file line at the center of the cell.
In the fourth step, anaphase, the mitotic spindles pry each chromatid apart from its copy, and drag them to the opposite side of the cell. Four bivalent chromosomes become two groups of 4 monovalent chromosomes. 
\begin{figure}[t!]\centering
\includegraphics[width=\columnwidth]{{mitosis1}.pdf}
\caption{Overview of Mitosis.}
\label{fig:mitosis1}
\end{figure}

\begin{figure}[t!]\centering
\includegraphics[width=\columnwidth]{{mitosis}.pdf}
\caption{Phases of cell cycle and mitosis}
\label{fig:mitosis}
\end{figure}

\paragraph{Concept of Meiosis}
The purpose of meiosis is to make haploid gametes. In order to explain the difference between mitosis and meiosis quickly and easily, consider the following analogy: 
You own a restaurant, and you keep 46 cookbooks on hand, to store all the recipes you need to make the food you sell. 
If you opened a new restaurant that you wanted to make the same food as the one that already exists, what would you do? Copy all 46 cookbooks, and take them to the new restaurant. 
That's like what happens in mitosis. Consider that the cookbooks are chromosomes, each containing lots of recipes that cells use to make “dishes,” called proteins.
When cell division occurs, each cell wants to ensure that each new cell can make the same proteins as the original. So each of the chromosomes are copied and evenly distributed to both new cells—both cells get a copy of each “cookbook.”
Meiosis is different. Whereas as mitosis makes a new cell with the same number of chromosomes, meiosis is a reductive type of cell division: it results in cells with fewer chromosomes. 

Figure~\ref{fig:meiosis1} is a simplified diagram illustrating the overall process and products of meiosis.

Meiosis is split into two separate parts, called meiosis I and meiosis II.
Meiosis I starts with the copying of chromosomes and their condensation into compact forms (just like mitosis). 
The metaphase of meiosis I is different, though: Instead of lining up in single file, the bivalent chromosomes line up two-by-two. 
These groups, called homologous chromosomes, are what separate during the anaphase of meiosis I (compare this to the anaphase of mitosis, where chromatids separate). 

If we look at the anaphase of meiosis I in nematodes (diploid number 4), the result is two groups of two bivalent chromosomes, rather than two groups of four monovalent ones. 
This difference in chromosome number in the post-anaphase groupings is really the only big difference between meiosis I and mitosis. 
Membranes form around the two groups, during telophase, and then the cell splits down the middle creating two non-clones. 
Each clone has half the number of chromosomes as the initial cell.

Meiosis II applies the process of mitosis to the two cells created by meiosis I. Since the chromosomes already exist in the bivalent form, interphase is skipped. The result is four cells, called gametes, each with two monovalent chromosomes.
% http://bio1510.biology.gatech.edu/module-4-genes-and-genomes/4-1-cell-division-mitosis-and-meiosis/
Meiosis sets the stage for Mendelian genetics. Students need to know that most of the genetics action occurs in the first meiotic division:
\begin{enumerate}
 \item homologous chromosomes pair up and align end-to-end (synapsis) in prophase I
 \item crossing over occurs between homologous chromosomes in prophase I, before chromosomes line up at the metaphase plate
 \item homologous chromosomes separate to daughter cells (sister chromatids do not separate) in the first division, creating haploid (1N) cells
 \item the separation of each pair of homologous chromosomes occurs independently, so all possible combinations of maternal and paternal chromosomes are possible in the two daughter cells – this is the basis of Mendel’s Law of Independent Assortment
 \item the first division is when daughter cells become functionally or genetically haploid
\end{enumerate}

Let's discuss the last point more in detail. Consider the X and Y chromosomes. They pair in prophase I, and then separate in the first division. The daughter cells of the first meiotic division have either an X or a Y; they don’t have both. Each cell now has only one sex chromosome, like a haploid cell.

One way of thinking about ploidy is the number of possible alleles for each gene a cell can have.  
Right after meiosis I, the homologous chromosomes have separated into different cells.  
Each homolog carries one copy of the gene, and each gene could be a different allele, but these two homologs are now in two different cells.  
Though it looks like there are two of each chromosome in each cell, these are duplicated chromosomes; ie, it is one chromosome which has been copied, so there is only one possible allele in the cell (just two copies of it).

The second meiotic division is where sister (duplicated) chromatids separate. 
It resembles mitosis of a haploid cell. At the start of the second division, each cell contains 1N chromosomes, each consisting of a pair of sister chromatids joined at the centromere.
\begin{figure}[t!]\centering
\includegraphics[width=\columnwidth]{{meiosis1}.pdf}
\caption{Overview of Meisosis.}
\label{fig:meiosis1}
\end{figure}
\paragraph{Genetic recombination}
Recombination occurs during meiosis and is a process that breaks and recombines pieces of DNA to produce new combinations of genes.
Recombination scrambles pieces of maternal and paternal genes, which ensures that genes assort independently from one another. 
It is important to note that there is an exception to the law of independent assortment for genes that are located very close to one another on the same chromosome because of genetic linkage.

% https://www.sciencelearn.org.nz/resources/208-meiosis-inheritance-and-variation
During fertilisation, 1 gamete from each parent combines to form a zygote. Because of recombination and independent assortment in meiosis, each gamete contains a different set of DNA. This produces a unique combination of genes in the resulting zygote.

Recombination or crossing over occurs during prophase I. Homologous chromosomes – 1 inherited from each parent – pair along their lengths, gene by gene. 
Breaks occur along the chromosomes, and they rejoin, trading some of their genes. The chromosomes now have genes in a unique combination.

Independent assortment is the process where the chromosomes move randomly to separate poles during meiosis. 
A gamete will end up with 23 chromosomes after meiosis, but independent assortment means that each gamete will have 1 of many different combinations of chromosomes.

This reshuffling of genes into unique combinations increases the genetic variation in a population and explains the variation we see between siblings with the same parents.



% \subsection{Haplotyping and its importance}
% As we also saw above, humans are diploid organisms and there are two copies of chromosome -- one inherited from father and other inherited from mother.
% Each of those copies are called \textit{homologous}
% chromosomes or \textit{haplotypes} and the DNA sequences for these homologous chromosomes is known as \textit{diploid assemblies}. 
% The organisms that have this arrangement are called
% diploid organisms or \textit{diploids}. 
% 
% \todo{write importance of haplotyping.}
% Haplotype-resolved genetic data can be used, for instance, for population genetic analyses of admixture, migration, and selection, but also to study allele-specific gene regulation, compound heterozygosity, and their roles in human disease.
% We refer the reader to \cite{Tewhey2011} and \cite{Glusman2014} for detailed reviews on the relevance of haplotyping.
% 
% \paragraph{Genetic variants and the reference genome}
% During reproduction, specialized cells recombine the pairs of
% chromosomes generating gametes: cells that have only one copy of each
% chromosome and that later can combine with a gamete from another individual
% to create a new individual. These replication and recombination
% processes are not exact, and sometimes an “error” occurs, generating genetic
% variations that can be later inherited by the offspring. The genetic variations represents the genetic uniqueness for every organism present on this planet.
% Because most of the genetic content is the same among individuals
% within the same species, it is possible to define a \textit{reference genome} for a
% population of species. For example, the Human Reference Genome \todo{HGD} and \todo{SGD} are available for yeast and humans.
% \todo{write how these reference genomes are constructed.}
% 
% \todo{structural variations, genotypes and haplotypes.}
% New developments in genome-scanning technologies and computational methodologies, and the availability of a reference sequence for comparison, have made possible the large-scale discovery of structural variants.
% Then, we can characterize the genetic variations of each individual
% in terms of how they differ from the reference genome. A variant that
% occurs in a single nucleotide, where, for instance a specific T in the reference
% genome is replaced by a G, is called single nucleotide polymorphism (SNP),
% 1.2 NGS and variation calling 3
% or single nucleotides variants (SNV). Those account for a large amount
% of the known variations in human genomes, however there are also more
% complex variants such as (large) insertions, deletions, inversions, among
% others. When a variant is present in only one of the chromosome copies,
% it is said to be a heterozygous variant. A variant that is present in both
% copies is known as a homozygous variant. 
% 
% % https://www.nature.com/articles/nrg1767
% % http://www.mi.fu-berlin.de/wiki/pub/ABI/GenomicsLecture10Materials/structural-variation.pdf
% \todo{figures for SVs}
% Structural variants are operationally defined as genomic alterations that involve segments of DNA that are larger than 1 kb, and can be microscopic or submicroscopic. 
% Different structural variants are thought to be disease causing or is discovered as part of a disease study. 
% Here we generally refer to smaller (<1 kb) variations or polymorphisms that involve the copy-number change of a segment of DNA as insertions or deletions (indels).
% 
% Below are the following types of structural variant.
% \begin{enumerate}
%  \item Copy-number variant (CNV). A segment of DNA that is 1 kb or larger and is present at a variable copy number in comparison with a reference genome. 
%  Classes of CNVs include insertions, deletions and duplications. 
%  This definition also includes large-scale copy-number variants, which are variants that involve segments of DNA $\geq 50$ kb.
%  \item Segmental duplication or low-copy repeat. A segment of DNA $\ge 1$ kb in size that occurs in two or more copies per haploid genome, with the different copies sharing >90\% sequence identity. 
%  They are often variable in copy number and can therefore also be CNVs.
%  \item Inversion. A segment of DNA that is reversed in orientation with respect to the rest of the chromosome. Pericentric inversions include the centromere, whereas paracentric inversions do not.
%  \item Translocation. A change in position of a chromosomal segment within a genome that involves no change to the total DNA content. Translocations can be intra- or inter-chromosomal.
% \end{enumerate}


\subsection{The Era of NGS}
% Next-Generation Sequencing (NGS) is an expression that refers to many
% technologies that have parallelized the sequencing process, producing vast
% amounts of sequences concurrently.
% What these have in common is that
% they cut DNA molecules from the donor into small pieces, known as reads,
% which are what is actually sequenced in great amounts. Reads length can
% range from about a hundred to a couple of thousands nucleotides. When
% reads are sequenced there is no information about their original position in
% the genome, so the output is essentially a large set of small pieces of the
% genome.2 The problem of reconstructing a genome starting from this set of
% reads is known as de novo genome assembly. This is known to be a hard
% problem, and its simplest mathematical formulation is NP-Hard [67].
% When we already posses a reference genome of the same species as the
% donor, the enterprise to sequence the donor genome is much more accessible
% through what is known as variation calling. A simplified summary of
% this process is as follows: Starting from a biological sample from the donor,
% NGS technology is used to obtain a massive set of reads. A read aligner
% maps the reads to the reference genome, ideally to positions with a high
% similarity score with the read. The number of reads whose alignment covers
% a position is known as the coverage. As the number of NGS reads is large,
% the ideal situation is to have high coverage through the genome, so that
% any position on the reference genome will have many reads piled up. This
% pile-up is analyzed, typically using statistical methods [59, 36] that can
% discover variations in the donor genome. With this process SNPs are relatively
% easy to detect, but more complex variants pose a bigger challenge.
% Nowadays, variation calling is routinely performed to sequence genomes,
% using a wide variety of methods [4, 59, 85, 74]. However, their results are
% not always consistent [74]. This in itself is motivation enough to study
% possible improvements for variation-calling mechanisms.
% 
% When variation-calling tools report a variant, they will indicate whether
% it is homozygous or heterozygous. However, this is not enough to completely
% reconstruct the donors genome: we still do not know to which of
% the two copies of each chromosome each variant belongs to. The problem of
% deciding whether two variants belong to the same copy of the chromosome
% or not is called haplotype phasing. There is a wide repertoire of methods
% to address this problem

% \paragraph{Different types of sequencing datasets and their protocols (also figures).}
% can try even writing from this: https://en.wikipedia.org/wiki/DNA_sequencing 
% https://www.ncbi.nlm.nih.gov/pubmed/19246620 
% https://www.sciencedirect.com/science/article/pii/B9780128006818000037
Since the completion of the human genome project in 2003, extraordinary progress has 
been made in genome sequencing technologies, which has led to a decreased cost per megabase 
and an increase in the number and diversity of sequenced genomes. An astonishing complexity of 
genome architecture has been revealed, bringing these sequencing technologies to even greater 
advancements. Some approaches maximize the number of bases sequenced in the least amount 
of time, generating a wealth of data that can be used to understand increasingly complex 
phenotypes. Alternatively, other approaches now aim to sequence longer contiguous pieces of 
DNA, which are essential for resolving structurally complex regions. These and other strategies 
are providing researchers and clinicians a variety of tools to probe genomes in greater depth, 
leading to an enhanced understanding of how genome sequence variants underlie phenotype 
and disease.

The different advancements come with their own limitations.
As new technologies emerge, existing problems are exacerbated or new problems arise.
NGS platforms provide vast quantities of data, but with associated error rates (~0.1-15\%)
are higher and the read-lengths generally shorter (35-700) bps for short-read approaches, 
than those of traditional \textit{Sanger} sequencing platforms, require careful 
examination of the results, particularly for variant discovery and clinical applications.
Although long-read sequencing overcome the length limitation of other NGS platforms,
it remains considerably more expensive and has lower throughput than other platforms,
limiting the wide-spread adoption of this technology in favour of less-expensive approaches.
% Finally, NGS is also competing with alternative technologies that can carry 
% out similar tasks, shown in figure; it is not clear how these disparate approaches to genomics, 
% medicine and research will interact in the years to come.
% % https://www.nature.com/articles/nrg.2016.49.pdf
% \todo{make figure from BOX 1 }

\subsubsection{Short Read NGS}
Illumina's suite of instruments for short-read sequencing range from small, low-throughput benchtop units 
to large untra-high through instruments dedicated to population-level whole genome sequencing.
dNTP identification is achieved through total internal reflection flourescence microscopy using two or four laser channels.
In most Illumina platforms, each dNTP is bound to a single flourescence that is specific to that base type and requires four different
imaging channels, whereas NextSeq and Mini-Seq systsme use a two-fluorophone system.
\begin{figure}[t!]\centering
\includegraphics[width=\columnwidth]{{ngs1}.pdf}
\caption{Overview of 454 pyrosequencing.}
\label{fig:ngs1}
\end{figure}
The first NGS instrument developed was the 454 pyrosequencing \citep{margulies2005genome} device. This SNA system distributes template-bound beads into a PicoTiterPlate along with 
beads containing an enzyme cocktail. As a dNTP is incorporated into a strand, an enzymatic cascade occurs, resulting in a bioluminescence signal. Each burst of light, 
detected by a charge-coupled device  (CCD) camera, can be attributed to the incorporation of one or more identical dNTPs at a particular bead (Figure~\ref{fig:ngs1}).

The Ion Torrent  was the first NGS platform without optical sensing \citep{rothberg2011integrated}. Rather than using an enzymatic cascade to generate a signal, the Ion Torrent platform 
detects the H + ions that are released as each dNTP is incorporated. The resulting change in pH is detected by 
an integrated complementary metal-oxide- semiconductor (CMOS) and an ion-sensitive field-effect transistor (ISFET) (Figure~\ref{fig:ngs2}).  The  pH  change  detected  by  the  sensor  is  
imperfectly proportional to the number of nucleotides detected, allowing for limited accuracy in measuring homopolymer lengths.

\begin{figure}[t!]\centering
\includegraphics[width=\columnwidth]{{ngs2}.pdf}
\caption{Overview of Ion Torrent sequencing.}
\label{fig:ngs2}
\end{figure}
The Illumina CRT system (Figure~\ref{fig:ngs3}) accounts for the largest market share for sequencing instruments compared to other platforms. 
Illumina’s suite of instruments for short-read sequencing range from small, low-throughput benchtop units to large ultra-highthroughput instruments dedicated to population-level
whole-genome sequencing (WGS). dNTP identification is achieved through total internal reflection fluorescence (TIRF) microscopy using either two or four laser
channels. In most Illumina platforms, each dNTP is bound to a single fluorophore that is specific to that base type and requires four different imaging channels, whereas the NextSeq and Mini-Seq systems use a two-fluorophore system.
\begin{figure}[t!]\centering
\includegraphics[width=\columnwidth]{{ngs3}.pdf}
\caption{Overview of Illumina sequencing.}
\label{fig:ngs3}
\end{figure}

\paragraph{Comparison of short-read platforms.}
Individual short-read sequencing platforms vary with respect to throughput, cost, error profile and read structure (\todo{TABLE 1}). 
Despite the existence of several NGS technology providers, NGS research is increasingly being conducted within the Illumina suite of instruments. Although 
this implies high confidence in their data, it also raises concerns about systemic biases derived from using a single sequencing approach. As a consequence, new 
approaches are being developed and researchers increas-ingly have the choice to integrate multiple sequencing methods with complementary strengths.
The SBL technique used by both the SOLiD and Complete Genomics systems affords these technologies a very high accuracy (~99.99\%)\citep{liu2012comparison, drmanac2010human}, as each base is probed 
multiple times. Although accurate, both platforms also show evidence of a trade-off between sensitivity and specificity, such that true variants are missed while 
few false variants are called. There is also evidence that the platforms share some under- representation of AT-rich regions, and the SOLiD platform displays 
some substitution errors and some GC-rich under-representation. Perhaps the feature most limiting to the widespread adoption of these technologies is the very 
short read lengths. Although both platforms can generate single-end and paired-end sequencing reads, the maximum read length is just 75 bp for SOLiD and 28–100 bp for Complete Genomics, limiting their use for genome 
assembly and structural variant detection applications. Unfortunately, owing to these limitations, along with runtimes on the order of several days, the SOLiD system has been relegated to a small niche within the industry. 
Illumina  dominates  the  short-read  sequencing industry owing, in part, to its maturity as a technology, a high level of cross-platform compatibility and 
its wide range of platforms. The suite of instruments available ranges from the low-throughput MiniSeq to the ultra-high-throughput HiSeq X, which is capable 
of sequencing ~1,800 human genomes to 30$\times$ coverage per year. Further diversification is derived from the many options available for runtime, read structure and 
read length (up to 300 bp). As the Illumina platform relies on a CRT approach, it is much less susceptible to the homopolymer errors observed in SNA platforms. 
Although it has an overall accuracy rate of >99.5\%,the platform does display some under-representation in AT-rich and GC-rich regions, as well as a tendency towards substitution errors. In 2008, \cite{bentley2008accurate} reported a very high concordance rate between human  single-nucleotide  polymorphisms  (SNPs)  identified with Illumina and SNPs identified from genotyping  microarrays. 
However, this high sensitivity came with a false-positive rate of around 2.5\%, leading this and other groups to consider using Sanger
sequencing to resequence the called SNPs in order to distinguish between true SNPs and false positives. With all of the possible options available, the Illumina 
suite allows for a wide range of applications: genome sequencing through WGS or exome sequencing; epi-genomics applications, such as 
ChIP–seq (chromatin immunoprecipitation followed by sequencing), ATAC –seq (assay for transposase- accessible chromatin using sequencing) or DNA methylation sequencing (methyl-seq); and transcriptomics applications through 
RNA sequencing (RNA-seq), to name a few. HiSeq X is currently the highest 
throughput instrument available; however, as a consequence of its optimization, it is limited to just a few applications, such as WGS and whole-genome bisulfite 
sequencing. HiSeq X is further limited as an all-purpose instrument owing to a required initial purchase of five or ten instruments (additional single instruments can 
be purchased after the initial commitment), placing this system out of reach of most facilities.

Both the 454 and the Ion Torrent systems offer superior read lengths compared to other short-read sequencers with reads up to an average of 700 
bp and 400 bp, respectively, providing some advantages for applications that focus on repetitive or complex DNA. However, as 
both of these platforms rely on SNA, they share many of the same drawbacks. Insertion and deletion (indel) errors dominate, although the overall error rate is on par with other NGS platforms in non-homopolymer regions. 
Homopolymer regions are problematic for these platforms, which lack single-base accuracy in measuring homopolymers larger than 6–8 bp \citep{forgetta2013sequencing, loman2012performance}. Unfortunately, whereas the Ion Torrent platform has kept pace with the 
rapidly evolving NGS field, the 454 platform has been unable to complete with other platforms in terms of yield or cost. 

\subsubsection{Long-read sequencing}
 It has become apparent that genomes are highly complex with many long repetitive elements, copy number alterations and structural variations that are relevant to evolution, adaptation and disease \citep{mccarroll2007copy, stankiewicz2010structural}. 
However, many of these complex elements are so long that short-read paired-end technologies are insufficient to resolve them. Long-read sequencing delivers reads in 
excess of several kilobases, allowing for the resolution of these large structural features. Such long reads can span complex or repetitive regions with a single continuous 
read, thus eliminating ambiguity in the positions or size of genomic elements. Long reads can also be useful for 
transcriptomic research, as they are capable of spanning entire mRNA transcripts, allowing researchers to identify the precise connectivity of exons and discern gene isoforms.

Currently, there are two main types of long-read technologies: single-molecule real-time sequencing approaches and synthetic approaches that rely on existing short-
read technologies to construct long reads in silico. The single-molecule  approaches  differ  from  short-read  approaches in that they do not rely on a clonal population of amplified DNA fragments to generate detectable 
signal, nor do they require chemical cycling for each dNTP added. Alternatively, the synthetic approaches do not generate actual long-reads; rather, they are an approach to library preparation that leverages 
barcodes to allow computational assembly of a larger fragment.

\begin{figure}[t!]\centering
\includegraphics[width=\columnwidth]{{ngs4}.pdf}
\caption{Overview of long-read sequencing and synthetic long reads.}
\label{fig:ngs4}
\end{figure}

\paragraph{Single-molecule long-read sequencing (PacBio and ONT)}
Currently, the most widely used long-read platform is the single-molecule real-time (SMRT) sequencing approach used by Pacific Biosciences (PacBio) \citep{eid2009real} (Figure~\ref{fig:ngs4}a). The instrument uses a specialized flow cell
with many thousands of individual picolitre wells with transparent bottoms — zero-mode waveguides (ZMW) \citep{levene2003zero}. Whereas short-read SBS technologies bind the DNA 
and allow the polymerase to travel along the DNA template, PacBio fixes the polymerase to the bottom of the well and allows the DNA strand to progress through the 
ZMW. By having a constant location of incorporation owing to the stationary enzyme, the system can focus on a single molecule. dNTP incorporation on each single-
molecule template per well is continuously visualized with a laser and camera system that records the colour and duration of emitted light as the labelled nucleotide 
momentarily pauses during incorporation at the bottom of the ZMW. The polymerase cleaves the dNTP-bound 
fluorophore during incorporation, allowing it to diffuse away from the sensor area before the next labelled 
dNTP is incorporated. The SMRT platform also uses a unique circular template that allows each template to be 
sequenced multiple times as the polymerase repeatedly traverses the circular molecule. Although it is difficult 
for DNA templates longer than ~3 kb to be sequenced multiple times, shorter DNA templates can be sequenced 
many times as a function of template length. These multiple passes are used to generate a consensus 
read of insert , known as a circular consensus sequence (CCS).


In 2014, the first consumer prototype of a nanopore sequencer — the MinION from Oxford Nanopore Technologies (ONT) — became available. Unlike other 
platforms, nanopore sequencers do not monitor incorporations or hybridizations of nucleotides guided by a template DNA strand. Whereas other platforms use a 
secondary signal, light, colour or pH, nanopore sequencers directly detect the DNA composition of a native ssDNA molecule. To carry out sequencing, DNA is 
passed through a protein pore as current is passed through the pore \citep{clarke2009continuous}(Figure~\ref{fig:ngs4}b). As the DNA translocates through the action of a secondary motor protein, a voltage blockade occurs that modulates the current passing 
through the pore. The temporal tracing of these charges is called squiggle space, and shifts in voltage are characteristic of the particular DNA sequence in the pore, which can then be interpreted as a 
k-mer. Rather than having 1–4 possible signals, the instrument has more than 1,000 — one for each possible k-mer, especially when modified bases present on native DNA are taken into account. The current MK1 MinION flow cell structure is composed of an application-specific integrated 
circuit (ASIC) chip with 512 individual channels that are capable of sequencing at ~70 bp per second, with an expected increase to 500 bp per second in 2016. The upcoming PromethION instrument is intended to be an ultra-high-throughput platform reported to include 
48 individual flow cells, each with 3,000 pores running at 500 bp per second. This works out to ~2–4 Tb for a 2-day run on a fully loaded device, placing this device in potential competition with Illumina’s HiSeq X. Similar to 
the circular template used by PacBio, the ONT MinION uses a leader-hairpin library structure. This allows the forward DNA strand to pass through the pore, followed by a hairpin that links the two strands, and finally the 
reverse strand. This generates 1D and 2D reads in which both ‘1D’ strands can be aligned to create a consensus sequence ‘2D’ read.

\paragraph{Synthetic long reads.}
Unlike true sequencing platforms, synthetic long-read technology relies on a system of barcoding to associate fragments that are sequenced on existing short-read sequencers. These approaches partition large DNA fragments into either microtitre wells 
or an emulsion such that very few molecules exist in each partition. Within each partition the template fragments are sheared and barcoded. This approach allows for sequencing on existing short-read instrumentation, 
after which data are split by barcode and reassembled with the knowledge that fragments sharing barcodes are derived from the same original large fragment \citep{mccoy2014illumina}. Similar to an earlier technology, 
BAC-by-BAC sequencing, synthetic barcoded reads provide an association among small fragments derived from a larger one. By segregating the fragments, repetitive or complicated regions can be isolated, allowing each to be assembled locally. 
This prevents unresolvable branch points in the assemblies, which lead to breaks (gaps) and shorter assembled contiguous  sequences.

There are currently two systems available for generating synthetic long-reads: the Illumina synthetic long-read sequencing platform (Figure~\ref{fig:ngs4}c) and the 10X Genomics emulsion-based system (Figure~\ref{fig:ngs4}d). The Illumina 
system (formerly Moleculo) partitions DNA into a microtitre plate and does not require specialized instrumentation. However, the 10X Genomics instruments (GemCode and Chromium) use emulsion to partition DNA and require the use of a microfluidic instrument 
to perform pre-sequencing reactions. With as little as 1 ng of starting material, the 10X Genomics instruments can partition arbitrarily large DNA fragments, up to ~100 kb, into micelles called ‘GEMs’, which typically contain $\leq 0.3\times$ copies of the genome and one unique 
barcode. Within each GEM, a gel bead dissolves and smaller fragments of DNA are amplified from the original large fragments, each with a barcode identifying the source GEM. After sequencing, the reads are aligned and linked together to form a series of anchored 
fragments across the span of the original fragment. Unlike the Illumina system, this approach does not attempt gapless, end-to-end coverage of a single DNA fragment. Instead it relies on linked reads, in which dispersed, small fragments that are derived from a single long molecule 
share a communal barcode. Although these fragments leave segments of the original large molecule without any coverage, the gaps are overcome by ensuring that there are many long fragments from the same genomic region in the initial preparation, thus generating a 
read cloud wherein linked reads from each long fragment can be stacked, combining their individual coverage into an overall map(Figure~\ref{fig:ngs4}d).

\paragraph{Comparison of single-molecule and synthetic long-read sequencing.}
There is growing interest in the field of long-read sequencing, and each system has its own advantages and drawbacks (\todo{TABLE 1}).
Currently, the most widely used instrument in long-read sequencing is the PacBio RS II instrument. This device is capable of generating single 
polymerase reads in excess of 50 kb with average read lengths of 10–15 kb for a long-insert library. Such properties are ideal for 
de novo genome assembly applications, for revealing complex long-range genomic structures and for full-length transcript sequencing. There are, however, several notable limitations. The 
single- pass error rate for long reads is as high as 15\% with indel errors dominating, raising concerns about the utility of the instrument. Fortunately, these 
errors are randomly distributed within each read and hence sufficiently high coverage can overcome the high error rate. The use of a circular template by PacBio also 
provides a level of error correction. The more frequently a single molecule is sequenced, the higher the resulting accuracy — up to ~99.999\% for insert sequences 
derived from at least 10 subreads.This high accuracy rivals that of Sanger sequencing, leading researchers to speculate that this technology can be used in a manner analogous to Sanger-based SNP validation. The 
runtimes and throughput of this instrument can be tuned by controlling the length of time for which the sensor monitors the ZMW; longer templates require longer times. For example, a 1 kb library that is run for 
1 hour will generate around 7,500 bases of sequence per molecule, with an average of 8 passes, whereas a 4-hour run will generate around 30,000 bases per molecule 
and ~30 passes. Conversely, a 10 kb library requires a 4-hour run to generate ~30,000 bases with ~3 passes. The limited throughput and high costs of PacBio RS II 
(around 1,000 per Gb), in addition to the need for high coverage, place this instrument out of reach of many small laboratories. However, in an attempt to ameliorate 
these concerns, PacBio has launched the Sequel System, which reportedly has a throughput 7$\times$ that of the RS II, thus halving the cost of sequencing a human genome at 30$\times$ coverage.


The ONT MinION is a small (~3 cm$\times$ 10 cm for the MK1) USB-based device that runs off a personal computer, giving it the smallest footprint of any current 
sequencing platform. This affords the MinION superior portability, highlighting its utility for rapid clinical responses and hard-to-reach field locations. Although 
substantial adjunct equipment is still required for library preparation (for instance, a thermocycler), improvements in library preparation and equipment optimization could conceivably reduce the space required for a 
fully functional sequencing system to the size of a single bag of luggage. Unlike other platforms, the MinION has few constraints on the size of the fragments to be sequenced. In theory, a DNA molecule of any size can 
be sequenced on the device, but in practice there are some limitations when dealing with ultra-long fragments. As a consequence of the unique nature of the ONT technology, in which there are more than 1,000 
distinct signals, ONT MinION has a large error rate — up to 30\% for a 1D read — and is dominated by indel errors. Effective homopolymer sequencing also remains a challenge for ONT MinION. When a homopolymer 
exceeds the k-mer length, it can be difficult to identify when one k-mer leaves the pore and another k-mer enters. Modified bases also pose a challenge to the device, as a modified base will alter the typical voltage 
shift for a given k-mer. Fortunately, recent improvements in the chemistry and the base calling algorithms are improving accuracy.

The Illumina synthetic long-read approaches are a direct response to the costs, error rates and throughput of true long-read sequencers. Relying on the existing 
Illumina infrastructure affords researchers the ability to simply purchase a kit for long-read sequencing. Accordingly, the throughput and error profile are identical to those of current Illumina devices. However, 
as  a  consequence  of  how  the  DNA  is  partitioned, the system requires more coverage than is required for a typical short-read project, thus increasing the costs associated with this technology relative to other 
Illumina applications.


Like  the  Illumina  synthetic  long-read  platform, the 10X Genomics emulsion-based platform relies on an existing short-read infrastructure to provide the 
sequencing. The microfluidic instrument is a one-time additional equipment cost, and the emulsion approach used allows for as little as 1 ng of starting material, which 
can be beneficial for situations in which the DNA is precious, such as biopsy samples. Currently, data output from the GemCode instrument is partially limited by 
the number of barcodes used and the somewhat inefficient DNA partitioning. Inefficient partitioning can lead to a surplus of DNA fragments within a droplet, thus 
complicating sequence deconvolution, which is further exacerbated by the limited number of barcodes. Both of these conditions lead to ambiguity regarding the 
positional relationship between reads sharing the same barcode, making analysis more difficult.

\subsubsection{Strand-Seq sequencing}
% https://www.nature.com/articles/nprot.2017.029
% https://www.nature.com/articles/nmeth.2206
 Strandseq \citep{falconer2012dna, sanders2017single} is a  single ­cell sequencing technique that identifies the original parental DNA template  strands  in  daughter  cells  following  cell  division. 
 The method  uses  bromodeoxyuridine  (BrdU)  incorporation  in  the nascent strand during DNA replication followed by selective degradation of the nascent strand to isolate the template strand for construction of directional sequencing libraries.

Strand-seq is a single-cell template strand sequencing technology
that generates directional genomic libraries that, when
aligned to the reference genome, permits a clear distinction
between the individual homologs of a chromosome. Homolog
resolution allows diverse types of variants to be located in a single
cell that would otherwise be very challenging to detect using
conventional approaches. This includes inversions, translocations,
copy-number changes, aneuploidy events and complex structural
variants, with haplotype awareness. By pooling data from several
dozen Strand-seq cells, each homologous chromosome can be
uniquely characterized and analyzed. 

\begin{figure}[t!]\centering
\includegraphics[width=\columnwidth]{{strandseq1}.pdf}
\caption{Overview of Strand-Seq sequencing protocol.}
\label{fig:strandseq1}
\end{figure}


In Strand-seq (Fig.~\ref{fig:strandseq1}), parental DNA template strands inherited by
daughter cells are sequenced following construction of a modified
Illumina genomic library20. The method takes advantage of the
directionality of single-stranded DNA molecules, which are distinguished
as either Crick (C; forward, or plus, strand of the reference
assembly) or Watson (W; reverse, or minus, strand) based 
on their 5\′–3\′ orientation (Fig.~\ref{fig:strandseq1}a). Daughter cells, generated following
one mitotic division of a parental cell in the presence of
BrdU, are isolated and deposited as single cells (or nuclei) into 
single wells of a 96-well plate (Fig.~\ref{fig:strandseq1}b). Forgoing any whole-genome
preamplification step, the genomic DNA is fragmented using
micrococcal nuclease (MNase) to generate mononucleosomal
fragments of ~150 bp in length. The MNase-digested chromatin
is then blunt-ended and 5′-phosphorylated by an end repair reaction,
and a single adenine is attached to the 3′ end of the fragment
by an A-tailing reaction. This adenine serves as a substrate for
standard Illumina forked adaptors, which have a single 3′ thymidine
overhang. After adaptor ligation,
the BrdU-substituted DNA strands are nicked by photolytic cleavage.
This is accomplished by treating samples with Hoechst 33258,
followed by UV radiation, which induces a single-stranded nick
at sites of BrdU incorporation. Nicking the BrdU+ strand
interferes with subsequent PCR amplification, and as a result, only
the BrdU– (i.e., the original DNA template) strand is amplified to
produce a directional single-cell library.
During PCR amplification, a custom multiplexing PCR primer
is added to the reaction, such that each
cell receives a unique hexamer barcode. This barcode allows
multiple cells to be pooled for gel size selection and sequencing.
Following Illumina HTS, the template strands in each library are
distinguished by the first sequencing read generated from the
A2 Illumina adaptor, and the single-cell barcode is read using a
custom sequencing primer generated by the second sequencing
read. By demultiplexing and aligning sequencing data to a reference
assembly, template strand identity is determined for each
chromosome in the cell (Fig.~\ref{fig:strandseq1}b).

The key difference between Strand-seq and other single-cell
sequencing protocols is that it aims to sequence only one strand of
DNA in each cell, by isolating daughter cells after one cell division.
While many steps mirror other single-cell sequencing methodologies,
the ability to produce directional single-strand libraries is
dependent on nicking the BrdU-incorporated strands through incubation
with Hoechst and subsequent UV irradiation. In addition
to introducing a photolytic cleavage step, the Strand-seq protocol
also bypasses any whole-genome amplification in order to preserve
the specific labeling of the DNA strands. This means that genomic
libraries are generated from only 6–10 pg of input DNA. Because
of this, and the multiple enzymatic cleanup steps that are required
during library preparation, we typically obtain genomic coverage
in the range of a few percentage points (1–5\%)—depending on the
number of libraries pooled and the depth of sequencing achieved.
The Strand-seq technology sacrifices coverage (depth and breadth)
for long-range structural information that offers many new applications
for genome biology, as discussed below. Finally, because the
resulting sequencing data are directional, bioinformatic analysis
must consider whether output reads map to the plus or the minus
strand of the reference genome, meaning most standard genomic
analysis methods cannot be immediately applied to the data.


% Now, we introduce some basic definitions that help in understanding the second part of this chapter.
% \todo{define read-length, coverage, variants or other basic definitions.}
% 
% \begin{definition}[Read]
%  The sequence of bases from a single molecule of DNA. Formally, it is a sequence of characters from alphabet $\{A, C, G, T\}$.
% \end{definition}
% 
% \begin{definition}[Structural variants]
%  A variation larger than single-nucleotide polymorphisms (SNPs). This can include the insertion or deletion of blocks of DNA, inversions or translocations of DNA segments, and copy-number differences.
% \end{definition}
% 
% \begin{definition}[Single-end and pair-end sequencing]
%  In single-end sequencing, a DNA template is sequenced only in one direction. In paired-end sequencing, a DNA template is sequenced in from both directions;
%  the forward and reverse reads may or may not overlap. A deviation in the expected genome alignment between two ends of a paired-end reads can indicate a structural variation.
% \end{definition}
% 
% \begin{definition}
% Chromosome: A collection of DNA and protein which organizes the human genome. Each human cell contains 23 sets of chromosomes; 22 pairs of autosomes (non sex determining chromosomes) and one pair of sex determining chromosomes. The human genome within the 23 sets of chromosomes is made of approximately 30,000 to 100,000 genes which are built from over 3 billion base pairs. While eukaryotic chromosomes are complex sets of proteins and DNA, prokaryotic chromosomal DNA is circular with the entire genome on a single chromosome
% \end{definition}
% 
% \begin{definition}
%  Assembly: The process of placing fragments of DNA that have been sequenced into their correct position within the chromosome.
% \end{definition}
% 
% \begin{definition}
%  Allele: Different forms of a gene which occupy the same position on the chromosome.
% \end{definition}
% 
% \begin{definition}
%  Repeated sequences (also known as repetitive elements, or repeats) are patterns of nucleic acids (DNA or RNA) that occur in multiple copies throughout the genome. There are different types of repeats: tandem, interspersed repeats and others like inverted repeats.
% \end{definition}
% 
% \begin{definition}[Genotype]
%  
% \end{definition}
% 
% \begin{definition}[Sequence alignment]
%  
% \end{definition}


% http://www.pacb.com/products-and-services/pacbio-systems/sequel/
% https://www.10xgenomics.com/technology/
% https://www.ncbi.nlm.nih.gov/pmc/articles/PMC3149993/
% https://www.biorxiv.org/content/biorxiv/early/2016/11/27/090001.full.pdf
% https://www.illumina.com/systems/sequencing-platforms/hiseq-x.html
% (write from introduction of david williamson and scheduling jobs on parallel machines.)
\section{Algorithmic Background}
Broadly, the algorithms to solve different computational problems are categoried into the following categories based on the type of the solution.
\begin{enumerate}
 \item Parameterized or FPT algorithms
 \item Approximation algorithms 
 \item Randomized algorithms
\end{enumerate}

\subsection{Parameterized Algorithms}
Imagine that you are an exceptionally tech-savvy security guard of a bar
in an undisclosed small town on the west coast of Norway. Every Friday,
half of the inhabitants of the town go out, and the bar you work at is well
known for its nightly brawls. This of course results in an excessive amount
of work for you; having to throw out intoxicated guests is tedious and rather
unpleasant labor. Thus you decide to take preemptive measures. As the town
is small, you know everyone in it, and you also know who will be likely to
fight with whom if they are admitted to the bar. So you wish to plan ahead,
and only admit people if they will not be fighting with anyone else at the
bar. At the same time, the management wants to maximize profit and is not
too happy if you on any given night reject more than  k  people at the door.
Thus, you are left with the following optimization problem. You have a list
of all of the  n  people who will come to the bar, and for each pair of people
a prediction of whether or not they will fight if they both are admitted. You
need to figure out whether it is possible to admit everyone except for at most
k troublemakers, such that no fight breaks out among the admitted guests.
Let us call this problem the Bar Fight Prevention problem. Figure~\ref{fig:parameter_dia1} shows an instance of the problem and a solution for
 k = 3.  One can easily check that this instance has no solution with k = 2.
 
\begin{figure}[t!]\centering
\includegraphics[width=\columnwidth]{{parameter_dia1}.pdf}
\caption{An instance of the  Bar Fight Prevention problem with a solution for k = 3. An edge between two guests means that they will fight if both are admitted}
\label{fig:parameter_dia1}
\end{figure}
 
\paragraph{Efficient algorithms for BAR FIGHT PREVENTION}
Unfortunately, the bar fight prevention is a classic NP-Complete problem (the read might have heard it under the name VERTEX COVER),
and so the best way to solve the problem is by trying all possibilties, right?
If the problem instance is small, for example, $n=100$ people, then the number of possibilties are $2^{1000}$ in a brute-force manner.
Unfortunately, this program would not finish before the guests arrive. 
There is a good news that the number $k$ of guests that should be rejected is not that large, $k \le 10$.
In this case, the total number of possibilties are $1000 \choose 10$. This is better compared to the earlier approach, but this infeasible to do it even on supercomputers.

Let's try different way, how about we identify some peaceful souls to accept, and some troublemakers we need to refuse at the door for sure.
Another cases could be identifying those persons who do not conflict.
The other case could be identifying a person who fight with atleast $k+1$ other guests.
If you identify such a person, for example,  person D in this example, we wanna strick him out, therefore, reducing the number $k$ of people you can reject by one.

Proceeding based on above observation, if there is no such person left, then we know that each guest will fight at most $k$ other guests.
Thus, rejecting any single guest will resolve at most $k$ potential conflicts.
Also, if there are more than $k^2$ potential conflict, then there is no way to ensure a peaceful night at the bar by rejecting only $k$ guests at the door.
So the guests belong to one of the two categories, one those participating in atleast one and other is at most $k$ potential conflicts,
therefore there are at most $k^2$ potential conflicts, there are at most $2k^2$ guests whose fate is not yet decided.
Now, if we try all the possibilties of $2k^2 \choose k$ also takes quite long time even on supercomputers.

After all these cases, it turns out that a simple observation yields a feasible algorithm for BAR FIGHT PREVENTION.
The crucial point is that every conflict has to be resolved, and the only way to resolve a conflict is to refuse at least one of the two participants.
The way we can solve this problem is as follows. Let's say there is atleast one unresolved conflict between person X and Y.
Trying moving X to the reject list and run the algorithm recursively by rejecting at most $k-1$ guests. If this succeeds, we are done.
In case it fails, then move X back to the undecided list and move Y to the reject list and run the algorithm recursively to check
whether the remaining conflicts can be resolved by rejecting  at most $k-1$ additional guests.
If this recursion fails, then we can not solve this instance by rejecting at most $k$ guests.

\textit{Time complexity.} There are a total of $2^k$ recursion calls and each recursion call can be solved in linear time $O(m+n)$, where $m$ is the total number of possible conflicts.
The good news is that this algorithm is even feasible on a laptop.


The above algorithm runs in time $O(2^k \cdot k \cdot n)$, while the brute-force takes $O(n^k)$ time.
There is quite some drastic difference between the running times of both the algorithms.

In $O(2^k \cdot k \cdot n)$--- time algorithm, the combinatorial explosion is restricted to the parameter $k$,
the running time is exponential in $k$, but linear in $n$.
Our goal is to find algorithms of this form.

\begin{definition}[Parameterized algorithms]
 Algorithms with running time $f(k)\cdot n^c$, for a constant $c$ independent of both $n$ and $k$, are called \textit{fixed-parameter algorithms} or FPT algorithms.
 Typically, the goal in parameterized algorithms is to design FPT algorithms, trying to make both $f(k)$ factors and the constant $c$ in the bound on the running time as small as possible.
 
 In parameterized algorithms, $k$ is simply a \textit{relevant secondary measurement} that encapsulates some aspect of the input instance, be it the size of the solution or the structure and other characterics of the input instance.
\end{definition}

\subsubsection{The art of parameterization}
For most of the biological problems, we model them as combinatorial optimization problems.
We frame an algorithm to solve them using relevant parameters as explained in previous section.
For some examples, it is easy to find a parameters from the problem input instance, while it is very difficult for others and require some important insights.
For example, consider the variant of BAR FIGHT PREVENTION problem where we want to reject at most $k$ guests
such that the number of conflicts (as we believe that the bouncers at the bar can handle $l$ conflicts, but not more).
Then we can parameterize either by $k$ or by $l$. We may even parameterize by both: then the goal is to find an FPT algorithm with running time $f(k,l) \cdot n^c$ for some computable function $f$
depending only on $k$ and $l$. In this way, the theory of parameterization and FPT algorithms can be extended to considering a set of parameters at the same time.
\begin{definition}[Parameterized algorithm (more than one parameter)]
Formally, however, one can express parameterization by $k$ and $l$ simply by defining the value $k+l$ to be the parameter:
an $f(k,l)\cdot n^c$ algorithm exists if and only if an $f(k+l) \cdot n^c$ algorithm exists.
\end{definition}

We can now formulate the extended BAR FIGHT PREVENTION in terms of parameters $k$ and $l$ as follows.
These parameters are explicity given in the input, defining properties of the solution we are looking for.

We can have more variants of BAR FIGHT PREVENTION problem, where we need to reject at most $k$ guests such that,
say, the numbers of conflicts decreases by $p$, or such that each accepted guest has conflicts with at most $d$ other accepted guests,
or such that the average number of conflicts per guest is at most $a$. Then the parameters $p, d, a$ are again explicity given in the input,
telling us what kind of solution we need to find. 

For the string or sequence problems related to genomics, one can parameterize by the maximum read-length, by the maximum coverage,
by the size of alphabet, by the number of alleles in a variant.

Parameterized complexity allows us to study how different parameters influence the complexity of the problem.
A successful parameterization of a problem needs to satisfy two properties:
First, there should be specific reason for the choice of parameters for different applications.
Second, we need efficient algorithms where the combinatorial explosion is restricted to the parameter(s), that is, we want the problem to be in FPT with this parameterization.

In conclusion, for the same problem, there can be multiple choices of parameters. 
Selecting the right parameter(s) for a particular problem is an art.

% \subsubsection{Formal definitions}
We now introduce the formal foundation of parameterized complexity.

\begin{definition}
 A parameterized problem is a language $L \in \sum* \times N$, where $\sum$ is a fixed, finite alphabet. For an instance $(x,k) \in \sum* \times N$, $k$ is called the parameter.
\end{definition}

We define the size of an instance $(x,k)$ of a parameterized problem as $|x| + k$. 

\begin{definition}
 A parameterized problem $L \subset \sum* \times N$ is called \textit{fixed-parameter tractable} (FPT) if there exists an algorithm $\mathcal{A}$ (called a fixed-parameter algorithm),
 a computable function $f: N \rightarrow N$ and a constant $c$ such that, given $(x,k) \in \sum* \times N$, the algorithm $\mathcal{A}$ correctly decides
 whether $(x,k) \in L$ in time bounded by $f(x). |(x,k)|^c$.
 The complexity class containing all fixed-parameter problems is called FPT.
\end{definition}


Observe that, given some parameterization problem $L$, the algorithm designer has essentially two different optimization goals when designing FPT algorithms for $L$.
Since the running time has to be of the form of $f(k)\cdot n^c$, one can:

\begin{enumerate}
 \item optimize the \textit{paramteric dependence} of the running time, i.e., try to design an algorithm where function $f$ grows as slowly as possible; or
 \item optimize the \textit{polynomial factor} in the running time, i.e. try to design an algorithm where constant $c$ is as small as possible.
\end{enumerate}

\subsection{Randomized Algorithms}
We define a randomized algorithm as an algorithm that is allowed access to a source of independent, 
unbiased, random bits; it is then permitted to use these random bits to influence its computation.
It is easy to sample a random element from a set $S$ by choosing $O(log |S|)$ random bits and then using these
bits to index an elememnt in the set. 

There are two principal advantages to randomized algorithms. The first is performance --- for many problems, 
randomized algorithms run faster than the best known deterministic algorithms.
Second, many randomized algorithms are simpler to describe and implement than deterministic algorithms of
comparable performance. 

We define a very useful tool from probability theory that is used in the analysis of randomized algorithm:
\textit{linear of expectation}. For random variable, $X_1, X_2 \ldots$

\begin{equation}
 E[\sum_i X_i] = \sum_i E[X_i].
\end{equation}

Let us consider the example of \textit{binary planar partition} of a set of $n$ disjoint line segments in the plane.
A binary planar partition consists of a binary tree together with some additional information.
Every internal node of a tree has two children. Associated with each node $v$ of a tree in a region $r(v)$ of the plane.
Associated with each internal node $v$ of a tree is a line $l(v)$ that intersects $r(v)$.
The region corresponding to the root is the entire plane.
The region $r(v)$ is partitioned by $l(v)$ into two regions $r_1(v)$ and $r_2(v)$,
which are the regions associated with the two children of $v$. Thus, any region $r$ of the partition
is bounded by the partition lines on the path from the root to the node corresponding to $r$ in the tree.
% \todo{think of better real-world example than in computer graphics.}

\begin{figure}[t!]\centering
\includegraphics[width=\columnwidth]{{random_dia1}.pdf}
\caption{An example of a binary planar partition for a set of segments (dark lines). Each leaf is labeled by the line segment it contains. The labels r(v) are omitted for clarity.}
\label{fig:random_dia1}
\end{figure}

Given a set $S = \{s_1, s_2, \ldots, s_n\}$ of non-intersecting line segments in the plane,
we wish to find a binary planar partition such that every region in the partition contains at most one line segment.
Notice that the definition allows us to divide the input line segment $s_i$ into several segments $s_{i1}, s_{i2}, \ldots,$
each of which lies in a different region. The example shows such a partition of a set into three line segments in Figure~\ref{fig:random_dia1}.

For a line segment $s$, let $l(s)$ denote the line obtained by extending $s$ on both sides to infinity.
For the set $S = \{s_1, s_2, \ldots, s_n\}$ of line segments, a simple and natural class of partition 
is the set of autopartitions, which are formed by only using lines from the set $\{l(s_1), l(s_2), \ldots, l(s_n)\}$
in constructing the partition. 

In the algorithm, the input is the set $S = \{s_1, s_2, \ldots, s_n\}$ of non-intersecting line segments.
The output is the binary autopartitions of $S$. To solve it, we pick a permutation $\pi$ of $\{1,2, \ldots, n\}$ uniformly
at random from the $n!$ possible permutations. Then we cut it with $l(s_i)$ where $i$ is first in the ordering $\pi$
such that $s_i$ cuts the region. We repeat this process till a region contains more than one segment.

For the analysis of randomized algorithm, we show that the expected size of autopartitions produced by the algorithm is $O(n log n)$.
We can easily show this using linear of expectation of the intersections.

\subsection{Approximation Algorithms}

Most of the real-world problems can be modelled as optimization problems and unfortunately, most interesting discrete optimization problems are NP-hard.
Thus unless P=NP, there are no efficient algorithms to find optimal solutions to such problems. We define an efficient algorithm is the one that is solved in time polynomial in input size.
What should we do next? How about we find a near-optimal solution which can be found in polynomial time? Furthermore, we focus on finding 
polynomial-time algorithms for some special cases of the problem, instead of any instance.


Here, we present the \textit{approximation algorithms} for discrete optimization problems.
We try to find a solution that closely approximates the optimal solution in terms of its \textit{value}.
We assume that there is some \textit{objective function} mapping each possible solution of an optimization problem 
to some nonnegative value, and an \textit{optimal solution} to the optimization problem is the one 
that either minimizes or maximizes the value of this objective function. We define an approximation algorithm as follows.

\begin{definition}
 An $\alpha$-approximation algorithm for an optimization problem is a polynomial-time algorithm that for all instances of the problem produces a solution whose value 
 is within a factor $\alpha$ of the value of an optimal solution.
 \end{definition}
 
 For an $\alpha$-approximation algorithm, we will call $\alpha$ the \textit{performance guarantee} of the algorithm.
 
 There are several advantages of devising approximation algorithm as follows.
 \begin{enumerate}
  \item An approximation algorithm provides a way to find the near-optimal solutions when the optimal solution is not required and NP hard to find.
  \item It give us an idea about how to devise an heuristic that will perform well in practice for the actual problem.
  \item It provides a mathematically rigorous basis on which to study heuristics.
  \item The field of approximation algorithms gives us a means of distinguising between various optimization problems in terms of how well they can be approximated.
 \end{enumerate}
 
 \begin{figure}[t!]\centering
\includegraphics[width=.9\columnwidth, height=.8\columnwidth]{{approx_dia1}.pdf}
\caption{An instance of Vertex Cover problem. An optimal vertex cover is {b, c, e, i, g}.}
\label{fig:approx_dia1}
\end{figure}
 
 The next question is about the quality of approximation algorithms, which means does the problems of our interest have \textit{polynomial-time approximation schemes}.
 
 \begin{definition}
  A polynomial-time approximation schemes (PTAS) is a family of algorithm $\{\mathcal{A_\epsilon}\}$,
  where there is an algorithm for each $\epsilon > 0$, such that $A_\epsilon$ is a $(1+\epsilon)$-approximation
  algorithm (for minimization problems) or a $(1-\epsilon)$-approximation algorithms (for minimization problems). 
 \end{definition}
 
 Let us consider an example of vertex cover in Figure~\ref{fig:approx_dia1} to explain approximation algorithm. 
%  https://www.cs.dartmouth.edu/~ac/Teach/CS105-Winter05/Notes/wan-ba-scribe.pdf
Given a $G= (V,E)$, find a minimum subset $C \subset V$, such that $C$ \textit{covers} all edges in $E$, i.e., 
every edge $\in E$ is incident to at least one vertex in $C$.

We consider the following steps to solve this problem. Let us consider set $C$ as empty. 
We pick any edge $\{u,v\} \in E$ and add it to the set $C$. Afterwards we delete all the edges incident to either $u$ or $v$.
We repeat this process till $E$ is not empty.

The above algorithm is a 2-approximation for vertex cover problem.






\chapter{Approximation algorithm for phasing individual genome}

In this chapter, we study the approximation status of \GMEC, which is a variant of \MEC problem (see Problem~\ref{prob:mec}) introduced in Chapter 1.
In other words, the minimum error correction problem (\MEC) is a segmentation problem where we have to partition a set of length $m$ strings into two classes.
A \MEC instance is an $n \times m$ matrix $M$ with entries from $\{0,1,-\}$. 
    Feasible solutions are composed of two binary $m$-bit strings, together with an assignment of each row of $M$ to one of the two strings.
    The objective is to minimize the number of mismatches (errors) where the row has a value that differs from the assigned solution string.
    The symbol ``$-$'' is a wildcard that matches both $0$ and $1$.
    A \MEC instance is \textit{gapless}, if in each row of $M$ all binary entries are consecutive.

 \GMEC is a relevant problem in computational biology, and it is closely related to segmentation problems that were introduced by {[}Kleinberg--Papadimitriou--Raghavan STOC'98{]} in the context of data mining.
 
     Without restrictions, it is known to be $\UG$-hard \citep{trevisan2012khot} to compute an $O(1)$-approximate solution to \MEC. For both \MEC and \GMEC, the best polynomial time approximation algorithm has a logarithmic performance guarantee.
    We partially settle the approximation status of \GMEC by providing a quasi-polynomial time approximation scheme (QPTAS).
    Additionally, for the relevant case where the binary part of a row is not contained in the binary part of another row, we provide a polynomial time approximation scheme (PTAS).

\section{Our results.}
Our main result is the following theorem.
\begin{theorem}\label{thm:qptas}
    There is a quasi-polynomial time approximation scheme (QPTAS) for \GMEC.
\end{theorem}
We therefore partially settle the approximability for this problem: \GMEC is not $\APX$-hard unless $\NP \subseteq \QP$ (cf.~\cite{RS09_approximation}).
Thus our result reveals a separation of the hardness of the gapless case and the case where we allow a single gap.
Furthermore, already \BMEC is strongly $\NP$-hard since the input does not contain numerical values. 
Therefore we can exclude the existence of an FPTAS for both $\BMEC$ and $\GMEC$ unless $\P = \NP$.

Additionally, we address the class of \emph{subinterval-free} \GMEC instances where no string is contained in another string.
More precisely, for each pair of rows from $M$ we exclude that the set columns with binary entries from one row is a strict subset of
the set of columns with binary entries from the other row.

\begin{theorem}\label{thm:ptas}
    There is a polynomial time approximation scheme (PTAS) for $\GMEC$ restricted to instances such that no string is the substring of another string.
\end{theorem}

\section{Overview of our approach.}

Our algorithm is a dynamic program (DP) that is composed of several levels.
Given a general \GMEC instance, we decompose the rows of the instance into length classes according to the length of the contiguous binary parts of the rows.
For each length class we consider a well-selected set of columns such that each row crosses at least one the columns and at most two.
(We say that a row crosses a column if the binary part of the row contains that column.)

We further decompose each length class into two sub-classes, one that crosses exactly one column and one that crosses exactly two columns.
For the second class, it is sufficient to consider every other column, which leaves us with many \emph{rooted} instances.
Thus for each sub-instance there is a single column (the root) which is crossed by all rows of the instance.

We further decompose rooted sub-instances into the left hand side and the right hand side of the root.
Since the two sides are symmetric, we can arrange the rows and columns of these sub-instances in such a way that all rows cross the first column.
We order the rows from top to bottom by increasing length in order to be able to further decompose the instance.

The first level of our DP solves these highly structured instances.
The basic idea that we would like to apply is that we select a constant number of rows from the instance that represents the solution.
Without further precautions, however, this strategy fails because of differing densities within the instance: 
the selected rows have to represent both the entries of columns crossed by many short rows and entries of arbitrarily small numbers of rows crossing many columns.
To resolve this issue, we observe that computing the solution strings $\sigma$ and $\sigma'$ is equivalent to finding a partition of $M$ into two row sets, one assigned to $\sigma$ and the other assigned to $\sigma'$.
If we assume to have the guarantee that for both solution strings $\sigma$ and $\sigma'$ an $\epsilon$ fraction of rows of the matrix $M$ forms a \BMEC sub-instance, we show that the idea above works.

This insight motivates to separate the instance from left to right into sub-problems with the required property and to assemble them from left to right using a DP.
There are, however, several complications.
In order to choose the right sub-instances, we have to take into account that the choice depends on which rows are assigned to $\sigma$ and which are assigned to $\sigma'$.
Therefore the DP has to take special care when identifying the sub-instances.

Furthermore, in order to stitch sub-instances together to form a common solution, the solution computed in the left sub-instance has to compute its solution oblivious of the choices of the right sub-instance.
This means that we have to compute a solution to the left sub-instance without looking at a fraction of rows.
We present an algorithm for these sub-instances in Section~\ref{sec:swc}.

%\sgnote{sub-instances of?}
In order to combine the sub-instances, we face further technical complications due to having distinct sub-instances for those rows assigned to $\sigma$ and those rows assigned to $\sigma'$.
In Section~\ref{sec:second}, we introduce a DP whose DP cells are pairs of simpler DP cells, one of $\sigma$ and one for $\sigma'$.
%In order to separate these complications from the actual combination of sub-problems, we first consider instances for which we only want to have one solution string, in Section~\ref{sec:single}.
%We note that these instance would be easily solvable by simply selecting the majority value in each column, but our aim is something different.
%As needed for the actual instances, in the DP we restrict our view and only considering constantly many rows at a time.
%Afterwards, in Section~\ref{sec:second}, we show how to combine our insights with taking care of both $\sigma$ and $\sigma'$.

Before we consider general instances, we first develop our techniques by considering subinterval-free instances which are easier to handle (Section~\ref{sec:subinterval-free}).
Observe that the instances considered until now are special rooted sub-interval-free instances.
We show how to solve arbitrary rooted sub-interval-free instances by combining the DP with additional information about the sub-problems that contain the root.
We then introduce the notion of domination in order to combine rooted sub-interval-free instances with a DP proceeding from left to right.
The main idea is that a dominant sub-problem dictates the solution.
At the interface of two sub-instances, there can be a (contiguous) region where none of the two sub-problems is dominant.
We show that these regions can be solved directly by considering a constant number of rows (using the results from Section~\ref{sec:swc}).

Until this point, all parts of our algorithm run in polynomial time.
%We lose this property when considering arbitrary rooted instances, in Section~\ref{sec:QPTAS}.
We lose this property when considering length classes, in Section~\ref{sec:length-classes}.
The length classes allow us to separate an instance into rooted sub-instances.
%The reason is that the left hand side of the root may have a completely different structure than the right hand side of the root.
The difficulty is that the left hand side of a separating column may have a completely different structure than the right hand side of that column.
We do not know how to combining the two sides by considering only a polynomial number of possibilities.
If we allow, however, quasipolynomial running time, we can solve the problem. 
We use that each of the two sub-instances (the one on the left and the one on the right) is composed of at most logarithmically many parts.
Considering all parts simultaneously allows us to take care of dependencies between the left hand side and the right hand side and still solve them as if they were separate instances.
%
%We next consider \GMEC instances where all strings are in the same length class, \ie, the longest binary part is at most twice as long as the shortest binary part of a row (Section~\ref{sec:length-classes}).
%As described earlier, we consider two sets of rows such that for each of these sets there is a sequence of columns such that each row crosses exactly one of the columns.
Combining such rooted instances from left to right then can be done in the same spirit as combining rooted sub-interval-free instances.
% \sgnote{why quadruples? DP cells from sub-internal free?}
To solve the entire length-class, we combine both solutions by running a new DP that considers quadruples of DP cells.

Finally, in Section~\ref{sec:generalQPTAS}, we are able to handle all length classes simultaneously. 
We solve general instances in the same spirit as the combined sub-instances of a single length class.
Instead of considering quadruples of cells, however, we form collections of quadruples that are -- figuratively speaking -- stacked on top of each other.
The key insight is that there are only $O(\log(n))$ different length classes and each collection has at most one quadruple of each length class.
Considering all possible collections adds another power of $\log(n)$ to the running time, which is still quasi-polynomial.

\section{Further related work.}
Binary-MEC is a variant of the Hamming $k$-Median Clustering Problem when $k = 2$ and there are PTASs known~\cite{JXL04_k, OR02_polynomial}. 
\cite{LMW02_finding} provided a PTAS for the general consensus pattern problem which is closely related to \MEC.
Additionally, they provided a PTAS for a restricted version of the star alignment problem aligning with at most a constant number of gaps in each sequence.
More recently, ~\cite{BDK+16_minimum} showed that it is unique games hard to approximate \MEC with constant performance guarantee, whereas it is approximable within a logarithmic factor in the size of the input. 
\GMEC was shown to be $\NP$-hard by ~\cite{Cilibrasi2007}.\footnote{Their result predates the hardness result of ~\cite{Fei14_np} for H2S. The proof of the claimed $\NP$-hardness of H2S by ~\cite{KPR98_segmentation} was never published.}
Additionally, they showed that allowing a single gap in each string renders the problem $\APX$-hard.


~\cite{AS99_two} provided a PTAS for H2S, the maximization version of \BMEC and ~\cite{WUB13_monochromatic} showed that there is also a PTAS for the maximization version of \MEC.
For \GMEC, ~\cite{he2010optimal} studied the fixed-parameter tractability in the parameter of fragment length with some restrictions.
These restrictions allow their dynamic programming algorithm to focus on the reconstruction of a single haplotype and, hence, to limit the possible combinations for each column.
There is an FPT algorithm parameterized by the coverage~\cite{Patterson2015,garg2016read} (and some additional parameters for pedigrees). 
~\cite{BDK+16_minimum} provided FPT algorithms parameterized by the fragment length and the total number of corrections for Gapless-MEC.
There are some tools which can be used in practice to solve Gapless-MEC instances~\cite{Pirola2015, Patterson2015}.

Most research in haplotype phasing deals with exact and heuristic approaches to solve \BMEC.
Exact approaches, which solve the problem optimally, include integer linear programming~\cite{Fouilhoux2012} and fixed-parameter tractable algorithms~\cite{he2010optimal, Pirola2015}.
There is a greedy heuristic approach proposed to solve Binary-MEC~\cite{Bansal2008}. 

~\cite{Lancia2001} obtained a network-flow based polynomial time algorithm for Minimum Fragment Removal (MFR) for gapless fragments.
Additionally, they found the relation of Minimum SNPs Removal (MSR) to finding the largest independent set in a weakly triangulated graph.

\section{Preliminaries and notation.}\label{sec:prelim}

We consider a \GMEC instance, which is a matrix $M \in \{0,1, -\}^{n \times m}$.
The $i$th row of $M$ is the vector $M_{i,*} \in \{0,1, -\}^{1 \times m}$ and the $j$th column is the vector $M_{*,j} \in \{0,1, -\}^{n \times 1}$.
The length of the binary part in $M_{i,*}$ is $|M_{i,*}|$. 
We say that the $i$th row of $M$ \emph{crosses} the $j$th column if $M_{i,j} \in \{0,1\}$.

For each feasible solution $(\sigma,\sigma')$ for $M$, we specify an assignment of rows $M_{i,*}$ to solution strings.
The default assignment is specified as follows.
For a row $M_{i,*}$, we assign $M_{i,*}$ to $\sigma$ if $\dist(\sigma,M_{i,*}) \le \dist(\sigma',M_{i,*})$. For $\dist(\sigma,M_{i,*})$, see Definition~\ref{eq:distance}
Otherwise we assign $M_{i,*}$ to $\sigma'$.
For the rows of $M$ assigned to $\sigma$ we write $\sigma(M)$ and for the rows assigned to $\sigma'$ we write $\sigma'(M)$.
For a given instance, $\Opt = (\tau, \tau')$ denotes an optimal solution.
Observe that knowing $\Opt$ allows us to obtain an optimal assignments $\tau(M)$ and $\tau'(M)$ by assigning each row to the solution string with fewest errors and knowing $\tau(M)$ and $\tau'(M)$ allows us to obtain an optimal solution by selecting the column-wise majority values.

\section{Simple instances with wildcards.}\label{sec:swc}
In this section, we consider instances of \GMEC where all entries of column one in $M$ are zero or one, i.e., $M_{i,1} \in \{0,1\}$ for each index $i$.
Observe that the wildcards now have a simple structure which we refer to as SWC-structure (for ``simple wildcards'').
An instance with SWC-structure is an SWC-instance.

\begin{definition}[Standard ordering of SWC-instances]
    We define the \emph{standard ordering} of rows in $M$ such that $|M_{i,*}| \le |M_{i+1,*}|$ for each $i$, \ie, we order them from top to bottom in increasing length of the binary part. 
    \label{def:order-SWC}
\end{definition}

\begin{definition}[Good SWC-instances]
    \label{def:good-SWC}
    We call an SWC-instance $M$ \emph{good}, if it is in standard ordering and there are at least $\epsilon |\tau(M)|$ rows of $\tau(M)$ and at least $\epsilon |\tau'(M)|$ rows of $\tau'(M)$ that have only entries from $\{0,1\}$.
\end{definition}

To solve good SWC-instances, we generalize the PTAS by Jiao et al.~\cite{JXL04_k} for \BMEC to solve a specific class of \GMEC instances. 
Our algorithm requires partitions of the set of rows.
In the following two definitions, the required number of rows may be a fractional number. 
To solve the problem, we allow the assignment of fractional rows, i.e., for a row $i$, we can choose an $x \in [0,1]$ and assign an $x$ fraction of $i$ to one set and a $1-x$ fraction to the other set. 
\begin{definition}[Trisection]
    An \emph{$\epsilon$-trisection} of an instance $M$ for $\tau$ is a partition of the rows into three consecutive ranges that have the following properties.
    \begin{enumerate}
        \item The first range $U$ contains row $M_{1,*}$ and $(1-\epsilon) |\tau(M)|$ rows of $\tau(M)$.
        \item The second range $L$ is consecutive to first row set containing $(\epsilon - \epsilon^2)|\tau(M)|$ rows of $\tau(M)$.
        \item The third range $X$ contains the remaining rows in $M$.
    \end{enumerate}
    To avoid ambiguity, we choose $L$ and $X$ such that the first row is in $\tau(M)$.

    We define an $\epsilon$-trisection $U'$, $L'$, and $X'$ for $\tau'$ analogously, replacing $\tau(M)$ by $\tau'(M)$.
    \label{def:trisection}
\end{definition}

\begin{definition}[Subdivision of trisections]
    We consider the rows sets $U, L, U', L'$ from Definition~\ref{def:trisection} and additionally, we divide each of these sets into $1/\epsilon^2$ disjoint subsets denoted as $R_i, S_i, R'_i, S'_i$.
    For each $i$, $R_i$ contains $\epsilon^2 \cdot |R|$ rows from $\tau(M)$ and $S_i$ contains $\epsilon^2 \cdot |S|$ rows from $\tau(M)$.
    Analogously, each $R'_i$ contains $\epsilon^2 \cdot |R'|$ rows from $\tau'(M)$ and $S'_i$ contains $\epsilon^2 \cdot |S'|$ rows from $\tau'(M)$.
    To avoid ambiguity, each set $R_i$ and $S_i$ starts with a (fractional) row of $\tau(M)$ and each set $R'_i$ and $S'_i$ starts with a (fractional) row of $\tau'(M)$. 
    \label{def:subdivision}
\end{definition}

\begin{definition}[Error difference]
   Let $i$ be a row of instance $M$ and let $(\sigma,\sigma')$ be a solution pair.
   Then the error difference of $\sigma$ and $\sigma'$ is defined as
   $d_i := \dist(\sigma,M_{i,*}) - \dist(\sigma',M_{i,*})$.
   \label{def:errors}
\end{definition}
%
We introduce a new algorithm $\textsc{SWC}_\delta$ for our setting.
For an instance $M$, we consider the rows sets $U, L, U', L'$ shown in Fig.~\ref{fig:section 2.1} from the $\epsilon$-trisections of $M$ and their subsets according to Definition~\ref{def:subdivision}.
We further select a multi-set of strings from $R'_i \cap \tau(M)$ and $S'_i \cap \tau'(M)$.
We then compute the majority weighting according to Definition~\ref{def:weighted-majority} for each column $j$ using 
multisets based on the minimum number of errors (according to Definition~\ref{def:errors}).
The main idea to find two small row sets that represent the whole instance $M$.
The intuitive meaning is that we select rows from the upper part with a much lower density then the rows of the lower part.
We therefore introduce a bias such that all rows are equally important.

\begin{figure}[h]
    \begin{center}
        \begin{tikzpicture}[scale=0.7]
            \draw (-2,-2) rectangle (2,3);      % A
            \draw (-2,-3) rectangle (9,-2);     % B
            \draw[dashed] (2,3) -- (9,-2);
            \draw[dashed] (9,-2) -- (18,-3);
            \draw[pattern=north east lines] (-2,-3.5) rectangle (18,-3); % Bottom area
            \node at (8,0){$U \cup U'$};
            \node at (14,-2.0) {$L \cup L'$};
            \node at (4,-3.3) {$X \cup X'$};
            \node at (-2.5,0){\(\left(\rule{0cm}{2.4cm}\right.\)};
            \node at (19,0){\(\left.\rule{0cm}{2.4cm}\right)\)};

        \end{tikzpicture}
        \caption{\label{fig:section 2.1} Row sets on an instance $M$ with partitions $U \cup U'$ (top), $L \cup L'$ (middle) and $X \cup X'$ bottom. For an error analysis, the instance is divided into blocks. All blocks have binary values without wildcards.}
    \end{center}
\end{figure}

\begin{definition}[Weighted majority]
    Let $j$ be an integer and let $\tilde{U}$ and $\tilde{L}$ be two matrices with at least $j$ columns.
    In $\tilde{U}_{*,j}$ and $\tilde{L}_{*,j}$, we replace all zeros by $-1$ and then all wildcard symbols by zero.
    We then compute the number
    $\nu := \sum_{i \in \tilde{U}_{i,j}} (1-\epsilon)i/(\epsilon - \epsilon^2) + \sum_{i \in \tilde{L}_{i,j}} i$. 
    Then $\textsc{Majority}_j(\tilde{U}, \tilde{L}) = 0$ if $\nu <0$ and $\textsc{Majority}_j(\tilde{U}, \tilde{L}) = 1$ if $\nu \ge 0$.
    \label{def:weighted-majority}
\end{definition}
With this preparation, we are now ready to present the algorithm.
The input has a long list of parameters that will allow our dynamic programs later on to control the execution.
The reason is that we do not know $\tau$ and $\tau'$. 
Therefore the algorithm takes \emph{guesses} of row sets as input.
The values $r$ and $r'$ are guesses of $|\tau(M)|$ and $|\tau'(M)|$.
\smallskip

\begin{algorithm}[H] 
    \caption{
        \label{alg:SWC}
        $\textsc{SWC}_\delta$}
    \SetAlgoNoLine
    \SetNlSkip{1em}
    \SetKwInOut{Input}{Input}
    \SetKwInOut{Output}{Output}
    \Input{Row sets $R_i$, $S_i$, $R'_i$ and $S'_i$ of a good SWC-instance $M$,  numbers $r, r'$.\\ Optional: selection of rows $\tilde{U}_i,\tilde{L}_i,\tilde{U}'_i,\tilde{L}'_i$, see below.}
    \Output{A pair of solution strings $(\sigma,\sigma')$.}
    Run the algorithm for each possible selection of the following type and keep the best outcome (minimum number of errors)\tcp*{If provided as input, skip selection.}
    For each $i$, select (with repetition) a multi-set $\tilde{U}_i$ of $1/\delta$ strings from $R_i$ and $\tilde{L}_i$ from $S_i$\;
    For each $i$, select (with repetition) a multi-set $\tilde{U}'_i$ of $1/\delta$ strings from $R'_i$ and $\tilde{L}'_i$ from $S'_i$ such that $\tilde{U}' \cap \tilde{U} = \tilde{L}' \cap \tilde{L} = \emptyset$\;
    For each column $j$, set $\sigma_j := \textsc{Majority}_j(\tilde{U}, \tilde{L})$ and $\sigma'_j := \textsc{Majority}_j(\tilde{U}', \tilde{L}')$\;
    For each row $i$ of $M$, determine the value $d_i := \dist(\sigma,M_{i,*}) - \dist(\sigma',M_{i,*})$\;
    Assign the $r$ rows with minimal values $d_i$ to $\sigma$ and the remaining $r'$ rows to $\sigma'$.
\end{algorithm}
\smallskip

Observe that for small (\ie, constant) values of $r$ or $r'$, the algorithm $\textsc{SWC}_\delta$ can be replaced by an exact algorithm since we know $\tau(M)$ if and only if we know $\tau'(M)$, and we are able to guess constantly many rows.

%In order to be able to show our results for arbitrary good SWC-instances, we first have to consider special instances.
\begin{lemma}\label{lem:SWC}
    Let $M$ be a good SWC-instance.
    For sufficiently large $r = |\tau(M)|$ and $r' = |\tau'(M)|$, let $R_i, S_i, R'_i, S'_i$ be a subdivision (Definition~\ref{def:subdivision}) of an $\epsilon$-trisection $U,L,X,U',L',X'$ of $M$.

    Then $\textsc{SWC}_{\epsilon^3}$ is a $(1 + O(\epsilon))$-approximation algorithm for $M$. 
\end{lemma}
The proof is based on a randomized argument using Chernoff bounds and can be found in Appendix~\ref{app:SWC}.
%
Lemma~\ref{lem:SWC} shows that the set of solutions considered by $\SWC$ contains at least one solution that is good enough even though we do not look at $X$. 
It does not say that we finally compute that solution, since other solutions may have fewer errors in $U\cup L$ or $U' \cup L'$.
% \sgnote{Punch line is missing why we need a stronger statement, probably the answer is exactly what you explained me yesterday.\tgm{added a punch line, please check if you like it}}
For our dynamic programs, we need a stronger statement.
We would like to be able to compute a solution for a \emph{partial} instance and afterwards change a fraction of lines without losing the approximation guarantee.
%Intuitively, the following lemma states that we can find a $(1+O(\epsilon))$ approximation for good SWC-instances, even if a fraction of rows (depending on $\epsilon$) is hidden.
The next lemma is a key ingredient of our result.

\begin{lemma}
    \label{lem:swc-gap}
    Let $M$ be a good SWC-instance and $\epsilon > 0$ sufficiently small.
    Let $U,L,X$ be an $\epsilon$-trisection for $\tau(M)$ and $U',L',X'$ an $\epsilon$-trisection for $\tau'$, with subdivisions $R_i,S_i,R'_i,S'_i$ according to Definition~\ref{def:subdivision}.
    Let $(\sigma, \sigma')$ be the solution computed by $\textsc{SWC}_{\epsilon^3}$ with $r = |\tau(M)|$, $r' = |\tau'(M)|$.
    Then re-assigning the rows $\sigma(X)$ to $\tau(X)$ and $\sigma'(X')$ to $\tau'(X')$ gives a $(1 + O(\epsilon))$-approximation for the instance $M$.
\end{lemma}
\begin{proof}
    For ease of presentation, we assume that all appearing numbers are integers.
    It is easy to adapt the proof by rounding fractional numbers appropriately.

    We first analyze the computed solution string $\sigma$. 
    Let $\eta$ be the total number of errors of $(\tau,\tau')$ within $M$ and
    let $\eta_P$ be the total number of errors of $(\sigma,\sigma')$ within $P := U \cup L$.
    Due to Lemma~\ref{lem:SWC}, we have $\eta_P \le (1+O(\epsilon)) \eta$.

    We may assume $r \ge r'$ since otherwise we can simply rename the two strings $\tau$, $\tau'$.
    Additionally, by renaming of $\sigma$ and $\sigma'$, we may assume that $|\sigma(P) \cap \tau(P)| \ge |\sigma'(P) \cap \tau(P)|$.
    Therefore $|\tau(P)| \ge n/3$ and $|\sigma(P) \cap \tau(P)| \ge n/6$.
    (Recall that the matrix $M$ has $n$ rows and $m$ columns.)

    \begin{claim}\label{claim:agree}
        There is a set $I$ of $m - 25\eta/n$ indices $i$ such that
        $\sigma_i = \tau_i$ for all $i \in I$.
    \end{claim}
    \begin{subproof}
        We concentrate on the columns of $M$ where both strings $\tau$ and $\sigma$ have at most $n/12$ errors within $P$.
        By counting the errors, there are at most $12 \eta/n$ columns where $\tau$ has at least $n/12$ errors.
        Similarly, there are at most $12 (1+O(\epsilon)) \eta_P/ n < 13 \eta/ n$ many columns where $\sigma$ has at least $n/12$ errors.
        Therefore there is a set $I$ of at least $m - 25\eta/n$ columns where simultaneously both $\tau$ and $\sigma$ have less than $n/12$ errors each.

        Now suppose that the claim was not true and there was an index $i \in I$ with $\tau_i \neq \sigma_i$.
        Then, since $|\tau(P) \cap \sigma(P)| \ge n/6$, either $\sigma_i$ or $\tau_i$ is erroneous in at least $n/12$ rows of $\tau(P) \cap \sigma(P)$, a contradiction.
    \end{subproof}

    Next we analyze $\sigma'$ for the columns $I$.
    Let $j$ be a column (\ie, an index) from $I$.
    By symmetry, we may assume $\sigma_j = \tau_j = 0$. 
    We aim to show that an optimal solution has always sufficiently many errors to pay for wrong entries of $\sigma'$.

    Let $\eta_j$ be the number of error of $(\tau,\tau')$ in column $j$ of $M$ and
    let $\eta_{P,j}$ be the number of errors of $(\sigma,\sigma')$ in column $j$ of $P$.
    Let $\eta''_j = \eta_j + \eta_{P,j}$.
    \begin{claim}\label{claim:errors}
        For each column $j$ of $I$, either $\sigma'_j = \tau'_j$ or $\eta''_j \ge (\epsilon-\epsilon^2)|\tau'(M)|/2$.
    \end{claim}
    \begin{subproof}
        We distinguish two cases.
        We first assume $\tau'_j = 0$.
        If also $\sigma'_j = 0$, we are done. 
        We therefore assume $\sigma'_j = 1$.
        If there are more than $|\tau'(L')|/2$ ones in column $j$ of $L'$, $(\tau,\tau')$ has more than $|\tau'(L')|/2$ errors in column $j$ and thus 
        $\eta_j \ge |\tau'(L')|/2$.
        Otherwise $\sigma'(L')$ has at least $|\tau'(L')|/2$ zeros in column $j$ and therefore $\eta_{P,j} \ge |\tau'(L')|/2$.
        We obtain $\eta''_j \ge |\tau'(L')|/2 \ge (\epsilon-\epsilon^2) |\tau'(M)|/2$ as claimed.

        In the second case, $\tau'_j = 1$ and we assume that $\sigma'_j = 0$.
        If there are more than $r'/2$ ones in column $j$ of $U'$, $(\sigma,\sigma')$ has more than $r'/2$ errors in column $j$ and thus $\eta_{P,j} \ge |\tau'(U')|/2$.
        Otherwise $\tau'(U')$ has at least $r'/2$ zeros in column $j$ and therefore $\eta_j \ge |\tau'(U')|/2$.
        Again, we obtain $\eta''_j \ge |\tau'(U')|/2 \ge (1-\epsilon) |\tau'(M)|/2$ as claimed.
    \end{subproof}

    Since by our assumption $|\tau'(X')| < \epsilon^2 |\tau'(M)|$,
    Claim~\ref{claim:errors} implies that within $I$, after reassigning the rows we still have a $(1+ O(\epsilon))$-approximation.

    To finish the proof, we argue that $\eta$ is large enough to pay for all errors in $X$ and $X'$ outside of $I$.
    Let $\eta_I$ be the number of errors due to assigning $\sigma$ to $\tau(X)$ and $\sigma'$ to $\tau'(X')$ within the interval $I$.

    Then, using the size of $I$ stated in Claim~\ref{claim:agree}, the total number of errors of $(\sigma,\sigma')$ in $M$ is at most
    $(1+O(\epsilon)) \eta + \eta_I + \epsilon^2 n \cdot 25\eta/n$, \ie, the errors of $\textsc{SWC}_{\epsilon^3}$ within $P$, the errors within $X$ and $X'$ in the columns of $I$, and all other entries of $X \cup X'$.
    The obtained approximation ratio is
    %\[
    %    \frac{(1+O(\epsilon)) \eta + \eta_I + \epsilon^2 n \cdot 25\eta/n}{\eta} 
    %    \le \frac{\eta + O(\epsilon) \eta + 25 \epsilon^2 \eta}{\eta}
    %    = 1 +  O(\epsilon)\,.
    %\]
    $((1+O(\epsilon)) \eta + \eta_I + \epsilon^2 n \cdot 25\eta/n/\eta 
        \le (\eta + O(\epsilon) \eta + 25 \epsilon^2 \eta)/\eta
        = 1 +  O(\epsilon)$.

    The first inequality uses that for some constant $k$, $(1 + k \epsilon) \eta \ge \eta + \eta_I$.
\end{proof}

\subsection{A DP for SWC-instances.}%a pair of solution strings.}
\label{sec:second}

Let $M$ be an SWC-instance with rows  $\{1, 2, \ldots n\}$. 
We define $\st_i$ to be the start and $\en_i$ the end of string number $i$ of $M$, i.e., the column number of the matrix where the binary part starts and ends.
For a sub-matrix $M'$ of $M$, $\st_{M'}$ determines the index of the first column of $M'$ and $\en_{M'}$ the index of the last column of $M'$.

We next specify the parts of which the DP cells are composed.
We divide the input instance into blocks defined as follows.
\begin{definition}[Block]
    Given a good SWC-instance $M$, a block $B$ is a sub-instance determined by four numbers $1 \le a < b < c \le n$ as follows.
    The first column of $B$ is column $1$ of $M$. 
    The last column of $B$ is $\en_{b}$.
    We write $U_B$ for the rows from $a$ to $b - 1$, $L_B$ for the rows from $b$ to $c - 1$, and $X_B$ for the rows from $c$ to $n$. 
    \label{def:sets-one-solution-string}
\end{definition}
The idea is that a block determines a trisection. 
We subdivide each block into chunks and select rows from these chunks. 
\begin{definition}[Chunk] 
    Let $B$ be a block determined by the numbers $a,b,c$.
    We partition $B$ into $2/\epsilon^2$ many \emph{chunks} (ranges or rows).
    These chunks are determined by numbers 
   % \[
   $a = a_1 < a_{2} < \dotsm < a_{1/\epsilon^2 + 1} = b = b_{1} < b_{2} < \dotsm < b_{1/\epsilon^2 + 1} = c$. %\,.
   % \]
    The $\ell$th chunk of $U_B$ is the submatrix composed of the rows $a_{\ell}$ to $a_{\ell+1}-1$ and the $\ell$th chunk of $L_B$ is the submatrix composed of the rows $b_{\ell}$ to $b_{\ell+1}$.
    \label{def:subsets-one-solution-string}
\end{definition}

\begin{definition}[Selection]
    \label{def:selection}
    For each block $B$ with a set of chunks $C$, we consider multiset $T$ of rows of size $2/\epsilon^5$.
    We require that $T$ contains $1/\epsilon^3$ rows from each chunk in $C$.
\end{definition}
The selection $T$ will take the role of $\tilde{U}$ and $\tilde{L}$ in $\textsc{SWC}_\delta$.
With these preparations we can define a DP cell.
\begin{definition}[DP cell]
    For each block $B$, each set of chunks $C$ of $B$ and each selection $T$ of rows from $B$, there is a DP cells are represented by $D(B,C,T)$. 
    A DP cells $D(B,C,T)$ is a \emph{predecessor} of $D(\hat B,\hat C,\hat T)$ if the following conditions hold.
    \begin{itemize}
        \item $\hat{a} = b$ and $\hat{b} = c$, where $b,c,\hat{a},\hat{b}$ are the numbers from Definition~\ref{def:sets-one-solution-string}.
        \item The chunks from $C$ between $b$ and $c$ are exactly the chunks from $\hat{C}$ between $\hat{a}$ to $\hat{b}$.
        \item For each pair of chunks from $T \times \hat{T}$ with the same range of rows, the selection $T$ matches the selection $\hat{T}$. 
            \label{def:dp-cell-one-solution-string}
    \end{itemize}
\end{definition} 

The value of $D(B,C,T)$ will be an approximation of the minimum number of errors
that we can have in $M$ until the last column of $B$.

\begin{figure}[h]
    \begin{center}
        \begin{tikzpicture}[scale=0.6]
            \draw (-2,-2) rectangle (2,3);      % A
            \node at (-2.3,3) {$a$};
            \draw (-2,-3) rectangle (9,-2);     % B
            %         \draw[dashed] (2,3) -- (9,-2);
            %          \draw[dashed] (9,-2) -- (18,-3);
            \node at (-2.3,-2) {$b$};
            \node at (-2.3,-3) {$c$};
            \node at (-2.3,-3.5) {$d$};
            \draw[red,dashed] (-2,-1.3) -- (8,-1.3);
            \node at (8, -1) {$b'$};
            \node at (9.5, -1.7) {$b$};
            \node at (13.5, -2.2) {$c'$};
            \draw (-2,-3.5) rectangle (18,-3);
            \draw[red,dashed] (-2,-2.6) -- (13,-2.6);
            \draw[red,dashed] (-2,-3.3) -- (18,-3.3);
            \node at (18.5, -3) {$d'$};
            \node at (18, -2.8) {$c$};
            %         \node[red] at (18.5, -3) {$B'_2$};
            %          \draw (2,-3.7) rectangle (20,-3.5); % Bottom area
            %         \node at (0,0){$A_0$};
            %         \node at (0,-2.7) {$B_0$};
            %         \node at (5.5,-1) {$B_0', T_U = T'_L$};
            % 	\node[red] at (5.5,-1) {$B_0'$};
            %        \node at (0,-3.3) {$B_1$};
            %         \node[red] at (7,-2.8) {$B'_1$};
            \node at (-2.7,0){\(\left(\rule{0cm}{2.4cm}\right.\)};
            \node at (21,0){\(\left.\rule{0cm}{2.4cm}\right)\)};
            \draw (2,-3) to (2,-3.5);
            %          \draw (9,-3.5) to (9,-3.7);
            %          \node at (20.5,-3.6) {$B_2$};
            \draw[blue,dashed] (-2,1) -- (4.7,1);
            \draw[blue,dashed] (-2,1.2) -- (4.6,1.2);
            \draw[blue,dashed] (-2,1.3) -- (4.5,1.3);
            \draw[blue,dashed] (-2,1.4) -- (4.4,1.4);
            %             \node[blue,dashed] at (5.5, 1.2) {$T'_U$};

            \draw[blue] (-2,1.8) -- (3.5,1.8);
            \draw[blue] (-2,2) -- (3.2,2);
            \draw[blue] (-2,2.1) -- (3.1,2.1);
            \draw[blue] (-2,2.2) -- (3,2.2);
            %             \node[blue] at (4, 2.2) {$T_U$};

            \draw[blue] (-2,-2.3) -- (10.8,-2.3);
            \draw[blue] (-2,-2.4) -- (11.7,-2.4);
            \draw[blue] (-2,-2.45) -- (12.2,-2.45);
            %             \node[blue] at (11, -2) {$T_L$};

            \draw[blue,loosely dashed] (-2,-1.5) -- (8,-1.5);
            \draw[blue,loosely dashed] (-2,-1.6) -- (8.1,-1.6);
            \draw[blue, loosely dashed] (-2,-1.7) -- (8.1,-1.7);
            %             \node[blue] at (8.5, -1.5) {$T'_L$};


            %         \draw (9,-3) to (9,-3);
        \end{tikzpicture}
        \caption{\label{fig:DP-crux1} Example for a pair of strings with $b' > b$. The blue lines and dashed blue ones represent sets $T$ and $T'$, and $T \cap T' = \emptyset$.}
    \end{center}
\end{figure}


We now describe the dynamic program for a pair of solution strings $(\sigma, \sigma')$ by using joint DP cells $(\zeta,\zeta')$ (see also Fig.~\ref{fig:DP-crux1}). 
For $\sigma'$, we use the same notation as in Definitions~\ref{def:sets-one-solution-string}, \ref{def:subsets-one-solution-string} and \ref{def:selection}, but we use the symbol prime (\,$\cdot '$\,) for all occurring variables.

\begin{definition}[DP cell for a pair]
    We define joint DP cell $(\zeta,\zeta') = (D(B,C,T), D'(B',C',T'))$ with the two single cells defined as in Definition~\ref{def:dp-cell-one-solution-string}. 
    We require that
    \begin{itemize}
        \item the rows of $C$ and $C'$ where chunks start are pairwise distinct, and
        \item $T \cap T' = \emptyset$.
    \end{itemize}
    \label{def:dp-cell-two-solution-string}
\end{definition}
\smallskip

%One of the selections for chunks in $B$ and $B'$ that corresponds to a joint DP cell is shown in Fig.~\ref{fig:DP-crux1}.
%Similar to a single cell, the value of a pair of cells will be an approximation of the minimum number of errors.

\begin{definition}[Predecessor of a joint DP cell]
    A DP cell $(\hat{\zeta}, \hat{\zeta}')$ is a \emph{predecessor} of $(\zeta, \zeta')$ if (a) $\hat{\zeta} = \zeta$ and $\hat{\zeta}'$ is a predecessor of $\zeta'$; or (b) $\hat{\zeta}$ is a predecessor of $\zeta$ and $\hat{\zeta}' = \zeta'$.
    \label{def:predecessor-two}
\end{definition}

\subparagraph{Algorithm ($\textsc{SWC}^{\sigma, \sigma'}$).}
%Additionally to guessing $r$, we globally guess a second number $r'$ such that $r$ and $r'$ represent $|\tau(M)|$ and $|\tau'(M)|$.
We globally guess two numbers $r$ and $r'$ that represent $|\tau(M)|$ and $|\tau'(M)|$.
We split the processing into an initialization phase and an update phase.
In the initialization phase, we assign values to each DP cell  $(\zeta,\zeta')$ based on $\SWC$ with the following parameters.
We obtain $R_i, S_i$ from the chunks $C$ and $R'_i,S'_i$ from the chunks $C'$.
In the execution of $\SWC$, we use the selections $T,T'$ instead of trying all possible selections, i.e., $T$ and $T'$ determine all $\tilde{U}_i$, $\tilde{L}_i$, $\tilde{U}'_i$, and $\tilde{L}'_i$ in the algorithm.
Let $\tilde{B}$ be the matrix with rows from $1$ to the $\min\{c - 1,c' - 1\}$ and columns one to $\min\{\en_B, \en_{B'}\}$.
The solution of the computation is a pair of strings $(\sigma_{\zeta,\zeta'},\sigma'_{\zeta,\zeta'})$, the prefixes of the two computed strings until $\en_{\tilde{B}}$.
The value of $(\zeta,\zeta')$ is $\cost_{\tilde{B}}(\sigma_{\zeta,\zeta'},\sigma'_{\zeta,\zeta'})$. For $\cost$ computation, see Definition~\ref{eq:cost}.

In the update phase, we compute the value and the pair of strings of the DP cell $(\zeta,\zeta')$ as follows. 
We inductively assume that all DP cells for predecessors of $(\zeta,\zeta')$ have been updated already.
As in Section~\ref{sec:single}, we try all predecessor pairs of DP cells and keep the one that gives the best result.
Let $(\overline{\zeta},\overline{\zeta}')$ be a predecessor of $(\zeta,\zeta')$.
By symmetry, we assume without loss of generality that $b' < b$. 
There are two cases how the two pairs interact.
The first case is $\zeta = \overline{\zeta}$.
We run $\SWC$ on the columns $\en_{\overline{B}'}+1$ to $\en_{B}$ with the parameters from $(\zeta,\zeta')$ (see initialization).
To obtain the full solution, we append the computed string for $B'$ to the string $\sigma'_{\zeta,\overline{\zeta}'}$ (which is one of the solution strings of the predecessor pair).
Let $\tilde{B}$ be the matrix with rows from $1$ to the $\min\{c - 1,c' - 1\}$ and columns one to $\en_{B'}$.
The solution of the computation is a pair of strings $(\sigma_{\zeta,\zeta'},\sigma'_{\zeta,\zeta'})$, the prefixes of the two computed strings from column one to $\en_{\tilde{B}}$.
The potential new value of $(\zeta,\zeta')$ is $\cost_{\tilde{B}}(\sigma_{\zeta,\zeta'},\sigma'_{\zeta,\zeta'})$.
We replace the stored solution with the potential new solution if the cost has decreased.

The second case is $\zeta' = \overline{\zeta}$.
This case is the crux of the joint DP, since we have a ``switch'' the role of $\sigma$ and $\sigma'$.
%
We run $\SWC$ on the columns $\en_{\overline{B}}$ to $\en_{B'}$ with the parameters from $(\zeta,\zeta')$ (see initialization).
To obtain the full solution, we then append the computed string for $B$ to the string $\sigma_{\overline{\zeta},\zeta'}$ (which is one of the solution strings of the predecessor pair).
Let $\tilde{B}$ be the matrix with rows from $1$ to the $\min\{c - 1,c' - 1\}$ and columns one to $\en_{B'}$.
The solution of the computation is a pair of strings $(\sigma_{\zeta,\zeta'},\sigma'_{\zeta,\zeta'})$, the prefixes of the two computed strings until $\en_{\tilde{B}'}$.
The potential new value of $(\zeta,\zeta')$ is $\cost_{\tilde{B}}(\sigma_{\zeta,\zeta'},\sigma'_{\zeta,\zeta'})$.
We replace the stored solution with the potential new solution if the cost has decreased.

%As in the previous section, 
For the last strings,
we additionally consider special cells that are defined as before, but with $c=n$ or $c'=n$. 
Intuitively, we use these cells when only at most $1/\epsilon^4$ rows of $\tau(M)$ or $\tau'(M)$ are left.
For pairs of cells containing such $\zeta$ or $\zeta'$, our computation considers the optimal solution within the computation instead of $\SWC$.

\begin{theorem}\label{thm:column1}
    The algorithm $\textsc{SWC}^{\sigma,\sigma'}$ is a PTAS for $SWC$-instances.
\end{theorem}
\begin{proof}
    To see that the DP works in polynomial time, we observe that instead of simple DP cells in Lemma~\ref{lem:simpleDP} here we consider pairs of DP cells.
    Therefore the number of cells is squared and thus stays polynomial.
    During the recursive construction of the solution, we compare each cell to be computed with one compatible cell at a time.
    Therefore the construction of the solution also takes only polynomial time.
    As in Lemma~\ref{lem:simpleDP}, the computed solution is vacuously feasible.

    We continue with analyzing the quality of the computed solution.
    Let $(\tau,\tau')$ be an optimal solution.
    We set $r = |\tau(M)|$ and $r' := |\tau'(M)|$.
    By renaming the two strings we may assume that the last row of the first $(1-\epsilon^2) r$ rows of $\tau(M)$ is below the first row of the last $\epsilon^2 r'$ rows of $\tau'(M)$.

    We consider DP cells similar to the proof of Lemma~\ref{lem:simpleDP}.
    Starting from the top-most row of $\tau(M)$, for each $i \ge 0$, the $i$th range $Y_i$ contains the next 
    $(\epsilon^{2i} - \epsilon^{2i+2})r$ rows of $\tau(M)$.
    We assign the rows not in $\tau(M)$ such that the first row of each $Y_i$ is contained in $\tau(M)$.
    Then we choose $Y_i$ such that all rows of $M$ until $Y_{i+1}$ are contained in $Y_i$.

    We consider the DP cells $\zeta_i$ for each $i$ with the parameters $B_i$, $C_i$, and $T_i$.
    The block $B_0$ contains the rows of $Y_0$ and $Y_1$, and the columns one to the end of the first row of $\tau(B_0)$.
    For each $i > 0$, block $B_i$ contains the rows of $Y_i$ and $Y_{i+1}$, and the columns after those of $B_{i-1}$ to the end of the first row of $B_i$.

    If only a constant number of rows of $\sigma(M)$ are left, we can compute the partial solutions optimally and
    there are DP cells for exactly this purpose:
    there is a DP cell $\zeta_i$ such that the last $2/\epsilon^5$ rows of $\tau(M)$ are located between $a_i$ and $c_i$ and $Y_i$ contains exactly these rows.
    As before, to keep a clean notation, in the following we implicitly assume that cells with constantly many rows of $\sigma(M)$ are handled separately.

    The chunks of $C_i$ are the ranges that equally distribute $\tau(B)$. The selection $T_i$ is the best possible selection as specified in $\SWC$.
    Analogously we define $B'_i$, $C'_i$, and $T'_i$ for $\zeta'_i$.

    We construct a solution SOL and inductively show that the value of each considered cell $(\zeta_i,\zeta'_j)$ and $(\zeta'_i,\zeta_j)$ is at most a factor $(1+O(\epsilon))$ larger than the number of errors of an optimal solution restricted to the considered prefix and the considered rows. 
    Afterwards we show that our algorithm computes a solution at least as good as SOL.

    We first consider the DP cell $(\zeta_0,\zeta'_0)$.
    Recall that we assumed \WLOG that $i'_0 > i_0$.
    We apply Lemma~\ref{lem:swc-gap} with the parameters of the pair of cells to obtain the prefixes $\sigma_{\zeta_0}$ and $\sigma'_{\zeta'_0}$.
    The total number of errors within the columns of $M$ at the prefixes is therefore at most a factor $(1+O(\epsilon))$ larger than in $(\tau,\tau')$.
    There are two possibilities for the subsequent steps with $i \ge 0$.

    We first assume that $b_{i+1}' > b_{i}$ and consider the cell $(\zeta_{i},\zeta'_{i+1})$.
    Then, similar to the proof of Lemma~\ref{lem:simpleDP}, we apply $\SWC$ to obtain the suffix of $({\sigma}_{\zeta_i,\zeta'_{i+1}},{\sigma'}_{\zeta_i,\zeta'_{i+1}})$ after $\en_{B_{i'}}$.
    By Lemma~\ref{lem:swc-gap}, considering the suffix alone we have at most a factor $(1+O(\epsilon))$ more errors within these columns than $(\tau,\tau')$.

    Since $\zeta_i$ is a predecessor of $\zeta_{i+1}$, all newly assigned rows were not considered in $(\zeta_i,\zeta'_i)$.
    Note that $\zeta_i$ did not change. Even though we looked at the same chunks, we used the same selections and therefore did not change $\sigma_{\zeta_i,\zeta'_i}$.

    The second possibility is that $b_{i+1}' < b_{i}$ and we consider the cell $(\zeta_{i+1},\zeta'_{i})$. The instance is shown in Fig.~\ref{fig:DP-crux2}.
    We then apply $\SWC$ to obtain the suffix of $({\sigma}_{\zeta_{i+1},\zeta'_{i}},{\sigma'}_{\zeta_{i+1},\zeta'_{i}})$ after $\en_{B_{i}}$.
    We obtain a $(1+O(\epsilon))$-approximation analogous to the case $b_{i+1}' > b_{i}$.
\end{proof}

\section{Subinterval-free instances.}\label{sec:subinterval-free}
We show how to generalize the results of the previous section in order to handle instances where no interval of a string $s$ is a proper subinterval of a string $s'$ and thus show Theorem~\ref{thm:ptas}.
To this end, we first show how to handle the rooted version of sub-interval free instances, where there is one column $j$ such that each string of the instance crosses $j$.

We order the rows of a subinterval-free instance $M$ from top to bottom such that for each pair $i,i'$ of rows with the binary part of $i$ starting on the left of the binary part of $i'$, $i$ is above $i'$.
In other words, the binary strings are ordered from top to bottom with increasing starting position (\ie, column).
Observe that the sub-string freeness property ensures that the last binary entry of $i'$ is not on the left of the last binary entry of $i$.

\begin{lemma}\label{lem:second_instance_setting}
    Let $M$ be a $\GMEC$ instance such that no string is the substring of another string.
    Furthermore we assume that there is a column $j$ of $M$ such that each string of the instance crosses $j$.
    Then there is a PTAS for $M$.
\end{lemma}
\begin{proof}
    Let $s$ and $t$ be the first and the last row of $M$.
    The column $j$ determines a block $W$ of $M$ that spans all rows and the columns from the first binary entry of $t$, $j_t$, to the last binary entry of $s$, $j_s$.
    In particular, $W$ has only binary entries.

    The right hand side of $j_t$ (the submatrix of $M$ composed of all columns with index at least $j_t$) forms a $\GMEC$ instance as required in Theorem~\ref{thm:column1}.
    The submatrix of $M$ that contains all rows of $M$ and columns $1$ to $j_s$ forms a \GMEC instance as required in Theorem~\ref{thm:column1} if we invert both the order of the rows and the columns.
    Instead of changing the ordering of the matrix, we can run the algorithm from right to left and from bottom to top.

    We would like to apply Theorem~\ref{thm:column1} independently to the two specified sub-problems.
    To this end we define a special set of DP cells $\gamma$ with cells $(\zeta_W,\zeta'_W) \in \gamma$.
    The content of these cells is similar to the regular cells, but it contains the information for both sides simultaneously.
    More precisely, a cell $\zeta_W$ has the following entrees.

    (a) Three consecutive ranges of rows determined by numbers $1\le \overleftarrow{c} < \overleftarrow{b} < \overrightarrow{b} < \overrightarrow{c} \le n$.
    These numbers determine an upper range $R_U$ from row $\overleftarrow{b}+1$ to row $\overrightarrow{b}-1$ and the following further ranges.
    A left lower range $\overleftarrow{R}_L$ from row $\overleftarrow{b}$ to row $\overleftarrow{c}+1$,
    as well as a right lower range $\overrightarrow{R}_L$ from row $\overrightarrow{b}$ to row $\overrightarrow{c}-1$.
    (b) A separation into chunks $C$. There are $3/\epsilon^2$ chunks in $C$: $1/\epsilon^2$ for $R_U$,  $1/\epsilon^2$ for $\overrightarrow{R}_L$, and  $1/\epsilon^2$ for $\overleftarrow{R}_L$.
    (c) A selection $T$ of $3/\epsilon^5$ rows (with repetition): $1/\epsilon^2$ for each chunk.

    We analogously obtain $\zeta'_W$ with the same variables but marked with the symbol prime.
    The rows selected in $\zeta'_W$ are required to be disjoint from those in $\zeta_W$, i.e., $T \cap T' = \emptyset$.
    Also the boundaries of chunks in $\zeta_W$ and $\zeta'_W$ have to be disjoint.

    \begin{definition}[Center cells]
        The cells $(\zeta_W,\zeta'_W) \in \gamma$ are called \emph{center cells}.
        \label{def:center-cells}
    \end{definition}
    The reason is that they take a special role as common ``centers'' of two separate runs of the DP: one run to the left and one run to the right.
    Observe that for each feasible entry of $(\zeta_W,\zeta'_W)$, we can apply Theorem~\ref{thm:column1} independently to the left and to the right, since the DP cells $(\zeta_W,\zeta'_W)$ takes the role of the left-most cell in Theorem~\ref{thm:column1}.
    The strings only overlap between the columns $j_t,j_s$ where we obtain an instance of \BMEC, which in particular is a good SWC-instance.
    Note that for each column $\hat{j}$ on the right hand side of $j_t$, all rows of $W$ located above $\overleftarrow{b}$ with binary entry at column $\hat{j}$  have also a binary entry at all rows between $\overleftarrow{b}$ and $\overrightarrow{b}$, due to the subinterval-freeness. 
    The properties of $\hat{j}$ on the left hand side of $j_s$ are analogous.
    We will choose $\overleftarrow{b}$ and $\overrightarrow{b}$ in such a way that by Lemma~\ref{lem:swc-gap}, it is therefore sufficient to consider the rows between $\overleftarrow{b}$ and $\overrightarrow{b}$ in order to handle all rows crossing $j$.

    %The two solution strings within this block are determined by the rows from $\overleftarrow{c}$ to $\overrightarrow{c}$ and from $\overleftarrow{c}'$ to $\overrightarrow{c}'$, depending only on $(\zeta_W,\zeta'_W)$.
    None of the remaining steps from Section~\ref{sec:second} interfere with each other.
    We therefore run the following $DP$.
    We first compute all center cells $(\zeta_W,\zeta'_W) \in \gamma$.
    For each cell, we store an infix of $\sigma$ and an infix of $\sigma'$.
    The infix of $\sigma$ starts at $j_t$ and ends at $j_s$.
    The entries of the two strings are those that we obtain from $\SWC$ with the parameters of $(\zeta_W,\zeta'_W)$.
    Each cell $(\zeta_W,\zeta'_W)$ forms a starting point for Algorithm $\textsc{SWC}^{\sigma, \sigma'}$, applied independently towards the left hand side and the right hand side.

    To see that the DP yields a good enough approximation, again we compare against an optimal solution $(\tau,\tau')$.
    Clearly we get a $(1+O(\epsilon))$-approximation for the infix between column $j_t$ and $j_s$ if for a DP cell $(\zeta_W,\zeta'_W)$, by Lemma~\ref{lem:swc-gap}.
    Note that the computed solution does not consider the rows above $\overleftarrow{c}$ or below $\overrightarrow{c}$.
    Since the further processing respects our choice between $\overleftarrow{c}$ and $\overrightarrow{c}$, the claim follows from Theorem~\ref{thm:column1}. 
\end{proof}

\subparagraph{General sub-interval-free instances} 
We use Lemma~\ref{lem:second_instance_setting} to handle general sub-interval free instances.
Instead of a single column $j$ crossed by all strings, we determine a sequence $q = (q_1,q_2,\dotsc)$ of columns with the property that each string crosses exactly one of them.
Let $s_1$ be the first string in $M$.
Then we choose $q_1$ to be the column of the last entry of $s_1$.

We recursively specify the remaining columns.
For a given $j$ such that we know $q_j$, let $s_i$ be the last (\ie, bottom-most) string that crosses $q_j$.
Then we choose $q_{j+1}$ to be the last (\ie, rightmost) column of string $s_{i+1}$.
For each $q_i$ in the sequence $q$, we determine a block $W_i$ analogous to $W$ in Lemma~\ref{lem:second_instance_setting}.

A simple induction shows that by the no-substring property and the chosen order of strings, each string crosses at least one column of $q$ and none of them crosses more than one.
In particular, for each $j$, the solution on the left hand side of $q_j$ depends on rows of $M$ disjoint from the rows that determine the solution on the right hand side of $q_{j+1}$. 

In order to combine the solution on the right hand side of $q_j$ with the solution on the left hand side of $q_{j+1}$, we introduce a notion of dominance.
Let us consider two arbitrary sub-matrices $V_1$ and $V_2$ of $M$.
\begin{definition}[Dominance]
    \label{def:dominance}
    We say that $V_1$ $\tau$-dominates $V_2$ if for each column $c$ that is in both $V_1$ and $V_2$, either at least one of the two matrices has no binary entries or
    the number of binary entries in $\tau(V_1)$ is at least $1/\epsilon^2$ times the number in $\tau(V_2)$.
    We say that $V_1$ is $\tau$-dominant over $V_2$ for a column $c$, if the one column submatrix of $V_1$ determined by $c$ dominates $V_2$. 
%
    We analogously define $\tau'$-dominance.
\end{definition}

Consider a submatrix $\overrightarrow{V}$ of $M$ that only contains rows that cross $q_i$ and a submatrix $\overleftarrow{V}$ of $M$ that only contains rows that cross $q_{i+1}$.
We observe that if $\overrightarrow{V}$ is $\tau$-dominant over $\overleftarrow{V}$ for some column $c$, it is also $\tau$-dominant for all columns on the left hand side of $c$:
until $q_i$ is reached, when moving to the left the number of binary entries of $\tau(\overrightarrow{V})$ increases and the number of binary entries of $\tau(\overleftarrow{V})$ decreases.
Analogously, if $\overleftarrow{V}$ is $\tau$-dominant over $\overrightarrow{V}$ for some column $c$, it is also $\tau$-dominant for all columns on the right hand side of $c$.

We therefore have a possibly empty interval $I$ without $\tau$-dominance such that the columns of $\overrightarrow{V}$ on the left hand side of $I$ are $\tau$-dominant and the columns of $\overleftarrow{V}$ on the right hand side of $I$ are $\tau$-dominant.
The different cases for $I$ are shown in Figures~\ref{fig:second_instance1} and \ref{fig:second_instance2}.
\begin{figure}
    \begin{center}
        \begin{tikzpicture}[scale=0.7]
            \draw (0,0) -- (4,0);
            \draw (0.1,-.1) -- (4,-.1);
            \draw (0.2,-.2) -- (4,-.2);
            \draw[blue,thick] (4,.2) -- (4,-3);
            \node at (4,.5) {$q_0$};
            \node at (3.5,-1) {$W$};
            \draw (0.5,-.5) -- (4.5,-0.5);
            \draw (0.5,-.6) -- (4.6,-0.6);
            \draw (0.7,-.7) -- (4.6,-0.7);
            \draw (1,-1) -- (5,-1);
            \draw (1.2,-1.2) -- (5.1,-1.2);
            \draw (3,-1.5) -- (6,-1.5);
            \draw (3.1,-1.7) -- (6.2,-1.7);
            \draw (3.1,-1.8) -- (6.2,-1.8);
            \draw (3.1,-2) -- (7,-2);
            \draw (4,-2.3) -- (7.5,-2.3);
            \draw (4,-2.5) -- (8,-2.5);
            \draw (4,-2.6) -- (9,-2.6);
            \draw (4,-2.7) -- (9.4,-2.7);
            \draw (4,-2.8) -- (9.5,-2.8);
            \draw[blue,thick] (14,-2.5) -- (14,-5);
            \node at (14, -2) {$q_1$};
            % 	\draw (5,-3.1) -- (11,-3.1);
            % 	\draw (5.1,-3.2) -- (11.1,-3.2);
            % 	\draw (5.2,-3.3) -- (11.2,-3.3);
            \draw (10,-3.4) -- (14,-3.4);
            \draw[dashed] (9.5,-2.2) to (9.5,-3.6);
            \draw[dashed] (10,-2.2) to (10,-3.6);
            \draw (10,-3.7) -- (14,-3.7);
            \draw (10.2,-3.8) -- (17.2,-3.8);
            \draw (10.3,-4) -- (17.5,-4);
            \draw (10.5,-4.4) -- (17.8,-4.4);
            \draw (12,-4.5) -- (18,-4.5);
            \draw (12.5,-4.7) -- (19,-4.7);
            \node at (9.8,-3.1) {$I$};
            \node at (13,-4.1) {$W$};
            \draw[red, dashed] (0,0) rectangle (3.89,-.3);
            \draw[red, dashed] (3.2,-.3) rectangle (4.5,-2.2);
            \draw[red, dashed] (10,-3.4) rectangle (14,-3.8);
            \draw[red, dashed] (12.5,-3.8) rectangle (17.3,-4.7);
            \draw[red, dashed] (4.1,-2.4) rectangle (9,-2.9);
        \end{tikzpicture}
        \caption{\label{fig:second_instance1} Blocks represented by ranges shown in red on an instance $M$ and the blue lines are the columns, $I$ and $W$ shows the empty interval and central region respectively.}
    \end{center}
\end{figure}

\begin{figure}
    \begin{center}
        \begin{tikzpicture}[scale=0.7]
            \draw (0,0) -- (4,0);
            \draw (0.1,-.1) -- (4,-.1);
            \draw (0.2,-.2) -- (4,-.2);
            \draw[blue,thick] (4,.2) -- (4,-3);
            \node at (4,.5) {$q_0$};
            \node at (3.5,-1) {$W$};
            \draw (0.5,-.5) -- (4.5,-0.5);
            \draw (0.5,-.6) -- (4.6,-0.6);
            \draw (0.7,-.7) -- (4.6,-0.7);
            \draw (1,-1) -- (5,-1);
            \draw (1.2,-1.2) -- (5.1,-1.2);
            \draw (3,-1.5) -- (6,-1.5);
            \draw (3.1,-1.7) -- (6.2,-1.7);
            \draw (3.1,-1.8) -- (6.2,-1.8);
            \draw (3.1,-2) -- (7,-2);
            \draw (4,-2.3) -- (7.5,-2.3);
            \draw (4,-2.5) -- (8,-2.5);
            \draw (4,-2.6) -- (9,-2.6);
            \draw (4,-2.7) -- (9.4,-2.7);
            \draw (4,-2.8) -- (9.5,-2.8);
            \draw[blue,thick] (11,-2.5) -- (11,-5);
            \node at (11, -2) {$q_1$};
            \draw (5,-3) -- (11,-3);
            \draw (5.1,-3.2) -- (11.2,-3.2);
            \draw (5.2,-3.3) -- (11.2,-3.3);
            \draw (5.5,-3.4) -- (11.3,-3.4);
            \draw[dashed] (5.5,-2.2) to (5.5,-3.6);
            \draw[dashed] (9,-2.2) to (9,-3.6);
            \draw (8,-3.7) -- (12,-3.7);
            \draw (8.2,-3.8) -- (12.2,-3.8);
            \draw (8.3,-4) -- (12.5,-4);
            \draw (8.5,-4.4) -- (12.8,-4.4);
            \draw (10,-4.5) -- (13,-4.5);
            \draw (10.5,-4.7) -- (14,-4.7);
            \node at (7,-3.1) {$I$};
            \node at (10.8,-4.1) {$W$};
            \draw[red, dashed] (0,0) rectangle (3.89,-.3);
            \draw[red, dashed] (3.2,-.3) rectangle (4.5,-2.2);
            \draw[red, dashed] (10.5,-3.1) rectangle (11.2,-4.7);
            %   	\draw[red, dashed] (4.1,-2.4) rectangle (9,-2.9);
        \end{tikzpicture}
        \caption{\label{fig:second_instance2} This sketch shows a non-dominance example in region $I$.}
    \end{center}
\end{figure}

\begin{definition}[Dominance region]
    The \emph{dominance region} of $\overrightarrow{V}$  with respect to $\overleftarrow{V}$ is the set of columns where $\overrightarrow{V}$ is dominant over $\overleftarrow{V}$, and vice versa. 
    \label{def:dominance-region}
\end{definition}
Within the dominance region, our old DP can simply compute solutions without considering interferences: the dominated block is small enough to be ignored, applying Lemma~\ref{lem:swc-gap}.

Within the interval $I$, both cells have to ``cooperate.''
We obtain a \BMEC block in the middle with additional rows on the top and bottom. 
This sub-instance can be solved directly,
%We handle it similar to the center in Lemma~\ref{lem:second_instance_setting}.
% \sgnote{we had nested DP here, inner and outer. I think we need that intuition in main text, probably move proof of ``DP is a PTAS'' to appendix.}%
%We handle it as in Lemma~\ref{lem:second_instance_setting}.

We use DP cells similar to Lemma~\ref{lem:second_instance_setting}, but for more than one center.
For each $j$, we consider column $q_j\in q$ and a collection $\kappa_j$ of DP cells $(\zeta_{q_j},\zeta'_{q_j}) \in \kappa_j$.
Each cell $(\zeta_{q_j},\zeta'_{q_j})$ is a center cell with center $q_j$.
We refer to the cells in $\kappa_j$ as the $j$th center cells.

Additionally, for each center cell we also store the dominance information on the left and right of $q_j$, 
i.e., we store the intervals $\overleftarrow{I}, \overleftarrow{I}'$ between $q_{j-1}$ and $q_j$ and the intervals $\overrightarrow{I},\overrightarrow{I}'$ between $q_j$ and $q_{j+1}$ where no cell dominates another, once with respect to $\tau$ and  once with respect to $\tau'$.

Formally this means to extend the cells by four numbers that store the start and end points four intervals $\overleftarrow{I},\overrightarrow{I}',\overrightarrow{I}$, and $\overrightarrow{I}'$.

For each of the four intervals we store additional information.
The four intervals only differ in whether we consider $\sigma$ or $\sigma'$. The left and right version are symmetric.
Therefore it is sufficient to analyze the details for a generic $I \in \overleftarrow{I},\overrightarrow{I}',\overrightarrow{I},\overrightarrow{I}'$.
The interval ${I}$ determines a block ${B}$ that we subdivide into chunks ${C}$ and we select rows ${T}$.
There are several differences to previous trisections, subdivisions and selections.

We divide the rows of block ${B}$ into four regions: 
a middle part ${U^\uparrow}$ that has only binary entries (a \BMEC sub-instance) such that each row crosses $q_j$,
a middle part ${U^\downarrow}$ that has only binary entries such that each row crosses $q_{j+1}$,
the rows ${U}^{\uparrow\uparrow}$ above ${U}^\uparrow$, and the rows ${U}^{\downarrow\downarrow}$ below ${U}^\downarrow$.
We choose the two middle parts such that the number of rows is maximal.

It is not sufficient to use a globally guessed $r$. 
Instead, we add four numbers ${r}^{\uparrow\uparrow},{r}^\uparrow,{r}^\downarrow,{r}^{\downarrow\downarrow}$ to the DP cell in order to guess and store the values 
$\tau({U}^\uparrow)$, $\tau({U^\downarrow})$, $\tau({U^{\uparrow\uparrow}})$, and $\tau({U^{\downarrow\downarrow}})$.

Due to the non-dominance, we know that for each column of ${B}$, at least an $\epsilon^2$-fraction of rows from $\tau(B)$ are located in the middle part ${M}$.
Observe that there is no region that takes the role of $X$ in a trisection.
We obtain an instance similar to a good SWC-instance, but it has two non-binary regions and the binary region only has an $\epsilon^2$ fraction of rows instead of an $\epsilon$-fraction.
To be able to still apply Lemma~\ref{lem:swc-gap}, we subdivide each of the three regions into chunks and increase the number of chunks per region. 
The number of chunks depends on  the four versions of ${r}$. 
If our chunks do not contain more than $\epsilon^4 {r}$ rows of $\tau(M)$, the lemma is applicable.
We guess a number $k$ and set the size of chunks with root $q_j$ to contain $\epsilon^{4k} {r}^{\uparrow\uparrow}$ rows of $\tau(M)$ in ${U}^\uparrow$ and $\epsilon^{4k} {r}^{\uparrow\uparrow}$ in ${M}$ for the rows with root $q_j$.
There may be an additional chunk with fewer rows, if the numbers don't match.
We specify the remaining chunks symmetrically, based on a number guessed for root $q_{j+1}$.
The choice of $k$ will become clear in the description of the DP.

The increased precision also requires that we  increase the precision of the entire remaining DP: we replace each selection of $1/\epsilon^2$ chunks into selections of $1/\epsilon^4$ chunks.
Clearly, the increased precision cannot decrease the quality of the computed solution.

The idea of the DP is that for each $q_j$, we run the rooted DP as an \emph{inner} DP that determines solutions for their dominance regions that fit to solutions in the consecutive non-dominance regions.
The non-dominance regions then form interfaces that we can use to compute an overall solution from left to right with an \emph{outer} DP.

The inner DP works as follows.
For each cell $(\zeta_{q_1},\zeta'_{q_1}) \in \kappa_1$, we compute the prefixes of $\sigma,\sigma'$ until $q_1$ exactly as in Lemma~\ref{lem:second_instance_setting}.
We start the DP to the right hand side also the same way as before, but with the difference that as soon as we reach the row ranges for $\overrightarrow{I}$ or $\overrightarrow{I}'$, we use the choices already stored in $(\zeta_{q_1},\zeta'{q_1})$.
We have to ensure that our choices within the DP do not contradict the choices of  $(\zeta_{q_1},\zeta'_{q_1}) \in \kappa_1$. 
If $(\zeta_{q_1,i},\zeta'_{q_1,i})$ is the cell of the inner DP that overlaps with $\overrightarrow{I}$ first, we require that among the common rows, $\overrightarrow{B}$ contains the remaining rows from $\tau(M)$ restricted to the rows of the inner DP and does not contradict $B_i$. 
Each chunk of $C_i$ contains $\epsilon^k |\tau(M)|$ rows of $\tau(M)$, for some integer $k$ (the same $k$ that we guessed for the non-domination region). 
We have to ensure that the chunks of $\overrightarrow{U}^\uparrow$ and $\overrightarrow{M}^\uparrow$ match the chunks and the chunks of $\overrightarrow{C}_i$.
Furthermore, we have to check that the selection of rows matches.

For all $j > 1$ continue in the same manner starting from $(\zeta_{q_j},\zeta'_{q_j})$ and handle the processing of $\overleftarrow{I}$ as we did before with $\overrightarrow{I}$.
Observe that we can see the processing of $(\zeta_{q_1},\zeta'_{q_1})$ as a special case with empty interval $\overleftarrow{I}$, and to obtain the suffix of $\sigma,\sigma'$, the last interval $\overrightarrow{I}$ can be handled as empty interval.

The global DP proceeds from left to right. For each $q_j$, it considers all cells $(\zeta_{q_j},\zeta'_{q_j})$.
The value of $(\zeta_{q_j},\zeta'_{q_j})$ is its inner DP value plus the best value achievable on the left hand side with the same choice of parameters for the left non-domination region.
Among all cells from $\kappa_j$ with the same parameters for the right non-domination region, the global DP only keeps the best value (the smallest number of errors).

\subparagraph{The above DP is a PTAS for $\mathbf{M}$.} 
Let $(\tau,\tau')$ be an optimal solution.
For each separate $q_j \in q$, we run the same DP as in Lemma~\ref{lem:second_instance_setting} and thus we obtain a $(1+O(\epsilon))$-approximation.
For the intervals $\overleftarrow{I}$ and $\overrightarrow{I}$, there is a choice of parameters that matches the choices analyzed in Lemma~\ref{lem:second_instance_setting}.
We therefore only have to argue that the transition between sub-instances works correctly.
We consider the dominant regions determined by $(\tau,\tau')$ and consider the DP cells that guess these regions correctly from left to right.
Let $I$ be one of the guessed non-dominant regions.
We obtain the solution for $I$ by applying $\SWC$, which gives a $(1+O(\epsilon))$ approximation.
The transition between dominant and non-dominant regions uses that in both cases we create the solution strings from the same parameters in $\SWC$ and therefore creates the solution from the same instance strings.
This finishes the proof of Theorem~\ref{thm:ptas}.

\section{A QPTAS for general instances.}\label{sec:QPTAS}

To solve the general instances, the main observation is that we divide the rows into their at most $\log_2(m)$ length classes $L_i$, and the $i$th length class $L_i$ is the set of all strings of length $\ell$ with $\ell \in (m/2^{i+1}, m/2^{i}]$.
First we present an algorithm to solve each length class $L_i$ separately by constructing their corresponding columns.

\subsection{Length classes.}\label{sec:length-classes}
We show how we can handle length classes of strings.
To this end, let us assume \WLOG that $m$ (\ie, the number of columns in $M$) is a power of $2$.
Then for each $i \ge 0$, the $i$th length class $L_i$ is the set of all strings of length $\ell$ with $\ell \in (m/2^{i+1}, m/2^{i}]$. 
We observe the following known property of length classes.
\begin{lemma}\label{lem:half}
    For each $i \ge 0$ there is a set $q_i = \{q_{i,1},q_{i,2},\dotsc\}$ of columns such that (a) each string in $L_i$ crosses at least one column from $q_i$ and (b) no string from $L_i$ crosses more than two columns from $q_i$.
    Furthermore, we can choose the sets such that $q_i \subseteq q_{i+1}$.
\end{lemma}
\begin{proof}
    At level $i$, for each $k$ with $1 \le k \le 2^{i+1}$ we select the column with index $k \cdot m/2^{i+1}$.
    We observe that the distance between two consecutive columns from $q_i$ is $m/2^{i+1}$, which matches the shortest length of strings in $L_i$: 
    if a minimal string starts right after a column of $q_i$, its last entry will cross the next column of $q_i$. 

    Since strings do not start before column $1$ and column $m$ is contained in each $q_i$, claim (a) follows.
    To see (b), observe that a maximum length string of $L_i$ is at most $m/2^i$.
    Let $j$ be an index. 
    The distances from $q_{i,j}$ to the column right before $q_{i,j+1}$ and from $q_{i,j+1}$ to right before $q_{i+2}$ are exactly $m/2^{i+1}$ .
    If the string starts directly at a column $q_{i,j}$ from $q_i$, it would cross column $q_{i,j+1}$ and end right before column $q_{i,j+2}$ as shown in Fig.~\ref{fig:gen_instance1}.

    The last claimed property follows directly from the construction of the sets $q_i$ (Fig.~\ref{fig:gen_instance1}).
\end{proof}

\begin{figure}
    \begin{center}
        \begin{tikzpicture}[scale=0.7]
            \draw[blue,thick] (7,.2) -- (7,-6);
            \draw[blue,thick] (13,.2) -- (13,-6);
            \draw[blue,thick] (19,.2) -- (19,-6);
            \draw[blue,thick,dashed] (10,.2) -- (10,-6);
            \draw[blue,thick,dashed] (16,.2) -- (16,-6);
            \node at (2.5,-1.5) {$L_1$};
            \node at (7,0.5) {$q_{1,1}$};
            \node at (10,0.5) {$q_{1,2}$};
            \node at (13,0.5) {$q_{1,3}$};
            \node at (16,0.5) {$q_{1,4}$};
            \node at (19,0.5) {$q_{1,5}$};
            \draw (1,-0.5) -- (7,-0.5);
            \draw (3,-0.7) -- (8,-0.7);
            \draw (5,-0.9) -- (8,-0.9);
            \draw (6,-1.2) -- (11,-1.2);
            \draw (7,-1.5) -- (13,-1.5);
            \draw (8,-1.7) -- (11,-1.7);
            \draw (9,-1.9) -- (12,-1.9);
            \draw (9,-2.1) -- (13,-2.1);
            \draw (9,-2.3) -- (14,-2.3);
            \draw (8,-2.5) -- (12,-2.5);
            \draw (9.5,-2.9) -- (14,-2.9);
            \draw (11,-2.7) -- (16,-2.7);
            \draw (13,-3.1) -- (19,-3.1);
            \draw (14,-3.3) -- (19,-3.3);
            \draw (17,-3.5) -- (21,-3.5);
            \draw (15,-3.7) -- (21,-3.7);
            %                 \node at (13,-6.8);
        \end{tikzpicture}
        \caption{\label{fig:gen_instance1} For a single-length-class instance, the sketch shows the strings crossing each column either exactly once or exactly twice.}
    \end{center}
\end{figure}

For each $i$, we now separate $L_i$ into two sub-instances.
One sub-instance $L'_i$ is formed by those rows from $L_i$ that only cross one column of $q_i$ and the second sub-instance $L''_i$ is formed by those rows that cross exactly two columns of $L_i$.

We first show that we can handle each class separately.
\begin{lemma}\label{lem:half-length-class}
    There is a QPTAS for \GMEC if all strings are in the same class $L'_i$ or $L''_i$.
\end{lemma}
\begin{proof}
    We first note that by skipping all $q_{i,j}$ with even $j$, the strings in $L''_i$ cross exactly one column of the set.
    It is therefore sufficient to handle $L'_i$.

    For each column $q_{i,j}$, we create a set of DP cells (as defined in Definition~\ref{def:dp-cell-two-solution-string}) that stores information about a center region as defined in Definition~\ref{def:center-cells} and about non-domination intervals as defined in Definition~\ref{def:dominance}, exactly as in the proof of Theorem~\ref{thm:ptas}. 
    The next insight is that we can order the rows crossing column $c$ at the left and right side as defined below.
    \subparagraph{Left row ordering}
    We first order the rows with increasing starting positions of strings as in Lemma~\ref{lem:second_instance_setting}.
    At the left side of column $c$ , we obtain a similar instance as in Lemma~\ref{lem:second_instance_setting}.

    \subparagraph{Right row ordering}
    Afterwards we reorder the rows in order to handle the right hand side of column $c$. More precisely, we order the strings in increasing order based on the end of strings $e_i$.
    The obtained structure corresponds to the right hand side of column $c$ is similar to the instance handled in Lemma~\ref{lem:second_instance_setting}.

    Instead of running the DP of Lemma~\ref{lem:second_instance_setting}, we guess the sequence of blocks.
    An optimal solution $(\tau,\tau')$ determines a sequence of blocks $\overleftarrow{A}_1,\overleftarrow{A}_2,\dotsc,\overleftarrow{A}_k$ such that $|\tau(\overleftarrow{A}_{i+1})| =  \epsilon^2 |\tau(\overleftarrow{A}_{i})|$. 
    Instead of moving from $\overleftarrow{A}_1$ to $\overleftarrow{A}_k$ using a DP, we directly guess the strings for all $k$ sub-matrices \emph{simultaneously}.
    We do the same with the chunks and row selections.
    Additionally, we guess the sequence of sub-matrices $\overleftarrow{A}'_1,\overleftarrow{A}'_2,\dotsc,\overleftarrow{A}'_{k'}$ simultaneously 
    such that $|\tau'(\overleftarrow{A}'_{i+1})| =  \epsilon^2 |\tau'(\overleftarrow{A}'_{i})|$.
    We obtain a combined DP cell $\overleftarrow{\zeta}$ for $k+k'$ sub-matrices.

    Again we form the sub-matrices $\overrightarrow{A}_1,\overrightarrow{A}_2,\dotsc,\overrightarrow{A}_{k}$ and  $\overrightarrow{A}'_1,\overrightarrow{A}'_2,\dotsc,\overrightarrow{A}'_{k'}$ analogous to the left hand side and guess the selected strings of all matrices simultaneously 
    such that $|\tau(\overrightarrow{A}_{i+1})| =  \epsilon^2 |\tau(\overrightarrow{A}_{i})|$ and  $|\tau'(\overrightarrow{A}'_{i+1})| =  \epsilon^2 |\tau'(\overrightarrow{A}'_{i})|$. 
    We obtain a combined DP cell $\overrightarrow{\zeta}$ for all $k+k'$ sub-matrices on the right hand side.
%     \sgnote{I think here length class means $L'_i$ or $L''_i$, right? This definition should go in main text.}
    \begin{definition}[DP cell for sub-class of a length class $L_i$] %($L'_i$ or $L''_i$)]
        For each $q_{i,j}$, let $\xi_j$ be the set of super-cells $(\overleftarrow{\zeta}, \overleftarrow{\zeta'}, \overrightarrow{\zeta}, \overrightarrow{\zeta'})$, but with the additional center and non-domination information of Lemma~\ref{lem:second_instance_setting}.
        \label{def:dp-cell-length-class}
    \end{definition}

    For each $q_{i,j}$, let $\xi_j$ be the set of super-cells based on Definition~\ref{def:dp-cell-length-class}.
    We then design a DP that moves from left to right through the columns in $q_i$.
    The DP and its analysis now follow from the proof of Theorem~\ref{thm:ptas}, but we consider the left hand side and right hand side of each cell from $\xi_j$ simultaneously.

    To analyze the running time, we observe that $k$ and $k'$ are at most $O(\log_{1/\epsilon}(n))$ since for each $i$ we assume that $|\tau(\overleftarrow{A}_{i+1})| =  \epsilon |\tau(\overleftarrow{A}_{i})|$
    and $|\tau'(\overleftarrow{A}'_{i+1})| =  \epsilon |\tau'(\overleftarrow{A}'_{i})|$. 
    The number of instances $\overrightarrow{A}_i$ and $\overrightarrow{A}'_i$ are also at most $O(\log_{1/\epsilon}(n))$ each, for the same reason.

%     \sgnote{Maybe, we say selection and majority voting to generate solution strings work as before.}
    We thus obtain super-cells that are combined of logarithmically many sub-cells with polynomial complexity.
    We obtain an overall super-cell which is a quadruple $(\overleftarrow{\zeta},\overleftarrow{\zeta}',\overrightarrow{\zeta},\overrightarrow{\zeta}')$, and we have to distinguish $\bigl(n^{O(1)}\bigr)^{4 \log_{1/\epsilon}(n)} = n^{O(\log n)}$
    different cells, which is quasi-polynomial\footnote{We assume that $n$ and $m$ are polynomially related. This is justified because there are $n \cdot m$ entries of $M$ and therefore measuring in $m$ instead of $n$ would also give a quasi-polynomial complexity.}.

    We now analyze the performance guarantee. 
    For each column $j$, we obtain the values $\sigma_j$ and $\sigma'_j$ in almost the same way as we do in Lemma~\ref{lem:second_instance_setting}, but with the difference that we require consistency with all other rows sampled.
    For an optimal solution $(\tau,\tau')$, it is sufficient to only consider choices of rows such that all rows selected for $\sigma$ are in $\tau(M)$ and all rows selected for $\sigma'$ are in $\tau'(M)$.
    Such a selection of rows ensures consistency. 
    Note that we could apply the proof of Lemma~\ref{lem:second_instance_setting}  from the root to the left hand side and to the right hand side independently,
    if we knew $\tau(M)$ and $\tau'(M)$, just by avoiding wrong assignments.
    The simultaneous selection of all relevant rows ensures that we consider at least one selection of rows that satisfies these strong conditions.
    This solution is a $(1+\epsilon)$ approximation by the proof of Lemma~\ref{lem:second_instance_setting}, and
    our DP computes a solution of at least the same quality since we consider the overall number of errors with respect to all sampled rows.
\end{proof}

Combining the two sub-classes gives a QPTAS for an entire length class.

\begin{definition}[DP for a length class $L_i$]
    For each index $j$ let $\xi'_j$ be the sets of DP cells for $L'_i$ and for the odd indices $j$ let $\xi''_j$ be the set of cells for $L''_i$.
    We define a super-cells that starts in $j$, $(Z'_j,Z''_j,Z'_{j+1},Z''_{j+2}) \in \xi'_j \times \xi''_j \times \xi'_{j+1} \times \xi''_{j+2}$ and the super-cell that ends in $j$, $(Z'_{j-1},Z''_{j-2},Z'_j,Z''_j) \in \xi'_{j-1} \times \xi''_{j-2} \times \xi'_{j} \times \xi''_{j}$.
 \label{def:dp-whole-length-class}
\end{definition}


\begin{lemma}\label{lem:length-class}
    There is a QPTAS for \GMEC if all strings are in the same class $L_i$.
\end{lemma}
\begin{proof}
To prove it, we consider DP-cells according to Definition~\ref{def:dp-whole-length-class} and combine these cells from two consecutive columns such that they are compatible.

    To combine the PTAS for $L'_i$ and $L''_i$, we proceed from left to right.
    For each index $j$ let $\xi'_j$ be the sets of DP cells for $L'_i$ and for the odd indices $j$ let $\xi''_j$ be the set of cells for $L''_i$.
    For each column $c$ before $q_{i,1}$, each pair of cells $(Z,Z') \in \xi'_1 \times \xi''_1$ determines two subproblems for which we compute the two separate solutions.
%     \sgnote{Maybe, we define DP cell for $L_i$ to re-use it in next section.}
    Let $\mathcal{B}(Z,Z',z)$ be the set of all boxes (sets of rows) for $\sigma$ considered in the sub-cells of $(Z,Z')$ at column $z$ and
    let $\mathcal{B}'(Z,Z',z)$ be the set of all boxes (sets of rows) for $\sigma'$ considered in the sub-cells of $(Z,Z')$ at column $z$. 

    We now extend the DP as follows.
    We compose the solution from left to right, starting with the prefix of $(\sigma,\sigma')$ before $q_{i,1}$ and then, step by step, we fill the intervals between $q_{i,j}$ and $q_{i,j+1}$ for $j \ge 1$.
    The starting interval can be seen as the interval between a dummy-column $q_0$ and $q_1$.
    For each $j$, let us analyze its interval. 
    If $j$ is odd, we simultaneously consider the cells $\xi'_j,\xi'_{j+1},\xi''_j, \xi''_{j+2}$.
    Otherwise, we simultaneously consider the cells $\xi'_j,\xi'_{j+1},\xi''_{j-1}, \xi''_{j+1}$.

    For each column $c$ with index $\ell$ in the interval, in both cases the values of the DP cells reveal all $\SWC$ instances at $c$ that we would have to solve in order to obtain solutions for $L'_i$ and $L''_i$ separately.
    Instead of solving these instances separately, we solve them simultaneously.

    Let $\hat{C}_1, \hat{C}'_1$ and $\hat{C}_2,\hat{C}'_2$ be the chunks of the four DP cells at position $j$.
    In order to determine the value $\sigma_j$, we have to combine $\hat{C}_1$ with $\hat{C}_2$ and take care of the different densities of rows.
    To this end, we generalize the function $\textsc{Majority}_j$.
    \begin{definition}{Generalized Majority}
        \label{def:generalized-majority}
        For a single chunk $c$, let $r(c)$ be the number of rows in $c$ that are guessed to be in $\tau(M)$ and $t(c)$ the number of selected rows.
        As in Definition~\ref{def:weighted-majority}, we replace all values zero by $-1$.
        Then, for the given set of chunks $C$ with selection $T$, we compute 
        \[
            \rho := \sum_{c \in C} \sum_{i \in T\colon i \in c} (r(c) \cdot M_{i,j})\,.
        \]
        We set $\sigma_j = 1$ if the outcome is at least zero and $0$ otherwise.    
        The definition is analogous for $\sigma'_j$.
    \end{definition}

    By replacing the majority function by the generalized majority function of Definition~\ref{def:generalized-majority} in the proof of Lemma~\ref{lem:SWC}, we obtain a $(1 + O(\epsilon))$-approximation
    also if we consider different cells simultaneously.

    Since we consider all cells for the entire interval simultaneously, one of the choices is at least as good as sampling uniformly at random with knowledge of $\tau(M)$ and $\tau'(M)$.
    We therefore obtain a solution for the interval with at most a $(1+O(\epsilon))$ factor of errors compared to $(\tau,\tau')$

    Finally we have to join the results that we obtain for the intervals.
    Observe that for each pair of cells $(Z''_j,Z''_{j+2}) \in \xi''_j \times \xi''_{j+2}$ there are two consecutive pairs of cells
    $(Z'_j,Z'_{j+1}) \in \xi'_j \times \xi'_{j+1}$ and $(Z'_{j+1}, Z'_{j+2}) \in \xi'_{j+1} \times \xi'_{j+2}$.
    For a quadruple of cells $(Z'_j,Z''_j,Z'_{j+1},Z''_{j+2})$ we consider each quadruple on the left hand side ending with the matching cells $Z'_j,Z''_j$.
    Among these, we take the one with fewest errors.
    To obtain the value of the new quadruple, we add the errors in the interval $(q_{i,j} q_{i,j+1}]$ to the value of the selected predecessor quadruple.
    To compute the value $(Z'_{j+1},Z''_j,Z'_{j+2},Z''_{j+2})$, we consider all cells $(Z'_j,Z''_j,Z'_{j+1},Z''_{j+2})$, \ie, the cells that have the same $Z''_{j},Z''_{j+1},Z'_{j+1}$ for all choices of $Z'_j$.
    We add the errors between $q_{i,j+1}$ and $q_{i,j+2}$ to the smallest value found among the predecessors.

    The approximation ratio follows from Lemma~\ref{lem:half-length-class} and the quasi-polynomial running time from the fact that we only consider constantly many super-cells simultaneously.

\end{proof}

\begin{figure}
    \begin{center}
        \begin{tikzpicture}[scale=0.7]
            \node at (2.5, .5) {$q_{3,1}$};
            \node at (5, .5) {$q_{2,1}$};
            \node at (7.5, .5) {$q_{3,3}$};
            \node at (10, .5) {$q_{1,1}$};
            % 	\draw[blue,thick] (0,.2) -- (0,-6);
            \draw[blue,thick] (10,.2) -- (10,-6);
            \draw[blue,thick, dashed] (5,.2) -- (5,-6);
            \draw[blue,thick, loosely dashed] (2.5,.2) -- (2.5,-6);
            \draw[blue,thick, loosely dashed] (7.5,.2) -- (7.5,-6);
            \draw (.2,-4) -- (10,-4);
            \draw (1,-4.2) -- (11,-4.2);
            \draw (3,-4.4) -- (10,-4.4);
            \draw (6,-4.6) -- (13,-4.6);
            \draw (7,-4.8) -- (14,-4.8);
            \draw (7,-5) -- (12,-5);
            \node at (0,-4.5) {$L_1$};

            \draw (.2,-2.5) -- (5,-2.5);
            \draw (1,-2.6) -- (5.5,-2.6);
            \draw (3,-2.8) -- (7,-2.8);
            \draw (3.5,-3.0) -- (6,-3.0);
            \draw (4,-3.2) -- (8,-3.2);
            \node at (0,-3) {$L_2$};


            \draw (6,-2.7) -- (10,-2.7);
            \draw (6.5,-2.9) -- (10.5,-2.9);
            \draw (7,-3.4) -- (10.2,-3.4);

            \draw (.2,-0.5) -- (2.5,-0.5);
            \draw (1,-0.7) -- (3,-0.7);
            \draw (1.1,-0.9) -- (2.8,-0.9);
            \draw (1.2,-1.1) -- (2.7,-1.1);
            \draw (1.6,-1.3) -- (3.3,-1.3);

            \draw (2.6,-1.5) -- (5.1,-1.5);
            \draw (3.5,-1.7) -- (5.7,-1.7);
            \draw (3.7,-1.9) -- (5.5,-1.9);
            \draw (2.6,-1.0) -- (5.2,-1.0);
            \draw (5.3,-1.0) -- (7.8,-1.0);
            \node at (0,-1) {$L_3$};
        \end{tikzpicture}
        \caption{\label{fig:gen_instance2} Different length classes, $L_1$ with corresponding column $q_{1,1}$, 
            $L_2$ with corresponding columns $q_{2,1}, q_{2,2} = q_{1,1}$, 
            and $L_3$ with corresponding columns $q_{3,1}, q_{3,2} = q_{2,1}, q_{3,3}, q_{3,4} = q_{1,1}$.}
    \end{center}
\end{figure}

\subsection{The general QPTAS.}\label{sec:generalQPTAS}
Finally we combine our insights to an algorithm for general instances shown in Fig.~\ref{fig:gen_instance2} by combining different length classes.
%
For different length classes $L_i$, we construct their corresponding columns as explained in previous section. 
The main idea is that for each column $j$, we only have to consider those quadruple of super-cells according to Definition~\ref{def:dp-whole-length-class} that cross $j$ from all the length classes simultaneously.
We therefore consider at most $O(\log(n))$ quadruples of super-cells simultaneously. In the dynamic programming, we consider a joint quadruple of super-cells from all the length classes.
Then the overall complexity of a joint cell is quasi-polynomial: the number of different cells is $\bigl(n^{O(\log n)}\bigr)^{O(\log n)} = n^{O(\log^2 n)}$.

Let $Q_{i,j}$ be the set of quadruples of length class $i$ crossing column $j$ such that the strings are ordered from shortest length class to the largest.
For each length class $i$, a quadruple $q \in Q_{i,j}$ is the set of rows starting at $j$, cross $j$, or end in $j$.
If $j$ is the index of $q_{i,\ell}$,
The quadruple $q$ starts in $j$ if it is formed by cells $(Z'_\ell,Z''_\ell,Z'_{\ell+1},Z''_{\ell+2})$ and ends in $j$ if it is formed by $(Z'_{\ell-1},Z''_{\ell-2},Z'_\ell,Z''_\ell)$ (see proof of Lemma~\ref{lem:length-class}. 
If $j$ lies between $q_{i,\ell}$ and $q_i,\ell$, $j$ crosses those quadruples that contain $Z'_\ell$ and $Z'_{\ell+1}$.
If non of the cases are true, we do not consider $q$ in the cells for column $j$.

Let us consider a $\log(n)$ vector of quadruples $v$ with one quadruple $Q_{i,j}$ for each $i$ and consider quadruples starting at, ending at, or crossing column $j$ for length class $i$.
We require that if for some $i$, the quadruple $q \in Q_{i,j}$ ends at $j$, then for all the length classes $L_k$ with $k>i$ the same condition holds (with index larger than $i$).
This also implies that if for some $i$, the quadruple of length class $i$ starts at $j$, then the same also holds for all quadruples of shorter length classes(with index larger than $i$).
%This insight is required in the induction over $j$. 
In particular, in order to be able to combine neighboring vectors of quadruples, we do not allow to mix starting and ending quadruples.
Let $\phi$ be the set of all $\log(n)$ vectors of tuples as described above (with one tuple of each length class).
% \sgnote{If we break Lemma 7 into definitions, then we can refer more precisely. \tgm{The lemma is in the appendix, we cannot refer to it more precisely.}}
The tuple for each length class is defined as in Lemma~\ref{lem:length-class} and the DP for general instances follows the ideas of Lemma~\ref{lem:length-class}.
% \sgnote{Do we need to be more precise about the selections, dominance regions here?}
We move from left to right column by column. In the initialization step, the joint DP cell is initialized based on Algorithm~\ref{alg:SWC} using $\phi$.
We guess the strings from each length class and consider them jointly in a DP cell and compute the minimum errors and pair of solution strings from interval $[q_{1,j}, q_{1,j+1}]$.

For column $j$, let us consider a vector $v \in \phi$.
We distinguish whether $v$ has starting or ending quadruples. 
(One of the two cases must apply due to the shortest length class.)
For a $v \in \phi$ with starting quadruples,
let $d$ be the smallest number such that there is a quadruple of length class $d$ starting at $j$.
To compute $v$ we consider all $v' \in \phi$ with the following properties.
(a) $v'$ has the same quadruples for all length classes $d' < d$ and 
(b) for $d' \ge d$, the right hand sides of the quadruples of length class $d'$ in $v'$ compatible the left hand sides of the quadruples of $v$ (see proof of Lemma~\ref{lem:length-class}.
The super-cells from left and right hand side are compatible if the intersecting strings from left and right hand side are assigned to same types of solution string $\sigma$ or $\sigma'$.
%\sgnote{This sentence not clear.}
%Now we can apply weighted majority weighting similar to in Lemma~\ref{lem:length-class}, but by considering strings from all length classes such that the right hand cells remain compatible with left hand side cells and trying over all possibilities.
%Then the value of $v$ is the minimum value over all $v'$. 

For a $v \in \phi$ with ending quadruples,
let $d$ be the smallest number such that there is a quadruple of length class $d$ ending at $j-1$. (In the very first column of the instance, we do not need this value.)
To compute $v$ we consider all $v' \in \phi$ with the following properties.
(a) $v'$ has the same quadruples for all length classes $d' < d$ and 
(b) for $d' \ge d$, the right hand sides of the quadruples of length class $d'$ in $v'$ match the left hand sides of the quadruples of $v$ in column $j-1$.
Then the value of $v$ is the sum of the minimum value over all such $v'$ and the number of errors in column $j$ obtained by applying $\SWC$ exactly as in the proof of Lemma~\ref{lem:length-class}. 
% \sgnote{Maybe, we need to be precise about how we take care of dependencies.}
%Observe that in Lemma~\ref{lem:length-class} we combined two pairs of super-cells that we can see as two length classes $L'$ and $L''$. 
%The only difference for the general instances with $log(n)$ length classes is that here we combine $\log(n)$ quadruples of super-cells.
%
%The termination till last column $j$ follows analogous to as in Section 2.2. 

The approximation ratio follows 
by
%analogous to Lemma~\ref{lem:length-class}, 
arguing that the expected number of errors at each column is at most $(1+O(\epsilon))$ of OPT
(see Lemma~\ref{lem:length-class}).
This finishes the proof of Theorem~\ref{thm:qptas}.
\section{Discussion}
The approximation status of \GMEC instances has been open for 10 years. We have studied the problem by unraveling its structures and identifying cases where the PTAS algorithm for \BMEC can not be directly applied for \GMEC.
Based on the ideas of the algorithm for \BMEC, we build our own dynamic programming algorithm, by additionally including some deep insights. We have proved that the dynamic programming algorithm is in QPTAS.
Proving that the \GMEC in QPTAS rules out the possibility of this problem to be in APX-hard and also provides a hint that this problem can exhibit a PTAS.

Unfortunately, due to the nested nature of dynamic programming, the algorithm can not be used to solve the \GMEC instances in practice.
The \GMEC instances are of utmost importance because they have a biological significance.
Based on this motivation, we looked into other types of algorithms ---parameterized algorithms \citep{martin2016whatshap} --- which are shown to work well in practice.
Studying the structures for \GMEC instances in this chapter provided us good insights in deriving the parameter in parameterized algorithm which is explained in the next chapter.
 \chapter{Parameterized algorithm for phasing individual genomes}\label{ref:chp2}
Dense and accurate chromosome-length haplotypes are necessary to completely understand the biology of diploid organisms.  
In this chapter, we provide a parameterized algorithm, which is basically an integrative phasing framework using different datasets for solving the phasing problem. 
This phasing algorithm is applied to the combination of global, but sparse haplotypes obtained from strand-specific single cell sequencing (Strand-seq) with dense, yet local haplotype information available through long-read or linked-read sequencing.
This results in dense and accurate chromosome-length haplotypes at reasonable costs. 
Furthermore, we provide comprehensive guidance on the required sequencing depths and reliably assign more than 95\% of alleles (NA12878) to their parental haplotypes using as few as 10 Strand-seq libraries in combination with 10-fold coverage PacBio data or, alternatively, 10X Genomics linked-read sequencing data.

\section{The need for combining different sequencing technologies}
Constructing genome-wide chromosome-length haplotypes is important for different biological applications.
Currently, methods used to chart the unique variation of individual human genomes rely largely on 2nd and 3rd generation DNA sequencing and can include specialized experimental protocols \citep{Snyder2015, Porubsky2016, de2014targeted, amini2014haplotype, selvaraj2013whole}. 
Sequencing technologies sample the human genome in the form of relatively short molecules (reads) and every read that spans at least two heterozygous variants 
can essentially be considered as a 'mini haplotype' that can be assembled into longer haplotype segments by partially overlapping reads spanning the same variable locus \citep{Glusman2014}. 
To this end, haplotype-informative reads need to be partitioned into two disjoint sets that represent the two haplotypes. 
This process, however, is challenging due to errors in sequencing as well as genotyping. 
For these reasons assembling haplotypes directly from sequencing data is computationally challenging, and the resulting optimization problems are proven hard \citep{Cilibrasi2007, Lancia2001}. 
Notwithstanding, a number of computational approaches for read-based phasing have recently been developed \citep{rhee2016survey} 
and, particularly, progress on fixed-parameter tractable (FPT) algorithms has enabled solving read-based phasing in practice \citep{deng2013highly, Kuleshov2014, Patterson2015}, for instance through the implementations available in the software package WhatsHap \citep{martin2016whatshap}. 
% Beyond phasing reads aligned to a reference genome, various approaches for haplotype-resolved de novo assembly have been explored21–25.	
% \todo{related work for MEC is missing.}

However, all approaches to reconstruct haplotypes from sequencing reads, reference-based or reference-free, come with the intrinsic limitation that the distance between subsequent heterozygous markers can be larger than the read length itself. 
While long-read sequencing (such as PacBio SMRT \citep{steinberg2014single} and Oxford NanoPore MinION \citep{ammar2015long}), or linked read data (such as those provided by 10X Genomics \citep{zheng2016haplotyping}) help to mitigate this issue, 
these technologies fail to phase over longer stretches of homozygosity, repeat-rich areas including segmental duplications, and centromeres. 
Thus, specialized techniques that enable homologous chromosomes to be discriminated are required to physically connect alleles across whole chromosomes \citep{zheng2016haplotyping, ma2010direct, yang2011completely}. 
As an alternative to whole chromosome separation, chromatin capture (Hi-C) methods \citep{lieberman2009comprehensive} can be employed to infer long-range haplotype information, based on the assumption that a chromosome will be cross-linked to itself more often than to its homologue \citep{selvaraj2013whole}.
Recently, Hi-C data sets have been used in combination with other sequencing methods for long-range phasing \citep{edge2017hapcut2, ben2016extending}. 
However, it has been shown that to generate a reliable long-range haplotype scaffold, relatively high sequence coverage (ideally ~90-fold) is needed to reduce bias caused by cross-links between non-homologous chromosomes \citep{edge2017hapcut2}. 
In particular, because these haplotypes need to be inferred statistically, the probability that two heterozygous variants are correctly phased relative to each other, deteriorates with increasing chromosomal distances.

% In Fig~\ref{fig:ex_all_datas}, we see the reads from different technologies over the human genome, ranging from Illumina based short reads to PacBio based long reads.
% The main observation is that none of these technologies produce reads such that the set of reads independently span the whole chromosome, covering all the variants.
% % Therefore, we propose an integrative strategy which has the ability to combine these technologies into one framework and therefore, generating accurate and complete haplotypes.
Our aim is to obtain dense and global haplotypes that span centromeres, homozygosity regions and genome assembly gaps, while keeping error rates, costs and labor at minimum. 
To this end, we harness the long-range phasing information provided by single cell template strand sequencing (Strand-seq) \citep{falconer2012dna, sanders2017single}. Strand-seq is an effective method to assemble highly 
accurate chromosome-length haplotypes, albeit with lower density of phased alleles in comparison to read-based phasing \citep{Porubsky2016}. 
Unlike other haplotyping methods, Strand-seq, by design, distinguishes parental homologues based on the directionality of single-stranded DNA. 
Therefore, Strand-seq is able to deliver global haplotypes, and its capability to correctly phase two variants with respect to each other does not depend on their distance. 
To fully exploit this advantage, while at the same time generating dense haplotypes that contain virtually all heterozygous SNVs, we designed a novel unified statistical framework to combine 
Strand-seq data with short-read, long-read, or linked-read sequencing data. 
Previously, Strand-seq data had only been used on its own, resulting in global yet sparse haplotypes \citep{Porubsky2016}. 
We demonstrate how the long range phase information inherent to Strand-seq data can be leveraged to bridge phased segments obtained from Illumina, PacBio or 10X Genomics sequencing data into contiguous and global haplotypes that span whole chromosomes.
We further offer extensive experimental guidance on favorable combinations of the number of used Strand-seq libraries and the depth of PacBio or Illumina coverage, 
and thus enable considerable reductions in costs and labor – yielding a novel, affordable and scalable approach for reconstruction of haplotype-resolved individual genomes.
\begin{center}
\begin{table}
\centering
\begin{tabular}{ |l|c|c| } 
 \hline
Formulation &  Approach  & Authors \\ 
  \hline
 MEC & Branch and Bound & \cite{wang2005haplotype}, \cite{lim2012individual}\\
   & Genetic algorithm& \cite{wang2005haplotype}, \cite{wang2012using}\\
   & Satisfiability (SAT) & \cite{mousavi2011effective}, \cite{he2010optimal} \\
   & Probabilistic approach & \cite{chen2008linear}, \cite{bansal2008mcmc}, \\
   & & \cite{Bansal2008}, \cite{Kuleshov2014b} \\
   & Parameterized & \cite{deng2013highly}, \cite{Pirola2015}, \\
   & & \cite{Patterson2015}\\
   & ILP & \cite{CDW13_exact} \\
   & Clustering & \cite{wang2007clustering} \\
   \hline
   MLF &  & \cite{zhao2005haplotype}, \cite{xie2008model}, \\
   & & \cite{kang2010hapassembler}, \cite{wu2013heuristic}\\
   \hline
   MFC & Graph & \cite{Duitama2010} \\
   \hline
   Others & Mixture Model & \cite{matsumoto2013mixsih} \\
    & Heuristic dynamic programming & \cite{xie2012fast} \\
    & Graph (spanning tree) & \cite{aguiar2012hapcompass}, \cite{mazrouee2014fasthap} \\
 \hline
\end{tabular}
\\[10pt]
 \caption{ Related work on computational approaches to haplotyping for a single individual}
\label{tab:related_work}
\end{table}
\end{center}
\section{Further related work}
The methods for haplotype assembly are mainly categorized using their problem formulations: minimum error correction (MEC), minimum letter flip (MLF), maximum fragment cut (MFC), and others shown in Table~\ref{tab:related_work}.
The explanation of these problem formulations is given in Chapter~\ref{ref:chp1}.

In the work by \cite{wang2005haplotype}, a haplotype problem is modelled as a binary tree and haplotypes are viewed as the optimal path in this tree. 
They applied a branch
and bound algorithm to solve this haplotype problem. 
This algorithm searches an optimal path in a binary tree, in which
the node on the $j$-th level denotes the $j$-th fragment and the
branch on the path connecting its child denotes its
corresponding category of haplotype. 
A binary string is used to express an individual code which represents a classification of fragments (a feasible solution to the MEC model).
The algorithm starts from root node by adding the first fragment and calculates the MEC score. 
Then, if the calculated score is bigger than the previous score, the node will be divided. This process is continued until all the fragments are considered.
This results in a binary tree and then the optimal haplotype path is computed in this tree.


The branch and bound algorithm
can find the exact optimal solution, but the running time is exponential in the number of
fragments. Therefore, it is not useful on large datasets. 

\cite{lim2012individual} first identified the initial upper bound using a local search algorithm and 
reduced the search space of the branch and bound algorithm based on this computed upper bound.
Thus, they solved the MEC problem in reasonable time.

Instead of the MEC score, \cite{wang2012using} used Hamming distance to calculate the difference between a haplotype and a fragment, 
without encoding the SNP values to 0, 1, and ``-'', as introduced in Chapter~\ref{ref:chp1}. 
Therefore, their  approach works even for tri- or tetra-allelic loci and homozygous sites. 
Thus they reconstructed the haplotype considering the case marked as a homozygous locus by a sequence error in the original data. 

Another approach for solving haplotyping using the MEC is to reduce the problem into a satisfiability (SAT) problem. 
\cite{he2010optimal} proposed a partial Max-SAT formulation for haplotype assembly.
% Given a set of clauses (a clause is a disjunction of Boolean literals), the MaxSAT problem asks for a complete assignment of all variables that maximizes the number of clauses the assignment satisfies.
Solvers such as Clone and WBO are used to solve the resulting MaxSAT problems.
\cite{mousavi2011effective} suggested a Max-2-SAT problem,
which is more general than the partial Max-SAT. The general Max-SAT solver is used for solving the instances. Their formulation is more generalized, 
considering homozygous alleles that can appear due to sequencing errors.
Also, their Max-SAT formulation results in instances with fewer variables and clauses than those of the Partial Max-SAT formulation.
The heuristic Max-SAT solver irots provided in the UBCSAT package is used.

Some studies tried to model haplotyping problems by using probabilistic approaches. 
Since the fragments of the input SNP matrix stem from two haplotypes, 
\cite{chen2008linear} assumed that the fragments were generated according to two parameters representing errors. 
They designed a probabilistic function using the error parameters for the haplotype. 
In the input fragment matrix $\mathcal{F}$, the fragments were divided into two sets and the most frequent character 
in each SNP site was selected to determine the haplotype sequence. 
Therefore, the two haplotypes could be reconstructed with a possible high probability. 
HASH by \cite{bansal2008mcmc} and HAPCUT by \cite{Bansal2008} also used probabilistic models based on graph structure. 
They constructed a graph with the input matrix. The nodes are the columns of the matrix. 
If there is a fragment that includes two sites, the two nodes are connected by an edge. 
The weight of the edge is the difference between the number of fragments matched to the haplotype sites and the unmatched fragment. 
HASH used a graph-cut algorithm and constructs a Markov chain, 
but HAPCUT optimized the MEC score by using a Max-Cut algorithm. 
In these two methods, the distance to both haplotypes was calculated for all fragments, and the fragments with the lowest distance are selected. 
Using a greedy algorithm, the best pair of haplotypes was determined by using the best MEC score.
Although HASH and HapCut achieve reasonably good results, they are stochastic and therefore can not guarantee optimal solutions.

Another method used dynamic programming to determine the haplotypes for a single individual \citep{he2010optimal}. 
For the input reads encoded in binary values, this approach first solves the partial instances optimally and then extends the partial obtained haplotypes by one bit repeatedly to obtain the full-length haplotypes. 
They showed that the method can be applied to whole-genome sequencing datasets. However, the method does not scale with increasing SNP sites.
To solve this drawback, \cite{deng2013highly} combined this dynamic programming method with a heuristic approach. 
This method first applies a heuristic to obtain a subset of the input matrix $\mathcal{F}$ by using a randomized sampling approach and then carries out the dynamic programming. 
Once it produces an initial solution from the submatrix, it refines the haplotypes by comparing the initial haplotype with all fragments. 
By repeating the initial solution and refinement steps, the final haplotypes are determined.
They showed that the heuristic algorithm gives very accurate solutions.
More recently, \cite{Pirola2015} considered a restricted variant of MEC, in which up to $k$ corrections were allowed per SNP position, 
and presented an FPT algorithm that runs in time exponential in $k$.

In another approach, the first ILP formulation for MEC was given by \cite{Fouilhoux2012} and was based on a reduction to the maximum bipartite induced sub-graph problem. 
In another ILP approach by \cite{CDW13_exact}, the binary variable $x_k$ for column $\mathcal{F}(k)$ 
was considered such that its value is supposed to be 1, if and only if the $k^{th}$ bits of $h^0$ and $h^1$ are 1 and 0 respectively.
Moreover, the binary variable $y_j$ for row $\mathcal{F}(j)$ was considered such that its value is supposed to be 1, if and only 
if the read corresponding to $\mathcal{F}(j)$ is aligned to $h^0$ and otherwise 0.
The constraints on all the binary variables for all the rows and columns, were that the binary variables should belong to 0 or 1.
The objective function was to minimize the number of flips in each entry from all the rows such that all rows can be assigned to the original haplotypes $h^0$ or $h^1$ without any conflict. 
By using some auxiliary variables, ILP can be reduced to a linear program, that gives an optimal solution to the problem.

Another approach by \cite{wang2007clustering} considered a clustering algorithm that is used to split the rows of $\mathcal{F}$ in two sets. 
The main contribution consists in the combination of the two distance functions used by the clustering algorithm. 
The first distance is the Hamming distance as defined in Equation~\ref{eq:distance}, which basically computes the number of mismatches between two fragments. 
The second distance $D'$ considers the number of matches between the two fragments.
The main idea is based on the intuition that, given a certain fixed number of mismatches between two fragments, the more they overlap the closer they are.
Using the above distance functions, a simple iterative clustering procedure is given as follows.
The hamming distance is computed for each possible pair of fragments in the SNP matrix.
Let the two fragments $r_1$ and $r_2$ have the highest Hamming distance and the clusters are initialized as $C1 = r_1$ and $C2 = r_2$.
Let the computed consensus strings are H1 and H2 from these two clusters C1 and C2 such that all the fragments are compared with H1 and H2 and assigned to the corresponding closer set. The ambiguity in the assignment of fragments to both consensus strings is broken based on the distance $D'$.
Once all fragments are assigned, the consensus strings H1 and H2 are updated and the algorithm iterates. The procedure loops until a stable haplotype pair is found (i.e. when the consensus haplotypes are the same before and after the update).

There were some approaches to solve other objective functions such as MFR and MSR.
They mainly used graphs, dynamic programming and other heuristic methods to solve haplotyping problem.
\begin{figure}[t!]\centering
\includegraphics[width=\columnwidth]{{Figure2}.pdf}
\caption{Integration of global and local haplotypes by the WhatsHap algorithm. 
An example solution of the weighted minimal error correction problem (wMEC) using WhatsHap algorithm is shown. For simplicity base qualities used as weights are omitted from the picture. 
(a)The columns of the matrix represent 34 heterozygous variants (SNVs). Continuous stretches of zeros and ones indicate alleles supported by respective reads (0 -- reference allele, 1 -- alternative allele). 
First two rows of the wMEC matrix are represented by Strand-seq haplotypes, illustrated as one 'super read' connecting alleles along the whole length of the chromosome. 
(1st row haplotype 1 alleles, 2nd row haplotype 2 alleles). 
Subsequent rows of the matrix are represented by reads that map to the reference assembly in short overlapping segments.  
Sequencing errors (shown in red in read 2 and 7) are corrected when the cost for flipping the alleles is minimized. (b) Reads are then partitioned into two haplotype groups (Haplotype 1 -- dark blue, Haplotype 2 -- light blue) 
such that a minimal number of alleles are corrected (in red). As an illustration of long haplotype contiguity facilitated by Strand-seq 'super reads', 
we depict two non-overlapping groups of reads (gray rectangles) that can be stitched together by Strand-seq (dashed lines). (c) Final haplotypes are exported for both groups of optimally partitioned reads.}
\label{fig:fig2}
\end{figure}
\section{Integrative phasing framework}
In this section, we present a novel integrative phasing method that is a framework to solve phasing jointly using different sequencing datasets.
To solve it, we present MEC instances to jointly include the read alignment or initial haplotypes from different technologies.
For example, For example, the haplotypes are generated using strand-specific cells or 10x Genomics, are added to the matrix $\mathcal{F}$. Furthermore, the read-alignments over the variants using different technologies such as PacBio or Illumina are then additionally incorporated in $\mathcal{F}$.
The goal is to partition the rows in $\mathcal{F}$ into two non-conflicting sets.
The input matrix and the bipartition of the rows is illustrated in Fig.~\ref{fig:fig2}.

Mathematically, aligned reads from Illumina or PacBio (or pre-phased 10X Genomics haplotype segments) and sparse Strand-seq haplotypes are jointly represented in the form of a fragment matrix, where each  row represent either one read (in case of Illumina and PacBio), 
one pre-phased haplotype segment (in case of 10X Genomics) or one sparse global haplotype (in case of Strand-Seq data) and columns represent the variant sites (Fig.~\ref{fig:fig2}). 
The matrix is filled with 0, 1 and ‘-’ entries, where 0 and 1 indicate that the corresponding read supports the reference or alternative allele, respectively,  and ‘-’ means the information is missing 
(e.g. because a read does not cover this variant site). WhatsHap selects a subset of rows and solves the wMEC problem optimally on these rows. 
The result is a maximum likelihood bipartition of rows, which corresponds to the two sought haplotypes.
For all analyses, whatshap was provided with a reference genome (option --reference) to enable re-alignment-based allele detection when constructing the fragment matrix from sequencing reads. 
This has been shown to significantly improve performance for PacBio reads \citep{martin2016whatshap}.
The goal is to partition the rows into two sets and generate two haplotypes for diploid genomes.

Based on Definition~\ref{def:feasible-mec}, we would like to generate a conflict-free MEC instance such that we can create a bipartition of rows into two sets. 
% \todo{explain MEC matrix and its goal using example above.}
% 
% \todo{1. present MEC matrix example for different technologies like pacbio, illumina, 10xG haps, SS haps. 2. Now state the common observation from all the examples. 3. Then provide a intuition to solve it. 
% 4. Finally provide the DP algorithm like in book DPChange of chapter 6.}
\subsection{StrandPhaseR pipeline}
% 	To build whole genome haplotypes from Strand-seq data we used 
	A new sorting-based pipeline, called StrandPhaseR, is developed to generating whole genome haplotypes from Strand-seq data.
	In StrandPhaseR phasing algorithm, a binary sorting based strategy of parallel matrices is used, in order to storing haplotype information obtained from single cell Strand-seq libraries. 
% 	StrandPhaseR implements an improved phasing algorithm based on a binary sorting strategy of two parallel matrices, storing haplotype information obtained from single cell Strand-seq libraries.
	Due to sequencing errors, there are conflicts in each matrix and the goal of sorting is to minimizing these conflicts in each matrix.
% 	The goal of sorting is to minimize the conflicts at variable positions in each matrix.
        The whole genome is divided into windows of equal size and WC regions for every Strand-seq library were identified by counting the number of Crick (forward, ‘+’) and Watson (reverse, ‘-’) reads.
% 	Haplotype informative WC regions were localized in every Strand-seq library  by counting the number of Crick (forward, ‘+’) and Watson (reverse, ‘-’) reads in equally sized regions (default 1 Mb). 
% 	We used Fisher's exact test to calculate the probability that a region contained approximately equal numbers of Crick and Watson reads and agreed with the expected 50:50 ratio of a WC region \citep{sanders2016characterizing}. 
% 	We identify alleles at SNVs for W and C reads in every informative region to generate haplotypes and then the sorting algorithm is performed.
% 	Alleles at variable positions (supplied as set of SNVs obtained from Illumina platinum haplotypes) were identified separately for W and C reads 
% 	in every informative region to generate low density single cell haplotypes 
% 	that are then sorted by the phasing algorithm. 
 
	
\subsection{WhatsHap Algorithm}\label{sec:algorithm}
WhatsHap \citep{Patterson2015} is a dynamic programming (DP) algorithm to optimally solve the wMEC problem.
It runs in $\mathcal{O}(2^c\cdot M)$ time, where $M$ is the number of variants to be phased and $c$ is the maximum physical coverage (which includes internal segments of paired-end reads).
Since it is independent of the read-length, so it is suitable for even long sequencing technologies.
The general idea is to proceed column-wise from left to right while maintaining a set of active reads.
Each read remains active from its first non-dash position to its last non-dash position in $\mathcal{F}$.
Let the set of active reads in column $k$ be denoted by $A(k)$.
Note that $c=\max_{k}\{|A(k)|\}$.
For each column $k$ of $\mathcal{F}$, we fill a DP table column $C(k,\cdot)$ with $2^\abs{A(k)}$ entries, one entry for each bipartition $B$ of the set of active reads $A(k)$.
Each entry $C(k,B)$ is equal to the cost of solving wMEC on the partial matrix consisting of columns $1$ to $k$ of $\mathcal{F}$ under the assumption that the sought bipartition of the full read set $A(1)\cup\ldots\cup A(k)$ \emph{extends} $B$ according to the below definition.
\begin{definition}[Bipartition extension]
For a given set $A$ and a subset $A'\subset A$, a bipartition $B=(P,Q)$ of $A$ is said to \emph{extend} a bipartition $B'=(P',Q')$ of $A'$ if $P'\subset P$ and $Q'\subset Q$.
\label{def:bipartite-extend}
\end{definition}
By this semantics of DP table entries $C(k,B)$, the minimum of the last column $\min_B\{C(M,B)\}$ is the optimal wMEC cost.

Let us consider an example to understand how the algorithm works. We consider an example SNP matrix $\mathcal{F}$ and its corresponding weight matrix $\mathcal{W}$ as follows.

\begin{equation}\label{eq:snp_matrix}
  \mathcal{F}  = \kbordermatrix{
     & v_{1}       & v_{2}  \\
    r_{1}       & 0 & 0 \\
    r_{2}       & 1 & 0 \\
    r_{3}       & 1 & 1 \\
  }
\end{equation}

\begin{equation}\label{eq:snp_weight_matrix}
  \mathcal{W}  = \kbordermatrix{
     & v_{1}       & v_{2}  \\
    r_{1}       & q_1 &  q_2\\
    r_2 & q_3 & q_4 \\
    r_3 & q_5 & q_6 \\
  }
\end{equation}

Let us consider some values in the weight matrix~\eqref{eq:snp_weight_matrix}:
\begin{equation}\label{eq:snp_weight_matrix1}
  \mathcal{W}  = \kbordermatrix{
     & v_{1}       & v_{2}  \\
    r_{1}       & 3 &  10\\
    r_2 & 9 & 1 \\
    r_3 & 4 & 5 \\
  }
\end{equation}

So the goal is to partition the reads into two sets with minimum flipping cost. If we try in a brute-force manner, the possible bipartitions are as follows.

\begin{itemize}
 \item For bipartition $\{r_1\}, \{r_2, r_3\}$, the cost is 1. This is achieved by flipping entry $\mathcal{F}(22)$ with a cost $\mathcal{W}(22) = 1$.
 \item For the other bipartition $\{r_1, r_2, r_3\}, \emptyset$, the cost as 14. This is achieved by flipping entries $\mathcal{F}(11), \mathcal{F}(12), \mathcal{F}(22)$, with a cost of 3+10+1 = 14.
\end{itemize}
\begin{figure}[t!]\centering
\includegraphics[width=\columnwidth]{{quality_measures}.pdf}
\caption{Hypothetical phasing of 10 single nucleotide variants (SNVs) along a defined chromosomal region is shown
here. Each heterozygous SNV is represented in its two allelic forms (0 - reference allele, 1 - alternative allele).
True (reference) haplotypes are distinguished in blue colors and predicted haplotypes in red. a) To count the
number of switch errors (black crosses) between the true and predicted haplotypes, neighboring pairs of SNVs
are compared along each haplotype and recorded as a new binary string of 0's and 1's depending on whether the
allele state changes (see gray box). A zero value is assigned if the given pair of SNVs have the same value,
otherwise a value of 1 is assigned value 1. The absolute number of differences in the binary string generated for
the true and predicted haplotypes is reported as the total number of switch errors. b) To calculate the Hamming
distance, the absolute number of differences between reference and predicted haplotypes is calculated for all
SNV positions. In addition we calculate block-wise Hamming distance which represents a cumulative sum of all
Hamming distances across all phased segments}
\label{fig:quality}
\end{figure}

In a similar manner, we can compute the cost for other possible bipartitions.

Since this example is very small, it is do-able in a brute-force manner. For large input instances, we explain the FPT algorithm implemented in WhatsHap with parameter as ``coverage''.
As explained above, the initialization for first column can be computed as follows. We consider all possible bipartitions and store the best possible allele assignment for different bipartitions.

For instance, the DP column cell for the above example for different bipartitions, $\Delta_C(1,.)$ can be filled as follows:
\[C(1, (\{r_1,r_2,r_3\},\emptyset)) =\min\{3+0,13+0\} = 3\]

Similarly, we can compute $\Delta_C(1,.)$ for other bipartitions $(\{r_1,r_2\},\{r_3\}),(\{r_1,r_3\},\{r_2\}),$\\
$(\emptyset,\{r_1,r_2,r_3\}), (\{r_3\},\{r_1,r_2\}), (\{r_2\},\{r_1,r_3\})$.

Now, let us consider second column and see how to compute DP column cell $C(2,.)$ for different bipartitions:
\[C(2, (\{r_1,r_2,r_3\},\emptyset)) = \min\{11+0, 5+0\}  + \min\{C(1, (\{r_1,r_2,r_3\},\emptyset)\} = 5+3 = 8 \]

In the above recurrence, we make sure that we follow the conditions imposed by Definition~\ref{def:bipartite-extend}.
To fill DP column $C(2,.)$, we can analogously compute the cost for the remaining bipartitions $(\{r_1,r_2\},\{r_3\})$,
$(\{r_1,r_3\},\{r_2\})$, $(\emptyset,\{r_1,r_2,r_3\})$, $(\{r_3\},\{r_1,r_2\})$, and $(\{r_2\},\{r_1,r_3\})$.

Finding optimal haplotypes is similar to finding the optimal bipartition of reads. Once we reach the last column and know the optimal bipartition, we can backtrace to get the optimal haplotypes.
We obverse that the running time is linear in the number of variants, so therefore, it is independent of read length.

\section{Evaluation metrics}
To assess the quality of assembled haplotypes in this study, we calculated different metrics described in the following.

\begin{itemize}
 \item Completeness: The process of haplotyping establishes phase relations between pairs of consecutive heterozygous variants. We call each such pair a 'phase connection'. For each haplotype segment produced by a combination of technologies, we therefore count the number of phase connections, which is equal to the number of heterozygous markers in the haplotype segment minus one. 
To measure the completeness of a phasing, we sum the number of phase connections across all haplotype segments and divide by the maximum possible number of phase connections, which is equal to the number of heterozygous variants in the chromosome minus one.
\item Switch error rate: The switch error rate is the fraction of phase connections for which the phasing between the two involved heterozygous variants is wrong (see Figure~\ref{fig:quality})).
\item Largest haplotype segment: In this study we are interested in haplotypes that span the whole length of the chromosomes. To measure the completeness of phasing, we report the fraction of heterozygous variants that are part of the largest haplotype segment.
\item Largest haplotype segment Hamming rate: To assess whether haplotypes are correct over long genomic distances, we only consider the largest haplotype segment and compute the Hamming distance between true and predicted haplotypes (see Figure~\ref{fig:quality}),
divided by the total number of heterozygous variants in this haplotype segment. 
That is,the Hamming error rate is equal to the fraction of wrongly phased heterozygous variants. 
Note that, a switch error (e.g. in the middle of a chromosome) can result  into a very high Hamming distance and hence the Hamming distance is a much more stringent quality measure. 
While the switch error rate assesses whether haplotypes are correct locally, i.e. between pairs of neighboring heterozygous variants, the Hamming distance assesses whether haplotypes are correct globally.
\end{itemize}

\section{Results}
\subsection{Datasets} 
Illumina reads \citep{sudmant2015integrated, 10002015global} were obtained from the 1000 Genome Project Consortium \footnote{\url{ftp://ftp.1000genomes.ebi.ac.uk/vol1/ftp/phase3/data/NA12878/high_coverage_alignment/}}. 
PacBio reads \citep{giab} were obtained from Genome in a Bottle Consortium (GIAB) \footnote{\url{ftp://ftp trace.ncbi.nlm.nih.gov/giab/ftp/data/NA12878/NA12878_PacBio_MtSinai/sorted_final_merged.bam}}. 
10X Genomics haplotypes: Pre-assembled 10X Genomics haplotypes (produced on the Chromium platform with Chromium Genome v1 reagents, sequenced on an Illumina HiSeq X Ten and processed with LongRanger 2.1.0) were downloaded from 
10X Genomics website \footnote{\url{https://support.10Xgenomics.com/genome-exome/datasets/NA12878_WGS_210}} and filtered for heterozygous and PASS filter SNVs. 
Strand-seq libraries \citep{Porubsky2016}: We downloaded them from the European Nucleotide Archive \footnote{\url{http://www.ebi.ac.uk/ena}}, accession number: PRJEB14185. 
The same data can also be obtained at the Zenodo site \footnote{\url{doi:10.5281/zenodo.830278}}. 
Reference haplotypes \citep{eberle2017reference}: In this study we use as a gold standard, we downloaded reference triopedigree- based haplotypes of NA12878 obtained released as part of the Illumina platinum genomes (Version: 2016-1.0 from 6 June 2016) \footnote{http://www.illumina.com/platinumgenomes/}.

\subsection{Experimental design and dataset description}
	To explore a new integrative phasing strategy, with the aim of obtaining dense and accurate chromosome-length haplotypes, we used sequencing data available for a well-studied individual (NA12878). 
	The NA12878 genome has been extensively sequenced using multiple technologies, providing high-coverage public sources of sequence information (see “Methods” section). 	
	In this study, we focused on read-based phasing data generated from Illumina short-read sequencing and PacBio technology, as they represent current standards for short- and long-read sequencing, 
	respectively (Illumina short-read sequencing is for simplicity referred to as “Illumina data”). 
	The Illumina dataset was sequenced to an average depth of 49.5x coverage with a median insert size of 433bp, and the PacBio dataset was sequenced to 39.6x coverage with an average read length of ~15kb.
	In addition, we evaluated the performance of 10X Genomics, an emerging linked-read technology. Since none of these technologies alone provides chromosome-length haplotype information, 
	we additionally incorporated single cell Strand-seq data \citep{Porubsky2016}, which has the capacity to scaffold haplotype information obtained from other data types (Fig.~\ref{fig:fig1}(a)). 
	Here we used 134 single cell libraries sequenced to an average depth of 0.037x coverage per library using a paired-end sequencing protocol. 
	To evaluate the phasing accuracy of haplotypes reported in this study, we used the publicly available Illumina platinum haplotypes generated for the same individual (NA12878) as a 'reference' standard.
	NA12878 'reference haplotypes’ were completed by genetic haplotyping using highly accurate genotypes from seventeen individuals of a three-generation pedigree \citep{eberle2017reference}, which renders it an ideal gold-standard set for haplotype comparisons. 
	We confirmed that sites and genotypes are in very good agreement with Genome in a Bottle calls.
	However, it should be noted that, due to stemming from short reads, this SNV set most likely lacks some variants at repetitive or complex genomic loci (e.g. recent segmental duplications).

\subsubsection{Downsampling of sequencing datasets}
	To assess different combinations of Strand-seq libraries (w.r.t. number of single cell libraries) with read data (w.r.t. depth of coverage), 
	we performed a systematic analysis of the phasing performance for various subsets of each dataset. 
	To achieve this, we downsampled the  original publicly available datasets consisting of: 134 single cell Strand-seq libraries \citep{Porubsky2016}, 
	39.6x coverage long-read PacBio data \citep{giab}, and 49.6x coverage short-read Illumina data \citep{sudmant2015integrated, 10002015global}. 
	To simulate Strand-seq datasets consisting of reduced numbers of single cells, we randomly selected subsets of either 5, 10, 20, 40, 60, 80, 100, or 120 libraries from the original number of 134  libraries in the dataset. 
	Read data from the PacBio and Illumina datasets were downsampled using Picard (picard-tools-1.130) to meet a defined depth of coverage of either 2, 3, 5, 10, 15, 25, or 30-fold. 
	The downsampling was performed for 5 independent trials to account for variability in downsampled datasets, and the average phasing performance across all trials was reported (as described below).

\subsection{Phasing performance of individual technologies}
	To independently assess the phasing performance of each technology we assembled haplotypes directly from sequencing reads (Illumina or PacBio) using WhatsHap (see Methods). 
	The main advantage of this algorithm is that it solves the Minimum Error Correction (MEC) problem optimally with a run-time that scales linearly in the number of variants (alleles) and is independent of the read length. 
	Therefore, it performs well with short-read technologies (Illumina) and is especially suited for use with long reads (PacBio, Oxford Nanopore). 10X Genomics haplotype segments were assembled by the vendor using the 10X LongRanger pipeline. 
	To phase multiple Strand-seq libraries we have developed a new phasing algorithm, implemented in the R package StrandPhaseR.
	In comparison to our previously published phasing algorithm \citep{Porubsky2016}, the current algorithm is provided as an easy to use R package and implements more robust heuristic approach to solve the MEC problem in Strand-seq data.
	The haplotypes generated by each technology (i.e. Illumina, PacBio, 10X Genomics and Strand-seq) were compared to the Illumina platinum reference haplotypes, to establish the density, completeness and accuracy of the phase blocks delivered by each platform independently. 
	For a more streamlined exposition, we focus on the results obtained for Chromosome 1 in the following analysis and present numbers aggregated across all chromosomes in a concluding discussion.	
	
	We found both PacBio and 10X Genomics technologies capable of phasing nearly the complete set of variants listed in the reference haplotypes (98.8\% and 97.2\%, respectively), 
	whereas Illumina alone phased only 77.8\% and Strand-seq only 57.6\% of the reference SNVs (Fig.~\ref{fig:fig1}(b)). 
	Note that for 10X Genomics data, we used the variant set discovered, genotyped, and phased by the 10X LongRanger software and hence variants not discovered decrease our estimate of completeness.
	The comparatively low percentage for Strand-seq can be explained by the relatively low sequencing coverage employed, combined with a slight unevenness in genomic coverage (see Figure~\ref{fig:quality}). 
	For all technologies except Strand-seq, only short-range haplotypes were assembled using the read-based phasing, with a limited number of alleles phased per haplotype segment (Fig.~\ref{fig:fig1}(c)). 
	For instance, we found >30,000 unconnected haplotype segments assembled from Illumina data, with the largest segment of 16kb (median ~500bp) harboring only 0.06\% of the phased variants. 
	This is because heterozygous variants that are further apart than the length of the sequenced DNA fragments cannot be connected, resulting in multiple disjoint haplotype segments with an unknown phase between them. 
	Improvements were achieved using longer sequencing reads from PacBio technology, which effectively decreased the number of phased haplotype segments (1,927) and increased their size; the largest segment of 1.7Mb (median ~21kb) 
	containing 1.25\% of all SNVs on Chromosome 1 (Fig.~\ref{fig:fig1}(c)). 10X Genomics produced even longer haplotype segments than both Illumina and PacBio data (Fig.~\ref{fig:fig1}(c)). 
	The largest haplotype segment contained almost 5\% of the heterozygous SNVs and spanned more than 8.5Mb (median ~241kb). Still, the haplotypes of Chromosome 1 came in 199 disconnected segments and, 
	hence, an end-to-end phasing was not achieved (Fig.~\ref{fig:fig1}(c)). That is, the linked reads from the 10X Genomics were not able to connect distant neighboring heterozygous sites, for instance at centromeres, 
	genome assembly gaps or regions of low heterozygosity (Fig.~\ref{fig:fig1}(a)). 
	This is in contrast to the global, albeit sparse, haplotypes produced by Strand-seq. 
	Although the completeness of Strand-seq haplotypes was lower compared to the other technologies, all phased variants were placed into a single haplotype segment spanning the entire length of Chromosome 1 (Fig.~\ref{fig:fig1}(b), and (c)).	
	
	Finally, we assessed the accuracy of each technology by calculating the extent of switch errors in comparison to the reference haplotypes. High phasing accuracy of each technology was exemplified by the low percentage (<0.4\%) 
	of switch errors (Fig.~\ref{fig:fig1}(d)) with PacBio and 10X Genomics being the most accurate. 
	Since no single phasing technology was sufficient to generate both global and dense haplotypes, 
	we explored integrative phasing approaches that combine global, sparse haplotyping as afforded by Strand-seq technology with local high-density haplotypes from read-based phasing.
\begin{figure}[t!]\centering
\includegraphics[width=\columnwidth]{{Figure1}.pdf}
\caption{Phasing efficacy of read-based and experimental phasing approaches using Chromosome 1 as an example.
a) Two homologous chromosomes are shown (blue and black). 
Experimental phasing approaches like Strand-seq can connect heterozygous alleles along whole chromosomes, however, at higher costs (time and labor) and lower density of captured alleles. 
In contrast, read-based phasing can deliver high-density haplotypes, but only short haplotype segments are assembled with an unknown phase between them. 
b) Barplot showing the percentage of phased variants, for each sequencing technology, from the total number of reference SNVs (Illumina platinum haplotypes). 
c) Graphical summary of phased haplotype segments for Illumina, PacBio, 10X Genomics and Strand-seq phasing shown for chromosome 1. 
Each haplotype segment is colored in a different color with the longest haplotype colored in red. Side bargraph reports the percentage of SNVs phased in the longest haplotype segment. 
d) Accuracy of each independent phasing approach measured as percentage of short switch errors in comparison to benchmark haplotypes.}
\label{fig:fig1}
\end{figure}

\subsection{Integrative global phasing performance}
We found that the combination of Strand-seq haplotypes with any of the other data types markedly increased the number of variants that were phased in the largest haplotype segment, albeit to differing degrees (Fig.~\ref{fig:fig3}(a)). 
Specifically, for the Illumina data we observed the completeness of each haplotype increased gradually with the number of Strand-seq libraries used in the experiment, 
whereas the depth of coverage of Illumina data had only a minor but noticeable effect (Fig.~\ref{fig:fig3}(a)). 
In contrast, the PacBio data showed a significant improvement in haplotype completeness at 10-fold genomic coverage, 
regardless of the number of Strand-seq libraries used (Fig.~\ref{fig:fig3}(a), black arrowhead). 
Similar results were seen when we combined Strand-seq with the 10X Genomics haplotypes (Fig.~\ref{fig:fig3}(a)). 
In all cases, integration of Strand-seq phasing drastically improved the contiguity of the haplotype spanning Chromosome 1 (Fig.~\ref{fig:fig3}(b)). 
When combining Illumina data with 40 Strand-seq libraries >65\% of the reference variants could be phased accurately (Fig.~\ref{fig:fig3}(b), black asterisk); 
5497 haplotype segments (collectively representing 19.7\% of the phased SNVs), however, remained disconnected, even when integrating the complete (N=134) Strand-seq dataset. These results confirm that Illumina data are of limited utility for haplotype phasing.
    
    In contrast, as few as 10 Strand-seq cells combined with 10-fold PacBio coverage were sufficient to phase more than 95\% of all heterozygous SNVs into a single haplotype segment (Fig.~\ref{fig:fig3}(b), black asterisk), 
    and merely 5 Strand-seq single cell libraries were required to connect all 10X Genomics haplotypes. 
    However, we recommend at least 10 Strand-seq libraries (Fig.~\ref{fig:fig3}(b), black asterisk) to ensure that at least one haplotype-informative (i.e. Watson-Crick-type) cell exists for every chromosome with high probability (p=0.978).
    This global haplotyping was unique to Strand-seq, as the combination of 10X Genomics with PacBio reads proved inefficient to join locally phased segments (Fig.~\ref{fig:fig3}(b)). 
    That is, the added value of combining these two technologies is limited as the haplotype segments tend to break at similar locations.
    
    Finally, we assessed the phasing accuracy of the assembled haplotypes (the longest phased segment only) (Fig.~\ref{fig:fig3}(c)). 
    Similar to the completeness of the haplotype, the accuracy of Illumina phasing gradually increased with sequencing depth and Strand-seq library number, indicating that Illumina coverage of 30-fold and higher is advisable (Fig.~\ref{fig:fig3}(c)). 
    We further observed slightly elevated switch error rates at lower PacBio depths, which plateaued at 10-fold coverage (Fig.~\ref{fig:fig3}(c), black arrowhead). 
    This is likely caused by allele uncertainty resulting from error-prone PacBio reads, especially at lower sequencing depths (Fig.~\ref{fig:fig3}(c)). 
    The lowest switch error rate (< 0.2\%) was achieved by the combination of Strand-seq with 10X Genomics data (Fig.~\ref{fig:fig3}(c), switch error rate).
  \begin{figure}[t!]\centering
\includegraphics[width=\columnwidth, height = \columnwidth]{{Figure3}.pdf}
\caption{Various combinations of Strand-seq and read-based phasing (Illumina, PacBio, 10X Genomics) - Chromosome 1 as an example.
Plots show haplotype quality measures for various combinations of Strand-seq cells (5, 10, 20, 40, 60, 80, 100, 120, 134) 
with selected coverage depths of Illumina or PacBio sequencing data (2, 3, 4, 5, 10, 15, 25, 30, >30-fold), 
or in combination with 10X Genomics haplotypes. a) Assessment of the completeness of the largest haplotype segment as the \% of phased SNVs. 
Grey bars highlight  PacBio sequencing depth where completeness and accuracy of final haplotypes do not dramatically improve. 
b) Assessment of the contiguity of the largest haplotype segment as the length of the largest haplotype segment. 
Every phased haplotype segment is depicted as a different color, with the largest segment colored in red. 
Black asterisks point to a recommended depth of coverage of a given technology in combination with Strand-seq 
c) Assessment of the accuracy of the largest haplotype segment as the level of agreement with the ‘reference’ standard. 
Black arrowheads highlight Illumina and PacBio sequencing depth where accuracy of final haplotypes do not substantially improve.}
\label{fig:fig3}
\end{figure}

    Switch error rates reflect local inaccuracies expressed by the number of pairs of consecutive heterozygous variants that are wrongly phased with respect to each other. 
    These error rates are not necessarily informative about global haplotype accuracy, which largely depend on how switch errors are spatially distributed. 
    Note that one single switch error implies that all following alleles (up to the next switch error) are assigned to the wrong haplotype. 
    Since our goal is to generate dense and global haplotypes, we additionally report the Hamming error rate of the largest haplotype segment in comparison to the reference haplotypes. 
    Illumina reads are highly accurate and therefore we observed lower impact of sequencing depth on the global accuracy of the largest phased haplotypes (Fig.~\ref{fig:fig3}(c), Hamming error rate). 
    In contrast, PacBio reads exhibited higher sequencing error rates, which translated into higher switch error rates at low sequencing depths. 
    Using 10-fold PacBio coverage combined with at least 10 Strand-seq cells yielded highly accurate global haplotypes (Fig.~\ref{fig:fig3}(c), black arrowhead), 
    while lower coverages led to markedly worse results. Furthermore, the combination of Strand-seq with 10X Genomics haplotypes yielded highly accurate global haplotypes, 
    already at the minimal amount of Strand-seq libraries (Fig.~\ref{fig:fig3}(c), right panel).
    
    Taken together, these results illustrate that Strand-seq can be used to phase existing sequence data and build dense, global and highly accurate haplotypes. 
    Indeed, we found our approach highly efficient for genome-wide phasing (Fig.~\ref{fig:fig4}(a)). Using a combination of 40 Strand-seq libraries with 30-fold Illumina coverage, 
    or 10 Strand-seq libraries with either 10-fold PacBio coverage or the 10X Genomics haplotypes we successfully scaffolded chromosome-length haplotypes for every autosome of NA12878. 
    The completeness of the genome-wide haplotypes measured for the largest haplotype block reached 95.7\% and 69.1\% using PacBio and Illumina reads, respectively (Fig.~\ref{fig:fig4}(a)). 
    We further demonstrated the high accuracy of these haplotypes on the local and global scales, which showed low switch (<0.45\%) and Hamming error (<0.99\%) rates for both the PacBio and Illumina combination (Fig.~\ref{fig:fig4}(a)). 
    Whereas scaffolding the 10X Genomic haplotypes produced the most accurate local haplotypes (switch error rate of 0.05\%), global performance suffered, and the highest Hamming error rate (2.18\%) was calculated for this combination. 
    Nevertheless, using Strand-seq to scaffold any of the datasets remarkably improved the completeness, contiguity and accuracy of phasing for each chromosome, 
    highlighting our integrative phasing strategy as a robust method for building dense and accurate whole genome haplotypes.

\begin{figure}[t!]\centering
\includegraphics[width=\columnwidth]{{Figure4}.pdf}
\caption{Recommended settings to phase certain amounts of individuals. (a) Genome-wide phasing of NA12878 using combination of 40 Strand-seq libraries with 30$\times$ short Illumina reads, 
10 Strand-seq libraries with 10-fold long PacBio reads, or 10 Strand-seq libraries with 10X Genomics data. Plots show quality measures such as percentage of phased SNV pairs, switch error rate, 
and Hamming error rate for phased autosomal chromosomes. (b) A diagram providing the recommendations for the required number of Strand-seq libraries to be combined with recommended minimum of 10-fold PacBio and 30$\times$ Illumina coverage
in order to reach global and accurate haplotypes for a depicted number of individual diploid genomes.}
\label{fig:fig4}
\end{figure}
\section{Discussion}
	Strand-seq has been successfully prepared from a wide range of cell types taken from various organisms \citep{Porubsky2016, falconer2012dna, sanders2017single} and is currently being adopted by an increasing number of researchers. 
	The integrative phasing strategy, which is a parameterized algorithm, paves the way to leveraging Strand-seq to obtain chromosome-length dense and accurate haplotypes at a manageable cost and labor investment. 
	Based on the comprehensive evaluation presented above, we recommend three different combinations of Strand-seq with a complementary technology (Fig.~\ref{fig:fig4}(b)).
	
	As one option, one can combine Strand-seq with standard Illumina sequencing. Although the power of Illumina data for phasing is limited, mainly due to short insert sizes and read lengths, it still has some merit for adding additional variants to Strand-seq haplotypes. 
	This might be of interest to many researchers since Illumina sequencing still constitutes the most common technology and there is an abundance of Illumina sequence data currently available for many sample genomes. 
	To completely phase these preexisting data, we recommend generating at least 40 Strand-seq libraries for the sample genome, which is sufficient to phase >68\% of all heterozygous variants genome-wide with good accuracy 
	(switch error 0.45\%, Hamming error 0.99\%), see Fig.~\ref{fig:fig4}(a).
	
	To build more complete haplotypes, we recommend combining Strand-seq with either PacBio or 10X Genomic technologies. A minimum of 10-fold PacBio coverage coupled with 10 Strand-seq 
	libraries will phase >95\% of heterozygous variants genome-wide with excellent accuracy (switch error 0.25\%, Hamming error 0.91\%). 
	PacBio has been demonstrated to be particularly powerful for resolving structural variation \citep{huddleston2017discovery, chaisson2015genetic} and, 
	although not explored here, might hence be the best choice when the resolution of haplotypes, structural variation and repetitive regions is desired. 
	However, the cost of this platform is still comparatively high. Therefore, until long-read technologies have become standard practice, we recommend combining 10 Strand-seq 
	libraries with 10X Genomics technology. We found this combination yielded the most complete (>98\% heterozygous variants genome-wide) haplotypes with the lowest switch error rate (0.05\%). 
	We did observe a slightly increased Hamming error rate (2.18\%), however, which indicates that some genomic intervals are placed on the wrong haplotype, most likely due to switch errors 
	in the pre-phased haplotype segments (produced by 10X Genomics) used as input. Overall, combining Strand-seq with 10X Genomics is the most cost-effective (in terms of time and money) strategy to phase an individual genome at extraordinary accuracy.

	
	In this study, we used pre-phased 10X Genomics haplotype segments because using the raw sparse linked read data leads to algorithmically challenging wMEC problem instances, 
	which presently cannot be solved optimally by WhatsHap. This implies that variants that have not been discovered by LongRanger are considered unphased (and hence decrease “completeness”) 
	and that the error rates can likely be improved further by solving the combined instance resulting from Strand-seq and 10X data. We therefore consider processing the 10X Genomics raw data an important topic of future research.

	In this chapter, we focused on single individual haplotyping to avoid the biases and limitations of reference-panel based phasing as well as the need to have access to genetic material of the parents. 
        In cases when high-coverage sequencing data of the parents are available, such datasets can be used to enhance read-based phasing and provide long range 
	phase information. 
	
% 	Strand-seq relies on BrdU incorporation during DNA replication and its use is therefore restricted to dividing cells.
% 	To provide long-range phase information for single samples in situations where growing cells are not available, 
% 	Hi-C can constitute an alternative solution able to yield chromosome-spanning haplotypes \citep{edge2017hapcut2}. 
% 	However, the required coverage, and hence sequencing cost, is considerably higher for Hi-C than for Strand-seq. 
% 	While the 134 Strand-seq libraries we used here reach a cumulative sequencing coverage of around 5x, markedly higher coverages are needed for Hi-C \citep{edge2017hapcut2}. 
% 	
% 	Our results demonstrate that dense and accurate chromosome-length haplotypes can be generated at manageable costs. 
% 	This development brings haplotype-level analyses closer to a routine practice, which can be key for understanding disease phenotypes. 
% 	We emphasize that the strategy we present here works for single individuals without relying on other family members or statistical inference from haplotype reference panels. 
% 	In contrast to such population-based phasing approaches, the method we advocate here allows insights into rare and de novo variants and long-range epistatic effects.
	
	In the next chapter, we will analyze how we can generalize this parameterized algorithm to incorporate trio information to performing phasing. 
	This will provide possibilities to generate good quality haplotypes for pedigrees, which will have profound implications to the study of variability of personal genomes in health and disease.




% 
% 





% \section{Integrative global phasing strategy}
% 	To generate more complete and dense haplotypes, we sought to establish a novel and  integrative phasing approach using a combination of Strand-seq data with the other data types. That is, we aim to enrich the sparse yet global phasing from Stand-seq using the dense haplotype information provided by Illumina, PacBio or 10X Genomics. However, integrating phase information across platforms poses a non-trivial statistical and algorithmic challenge, which we resolved by treating the sparse Strand-seq haplotypes generated by StrandPhaseR as one row in the fragment matrix processed by WhatsHap (see Methods). The other rows correspond to sequencing reads (PacBio, Illumina) or pre-assembled haplotype segments (10X Genomics) (see Methods). This allows, for the first time, for integrative phasing by solving the corresponding optimization problem (weighted MEC) provably optimal (\todo{fig}2). We performed extensive experiments to demonstrate that this approach enables excellent results in practice, as we describe in the following section.
% 	To discover the most beneficial combinations of Strand-seq with Illumina or PacBio data, we explored combinations of variable numbers of Strand-seq libraries together with increasing depths of sequencing reads. To this end, we downsampled the number of Strand-seq libraries used in the analysis by randomly selecting subsets of libraries (5, 10, 20, 40, 60, 80, 100, or 120) from the original (N = 134) dataset. Similarly, we randomly downsampled the sequencing reads from the Illumina and PacBio datasets to a lower genomic coverage (2, 3, 4, 5, 10, 15, 25, and 30-fold). We applied our integrative phasing strategy to all pairs of downsampled Strand-seq libraries and the downsampled PacBio/Illumina datasets to assess the completeness (i.e. % of phased SNVs), contiguity (length of the largest haplotype segment) and accuracy (agreement with the 'reference' standard) of each assembled haplotype.
% 	We found that the combination of Strand-seq haplotypes with any of the other data types markedly increased the number of variants that were phased in the largest haplotype segment, albeit to differing degrees (\todo{fig}3a). Specifically, for the Illumina data we observed the completeness of each haplotype increased gradually with the number of Strand-seq libraries used in the experiment, whereas the depth of coverage of Illumina data had only a minor but noticeable effect (\todo{fig}3a, i). In contrast, the PacBio data showed a significant improvement in haplotype completeness at 10-fold genomic coverage, regardless of the number of Strand-seq libraries used (\todo{fig}3a i, black arrowheadgray rectangle). Similar results were seen when we combined Strand-seq with the 10X Genomics haplotypes (\todo{fig}3a, ii). In all cases, integration of Strand-seq phasing drastically improved the contiguity of the haplotype spanning Chromosome 1 (\todo{fig}3b). When combining Illumina data with 40 Strand-seq libraries >65% of the reference variants could be phased accurately (\todo{fig}3b i, black asterisk); 5497 haplotype segments (collectively representing 19.7% of the phased SNVs), however, remained disconnected, even when integrating the complete (N=134) Strand-seq dataset. These results confirm that Illumina data are of limited utility for haplotype phasing.
% 	In contrast, as few as 10 Strand-seq cells combined with 10-fold PacBio coverage were sufficient to phase more than 95% of all heterozygous SNVs into a single haplotype segment (\todo{fig}3b ii, black asterisk), and merely 5 Strand-seq single cell libraries were required to connect all 10X Genomics haplotypes. However, we recommend at least 10 Strand-seq libraries (\todo{fig}3b iii, black asterisk) to ensure that at least one haplotype-informative (i.e. Watson-Crick-type) cell exists for every chromosome with high probability (p=0.978). This global haplotyping was unique to Strand-seq, as the combination of 10X Genomics with PacBio reads proved inefficient to join locally phased segments (\todo{fig}3b iv). That is, the added value of combining these two technologies is limited as the haplotype segments tend to break at similar locations.
% 	Finally, we assessed the phasing accuracy of the assembled haplotypes (the longest phased segment only) (\todo{fig}3c). Similar to the completeness of the haplotype, the accuracy of Illumina phasing gradually increased with sequencing depth and Strand-seq library number, indicating that Illumina coverage of 30-fold and higher is advisable (\todo{fig}3c, i). We further observed slightly elevated switch error rates at lower PacBio depths, which plateaued at 10-fold coverage (\todo{fig}3Cc, black arrowhead). This is likely caused by allele uncertainty resulting from error-prone PacBio reads, especially at lower sequencing depths (\todo{fig}3C, i). The lowest switch error rate (< 0.2%) was achieved by the combination of Strand-seq with 10X Genomics data (\todo{fig}3c, ii, switch error rate).
% 	Switch error rates reflect local inaccuracies expressed by the number of pairs of consecutive heterozygous variants that are wrongly phased with respect to each other. These error rates are not necessarily informative about global haplotype accuracy, which largely depend on how switch errors are spatially distributed (see Methods, Supplementary \todo{fig}S4a). Note that one single switch error implies that all following alleles (up to the next switch error) are assigned to the wrong haplotype. Since our goal is to generate dense and global haplotypes, we additionally report the Hamming error rate of the largest haplotype segment in comparison to the reference haplotypes (see Methods, Supplementary \todo{fig}S4b). Illumina reads are highly accurate and therefore we observed lower impact of sequencing depth on the global accuracy of the largest phased haplotypes (\todo{fig}3c, Hamming error rateiii). In contrast, PacBio reads exhibited higher sequencing error rates, which translated into higher switch error rates at low sequencing depths. Using 10-fold PacBio coverage combined with at least 10 Strand-seq cells yielded highly accurate global haplotypes (\todo{fig}3c, iii black arrowheadgray rectangle), while lower coverages led to markedly worse results. Furthermore, the combination of Strand-seq with 10X Genomics haplotypes yielded highly accurate global haplotypes, already at the minimal amount of Strand-seq libraries (\todo{fig}3c, Hamming error rateiv).
% 	Taken together, these results illustrate that Strand-seq can be used to phase existing sequence data and build dense, global and highly accurate haplotypes. Indeed, we found our approach highly efficient for genome-wide phasing (\todo{fig}4a). Using a combination of 40 Strand-seq libraries with 30-fold Illumina coverage, or 10 Strand-seq libraries with either 10-fold PacBio coverage or the 10X Genomics haplotypes we successfully scaffolded chromosome-length haplotypes for every autosome of NA12878. The completeness of the genome-wide haplotypes measured for the largest haplotype block reached 95.7% and 69.1% using PacBio and Illumina reads, respectively (\todo{fig}4a, i). We further demonstrated the high accuracy of these haplotypes on the local and global scales, which showed low switch (<0.45%) and Hamming error (<0.99%) rates for both the PacBio and Illumina combination (\todo{fig}4a, i,ii). Whereas scaffolding the 10X Genomic haplotypes produced the most accurate local haplotypes (switch error rate of 0.05%), global performance suffered, and the highest Hamming error rate (2.18%) was calculated for this combination. Nevertheless, using Strand-seq to scaffold any of the datasets remarkably improved the completeness, contiguity and accuracy of phasing for each chromosome, highlighting our integrative phasing strategy as a robust method for building dense and accurate whole genome haplotypes (see Data Availability to access phased SNVs for NA12878 using integrative phasing approach presented in this study.). 

% \section{METHODS}
% 
% \subsection{Integrative phasing using WhatsHap}
% 	As an input for integrative phasing, Strand-seq haplotypes were phased using StrandPhaseR (exported in VCF format) and combined with either PacBio or Illumina alignments (both stored in BAM format) or 10X Genomics pre-phased haplotype segments (stored in the VCF produced by LongRanger) to phase heterozygous variants obtained from Illumina platinum genomes (see Data
% Availability Access). We achieved this integrative phasing across platforms by solving the weighted minimum error correction (wMEC) problem using WhatsHap19,20.
% 	Mathematically, aligned reads from Illumina or PacBio (or pre-phased 10X Genomics haplotype segments) and sparse Strand-seq haplotypes are jointly represented in the form of a fragment matrix, where each  row represent either one reads (in case of Illumina and PacBio), one pre-phased haplotype segment (in case of 10X Genomics) or one sparse global haplotype (in case of StrandSeq data) and columns represent the variant sites (\todo{fig}2). The matrix is filled with 0, 1 and ‘-’ entries, where 0 and 1 indicate that the corresponding read supports the reference or alternative allele, respectively,  and ‘-’ means the information is missing (e.g. because a read does not cover this variant site). WhatsHap selects a subset of rows and solves the wMEC problem optimally on these rows, as described earlier20. The result is a maximum likelihood bipartition of rows, which corresponds to the two sought haplotypes.
% 	For all analyses, whatshap was provided with a reference genome (option --reference) to enable re-alignment-based allele detection when constructing the fragment matrix from sequencing reads. This has been shown to significantly improve performance for PacBio reads20.
% 
% \section{Quality metrics of assembled haplotypes}
% 	To assess the quality of assembled haplotypes in this study, we calculated different metrics described in the following.
% Completeness: The process of haplotyping establishes phase relations between pairs of consecutive heterozygous variants. We call each such pair a 'phase connection'. For each haplotype segment produced by a (combination of) technologies, we therefore count the number of phase connections, which is equal to the number of heterozygous markers that make part of such a haplotype segment minus one. To measure the completeness of a phasing, we sum the number of phase connections across all haplotype segments and divide by the maximum possible number of phase connections, which is equal to the number of heterozygous variants on a chromosome minus one.
% Switch error rate: The switch error rate is the fraction of phase connections for which the phasing between the two involved heterozygous variants is wrong (Supplementary \todo{fig}S3a).
% Largest haplotype segment: In this study we are interested in haplotypes that span the whole length of all chromosomes. To measure the completeness of phasing, we report the fraction of heterozygous variants that are part of the largest haplotype segment.
% Largest haplotype segment Hamming rate: To assess whether haplotypes are correct also over long genomic distances, we only consider the largest haplotype segment and compute the Hamming distance between true and predicted haplotypes (Supplementary \todo{fig}S3b), divided by the total number of heterozygous variants in this haplotype segment. That is,the Hamming error rate is equal to the fraction of wrongly phased heterozygous variants. Note that, only one switch error (e.g. in the middle of a chromosome) can result  into a very high Hamming distance and hence the Hamming distance is a much more stringent quality measure. While the switch error rate assesses whether haplotypes are correct locally, i.e. between pairs of neighboring heterozygous variants, the Hamming distance assesses whether haplotypes are correct globally.





 \chapter{Parameterized algorithm for phasing pedigrees}
Read-based phasing deduces the haplotypes of an individual from sequencing reads that cover multiple variants, while genetic phasing takes only genotypes as input and applies the rules of Mendelian inheritance to infer haplotypes within a pedigree of individuals.
Combining both into an approach that uses these two independent sources of information -- reads and pedigree -- has the potential to deliver results better than each individually.\\

In this chapter, we provide a theoretical framework combining read-based phasing with genetic haplotyping, and describe a fixed-parameter algorithm and its implementation for finding an optimal solution.
We show that leveraging reads of related individuals jointly in this way yields more phased variants and at a higher accuracy than when phased separately, both in simulated and real data.
Coverages as low as 2$\times$ for each member of a trio yield haplotypes that are as accurate as when analyzed separately at 15$\times$ coverage per individual.

\section{Introduction}\label{sec:intro}
With sequencing cost decreasing at an exponential rate, it has now become possible to sequence pedigree of genomes.
As a result, the sequencing datasets from pedigrees are becoming publicly available.
Thus to come up with an algorithm, that solves the haplotyping problem for pedigrees in a joint framework by considering both information sources, from sequencing reads and principles of Mendelian inheritance is very important.

Over the last few years, there has been considerable efforts devoted to phasing using reference panels, phasing pedigrees using genotype data and molecular haplotyping using NGS data.
Broadly, the methods for phasing are classified into three classes. 
First, haplotypes can be inferred from genotype information of large cohorts based on the idea that common ancestry gives rise to shared haplotype tracts, as reviewed by \cite{Browning2011, loh2016fast, loh2016reference}.
This approach is known as \emph{statistical} or \emph{population-based phasing}.
The idea is to explain the genotypes of the target genome by finding the maximum likelihood paths from the reference panel (large set of haplotypes from the population).
It can be applied to unrelated individuals and only requires genotype data, which can be measured at low cost.
While very powerful for common variants, this technique is less accurate for phasing rare variants and cannot be applied at all to private or \textit{de novo} variants.
%
Second, haplotypes can be determined based on genotype data of related individuals, known as \emph{genetic haplotyping} \citep{Glusman2014}.
To solve the phasing problem, one seeks to explain the observed genotypes under the constraints imposed by the Mendelian laws of inheritance, while being parsimonious in terms of recombination events.
For larger pedigrees, such as parents with many children, this approach yields highly accurate phasing \citep{Roach2011, abecasis2002merlin, williams2010rapid}.
On the other hand, it is less accurate for single mother-father-child trios and has the intrinsic limitation of not being able to phase variants that are heterozygous in all individuals.
%
Third, the sequences of the two haplotypes can be determined experimentally, called \emph{molecular haplotyping}.
Many techniques do not resolve the full-length haplotypes but yield blocks of varying sizes.
Approaches furthermore largely differ in the amount of work, DNA, and money they require.
As discussed in Chapter~\ref{ref:chp1}, on one end of the scale, next-generation sequencing (NGS) instruments generate local phase information of the length of a sequenced fragment at ever-decreasing costs.
On the other end, upcoming long-read technologies such as Pacific Biosciences and Oxford Nanopore technologies produce long reads in the order of magnitude of kilo-bases.
The sequencing cost of these technologies is decreasing at an exponential rate.
There is a lot of literature on several computational approaches that utilize data from these technologies to produce haplotypes \citep{rhee2016survey}.
As discussed in Chapter~\ref{ref:chp2}, WhatsHap is an efficient parameterized approach for phasing a single individual.
It basically uses long read data from PacBio technology and performs the Minimum Error Correction of reads in a haplotype aware manner, to produce two haplotypes.

Another sequencing approach consists in breaking both homologous chromosomes into (larger) fragments and 
separating them into a number of pools such that each pool is unlikely to contain fragments from the same locus of both haplotypes.
This can, for instance, be achieved by dilution followed by bar-coded short-read sequencing.
To achieve molecular haplotyping over the range of a full chromosome, 
protocols have been invented to physically separate the two homologous chromosomes, for example by microscopy-based chromosome isolation, fluorescence-activated sorting, or microfluidics-based sorting.
These and other experimental techniques for molecular haplotyping have been surveyed by \cite{Snyder2015}.
They are of great interest because they facilitate phasing of rare variants for single individuals.
Rare variants have been postulated to contribute considerably to clinical traits and are hence of major interest.

\paragraph{Hybrid Approaches.}
The ideas underlying population-based phasing, genetic haplotyping, and read-based phasing have been combined in many ways to create hybrid methods.
\cite{Delaneau2013a}, for instance, use local phase information provided by sequencing reads to enhance their population-based phasing approach SHAPEIT.
Exploiting pedigree information for statistical phasing has also been demonstrated to significantly improve the inferred haplotypes \citep{Marchini2006,CDW13_exact}.
Using their heuristic read-based phasing approach HapCompass, \cite{Aguiar2013} note that combining reads from parent-offspring duos increases performance in regions that are identical by descent (IBD). 
Beyond this approach, we are not aware of prior work to leverage family information towards read-based phasing.
% \paragraph{Haplotype Assembly.}
% When many haplotype fragments are available for one individual, for instance from sequencing, one can attempt to reconstruct the full haplotypes or at least to obtain larger blocks.
% This process is known as \emph{haplotype assembly}, \emph{single-individual haplotyping}, or \emph{read-based phasing} (in case the fragments indeed stem from sequencing reads).
% It requires reads that span two or more heterozygous variants.
% In order to be successful, reads covering as many pairs of consecutive heterozygous variants as possible are desirable.
% At present, third generation sequencing platforms, as marketed by Pacific Biosciences (PacBio) and Oxford Nanopore, become more widespread and offer reads spanning thousands to tens of thousands of nucleotides.
% Although error-rates are much higher than for common second generation technologies, the longer reads provide substantially more phase information and hence render them promising platforms for read-based phasing.
% 
% To formalize the haplotype assembly problem in the face of errors, we define \emph{operations} on the input matrix and ask for the minimum number of operations one needs to apply to render it feasible.
% Different such operations have been studied, in particular removal of rows, resulting in the \emph{Minimum Fragment Removal (MFR)} problem, removal of columns, resulting in the \emph{Minimum SNP Removal (MSR)} problem, and flipping of bits, resulting in the \emph{Minimum Error Correction (MEC)} problem.
% All three problems are NP-hard \citep{Lancia2001,Cilibrasi2007}.
% Flipping of bits corresponds to correcting sequencing errors and hence the MEC problem has received most attention in the literature and is most relevant in practice.
% A wealth of exact and heuristic approaches to solve the MEC problem exists.
% Exact approaches, which solve the problem optimally, include integer linear programming \citep{Fouilhoux2012,CDW13_exact}, and fixed-parameter tractable (FPT) algorithms \citep{he2010optimal,Patterson2015,Pirola2015}.
% Refer to the reviews by \cite{Schwartz2010} and \cite{Rhee2015} for further related approaches.

% \begin{figure}[t!]\centering
% \includegraphics[width=\columnwidth]{{Marschall.152.fig.1}.pdf}
% \caption{Seven SNP loci covered by reads (horizontal bars) in three individuals.
% Unphased genotypes are indicated by labels 0/0, 0/1, and 1/1.
% The alleles that a read supports are printed in white.}
% \label{fig:ex_pedigree}
% \end{figure}
\paragraph{Contributions.}
Here, we build upon our previous approach WhatsHap \citep{Patterson2014,Patterson2015} and generalize it to jointly handle sequencing reads of related individuals.
WhatsHap is an FPT approach that solves the (weighted) MEC problem optimally, exponentially in time with maximum coverage, but linearly in the number of variants.
In particular the run-time does not explicitly depend on the read length.
These properties make it particularly apt for current long-read data. 
This has also been observed by \citet{Kuleshov2014b}, who approached the weighted MEC problem in a message-passing framework and, by doing so, independently arrived at the same DP algorithm used in WhatsHap.
The exponential run-time in the maximum coverage does not constitute a problem in practice because reads can be removed in regions of excess coverage without loosing much information.
The evaluation by \cite{Patterson2015} suggests that pruning data to a maximum coverage of 15$\times$ yields excellent results while an even higher coverage does not deliver a significant additional improvement.

Here, we introduce a unifying formal framework to fully integrate read-based and genetic haplotyping.
To this end, we define the Weighted Minimum Error-Correction on Pedigrees Problem, termed PedMEC, which generalizes the (weighted) MEC problem and accounts for Mendelian inheritance and recombination.
This problem is NP-hard.
We generalize the WhatsHap algorithm for solving this problem optimally and thereby show that PedMEC is fixed-parameter tractable.
When the maximum coverage is bounded, the run-time of our algorithm is linear in the number of variants and does not explicitly depend on the read length, hence inheriting the favorable properties of WhatsHap.

We target an application scenario where related individuals are sequenced using error-prone long-read technologies such as PacBio sequencing.
As a driving question motivating this research, we ask how much coverage is needed for resolving haplotypes in related individuals as opposed to single or unrelated individuals.
Our focus is on phasing and we do not consider the genotyping step, which can either be done from the same data or from orthogonal and potentially cheaper data sources such as micro-arrays or short-read sequencing.
On simulated and real PacBio data, we show that sequencing each individual in a mother-father-child trio to 5$\times$ coverage is sufficient to establish a high-quality phasing.
This is in stark contrast to state-of-the-art single-individual read-based phasing, which yields worse results even for 15$\times$ coverage with respect to both error rates and numbers of phased variants.
We furthermore demonstrate that our technique also exhibits favorable properties of genetic haplotyping approaches:
Because of genotype relationships between related individuals, we are able to infer correct phases even \emph{between} haplotype blocks that are \emph{not connected} by any sequencing reads in any of the individuals.

\begin{table*}
\caption{Overview of common notation.}\label{tab:notation}
\begin{tabular}{l@{\hspace{1em}}l@{\hspace{1em}}l}
\hline
Notation & Meaning & Example \\\hline
$\mathcal{I}$ & Set of individuals & $\{1,2,3,4\}$ \\
$\mathcal{T}$ & Set of trio relationships & $\{(1,2,3),(1,2,4)\}$\\[.5em]
$\mathcal{F}_i\in\{0,1,-\}^{R_i\times M}$ & Input SNP matrix for individual $i\in\mathcal{I}$ & $\left[ \begin{array}{p{1em}p{1em}p{1em}p{1em}p{1em}}
- & - & 1 & 0 & 1 \\
0 & 1 & 1 & 1 & - \end{array} \right]$\vspace{.5em}\\
$\mathcal{W}_i\in\N^{R_i\times M}$ & Matrix of weights for individual $i\in\mathcal{I}$ & $\left[ \begin{array}{p{1em}p{1em}p{1em}p{1em}p{1em}}
0 & 0 & 10 & 21 & 7 \\
13 & 9 & 31 & 25 & 0 \end{array} \right]$\vspace{.5em}\\ 
$\mathcal{X}\in\N^M$  & Recombination cost vector & $(5,20,12,23,11)$\\
$g_i\in\{0,1,2\}^M$ & Input genotypes for individual $i$ & $(0,2,2,1,1)$ \\\hline
$A(k)$ & Set of reads active in column $k$ & $\big\{[\,\text{-}\ \text{-}\ 1\ 0\ 1], [0\ 1\ 1\ 1\ \text{-}\,]\big\}$ \\ 
$\Delta_C(k,B,t)$ & Local cost for column $k$, bipartition $B$, & 10 \\
 & and transmission tuple $t$ & \\
$C(k,B,t)$ & DP table entry for column $k$, bipartition $B$, & 37 \\
 & and transmission tuple $t$ & \\\hline
$h_i^0, h_i^1\in\{0,1\}^M$ & Sought haplotypes for individual $i$ & $(0,1,1,1,0), (0,1,1,0,1)$ \\
$t_{m\to c},t_{f\to c}\in\{0,1\}^M$ & Sought transmission vectors for trio $(m,f,c)\in\mathcal{T}$ & $(0,0,0,1,1)$\\
\hline
\end{tabular}
\end{table*}

\section{The Weighted Minimum Error Correction Problem on Pedigrees}
As discussed in Chapter~\ref{ref:chp1}, read-based phasing has predominantly been formulated as the Minimum Error Correction (MEC) problem \citep{Cilibrasi2007} and its weighted sibling wMEC \citep{Greenberg2004}.

In this section, we present a novel formulation for jointly phasing individuals in a pedigree.
To this end, we generalize wMEC (see Problem~\ref{prob:wmec}) to account for multiple individuals in a pedigree simultaneously while modeling inheritance and recombination.
An overview of notation we use is provided in Table~\ref{tab:notation}.
We assume our pedigree to contain a set of $N$ individuals $\mathcal{I}=\{1,\ldots,N\}$.
Relationships between individuals are given as a set of (ordered) mother-father-child triples $\mathcal{T}$.
For example, if $\mathcal{I}=\{1,2,3,4\}$, then $\mathcal{T}=\big\{(1,2,3),(1,2,4)\big\}$ corresponds to a pedigree where individuals $1$ and $2$ are the parents of individuals $3$ and $4$.
% Refer to Figure~\ref{fig:notation} for an illustration.
We only consider non-degenerate cases without circular relationships and where each individual appears as a child in at most one triple.
Furthermore, we assume all considered variants to be non-overlapping and bi-allelic.
Each individual $i$ comes with a \emph{genotype vector} $g_i\in\{0,1,2\}^M$, giving the genotypes of all $M$ variants.
Genotypes $0$, $1$, and $2$ correspond to being homozygous in the reference allele, heterozygous, and homozygous in the alternative allele, respectively.
In the context of phasing, we can restrict ourselves to the set of variants that are heterozygous in at least one of the individuals, that is, to variants $k$ such that $g_i(k)=1$ for at least one individual $i\in\mathcal{I}$.
For each individual $i\in \mathcal{I}$, a number of $R_i$ aligned sequencing reads is provided as input, giving rise to one SNP matrix $\mathcal{F}_i\in\{0,1,-\}^{R_i\times M}$ and one weight matrix $\mathcal{W}_i\in\N^{R_i\times M}$ per individual.
We seek to compute two haplotypes $h^0_i,h^1_i\in\{0,1\}^M$ for all individuals $i\in\mathcal{I}$.
As before in the MEC problem, we want these haplotypes to be consistent with the sequencing reads.

In addition, we want the haplotypes to respect the constraints given by the pedigree.
Recall that in each parent, the two homologous chromosomes recombine during meiosis to give rise to a haploid gamete that is passed on to the offspring.
Therefore, each haplotype of a child should be representable as a mosaic of the two haplotypes of the respective parent with few recombination events.
To control the number of recombination events, we assume a per-site recombination cost of $\mathcal{X}(k)$ to be provided as input.
Controlling the recombination cost per site is important because it is not equally likely to happen at all points along a chromosome.
Instead, \emph{recombination hotspots} exist, where recombination is much more likely to occur (and should hence be penalized less strongly in our model).
The cost $\mathcal{X}(k)$ should be interpreted as the (phred-scaled) probability that a recombination event occurs between variant $k-1$ and variant $k$.
To formalize the inheritance process, we define \emph{transmission vectors} $t_{m\to c}, t_{f\to c}\in \{0,1\}^M$ for each triple $(m,f,c)\in T$.
The values $t_{m\to c}(k)$ and $t_{f\to c}(k)$ tell which allele at site $k$ is transmitted by mother and father, respectively.
The haplotypes we seek to compute have to be \emph{compatible} with transmission vectors, defined formally as follows.

\begin{definition}[Transmission vector compatibility]
For a given trio $(m,f,c)\in\mathcal{T}$, the haplotypes $h^0_m,h^1_m,h^0_f,h^1_f,h^0_c,h^1_c\in\{0,1\}^M$ are \emph{compatible} with the transmission vectors $t_{m\to c}, t_{f\to c}\in \{0,1\}^M$ if
\[h^0_c(k)=
 \begin{cases}
  h^0_m(k) & \text{if}\quad t_{m\to c}(k)=0 \\
  h^1_m(k) & \text{if}\quad t_{m\to c}(k)=1 \\
 \end{cases}
\]
and
\[h^1_c(k)=
 \begin{cases}
  h^0_f(k) & \text{if}\quad t_{f\to c}(k)=0 \\
  h^1_f(k) & \text{if}\quad t_{f\to c}(k)=1 \\
 \end{cases}
\]
for all $k\in\{1,\ldots,M\}$.
\end{definition}
With this notion of transmission vectors, recombination events are characterized by changes in the transmission vector, that is, by positions $k$ with $t_{m\to c}(k-1)\neq t_{m\to c}(k)$ or $t_{f\to c}(k-1)\neq t_{f\to c}(k)$.
Given our recombination cost vector $\mathcal{X}$, the cost associated to a transmission vector can be written as follows (in slight abuse of notation).

\begin{definition}[Transmission cost]
For a transmission vector $t_{p\to c}\in\{0,1\}^M$ with $p\in\{m,f\}$ and a recombination cost vector $\mathcal{X}\in\N^M$, the \emph{cost} of $t_{p\to c}$ is defined as
\[
  \mathcal{X}(t_{p\to c}) := \sum_{k=2}^M\iverl t_{p\to c}(k-1)\neq t_{p\to c}(k)\iverr\cdot\mathcal{X}(k),
\]
where $\iverl S \iverr=1$ if statement $S$ is true and $0$ otherwise.
\end{definition}

To state the problem of jointly phasing all individuals in $\mathcal{I}$ formally, it is instrumental to consider the set of matrix entries to be flipped explicitly.
We will therefore introduce a set of index pairs $E_i \subset \{1,\ldots,R_i\}\times\{1,\ldots,M\}$ where $(j,k)\in E_i$ if and only if the bit in row $j$ and column $k$ of matrix $\mathcal{F}_i$ is to be flipped.
\begin{figure}[t!]\centering
\includegraphics[width=\columnwidth]{{pedmec4}.pdf}
\caption{Example shows the input instance and cheapest solution and the resultant haplotypes.}
\label{fig:pedmec4}
\end{figure}
\begin{problem}[Weighted Minimum Error Correction on Pedigrees, PedMEC]\label{prob:pedmec}
Let a set of individuals $\mathcal{I}=\{1,\ldots,N\}$, a set of relationships $\mathcal{T}$ on $\mathcal{I}$, recombination costs $\mathcal{X}\in\N^M$, and, for each individual $i\in I$, a sequencing read matrix $\mathcal{F}_i\in\{0,1,-\}^{R_i\times M}$ and corresponding weights $\mathcal{W}_i\in\N^{R_i\times M}$ be given.
Determine a set of matrix entries to be flipped $E_i \subset \{1,\ldots,R_i\}\times\{1,\ldots,M\}$ to make $\mathcal{F}_i$ feasible and two corresponding haplotypes $h_i^0, h_i^1\in\{0,1\}^M$ for each individual $i\in\mathcal{I}$ as well as two transmission vectors $t_{m\to c},t_{f\to c}\in\{0,1\}^M$ for each trio $(m,f,c)\in\mathcal{T}$ such that
\[\sum_{i\in\mathcal{I}}\sum_{(j,k)\in E_i}\mathcal{W}_i(j,k)+\sum_{(m,f,c)\in\mathcal{T}}\mathcal{X}(t_{m\to c}) + \mathcal{X}(t_{f\to c})\]
takes a minimum, subject to the constraints that all haplotypes are compatible with the corresponding transmission vectors, if existing.
\end{problem}

Note that for the special case of $\mathcal{I}=\{1\}$ and $\mathcal{T}=\emptyset$, PedMEC is identical to wMEC.
Therefore, the PedMEC problem is also NP-hard.
As discussed in Section~\ref{sec:intro}, we are specifically interested in an application scenario were the genotypes are already known.
By using genotype data, we aim to most beneficially combine the merits of genetic haplotyping and read-based haplotyping.
We therefore extend the PedMEC problem to incorporate genotypes and term the resulting problem PedMEC-G.
\begin{problem}[PedMEC with genotypes, PedMEC-G]
Let the same input be given as for Problem~\ref{prob:pedmec} (PedMEC) and, additionally, a genotype vector $g_i\in\{0,1,2\}^M$ for each individual $i\in\mathcal{I}$. Solve the PedMEC problem under the additional constraints that $h^0_i+h^1_i=g_i$ for all $i\in\mathcal{I}$, where ``$+$'' refers to a component-wise addition of vectors.
\end{problem}

For the classical MEC problem, additionally assuming that all sites to be phased are heterozygous is common \citep{CDW13_exact}.
This variant of the MEC problem is a special case of PedMEC-G with $\mathcal{I}=\{1\}$ and $\mathcal{T}=\emptyset$ and $g_1(k)=1$ for all $k$.
\begin{figure}[t!]\centering
\includegraphics[width=\columnwidth]{{pedmeccost1}.pdf}
\caption{Example shows bipartition cost for a particular transmission 00 at column one.}
\label{fig:pedmeccost1}
\end{figure}
\section{Example on PedMEC}
Let us take an example to illustrate on how to solve the PedMEC instance shown in Figure~\ref{fig:pedmec4}.
Consider a trio with three individuals, mother, father and child. As input, the sequencing reads of each individual are represented as SNP matrices $\mathcal{F}_i$
Also, given corresponding weight matrices $\mathcal{W}_i$ that represent the likelihood of sequencing errors in each entry of matrices.
Also, we have a recombination vector $\mathcal{X}$, representing the likelihood of recombination between two consecutive variants.
The sequencing errors present in reads create conflicts in these matrices. 
The objective here is to generate the conflict-free matrices such that we obey Mendelian laws of inheritance.
This objective is achieved by jointly minimizing the flipping cost of set of entries over all the individuals and, additionally 
allowing minimum recombination events in mother, father or both, for every trio relationships.
As an output, we obtain two haplotypes for each individual in a trio.

In this small example, it is easy to compute that the entries marked in red boxes create conflict in matrices. 
Therefore, the cheapest solution is to flip these bits by paying a cost of 3 and 5, plus allowing one recombination event in mother for a cost of 22.
In this way, when we get conflict matrices for an each individual, we can assemble an each individual separately to generate two haplotypes for them.

Let us see on how we can solve this PedMEC instance using a dynamic programming based approach. 
Let us assume there are two haplotypes marked in green and purple for mother, and brown and blue for father shown in Figure~\ref{fig:pedmeccost1}.
The haplotypes for child can be determined based on the haplotypes transmitted from mother and father.
% We now explain the algorithm to solve PedMEC instances to generate two haplotypes for each individual.

In a DP, we go column wise from left and right. 
Let us consider first column and compute DP cell value for partitions shown in Figure~\ref{fig:pedmeccost1} and transmission value 00; 
the least significant bit 0 represents the haplotype green is transmitted to child, otherwise purple; and the most significant bit 0 tells that the
brown haplotype from father is transmitted to child, otherwise blue. 
To compute the minimum allele assignment for each partition, we try different possibilities of alleles assignments.
For example, first, we flip all the entries in the partition to 1; and pay cost based on weights for the corresponding entries from weight matrices and, second we try to flip entries to 0 and pay cost accordingly.
For the green partition, we pay cost 0 if we flip all the entries in the partition to 0 and otherwise 15+7+32+27 = 81.
Similarly, we compute the allele assignment costs for other partitions too.
We further take the minimum allele assignment for each partition and add the costs from all partitions, which results in cost 5 for this example.
Let us consider a different transmission value 01. For this transmission value, the haplotype purple is transmitted to child.
Accordingly, the partitions change and their corresponding allele assignment costs. For the green partition, which now does not span across individuals, cost 0 if we flip bits to 0 and otherwise cost 22.
As before, we compute the minimum allele assignment cost for each partition and resultant cost is 37 for this example.
% \begin{figure}[t!]\centering
% \includegraphics[width=\columnwidth]{{pedmeccost2}.pdf}
% \caption{Example shows biparition cost for a particular transmission 01 at column one.}
% \label{fig:pedmeccost2}
% \end{figure}
\begin{figure}[t!]\centering
\includegraphics[width=\columnwidth]{{pedmeccost4}.pdf}
\caption{Example shows bipartition cost for a particular transmission 00 at column two, given DP column for column one.}
\label{fig:pedmeccost4}
\end{figure}
Similarly, we can compute partitions cost for all possible partitions by considering different transmission value and store them in DP column.

Let us compute the partition cost for column two shown in Figure~\ref{fig:pedmeccost4}, given the DP column for column one.
In Figure~\ref{fig:pedmeccost4}, shown are partitions with transmission value 00, we compute the initialization cost as we computed before in column one. 
For example, for green partition, we compute the initialization cost (=3) as before, but we now additionally consider different possibilities of recombination events between two consecutive columns.
Therefore, we pay additional cost 91 if the DP cell at column one has transmission values 01 or 10, by allowing one recombination event for both cases. We do not pay any recombination cost without any recombination event.
In this way, when we compute the DP cell cost, we try all possibilities of recombination events (00, 01, 10, 11) with the previous cells and then take the minimum cost. 
Additionally, we consider the recursion cost from the column one such that the partitions are consistent.
For this example partition, it is easy to see that the total cost is 8.
Similarly, we compute partition cost for other partitions by trying all transmissions values and store them in DP column two.

We recurse this process till the last column. Once we know the optimal partitions at last column, we can finally backtrace to get the haplotypes for each individual.

With this example, we are now ready to provide a formal dynamic programming algorithm.

\section{Algorithm}\label{sec:algorithm}
\paragraph{Algorithm Overview: Solving PedMEC and PedMEC-G.}
In the following, we will see how the ideas of WhatsHap algorithm can be extended for solving PedMEC and PedMEC-G.
The basic idea is to use the same technique on the union of the sets of active reads across all individuals $i\in\mathcal{I}$, while adding some extra book-keeping to satisfy the additional constraints imposed by pedigree and genotypes.
Let $A_i(k)$ be the set of active reads in column $k$ of $\mathcal{F}_i$.
We now define $A(k)=\bigcup_{i\in\mathcal{I}}A_i(k)$.
% For convenience, we write $i(r)$ to denote the individual where read $r\in A(k)$ originated from, i.e.\ $r\in A_{i(r)}(k)$ for all $r\in A(k)$.
A bipartition $B=(P,Q)$ of $A(k)$ now induces bipartitions for each individual: $B_i=\big(P\cap A_i(k), Q\cap A_i(k)\big)$.

As before, we consider all bipartitions of $A(k)$ for each column $k$, but now additionally distinguish between all possible transmission values.
We assume the set of trio relationships $\mathcal{T}$ to be (arbitrarily) ordered and use a tuple $t\in\{0,1\}^{2|\mathcal{T}|}$ to specify an assignment of transmission values.
Such an assignment $t$ can later (during backtracing) be translated into the sought transmission vectors:
Assuming $t$ to be an optimal such tuple at column $k$, its relation to the transmission vectors is given by
\[t=\big(t_{m_1\to c_1}(k),t_{f_1\to c_1}(k),t_{m_2\to c_2}(k),t_{f_2\to c_2}(k),\ldots\big).\]

The transmission tuples give rise to one additional dimension of our DP table for PedMEC(-G), as compared to the DP table for wMEC.
For each column $k$, we compute table entries $C(k,B,t)$ for all $2^\abs{A(k)}$ bipartitions of reads and all $2^{2|\mathcal{T}|}$ possible transmission tuples, for a total of $2^{\abs{A(k)}+2\abs{\mathcal{T}}}$ entries in this column.

\paragraph{Computing Local Costs.}
Along the lines of \cite{Patterson2015}, we first describe how to compute the cost incurred by flipping matrix entries in each column, denoted by $\Delta_C(k,B,t)$, and then explain how to combine them with entries in $C(k-1,\cdot,\cdot)$ to compute the cost $C(k,B,t)$.
The crucial point for dealing with reads from multiple individuals in a pedigree is to realize that matrix entries from haplotypes that are identical by descent (IBD) need to be identical (or need to be flipped to achieve this).
For unrelated individuals (i.e. $\mathcal{T}=\emptyset$), none of the haplotypes are IBD, giving rise to $2|\mathcal{I}|$ sets of reads for the $2|\mathcal{I}|$ unrelated haplotypes.
These $2|\mathcal{I}|$ sets of reads are given by $B$ and the cost $\Delta_C(k,B,t)$ can be computed by flipping all matrix entries of reads within the same set to the same value.

For a non-empty $\mathcal{T}$, the transmission tuple $t$ tells which parent haplotypes are passed on to which child.
In other words, $t$ identifies each child haplotype to be IBD to a specific parent haplotype.
We can therefore merge the corresponding sets of reads since all reads coming from haplotypes that are IBD need to show the same allele and need to be flipped accordingly.
In total, we obtain $2|\mathcal{I}| - 2|\mathcal{T}|$ sets of reads, since each trio relationship implies merging two pairs of sets.
We write $\mathcal{S}(k,B,t)$ to denote this set of sets of reads induced by bipartition $B$ and transmission tuple $t$ in column $k$.
The cost $W_{k,S}^a$ of flipping all entries in a read set $S\in \mathcal{S}(k,B,t)$ to the same allele $a\in\{0,1\}$ is given by 
\[W_{k,S}^a = \sum_{(i,j)\in S}\iverl \mathcal{F}_i(j,k)\neq a\iverr\cdot\mathcal{W}_i(j,k),\]
where we identify reads in $S$ by a tuple $(i,j)$, telling that it came from individual $i$ and corresponds to row $j$ in $\mathcal{F}_i$.
For PedMEC, i.e.\ if no constraints on genotypes are present, every set $S$ can potentially be flipped to any allele $a\in\{0,1\}$.
Hence, the cost is given by 
\begin{equation}\label{eqn:delta_c}
\Delta_C(k,B,t)= \min_{a\in \{0,1\}^{\mathcal{S}(k,B,t)}}\left\{\sum_{S\in\mathcal{S}(k,B,t)}W_{k,S}^{a(S)}\right\},
\end{equation}
that is, we minimize the sum of costs incurred by each set of reads $S\in\mathcal{S}(k,B,t)$ over all possible assignments of alleles to read sets.
For PedMEC-G, this minimization is constrained to only consider allele assignments consistent with the given genotypes.
To ensure that valid assignments exist, we assume the input genotypes to be free of Mendelian conflicts.

\paragraph{Computing a Column of Local Costs.}
To compute the whole column $\Delta_C(k,\cdot,\cdot)$, we proceed as follows.
In an outer loop, we enumerate all $2^{2\abs{\mathcal{T}}}$ values of the transmission tuple $t$.
For each value of $t$, we perform the following steps:
We start with bi-partition $B=(A(k),\emptyset)$ and compute all $W_{k,S}^a$ for all sets $S\in\mathcal{S}(k,B,t)$ and all $a\in\{0,1\}$, which can be done in $\mathcal{O}(\abs{A(k)}+|\mathcal{I}|)$ time.
Next we enumerate all bipartitions in Gray code order, as done previously \citep{Patterson2015}.
This ensures that only one read is moved from one set to another in each step, facilitating constant time updates of the values $W_{k,S}^a$.
The value of $\Delta_C(k,B,t)$ is then computed from the $W_{k,S}^a$'s according to Equation~\eqref{eqn:delta_c}, which takes $\mathcal{O}(2^{2|\mathcal{I}|}\cdot |\mathcal{I}|)$ time.
Computing the whole column $\Delta_C(k,\cdot,\cdot)$ hence takes 
$\mathcal{O}\big(2^{2\abs{\mathcal{T}}}\cdot(2^{|A(k)|} +  2^{2|\mathcal{I}|}\cdot |\mathcal{I}|)\big)$
time.

\paragraph{DP Initialization.}
The first column of the DP table, $C(1,\cdot,\cdot)$, is initialized by setting $C(1,B,t):=\Delta_C(1,B,t)$ for all bipartitions $B$ and all transmission tuples $t$.

\paragraph{DP Recurrence.}
Recall that $C(k,B,t)$ is the cost of an optimal solution for input matrices restricted to the first $k$ columns under the constraints that the sought bipartition extends $B$ and that transmission happened according to $t$ at site $k$.
Entries in column $C(k+1,\cdot,\cdot)$ should hence add up local costs incurred in column $k+1$ and costs from the previous column. 
To adhere to the semantics of $C(k+1,B,t)$, only entries in column $k$ whose bipartitions are \emph{compatible} with $B$ are to be considered as possible ``predecessors'' of $C(k+1,B,t)$.
\begin{definition}[Bipartition compatibility]
Let $B=(P,Q)$ be a bipartition of $A$ and $B'=(P',Q')$ be a bipartition of $A'$.
We say that $B$ and $B'$ are \emph{compatible}, written $B\simeq B'$, if
$P\cap(A\cap A') = P'\cap(A\cap A')$
and
$Q\cap(A\cap A') = Q'\cap(A\cap A')$.
\end{definition}
Two bipartitions are therefore compatible when they agree on the intersection of the underlying sets.
Besides ensuring that bipartitions are compatible, we need to incur recombination costs in case the transmission tuple $t$ changes from $k$ to $k+1$.
Formally, entries in column $k+1$ are given by
\begin{align}\label{eqn:recurrence}
& C(k+1,B,t)= \Delta_C(k+1,B,t)\\
& \quad + \min_{\substack{B'\in\mathcal{B}(A(k)):B'\simeq B\\t'\in\{0,1\}^{2\abs{\mathcal{T}}}}}\big\{C(k,B',t')+d_H(t,t')\cdot\mathcal{X}(k+1)\big\}, \nonumber
\end{align}
where $\mathcal{B}\big(A(k)\big)$ denotes the set of all bipartitions of $A(k)$ and $d_H$ is the Hamming distance.
The distance $d_H(t,t')$ hence gives the number of changes in transmission vectors and thus the term $d_H(t,t')\cdot\mathcal{X}(k+1)$ gives the recombination cost to be added.

\paragraph{Projection Columns.}
To ease computing $C(k+1,B,t)$ via Equation~\eqref{eqn:recurrence}, we use the same technique described by \cite{Patterson2015} and define intermediate \emph{projection columns} $C^\cap(k,\cdot,\cdot)$.
They can be thought of as being \emph{between} columns $k$ and $k+1$.
Consequently, they are concerned with bipartitions of the intersection of read sets $A(k)\cap A(k+1)$ and hence contain $2^{|A(k)\cap A(k+1)|+2|\mathcal{T}|}$ entries, which are given by
\begin{equation}\label{eqn:proj_col}
C^\cap(k,B',t)=\min_{\mathcal{B}(A(k)):B\simeq B'}\{C(k,B,t)\}.
\end{equation}
These projection columns can be created while computing $C(k,\cdot,\cdot)$ at no extra (asymptotic) run-time.
Using these projection columns, Equation~\eqref{eqn:recurrence} becomes
\begin{align}\label{eqn:recurrence_projcol}
& C(k+1,B,t)= \Delta_C(k+1,B,t)\\
& \quad + \min_{t'\in\{0,1\}^{2\abs{\mathcal{T}}}}\big\{C^\cap(k,B\cap A(k),t')+d_H(t,t')\cdot\mathcal{X}(k+1)\big\}, \nonumber
\end{align}
where $B\cap A(k) := (P\cap A(k), Q\cap A(k))$ for $B=(P,Q)$.
We have therefore reduced the run-time of computing this minimum to $\mathcal{O}(2^{2|\mathcal{T}|})$.

\paragraph{Runtime.}
Computing one column of local costs, $\Delta_C(k,\cdot,\cdot)$, takes $\mathcal{O}\big(2^{2\abs{\mathcal{T}}}\cdot(2^{|A(k)|} +  2^{2|\mathcal{I}|}\cdot |\mathcal{I}|)\big)$ time, as discussed above.
For each entry, we use Equation~\eqref{eqn:recurrence_projcol} to compute the aggregate value of cost incurred in present and past columns.
Over all columns, we achieve a run-time of 
$\mathcal{O}\big(M\cdot \big(2^{2|\mathcal{T}|+c+|\mathcal{I}|}|\mathcal{I}|+2^{4|\mathcal{T}|+c}\big)\big)$, where $c=\max_k\{\abs{A(k)}\}$ is the maximum coverage.

\paragraph{Backtracing.}
An optimal bipartition and transmission vectors can be obtained by recording the indexes of the table entries that gave rise to the minima in equations \eqref{eqn:recurrence_projcol} and \eqref{eqn:proj_col} when filling the DP table and then backtracing starting from the optimal value in the last column.
Optimal haplotypes are subsequently obtained using the bipartition and transmission vectors.


\section{Experimental Setup}
To evaluate the performance of our approach, we considered both real and simulated datasets.

\subsection{Real Data}
The Genome in a Bottle Consortium (GIAB) has characterized seven individuals extensively using eleven different technologies~\citep{giab}.
The data is publicly available.
Here we consider the Ashkenazim trio, consisting of three related individuals: NA24143 (mother), NA24149 (father) and NA24385 (son).
We obtained a consensus genotype call set (NIST\_CallsIn2Technologies\_05182015) provided by GIAB, containing variants called by two independent technologies.
For our benchmark, we consider all bi-allelic SNPs on Chromosome~1 called in all three individuals, amounting to 141,256 in total, and use the provided (unphased) genotypes.

\paragraph{Ground Truth via Statistical Phasing.}
To generate a ground truth phasing for comparison, we used the population-based phasing tool SHAPEITv2-r837~\citep{shapeit} with default parameters.
The program was given the 1000 Genomes reference panel\footnote{\scriptsize{\url{https://mathgen.stats.ox.ac.uk/impute/1000GP_Phase3.tgz}}}, the corresponding genetic map\footnote{\scriptsize{\url{http://www.shapeit.fr/files/genetic_map_b37.tar.gz}}}, and the unphased genotypes as input.
SNPs present in the GIAB call set but absent in the reference panel were discarded, resulting in 140,744 phased SNPs, of which 58,551 were heterozygous in mother, 57,152 in father and 48,023 in child.
We refer to this set of phased SNPs as \emph{ground truth phased variants}.
We emphasize that this phasing is solely based on genotypes and does not use phase information present in the reads in any way and hence is completely independent.
In the following, we refer to the original genotypes from the GIAB call set (without phase information) restricted to this set of SNPs as \emph{ground truth unphased genotypes}, which we use as input for read-based phasing experiments described below.

\paragraph{PacBio Data.} For each individual, we downloaded aligned Pacific Biosciences (PacBio) reads\footnote{\scriptsize{\url{ftp://ftp-trace.ncbi.nlm.nih.gov/giab/ftp/data/AshkenazimTrio/(HG002_NA24385_son|HG003_NA24149_father|HG004_NA24143_mother)/PacBio_MtSinai_NIST/MtSinai_blasr_bam_GRCh37/}}},  
which had an average coverage of 42.3$\times$ in mother, 46.8$\times$ in father and 60.2$\times$ in child, respectively.
The average mapped read length across mother was 8,328 bp, father was 8,471 bp and child was 8,687 bp.
For each individual, we separately downsampled the aligned reads to obtain data sets of 2$\times$, 3$\times$, 4$\times$, 5$\times$, 10$\times$, and 15$\times$ average coverage.

\paragraph{10XGenomics Data.}
The GemCode platform marketed by 10XGenomics uses a barcoding technique followed by pooled short-read sequencing and data analysis through a proprietary software solution to resolve haplotypes.
Data from this platform is available from the GIAB project and represents phase information obtained completely independently from either statistical phasing or PacBio reads.
We downloaded the corresponding files\footnote{\scriptsize{\url{ftp://ftp-trace.ncbi.nlm.nih.gov/giab/ftp/data/AshkenazimTrio/(HG002_NA24385_son|HG003_NA24149_father|HG004_NA24143_mother)/10XGenomics}}} for comparison purposes.

\subsection{Simulated Data}
Despite the high-quality data set provided by GIAB, we sought to complement our experiments by a simulated data set.
While the population-based phasing we use as ground truth is arguably accurate due to a large reference panel and the high-quality genotype data used as input, it is not perfect.
Especially variants with low allele frequency present challenges for population-based phasers.

\paragraph{Virtual Child.}
As basis for our simulation, we use the haplotypes of the two parents from our ground truth phased dataset.
We generated two haplotypes of a virtual child by applying recombination and Mendelian inheritance to the four parent haplotypes.
In reality, recombination events are rare: All of Chromosome~1 spans a genetic distance of approximately 292\,cM, corresponding to 2.9 expected recombination events along the whole chromosome.
To include more recombinations in our simulated data set, we used the same genetic map as above, but multiplied recombination rates by 10.
The recombination sites are sampled according to the probabilities resulting from applying Haldane's mapping function to the genetic distances between two variants.
In line with our expectation, we obtained 26 and 29 recombination sites for mother and father, respectively. 
The resulting child had 41,676 heterozygous variants.

\paragraph{Simulating PacBio Reads.}
We aimed to simulate reads that mimic the characteristics of the real PacBio data set as closely as possible.
For this simulation, we incorporate the variants of each individual into the reference genome (hg19) to generate two true haplotypes for each individual in our triple.
We used the PacBio-specific read simulator pbsim by \cite{pbsim} to generate a 30$\times$ data set for Chromosome~1.
The original GIAB reads were provided to pgsim as a template (via option \texttt{--sample-fastq}) to generate artificial reads with the same length profile.
Next, we aligned the reads to the reference genome using BWA-MEM 0.7.12-r1039 by \cite{bwa} with option \texttt{-x pacbio}.
As before for the real data, the aligned reads for each individual were downsampled separately to obtain data sets of 2$\times$, 3$\times$, 4$\times$, 5$\times$, 10$\times$, and 15$\times$ average coverage.

\subsection{Compared Methods}
Our main goal is to analyze the merits of the PedMEC-G model in comparison to wMEC; in particular with respect to the coverage needed to generate a high-quality phasing.
The algorithms to solve wMEC and PedMEC-G described in Section~\ref{sec:algorithm} have been implemented in the WhatsHap software package\footnote{\scriptsize{\url{https://bitbucket.org/whatshap/whatshap}}}, distributed as Open Source software under the terms of the MIT license.
We emphasize that WhatsHap solves wMEC and PedMEC-G optimally.
Since the focus of this paper is on comparing these two models, we do not include other methods for single-individual haplotyping.
We are not aware of other trio-aware read-based phasing approaches that PedMEC-G could be compared to additionally.

The run-time depends exponentially on the maximum coverage.
Therefore we prune the input data sets to a \emph{target maximum coverage} using the read-selection method introduced by \cite{Fischer2016}, which is implemented as part of WhatsHap.
This target coverage constitutes the only parameter of our method.
For PedMEC-G, we prune the maximum coverage to 5$\times$ for each individual separately.
For wMEC, we report results for 5$\times$ and 15$\times$ target coverage.
The respective experiments are referred to as \mbox{PedMEC-G-5}, wMEC-5, and wMEC-15.
For wMEC, we use the additional ``all heterozygous'' assumption (see Section~\ref{sec:algorithm}), to also give it the advantage of being able to ``trust'' the genotypes, as is the case for PedMEC-G.
Both wMEC and PedMEC-G were provided with the ground truth unphased genotypes for the respective data set.
PedMEC-G was additionally provided with the respective genetic map (original 1000G genetic map for real data and scaled by factor 10 for simulated data).
\begin{figure}%
\centering
\includegraphics[width=\columnwidth,height=.7\columnwidth,keepaspectratio]{{simulated}.pdf}\\
\centering
\includegraphics[width=\columnwidth,height=.7\columnwidth,keepaspectratio]{{pedmec1}.pdf}%
\caption{Simulated data set (top) and real dataset (bottom): phasing error rate ($x$-axis) versus completeness in terms of the fraction of unphased SNPs ($y$-axis) for \mbox{PedMEC-G-5} (solid line), wMEC-5 (dashed line), and wMEC-15 (dotted line).
Average coverage (per individual) of input data is encoded by circles of different sizes.}\label{fig:main-results}%
\end{figure}

As described above, the ground truth phased variants was generated by SHAPEIT with default parameters, implying that SHAPEIT treated the three samples as unrelated individuals.
For comparison purposes, we re-ran SHAPEIT and provided it with pedigree information.
We refer to the resulting phased data set as \emph{SHAPEIT-trio}.
Moreover, we ran duoHMM (v0.1.7) by \cite{OConnell2014} on the resulting files to further improve the phasing.
\begin{figure}[t]
\begin{center}
\includegraphics[height = .6\columnwidth, width=.6\columnwidth]{{Marschall.152.fig.3}.pdf}
\end{center}
\caption{Three-way comparison of phasings provided by SHAPEIT, 10XGenomics, and \mbox{PedMEC-G-5} (on 15$\times$ coverage data). 
Of all pairs of consecutive SNPs phased by all three methods, the percentages of cases where the phasing reported by one method disagrees with the other two are reported. 
Missing to 100\%: cases where all three methods agree.
Left: SHAPEIT run with default parameters, corresponding to our ``ground truth phasing''; right: SHAPEIT run with pedigree information.}\label{fig:threeway}
\end{figure}
\section{Performance Metrics}\label{sec:metrics}
We compare each phased individual to the respective ground truth haplotypes separately and only consider sites heterozygous in this individual.

\paragraph{Phased SNPs.}
For read-based phasing of a single individual (wMEC), we say that two heterozygous SNPs are directly connected if there exists a read covering both.
We compute the connected components in the graph where SNPs are nodes and edges are drawn between directly connected SNPs.
Each connected component is called a \emph{block}.
For read-based phasing of a trio (PedMEC), we draw an edge when two SNPs are connected by a read in any of the three individuals.
In both cases, we count a SNP as being \emph{phased} when it is not the left-most SNP in its block (for the left-most SNP, no phase information with respect to its predecessors exists).
All other SNPs are counted as \emph{unphased}.
Below, we report the average \emph{fraction of unphased SNPs} over all three family members.

\paragraph{Phasing Error Rate.}
For each block, the first predicted haplotype is expressed as a mosaic of the two true haplotypes, minimizing the number of switches.
This minimum is known as the \emph{number of switch errors}.
Note that the second predicted haplotype is exactly the complement of the first one, due to only considering heterozygous sites.
When two switch errors are adjacent, they are subtracted from the number of switch errors and counted as one \emph{flip error}.
The \emph{phasing error rate} is defined as the sum of switch and flip errors divided by the number of phased SNPs.

\paragraph{Three-Way Phasing Comparison.}
To simultaneously compare phasings from three different methods (e.g.\ SHAPEIT, 10XGenomics, and \mbox{PedMEC-G-5}), we proceeded as follows.
For an individual, we considered all pairs of consecutive heterozygous SNPs that have been phased by all three methods.
For each of these pairs, either all three methods agree or two methods agree (since only two possible phases exist).
Below, we discuss the fraction of these different cases in relation to the total number of considered SNP pairs.
\begin{figure}[t]\centering
\includegraphics[width=\columnwidth]{{Marschall.152.fig.4}.pdf}
\caption{Two disjoint unconnected haplotype blocks for which phase information can be inferred from the genotypes.}
\label{fig:genetic_molecular}
\end{figure}
\section{Results}
We report the results of  wMEC-5, wMEC-15, and \mbox{PedMEC-G-5} for both data sets, \emph{real} and \emph{simulated}.
All combinations of the three methods, two data sets, and six different average coverages (2$\times$, 3$\times$, 4$\times$, 5$\times$, 10$\times$ and 15$\times$) were run.
The predicted phasings are compared to the ground truth phasing for the respective data set.
That is, for the real data set, we compare to the population-based phasing produced by SHAPEIT; for the simulated data set, we compare to the true haplotypes that gave rise to the simulated reads.
Figure~\ref{fig:main-results} shows the fraction of unphased SNPs in comparison to the phasing error rate (see Section~\ref{sec:metrics}) for all conducted experiments.
A perfect phasing would be located in the bottom left corner.

\paragraph{The Influence of Coverage.}
Increasing the average coverage is beneficial for phasing.
For all three methods (wMEC-5, wMEC-15, and \mbox{PedMEC-G-5}) and both data sets, the phasing error rate and the fraction of unphased SNPs decrease monotonically when the average coverage is increased, as is clearly visible in Figure~\ref{fig:main-results}.
The effect is much more drastic for wMEC than for PedMEC, however.
Apparently, wMEC needs more coverage to compensate for PacBio's high error rate while PedMEC can resort to exploiting family information to resolve uncertainty.

\paragraph{The Value of Family Information.}
When operating on the same input coverage, \mbox{PedMEC-G-5} clearly outperformed wMEC-5 and even wMEC-15 in all cases tested.
This was true for phasing error rates as well as for the fraction of phased positions.
On the real data set with average coverage 10$\times$, for instance, wMEC-5 and wMEC-15 reached an error rate of 2.9\% and 1.9\%, while it was 0.5\% for \mbox{PedMEC-G-5}.

Most remarkingly, \mbox{PedMEC-G-5} delivers excellent results already for very low coverages.
When working with an average coverage as low as 2$\times$ for each family member, it achieves an error rate of 1.4\% and a fraction of unphased SNPs of 1.8\% (on real data).
In contrast, wMEC-15 needs 15$\times$ average coverage on each individual to reach similar values (1.4\% error rate and 1.3\% unphased SNPs).
When running on 5$\times$ data, \mbox{PedMEC-G-5}'s error rate and fraction of unphased positions decrease to 0.75\% and 0.85\%, respectively.
Therefore, it reaches better results while requiring only a third of the sequencing data, which translates into significantly reduced sequencing costs.

\paragraph{Comparison of Real and Simulated Data.}
When comparing results for simulated and real data, i.e.\ top and bottom plots in Figure~\ref{fig:main-results}, the curves appear similar, with some important difference.
In terms of the fraction of phased SNPs ($y$-axis), results are virtually identical.
This indicates that our simulation pipeline establishes realistic conditions regarding this aspect.
Differences in terms of error rates ($x$-axis) are larger.
In general, error rates in the real data are larger than in the simulated data, which might be partly caused by a too optimistic error model during read simulation.
On the other hand, the population-based phasing used as ground truth for the real data set will most likely also contain errors.
Especially low-frequency variants present difficulties for population-based phasers.
Next, we therefore compare our ground truth statistical phasing to the independent phasing provided by 10XGenomics.

\paragraph{Three-Way Comparison with 10XGenomics.}
Figure~\ref{fig:threeway}(left) shows the results of a three-way comparison of ground truth statistical phasing, 10XGenomics, and \mbox{PedMEC-G-5} on 15$\times$ coverage data.
We observe that the total fraction of cases where there is disagreement between the three methods is below 1\%.
Out of these, 10XGenomics and the statistical phasing agree in a sizeable fraction of cases, shown in blue.
In these cases, the PacBio-based \mbox{PedMEC-G-5} is likely wrong.
Given PacBio's high read error rate, the existence of such cases is not surprising.
On the other hand, there also is a significant fraction of cases where \mbox{PedMEC-G-5} and 10XGenomics agree, shown in red, indicating likely errors in the statistical phasing.
Cases where \mbox{PedMEC-G-5} and the statistical phasing agree but disagree with 10XGenomics are very rare, which is likely due to the low error rates of short-read sequencing underlying the 10XGenomics phasing and the resulting highly accurate phasing.

\paragraph{SHAPEIT-trio and duoHMM.}
Figure~\ref{fig:threeway}(right) shows the same three-way comparison, but uses the results obtained from SHAPEIT when run in trio mode.
We see that this improved the phasing for the child but dramatically worsened the agreement for the parents, with more than 4\% of all phased SNP pairs for which 10XGenomics and \mbox{PedMEC-G-5} agreed but disagreed with SHAPEIT-trio.
Running duoHMM \citep{OConnell2014} to improve the SHAPEIT-trio phasing did not lead to any changes, which might be related to that duoHMM is designed to be run for large cohorts of related individuals.

\paragraph{Phase Information Beyond Block Boundaries.}
Genetic phasing operates on genotypes of a pedigree, without using any sequencing reads.
Figure~\ref{fig:genetic_molecular} illustrates a case where we have two blocks that are not connected by reads in any individual. 
Nonetheless phase information can be inferred from the genotypes: Each block contains a SNP that is homozygous in both parents and heterozygous in the child, which immediately establishes which haplotype is maternal and which is paternal in both blocks.
Note that this, in turn, also implies the phasing of the parents.
By design, PedMEC-G implicitly exploits such information.
To demonstrate this, we used the real data set and merged all blocks reported by PedMEC-G into one chromosome-wide block and determined the fraction of cases where phases were correctly inferred between blocks---and hence between two SNPs that are not connected by reads in any individual.
This resulted in a fraction of 89.7\% correctly inferred phased (averaged over all individuals and coverages; standard deviation 1.4\%).
Repeating the same for wMEC yielded 50.4\% correctly inferred phased, as expected equalling a coin flip.

\paragraph{Runtimes.}
All experiments have been run on a server with two Intel Xeon E5-2670 CPUs (10 cores each) running at 2.5GHz.
The implementation in WhatsHap is sequential, i.e.\ only using one CPU core.
In all cases, the time spent reading the input files dominated the time spent in the phasing routine itself.
Processing all three individuals of the 5$\times$ coverage real data set took 31.1\,min, 31.2\,min, and 26.2\,min for wMEC-5, wMEC-15, and \mbox{PedMEC-G-5}, respectively.
This time included all I/O and further processing.
Of these times, 2.0\,s, 4.1\,s, and 101.0\,s were spent in the phasing routine, respectively.
For input coverage 15$\times$, total processing took 89.3\,min, 93.9\,min, and 65.4\,min for wMEC-5, wMEC-15, and \mbox{PedMEC-G-5}, respectively.
Of this, 2.5\,s, 149.9\,s, 321.5\,s were spent in the phasing routines, respectively.
We conclude that the phasing algorithm presented here is well suited for handling current data sets swiftly.
In the future, we plan to further optimize the implementation of I/O subroutines and provide automatic chromosome-wise parallelization of data processing.

\section{Discussion}
We have presented a unifying framework for integrated read-based and genetic haplotyping.
By generalizing the WhatsHap algorithm \citep{Patterson2015}, we provide a fixed-parameter tractable method for solving the resulting NP-hard optimization problem, which we call \emph{PedMEC}.
When maximum coverage and number of individuals are bounded, the algorithm's run-time is linear in the number of phased variants and independent of the read length, making it well suited for current and future long-read sequencing data.
This is mirrored by the fact that the run-time is dwarfed by the time required for reading the input files in practice.
PedMEC can use any provided costs for correcting errors in reads as well as for recombination events.
By using phred-scaled probabilities as costs, minimizing the cost can be interpreted as finding a maximum likelihood phasing in a statistical model incorporating Mendelian inheritance, read error correction, and recombination.

Testing the implementation on simulated and real trio data, we could show that the method is notably more accurate than phasing individuals separately, especially at low coverages.
Beyond enhanced accuracy, our method is also able to phase a greater fraction of heterozygous variants compared to single-individual phasing.

Being able to phase more variants is a key benefit of the integrative approach.
Whereas read-based phasing can in principle only phase variants connected by a path through the covering reads, adding pedigree information enables even phasing of variants that are not covered in all individuals since the algorithm can ``fall back'' to using genotype information.
Figure~\ref{fig:ex_pedigree} illustrates the increased connectivity while phasing a trio, resulting in more phased variants in practice.

Genetic haplotyping alone cannot phase variants that are heterozygous in all individuals, emphasizing the need for an integrative approach as introduced here.
We demonstrate that such an approach indeed yields better result and recommend its use whenever both reads and pedigree information are available.
Most remarkingly, the presented approach is able to deliver outstanding performance even for coverages as low as 2$\times$ per individual, on par with performance delivered by single-individual haplotyping at 15$\times$ coverage per individual.

The disadvantage of PedMec framework is that we can not phase variants that are heterozygous in all individuals.
Additionally, the variants that are unique to the genome can not be phased due to the reference genome being included in the pipeline.
The next step is to generalize this approach by excluding a reference genome from the pipeline and perform phasing directly from reads.
In the next chapter, we discuss the generalized approach to generate haplotypes directly from reads.

% \paragraph{Future work.}
% We plan to implement phasing of \emph{de novo} variants observed in the child, which would be impossible with pure genetic haplotyping but is straightforward with our approach.
% 
% Since runtime is exponential in the maximum physical coverage, pruning of datasets is required in practice.
% The read selection approach currently implemented in WhatsHap \citep{Fischer2016} aims to retain reads that both cover and connect many variants at the same time, in particular for heterogeneous combinations of datasets such as paired-end or mate-pair reads together with long reads.
% For pedigrees, each dataset is currently pruned individually, but results would likely improve if pedigree structure was taken into account in this step.
% Finally, since we show that it is possible and beneficial to integrate both read-based and genetic phasing, the next obvious question is whether it is possible to modify our unified theoretical framework to one that also includes statistical phasing.






%%% Local Variables:
%%% mode: latex
%%% TeX-master: "main"
%%% End:

 \chapter{A graph-based approach to diploid genome assembly}
Constructing high-quality haplotype-resolved \textit{de novo} assemblies of diploid genomes is important to reveal the full extent of structural variation and its role in health and disease.
Current assembly approaches often put the two sequences into one haploid consensus sequence and, therefore, fail to capture the diploid nature of the organism under study.
Thus, coming up with an assembler which is able to produce accurate and complete diploid assemblies, while being resource-efficient with respect to sequencing costs, 
is a key challenge to be addressed by the bioinformatics community.\\

In this chapter, we present a novel graph-based approach to diploid assembly, which integrates accurate Illumina data and long-read Pacific Biosciences (PacBio) data.
We demonstrate the effectiveness of our method on a pseudo-yeast diploid genome and show that we require as little as 50$\times$ coverage Illumina data 
and 10$\times$ PacBio data to generate accurate and complete assemblies.
Additionally, we show that our approach has the ability to detect and phase structural variants.\\


\section{Introduction}
% Determining the two genome sequences per chromosome of diploid organisms is important.
% Separate determination of the two haplotype sequences can in principle avoid genotyping errors in complex regions of the genome caused by incorrect models of variants at nearby sites being independent.
As introduced in Chapter 1, the process of assembling the two genome sequences from sequencing reads in a haplotype-aware manner is known as \textit{diploid} or \textit{haplotype-aware genome assembly} and the generated assemblies are called ``haplotigs''.
However, the characteristics of next generation sequencing (NGS) reads are variable, short length and have low error rates; therefore, solving the diploid assembly problem is fundamentally challenging.
Additional challenges inherent in the genome assembly problem include dealing with short and long genomic repeats, handling other rearrangements present in the genome, and scaling efficiently considering input size, genome size, and hardware availability.

Across the short, long, and hybrid categories, most current assemblers \citep{sohn2016present, simpson2015theory} require collapsing the two genome sequences of a diploid sample into a single haploid ``consensus'' sequence (or primary contig). 
The consensus sequence is obtained by merging together the distinct alleles at heterozygous regions into a single allele, and therefore losing a lot of information.
The resulting haploid de novo assembly does not represent the true characteristics of the diploid input genome. 

To generate diploid assemblies for heterozygous genomes, there are two standard linear approaches; 
one uses haploid contig sequences \citep{chin2016phased, pendleton2015assembly, seo2016novo, mostovoy2016hybrid}, 
while the other partitions the reads while using the reference genome as a backbone \citep{Glusman2014, martin2016whatshap, chaisson2017multi}.
% \todo{Maybe we should cite the HGSVC preprint. Tobi: I can insert this.}
In both approaches, the reads are first aligned (either to the reference genome or the contigs). Second, variants such as SNVs are detected based on the aligned reads. 
Finally, the detected variants are phased using long reads from a same or different sequencing technology.

For both reference-guided and contig-based assembly, this third step---solving the phasing problem---has been formulated as the minimum error correction (MEC) optimization problem \citep{lippert2002algorithmic,Cilibrasi2007}.
There are several disadvantages to reference-guided assembly; for example, the reads are initially aligned to the reference genome and therefore the process contains a reference bias. 
The sequences or structural variations that are specific to the genome are not detected by using these approaches. 
Thus, we can not find the cause of the disease, which is actually associated with this structural variant in the genome.

However, there are also several reasons why the set of sequences/contigs produced by contig-based assembly is not ideal.
First, the contigs produced by assemblers ignore the heterozygous variants in complex regions, opting instead to break contiguity to express even moderate complexity. 
Second, the contigs do not capture end-to-end information in the genome; the ordering or relationships between contigs are required in order to generate end-to-end chromosomal-length assemblies.
% \todo{TM: I do not understand this sentence. Can you explain / make this more clear?}
% \sgnote{Done.}

Some newer diploid assembly methods include \cite{weisenfeld2017direct}, where 10x Genomics linked read data is used to determine the actual diploid genome sequence. 
Their approach is based on de Bruijn graphs and applies a series of graph simplifications, in which simple bubbles are detected and phased by using (short) reads that stem from the same (long) input molecule, which is determined through barcoding.
There is also a recent study by \cite{chin2016phased}, who follows a linear phasing approach to generate diploid assemblies (\textit{haplotigs}) for diploid genomes by using PacBio reads.

\paragraph{Contributions.}
We propose a graph-based approach to generate haplotype-aware assemblies of single individuals.
Our contribution is twofold. First, we propose a hybrid approach to integrate accurate Illumina and long PacBio reads to generate diploid assemblies. 
The Illumina reads are used to generate an assembly graph that serves as a backbone for subsequent PacBio-based steps.
Second, we generalize the diploid assembly problem to encompass constructing the diploid assembly directly from the underlying assembly graph.
The two genome sequences can be seen as two paths over the regions of heterozygosity in the assembly graph.
We make some first steps towards performing read-based phasing on graphs.

% Phasing using an assembly graph has several advantages over other approaches. For example, it is possible to phase larger blocks at once, because paths in an assembly graph can span multiple variants.
% Moreover, it is easier to detect large structural variants, such as translocations and other rearrangements, in an assembly graph.
% The structural variants are reflected by bubbles in an assembly graph. The bubbles are defined as a set of disjoint paths that share the same start and end nodes. 
% Figure~\ref{fig:ex_sv} illustrates how such bubbles can represent both small variants (SNVs and indels up to 50 base pairs in length) and larger structural variants.
% The graph-based approach provides a way to both accurately detect all types of structural variation and perform further downstream analyses.
% Figure~\ref{fig:ex_graph_approach} shows the conceptual advantage of our graph-based approach over contig-based methods such as Falcon Unzip.
% Consider four SNVs separated by two large SVs and there are four reads spanning these variants.
% Out of those reads, the two reads $r_3$ and $r_4$ span the two SVs, but do not cover any of the two SNVs.
% In this case, Falcon Unzip generates a primary contig that spans from one end to the other, but generates incomplete and fragmented haplotigs (phased primary contigs in the language of Falcon Unzip).
% % Since we are interested in haplotypes, we talk in the language of \textit{haplotigs} in the rest of the article.
% % \todo{Commented out a sentence that disrupted the flow. Better define ``haplotig'' already above?}.
% % \sgnote{I added it in the second para of Introduction.}
% In contrast, our graph-based approach attempts to phase across all types of variation, including SVs, in order to produce end-to-end haplotigs.
% Therefore, the graph-based approaches are powerful to deliver more complete and contiguous haplotigs.
% In Section~\ref{sec:phasing} we discuss in detail how we can phase these bubbles using graph-based approach. 

We demonstrate the feasibility of our approach by performing a haplotype-aware de novo assembly of a whole pseudo-diploid yeast (SK1+Y12) genome.
We show that we generate more accurate, more contiguous, and more correctly phased diploid genomes compared to Falcon Unzip.
Through analysis of different coverage levels, we demonstrate that we require only 50$\times$ short-read coverage and as little as 10$\times$ long-read coverage data to generate diploid assemblies.
This illustrates that our hybrid strategy is a cost-effective way of generating haplotype-resolved assemblies.
% Our hybrid assembler, which operates on both long reads (with low coverage) and short reads (with higher coverage), has the potential to generate high-quality assemblies at reduced cost.
Finally, we show that we successfully detect and phase large structural variants.


% \begin{figure}[t!]\centering
% \includegraphics[width=\columnwidth]{ex_sv.pdf}
% \caption{Based on reads (middle) from the two sequences (top), the bubbles in the graph (bottom) show three different heterozygous variations; the first one is an SNV, the second one is an SV, and the third one is an indel. }
% \label{fig:ex_sv}
% \end{figure}

% \begin{figure}[t!]\centering
% \includegraphics[width=\columnwidth]{ex_graph_approach.pdf}
% \caption{Input: an assembly graph (top) (consisting of four SNVs and two SVs) and the PacBio reads $r_1, r_2, r_3, r_4, r_5, r_6$ (gray). Output: the phased reads (colored in blue and red) and haplotigs (bottom) using Falcon Unzip and our graph-based approach. 
% Our graph-based approach delivers end-to-end haplotigs. Contrarily, Falcon Unzip does not phase the central section, which does not contribute to the total haplotig size.}
% \label{fig:ex_graph_approach}
% \end{figure}
% https://dl.acm.org/citation.cfm?id=322075
% https://www.ncbi.nlm.nih.gov/pmc/articles/PMC5411783/
% http://journals.plos.org/plosone/article?id=10.1371/journal.pone.0019175
\section{Further related work}
The assembly problem was initially formalized as the shortest common super-string problem \citep{maier1978complexity} to generate consensus sequence.
The SCS problem turned out to be NP-hard, thus \cite{tarhio1988greedy} followed the greedy approach to find an approximate solution to solve the assembly problem.
The basic idea of this approach is to combine the two reads that overlap to a maximum extent in terms of length or base quality estimates.
This process is repeated until a predefined minimum quality threshold is reached.

The initial genome assemblers \citep{sutton1995tigr, Green99phrapdocumentation} were based on the greedy strategies and have been used during the Human Genome Project.
Greedy approaches were also explored for datasets generated by second-generation DNA sequencing technologies \citep{warren2006assembling, jeck2007extending}.

The major disadvantage of these greedy approaches is that the shortest common super-string problem neglects an essential feature of complex (e.g.,
vertebrate) genomes: repeats. This has led to the invention of graph-based models for sequence assembly to handle repeats and other complex genomes.
Graph-based models form the basis of most current genome sequence assemblers. 
As introduced in Chapter 1, there are basically two types of graph models such as de Bruijn and OLC-based string graphs.
Over the last few years, a lot of effort has been devoted to improving the graph based assemblers. 

The Edena assembler followed the string graph approach for short-read sequencing data \citep{hernandez2008novo}. To efficiently construct the string graph, several assemblers were developed using the FM index \citep{simpson2010efficient}, 
which led to memory-efficient assemblers for large genomes \citep{li2012exploring, simpson2012efficient}.
Furthermore, several fast string graph construction algorithms were developed \citep{dinh2011memory, gonnella2012readjoiner, ben2014string}.
 
\begin{figure}[t!]\centering
\includegraphics[width=\columnwidth]{{cropped_falcon-unzip}.pdf}
\caption{(a) An initial assembly graph is constructed by FALCON by error-correcting the reads. 
The bubbles are collapsed into a consensus sequence ``primary contig''.
(b) Heterozygous SNPs  are identified and phased, thus haplotype of reads is identified. (c) The phased reads are used to incorporate the haplotype-fused path into the initial assembly graph, thus finally a set of primary contigs and associated haplotigs are generated.
Figure from paper ``Phased diploid genome assembly with single-molecule real-time sequencing''.}
\label{fig:falcon_unzip}
\end{figure}
In the direction of de Bruijn graph, there were also quite a lot of efforts involved.
The ABySS assembler \citep{simpson2009abyss} introduced a representation of
the graph as a hash table of k-mers, with each k-mer storing a byte representing the presence or absence of its eight possible
neighboring k-mers. This representation helped in computational gain by allowing the hash table to be distributed across a cluster
of computers. In recent years, the use of Bloom filters \citep{bloom1970space} has been gaining popularity by representing a set of k-mers.
The Bloom filter was first applied to k-mer counting by \cite{melsted2011efficient}. 
The FM-index data structure \cite{ferragina2000opportunistic}, which was basically designed for mapping reads to the reference genome, was also used for
sequence assembly. 
The FM-index, with slight modifications, has been recently used for representing de Bruijn graphs \citep{bowe2012succinct, rodland2013compact}.
The main point that only a fraction of k-mers need to be directly stored as vertices was observed by \cite{ye2012exploiting}. A similar
technique was used to improve the memory consumption of the popular SOAPdenovo assembler \citep{luo2012soapdenovo2}.

Several algorithms that particularly consider the mate-pair information during assembly have been developed .
 In the works of \citep{butler2008allpaths, bankevich2012spades}, the assembler attempts to enumerate all the paths connecting the endpoints of a mate pair.
The paired de Bruijn graph \citep{medvedev2011paired} approach modifies the de Bruijn graph
structure to implicitly encode the mate-pair information, thereby resolving segments of the graph based on the mate-pair information.
\paragraph{Long-read assemblers}
Although these short-read assemblers can generate consensus sequences to span entire chromosomes, they are very fragmented and lack the finer resolution required to improve contig lengths. 
Instead, the biggest gains in contig lengths stem from single-molecule sequencing, which generates long-read datasets.
These long-read technologies such as PacBio and ONT can generate reads much longer than Illumina. 
More specifically, these reads are long enough to cover the most common repeats in both microbial and vertebrate genomes and can therefore generate highly continuous assemblies. 
Subsequent efforts have been devoted to handling long read lengths and decreasing error rates to generate good quality assemblies.
This includes assembly tools designed specifically for long-read PacBio and Nanopore data: Canu \citep{koren2017canu}, HINGE \citep{kamath2017hinge}, Racon \citep{vaser2017fast}, Falcon \citep{chin2016phased}, and Miniasm \citep{li2016minimap}.

Canu, which is based on Celera Assembler, has been specifically designed for noisy single-molecule sequences.
Canu follows an adaptive overlapping strategy based on tf-idf weighted MinHash \citep{berlin2015assembling} and a sparse assembly graph construction.
It was mainly designed to support long-read data, to be efficient in terms of run-time and coverage requirements, and additionally perform repetition and haplotype separation.
Racon corrects assemblies by finding a consensus sequence between reads and the assembly through the construction of partial order alignment (POA) graphs.
After aligning the reads by a mapper of choice (e.g. Minimap or Graphmap),
Racon segments the sequence and finds the best alignment between a POA graph of the reads and the assembly.
The HGAP (Hierarchical Genome Assembly Process) pipeline
was developed to assemble PacBio-generated data without requiring correction using short-read
data \citep{chin2013nonhybrid}. HGAP is a hierarchical pipeline; it selects the longest PacBio reads to form the basis
of the assembly and error-corrects this subset of reads using the remaining data. The corrected
long reads are then assembled, and a consensus sequence is generated using the complete data
set. This method often generates single-contig assemblies of bacterial genomes.
The FALCON assembler follows the design of the hierarchical genome assembly process (HGAP) but uses more computationally optimized components. It uses string graph algorithm for genome assembly and being best suited for PacBio reads.
It includes several important steps: error correction of raw reads, pre-assembly error correction, overlap filtering, graph construction from overlaps and contig construction from graph.

\paragraph{Hybrid assemblers}
Besides short and long-read assemblers, the hybrid approaches to combine different sequencing datasets provide another avenue for generating accurate, contiguous and complete assemblies. 
The hybrid assemblers, SPAdes by \cite{bankevich2012spades} and ALLPATHS-LG by \cite{gnerre2011high}, are DBG-based, which constructs the base graph using Illumina data and then scaffold the assemblies using longer, less accurate reads such as PacBio. 
Furthermore, SPAdes also supports ONT data as the long read complement to NGS data. 
Recently, \cite{fan2017hysa} demonstrates that the combination of PacBio and Illumina delivers accurate structural variant calls.

\paragraph{Other technologies}
An emerging trend to generate accurate, contiguous and complete assemblies is, to combine cost-effective Illumina sequencing dataset with other sequencing techniques. 
One powerful sequencing technique is chromatin conformation capture via proximity ligation and high-throughput sequencing (Hi-C) \citep{lieberman2009comprehensive}. 
This technique involves in generating a paired-read data type (two reads separated by some distance) but from a distribution of sizes that can span mega bases. 
The main advantage of these data sets is in the scaffolding step to connect contigs, which help in generating chromosome-length scaffolds, and phase haplotypes \citep{burton2013chromosome, selvaraj2013whole}. 
Along the similar lines, \cite{rice2017improved} demonstrates on using in vitro reconstituted chromatin and Illumina sequencing to assemble the American alligator genome. 
Another sequencing technique is the barcoding of short reads to tag groups of ``linked reads'' that all originate from a larger, single molecule of DNA. 
For these synthetic long-read data, \cite{weisenfeld2017direct} introduces a new assembler, Supernova, for the de novo assembly of diploid human genomes. 
Additionally, in the latest version of the ABySS assembler, \cite{jackman2017abyss} explores linked reads and optical mapping for improved scaffolding.
\paragraph{Diploid assemblers}
To date, there is only one diploid assembler, Falcon Unzip. It's pipeline is given in Figure~\ref{fig:falcon_unzip}.
Falcon Unzip first constructs a string graph by using PacBio reads and then collapse the bubbles representing divergent regions between homologous sequences (Fig.~\ref{fig:falcon_unzip}a) to generate sets of ``haplotype-fused contigs'' , also called ``primary contigs''. 
Next, Falcon-Unzip identifies phase of reads using aligned reads over heterozygous SNVs. (Fig.~\ref{fig:falcon_unzip}b). 
The aligned reads provides information on how alleles at heterozygous positions are connected. Based on the aligned reads, Falcon Unzip identifies haplotypes for the genome and phases each read using greedy approach.
Phased reads are then used to assemble haplotigs and primary contigs (Fig.~\ref{fig:falcon_unzip}c) 
that form the final diploid assembly with phased single-nucleotide polymorphisms (SNPs) and structural variants (SVs).
\begin{figure}[t!]\centering
\includegraphics[width=.8\columnwidth]{pipeline.pdf}
\caption{Overview of the diploid assembly pipeline. }
\label{fig:pipeline}
\end{figure}

\textit{Phasing using primary contigs.}
In Falcon Unzip, the reads are aligned to primary contigs and heterozygous SNPs (het-SNPs) are called by considering signal from sequence alignments over these hets.
A simple phasing algorithm is followed to identify phased SNPs using greedy approach. 
Along each contig, the algorithm assigns phasing blocks where ``chained phased SNPs'' can be identified. 
In a contig, the phase to aligned reads to that contig can be assigned unambiguously if read covers sufficient heterozygous SNVs.
Based on the information from primary contig and phased read set, Falcon Unzip assigns a phased tag.
Some reads might not have enough phasing information. For example, if there are not enough het-SNP sites covered by a read, it assigns a special 'un-phased tag' for each un-phased read.
The initial assembly graph is fused using phased reads and the haplotigs are generated in a greedy manner using local conservative approach.
% \todo{maybe add example how haplotigs from haplotype fused assembly graph works?}

\section{Diploid assembly pipeline}
Our assembly workflow uses short read (e.g.\ Illumina) and long read (e.g.\ PacBio) data combinedly, as illustrated in Figure~\ref{fig:pipeline}, and we describe the details of this process in the following.

\subsection{Sequence graph} 
Our first step is to construct a sequence graph using short read data with a low error rate, as provided by the Illumina platform.
% The diploid assembly problem can be modelled using a sequence graph (defined below). We construct the underlying sequence graph using Illumina data.
\begin{definition}[Sequence Graph]
We define a sequence graph $G_s (N_s, E_s)$ as a bidirected graph, consisting of a set of nodes $N_s$ and a set of edges $E_s$.
The nodes $n_i$ are sequences over an alphabet $\mathcal{A} = \{A,C,G,T\}$.
For each node $n_i \in N_s$, its reverse-complement is denoted by $n'_i$.
An edge $e_{i'j}$ connects the nodes $n'_i$ to $n_j$. 
Nodes may be traversed in either the forward or reverse direction, with the sequence being reverse-complemented in the reverse direction. 
% \todo{There was a left/right in the definition at some point. Where did it go (and why)?} \sgnote{Richard and I prefer to write in the language of n and n', it is easier and more mathematical. What do you think? TM: That's ok with me. (Although I disagree that it's more mathematical.)}
\end{definition}

In words, edges represent adjacencies between the sequences of the nodes they connect.
Thus, the graph implicitly encodes longer sequences as the concatenated sequences of the nodes along walks through the graph.

To illustrate this, we consider an example sequence graph $G_s$ in Figure~\ref{fig:wmec}. It consists of a node set $N_s = \{1, 1', 2, 2', 3, 3', \ldots\}$
and an edge set $E_s = \{1 \rightarrow 2', 1 \rightarrow 3' \ldots\}$.

To generate the sequence graph $G_s$, we first employ SPAdes \citep{bankevich2012spades}, which constructs and simplifies a de Bruijn graph, and we subsequently remove the overlaps between the nodes in the resulting graph in a process we call \textit{bluntification}, explained in the Supplement.

\subsection{Bubble detection in sequence graphs} To account for heterozygosity in a diploid genome, we perform bubble detection. The notion of \textit{bubble} we use is closely based on the \textit{ultrabubble} concept as defined by \cite{paten2017superbubbles}. Briefly, bubbles have the following properties:
\begin{itemize}
 \item \textit{2-node-connectivity}. A bubble is bounded by fixed start and end nodes.
 Removing both the start and end nodes disconnects the bubble from the rest of the graph.
 Note that a bubble can be viewed in either orientation.
 If the graph is traversed in one direction, and a bubble is encountered that starts at a node $n_i$  and ends at a node $n'_j$, then that bubble can also be described as the bubble with start node $n_j$ and end node $n'_i$, as it would be encountered when traversing the graph in the opposite direction.
  \item \textit{Directed acyclicity}. A bubble is directed and acyclic.
 \item \textit{Directionality}. All paths through the bubble flow from start to end.
 \item \textit{Minimality}. No vertex in the bubble other than the start node $n_i$ (with proper orientation) forms a pair with the end node $n'_j$ (with proper orientation) that satisfies the above properties. Similarly, no vertex in the bubble other than $n'_j$ forms such a pair with $n_i$.
%  \todo{Reword to avoid ``Only the smallest possible bubble ... is a bubble'' (because larger structures are no bubbles)}
\end{itemize}

A bubble can represent a potential sequencing error or genetic variation within a set of homologous molecules.
We represent bubbles as collections of alternative paths.

\begin{definition}[Path] We define path $a_i$ as a linear ordering of nodes $a_i= n_1, \ldots n_m$. 
% \todo{Why start from a complemented node $n'_1$?}
\label{def:allele-path}
\end{definition}

A bubble is a collection of paths with the same start and end node and can be defined as follows:
\begin{definition}[Bubble]
Formally, a bubble is represented as a collection of allele paths
 $l_k= \{a_1,a_2 \ldots\}$
 where 
 \[a_1=(n_1, n_2, \ldots n_m), a_2=(n_1, n_3 \ldots n_m)\] and so on. 
\end{definition}


For example, Figure~\ref{fig:wmec} shows a set of two bubbles $L=\{l_2, l_1\}$, and the set of allele paths for the
% \todo{TM: did you define ``complex bubble'' already?} 
bubble $l_2$ is $\{a_1, a_2, a_3\}$,
where $a_1 = (6, 7', 8', 11')$, $a_2 = (6, 9', 11')$, $a_3=  (6, 10, 11')$.

\subsection{PacBio alignments} 
For phasing bubbles, we consider long reads from third generation sequence technologies such as PacBio.
We align these long reads to the sequence graph $G_s$ to generate paths through the graph.
We perform graph alignment using a banded version of the algorithm described by \cite{rautiainen2017aligning}, which is a generalization of semi-global alignment to sequence-to-graph alignment\footnote{\url{https://github.com/maickrau/GraphAligner}}.

There are several advantages of aligning PacBio reads to graphs instead of to a reference genome or contigs.
SNPs often occur near larger variants such as insertions and deletions. SNPs are thus often missed in these regions when reads contain large mismatches with respect to the linear sequences they are aligned against. Graph alignment allows the alignment of reads to variants appropriate to each read's phase, and to other types of complex events.

\begin{definition}[Alignment]
We define a set of read alignments as $R=\{r_1, r_2, \ldots, r_j\}$, where each read alignment $r_{j}$ is given by a path of oriented nodes in graph $G_s$, written $r_{j}=(n_1, \ldots, n_m)$.
\end{definition}
For example, in Figure~\ref{fig:wmec}, $R = \{r_1, r_2, r_3, r_4\}$ and the read alignment path $r_1$ can be written as 
$r_1 = (1, 2', 5', 6, 7', 8', 11' ) $


\subsection{Bubble ordering}
The next stage of our algorithm is to obtain an ordering of the bubbles $L=(l_1, l_2, \ldots l_k)$, which we refer to as a \emph{bubble chain}. 
For example, in Figure~\ref{fig:wmec}, $L=(l_1, l_2)$ is a bubble chain.
A general sequence graph $G_s$ is cyclic, due to different types of repeats present in the genome that create both short and long cycles.
Ordering bubbles in such a graph is closely related to resolving repeats, which is a challenging problem.
In this study, we rely on the Canu algorithm \citep{koren2017canu} to provide a bubble ordering by aligning Canu-generated contigs to our sequence graph.
Furthermore, we detect repetitive bubbles---that is, bubbles that would need to be traversed more than once in a final assembly---based on the depth of coverage of aligned PacBio reads, and remove such bubbles.
% \todo{Shilpa, please check these sentence. I reworded them a bit.}
We deem a bubble repetitive if the number of PacBio reads aligned to its starting node is greater than a coverage threshold specified by the user over the genome.
For example, given a 30$\times$ ($=c$) data set and a repeat that occurs 20 ($=r$) times in the genome, then the coverage at the bubble on average is 600 ($=r \cdot c$).

\subsection{Graph-based phasing}
\label{sec:phasing} 
Given a sequence graph $G_s$, ordered bubbles $L$, and PacBio alignments $R$, the goal is to reconstruct two haplotype sequences $\{h_0, h_1\}$, called haplotigs, along each chain of bubbles.
\begin{definition}[Haplotype path]
Formally, a pair of haplotype paths $(h_0, h_1)$ can be defined as two paths through a bubble chain in the sequence graph and denoted as:
\[h_0=(n_s, n_2, \ldots n_e )\]

\[h_1=(n_s, n_3, \ldots n_e )\]

where $h_0$ and $h_1$ may differ at the heterozygous regions defined by bubbles, and $n_s$ and $n_e$ are the start and end of the bubble chain.
% \todo{I'm really confused by these $n'$s. I think we can just use plain $n$s.}
\end{definition}

The two genome sequences can be seen as two walks through the bubbles $L$ in the sequence graph $G_s$ that are consistent with the PacBio alignments $R$.
In maximum likelihood terminology, the goal is to find the most likely haplotype paths given the alignment paths traversing through the bubbles.
For example, in Figure~\ref{fig:wmec}, given bubbles $(l_1, l_2)$ and PacBio alignments $R=\{r_1,r_2,r_3,r_4\}$, the goal is to find two maximum likelihood haplotype paths $\{h_0, h_1\}$ such that each PacBio alignment is assigned to one of the haplotypes. 
% \todo{continue with ``that...'', because you do not just want to find any two paths, but ML ones.}

For a linear chain of bubbles $L$, the task of finding these two haplotype paths is equivalent to picking one allele path per haplotype for each bubble.
% We generate an association between bubbles $L$ and aligned PacBio reads $R$ in the sequence graph $G_s$ in order to encode alignment paths in the form of an allele path in bubbles.
To this end, we note that an alignment path $r_j$ for a given read can be viewed as a sequence of allele paths traversed in consecutive bubbles. 
% stores the information about the allele path $a_i$ in a bubble $l_k$ that the read $r_j$ is aligned.
We represent this association of reads to allele paths in the form of a \emph{bubble matrix} $\mathcal{F}\in\{0,1, \ldots m, -\}^{|R|\times |L|}$, where $|R|$ is the number of reads, $|L|$ is the number of bubbles along a chromosome, and $m = \max_k|l_k|$ is the maximum number of  paths (or alleles) in any bubble $l_k\in L$.
% Please note that the different paths in a bubble corresponds to possible alleles at a genetic variant.
The entry $\mathcal{F}(j,k) \in \{0, 1, \ldots m, -\}$ represents the allele path index in bubble $l_k$ that read $r_j$ is aligned to, where a value of ``$-$'' indicates that the read does not cover the bubble.
% For example, in Figure~\ref{fig:wmec}, the read alignment path $r_1$ traverses an allele path $a_1$ in a bubble $l_1$ and $a_1$ in $l_2$. \todo{TM: The allele paths are not shown in that figure, right? Can we add them?}
In Figure~\ref{fig:wmec}, note that the read alignment path $r_4$ does not cover all the nodes in any of the allele paths in $l_2$ and hence we set the corresponding value $\mathcal{F}(4,2)$ to ``$-$''.
As a result, this read covers only one bubble, which renders it uninformative for phasing, and we do not consider it further.
The remaining phasing-informative reads in Figure~\ref{fig:wmec} are represented as:

\begin{equation}\label{eq:bubble_matrix}
  \mathcal{F}  = \kbordermatrix{
     & l_{1}       & l_{2}  \\
    r_{1}       & 0 & 0 \\
    r_{2}       & 2 & 2 \\
    r_{3}       & 1 & 2 \\
  }
\end{equation}

Corresponding to $\mathcal{F}$, we have a weight matrix $\mathcal{W}\in W^{|R|\times |L|\times m}$. %, where $|R|$ is the number of reads, $|L|$ is the number of bubbles and $m$ is the maximum number of alleles in a bubble $l_k$.
Each entry in $\mathcal{W}(j,k)$ is a tuple storing a weight for each allele, which can for instance reflect ``phred-scaled'' (i.e. $-10\log(p)$) probabilities that the read supports a given allele.
The weight of ``0'' at the $i^\text{th}$ entry in the tuple $\mathcal{W}(j,k)$ encodes that the read $r_j$ is aligned to allele path index $i$ in bubble $l_k$.
The remaining non-zero values in tuple $\mathcal{W}(j,k)$ store the confidence scores of switching the aligned read $r_j$ to other alleles in bubble $l_k$.

For example, the corresponding weight matrix $\mathcal{W}(j,k)$ for $\mathcal{F}$~\eqref{eq:bubble_matrix} is given by:
\begin{equation}\label{eq:weight_matrix}
  \mathcal{W}  = \kbordermatrix{
     & l_{1}       & l_{2}  \\
    r_{1}       & [0,q_1,q_2] &  [0,q_3,q_4]\\
    r_2 & [q_9,q_8,0] & [q_{11},q_5,0] \\
    r_3 & [q_{10},0,q_7] & [q_5,q_6,0] \\
  }
\end{equation}
where the entry $\mathcal{W}(1,1)$ value $[0,q_1,q_2]$ means that the read $r_0$ is aligned to allele $a_0$ at bubble $l_1$.
Additionally, the cost of flipping it to other alleles is $q_1$ for $a_1$ and $q_2$ for $a_2$.
% \todo{We should use different $q$s for each entry.}

We are now ready to present the problem formulation.
The main insight is that solving phasing for bubble chains is similar to solving the phasing problem for multi-allelic SNVs in reference-based haplotype reconstruction.
Therefore, we build on the previous formulation of the Minimum Error Correction (MEC) problem \citep{Lancia2001} and its weighted version (wMEC) \citep{lippert2002algorithmic,Patterson2015} and further adapt it to work on a subgraph consisting of a chain of bubbles, defining the \emph{Minimum Error Correction for graphs} (gMEC) problem.

\begin{figure}[t!]\centering
\includegraphics[width=\columnwidth]{wmecfig.pdf}
\caption{For a subgraph of $G_s$, the example shows two bubbles $l_1$ and $l_2$, and their corresponding alleles. Reads $r_1,r_2,r_3,r_4$ traverse the bubbles.}
\label{fig:wmec}
\end{figure}

% \begin{definition}[Distance] 
% The quality of a solution relies on the measure $d(r_1,r_2)$ based on the Hamming distance between any two rows $r_1,r_2$. Formally,
% \[d(p_1,p_2):=\sum_{k=1}^{|L|} \big|\big\{k\,\big|\,r_1(k)\neq -\ \wedge\ r_2(k)\neq -\ \wedge\ r_1(k)\neq r_2(k)\big\}\big|.\]
% \end{definition}
% 
% \begin{definition}[Feasibility]
% A feasible solution to a bubble matrix $\mathcal{F}\in\{a_0, a_1 \ldots a_m, -\}^{|R|\times |L|}$ is a pair of haplotypes $h^0,h^1\in\{a_0, a_1, \ldots a_m\}^M$ such that 
% \[d(h_0,h_1):=\sum_{j=1}^{|R|} \min\{ d(\mathcal{F}(j), h_0), d(\mathcal{F}(j), h_1)\} \]
% and there exists a bi-partition of rows (i.\,e., reads) into two sets such that all pairwise distances of two rows within the same set are zero.
% \end{definition}

\begin{problem}[wMEC for bubble chains (gMEC)]\label{prob:gMEC}
Assume we are given a bubble chain $L=(l_1,\ldots,l_{|L|})$ and a set $R$ of aligned reads $r_j$ that pass through these bubbles, with $\mathcal{F}(j,k)$ indicating the index of the allele in bubble $l_k$ that the alignment of read $r_j$ passes through, or ``$-$'' if it does not pass through $l_k$, and that $\mathcal{W}(j,k,i)$ is the cost of flipping $\mathcal{F}(j,k)$ to new value $i$.  
We want to find two paths through $L$, each of which consists of a sequence of allele indices specifying which allele the path takes in each bubble $l_k$, and then to flip entries of $\mathcal{F}$ such that each row is equal to one of the paths for all non-dash entries while the incurred costs are minimized.
\end{problem}
% For simplicity, we represent the $i^{th}$ element of tuple $\mathcal{W}(j,k)$ as $\mathcal{W}(j,k,i)$.
Note that the wMEC problem constitutes a special case of gMEC, where the input graph is a chain of bi-allelic bubbles.
% In its linear counterpart, wMEC with each entry in $\mathcal{F}(j,k) \in \{0,1\}$ works for bi-allelic cases.
Next, we describe how to solve gMEC via dynamic programming (DP).

In the WhatsHap algorithm \citep{Patterson2015}, wMEC is solved in an exact manner for bi-allelic variants using a dynamic programming approach.
It runs in $\mathcal{O}(2^c\cdot |L|)$ time, where $|L|$ is the number of variants to be phased and $c$ is the maximum physical coverage.
The basic idea is to proceed column-wise from left to right over a set of active reads.
Each read remains active from its first non-dash position to its last non-dash position in $\mathcal{F}$.
In column $k$, we denote the set of active reads as $A(k)$, particularly, $c=\max_{k}\{|A(k)|\}$.
The algorithm now considers all \emph{bipartitions} of $A(k)$, that is, all pairs $B=(P,Q)$ of disjoint sets $P$ and $Q$ such that $P\cup Q=A(k)$.
We fill a DP table column wise and for each column $k$ of $\mathcal{F}$, we fill a DP table column $C(k,\cdot)$ with $2^\abs{A(k)}$ entries corresponding to these bipartitions of $A(k)$.
Each entry $C(k,B)$ is equal to the cost of solving wMEC on the partial matrix consisting of columns $1$ to $k$ of $\mathcal{F}$ such that the bipartition of the full read set $A(1)\cup\ldots\cup A(k)$ \emph{extends} $B$ according to the below definition.
\begin{definition}[Bipartition extension]
For a given set $A$ and a subset $A'\subset A$, a bipartition $B=(P,Q)$ of $A$ is said to \emph{extend} a bipartition $B'=(P',Q')$ of $A'$ if $P'\subset P$ and $Q'\subset Q$.
\end{definition}

Once all entries of the DP table have been computed, the minimum of the last column $\min_B\{C(|L|,B)\}$ indicates the optimal wMEC cost and the optimal bipartition can be obtained by backtracing.
We refer the reader to \cite{Patterson2015} for a more detailed explanation of this algorithm.

\textit{Solving gMEC for bubble chains}. The basic idea is to now extend the dynamic program to consider all possible path-pairs through each bubble. 
In the bi-allelic case, we have only two paths in every bubble and, therefore, there is only one pair of distinct paths.
In the multi-allelic case, we consider all possible path pairs in each bubble.
The goal is to find an optimal pair of paths from the sequence graph $G_s$.
% \todo{what does ``simplified version'' refer to? After repeat removal?} 
Analogously to the WhatsHap algorithm for wMEC, we proceed from left to right using dynamic programming.  
% Assuming we have completed the process up to bubble $l_k$, at bubble $l_{k+1}$ we consider each possible bipartition of all the reads crossing that bubble into two groups.\todo{TM: unclear wording}
% For each bipartition, we find the best pair of alleles and the best assignment at $l_k$.\todo{Can we be more precise here and mention costs in this column vs.\ costs from previous columns?}
% We continue this way until the last bubble $|L|$ and then backtrace to get the two optimal paths.

To explain the dynamic programming algorithm that we use, consider a toy example with the weight matrix~\eqref{eq:weight_matrix}:
\begin{equation}\label{eq:weight_matrix1}
  \mathcal{W}  = \kbordermatrix{
     & l_{1}       & l_{2}  \\
    r_{1}       & [0,10,5] &  [0,5,8]\\
    r_2 & [7,6,0] & [5,2,0] \\
    r_3 & [2,0,4] & [4,3,0] \\
  }
\end{equation}

\textit{DP cell initialization}. Along similar lines as \cite{Patterson2015}, we first compute the local cost incurred by bipartition $B= (R,S)$ in column $k$, denoted $\Delta_C(k,B)$, and later combine it with the corresponding costs incurred in previous columns.
% We write $\mathcal{S}(k,B)$ to denote this set of sets of reads induced by bipartition $B$ in column $k$.
The cost $W_{k,R}^i$ of flipping all entries in a read set $R$ to an allele index $i\in\{0,1,\ldots |l_k|\}$ is given by 
\[W_{k,R}^i = \sum_{j\in R}\iverl \mathcal{F}(j,k)\neq i\iverr\cdot\mathcal{W}(j,k,i),\]
In the same manner, we can compute costs $W_{k,S}^i$ for read set $S$ to an allele index $i$.

To compute the cost incurred by a bipartition in a particular column $k$, we minimize over all possible pairs of alleles in bubble $l_k$.
There are ${|l_k| \choose 2}$ such pairs.
So given the corresponding column vectors $\mathcal{F}(k)$ and $\mathcal{W}(k)$ of the bubble matrix and of the weight matrix, respectively, and the bipartition $B=(R,S)$ of active reads $A(k)$, the cost $\Delta_C(k,B)$ is computed by minimizing over all pairs of alleles
$A = \{(x,y) \in l_k \times l_k | x \ne y, x\textless y\}$:
\begin{equation}\label{eq:dp-cell}
  \Delta_C(k,B)= \min_{(p_0,p_1)\in {A}}\left\{\min\{W_{k,S}^{p_0} + W_{k,R}^{p_1}, W_{k,S}^{p_1} + W_{k,R}^{p_0}\}\right\},
\end{equation}
where the outer minimization considers all allele pairs and the inner minimization considers the two possibilities of assigning those two alleles to the two haplotypes.

   

% \begin{algorithm}
%     \caption{\label{alg:dp-cell}\textsc{DP CELL INITIALIZATION}}
%     \SetKwInOut{Input}{Input}
%     \SetKwInOut{Output}{Output}
%     \Input{The column vectors of the bubble matrix $\mathcal{F}(k)$ and a weight matrix $\mathcal{W}(k)$ of the bubble $k$, and the bipartition $S$ of active reads $R (k)$}
%     \Output{$\Delta_C(k,B)$}
%     All allele-pairs set $A = \{(x,y) \in l_k \times l_k | x \ne y, x\textless y\}$ from bubble $l_k = \{a_1, a_2, \ldots a_m\}$
%     \[\Delta_C(k,B)= \min_{i\in {A}^{\mathcal{S}(k,B)}}\left\{\sum_{S\in\mathcal{S}(k,B)}W_{k,S}^{i}\right\},\]
%     
% \end{algorithm}

\textit{DP column initialization}. 
We initialize the first DP column by setting $C(1,B):=\Delta_C(1,B)$ for all possible bipartitions $B$.
We enumerate all bipartitions in Gray code order, as done previously in \cite{Patterson2015}.
This ensures that only one read is moved from one set to another in each step, facilitating constant time updates of the values $W_{k,S}^i$.

For a variant matrix~\eqref{eq:bubble_matrix} and its corresponding weight matrix~\eqref{eq:weight_matrix1}, the DP column cell for bipartition $B=(R,S)$ is given by
\[\Delta_C(k,(R,S))= \min\big\{W^0_{k,R} + W^1_{k,S}, W^1_{k,R} + W^2_{k,S}, \]
\[W^0_{k,R} + W^2_{k,S}, W^1_{k,R} + W^0_{k,S},\]
\[W^2_{k,R} + W^1_{k,S}, W^2_{k,R} + W^0_{k,S} \big\}\]
Now, plugging values from~\eqref{eq:weight_matrix1} into the above equation for different bipartitions, $\Delta_C(1,.)$ can be filled as follows:
\[
\begin{split}
\Delta_C(1,& (\{r_1,r_2,r_3\},\emptyset)) =\\ & min\{9+0, 16+0, 9+0,16+0, 9+0, 9+0\} = 9 
\end{split}
\]
% \[C(1, (\{r_1,r_2\},\{r_3\})) = \min\{7+2, 16+0, 5+4, 7+0, 16+4, 7+4\} = 7\]
Similarly, we can compute $\Delta_C(1,.)$ for other bipartitions $(\{r_1,r_2\},\{r_3\}),$\\
$(\{r_1,r_3\},\{r_2\}), (\emptyset,\{r_1,r_2,r_3\}), (\{r_3\},\{r_1,r_2\}), (\{r_2\},\{r_1,r_3\})$.

\begin{algorithm}
    \caption{\label{alg:dp-column}\textsc{DP COLUMN INITIALIZATION}}
    \SetKwInOut{Input}{Input}
    \SetKwInOut{Output}{Output}
    \Input{Set $A(1)$ of reads covering bubble $l_1$.}
    \Output{$C(1,.)$}
    \For{ all bipartitions $B$ of column $k$}{
	Compute $\Delta_C(k,B)$ using Equation~\ref{eq:dp-cell} and store in $C(1,B)$.	
    }
\end{algorithm}
Due to the use of the Gray code order, we can perform this operation for one DP column in $\mathcal{O}( {|l_k| \choose 2} \cdot 2^{|A(k)|})$ time.

\textit{DP column recurrence}.
Note that $C(k,B)$ is the cost of an optimal solution of Problem~\ref{prob:gMEC} for input matrices restricted to the first $k$ columns under the additional constraint that the solution's bipartition of the full read set extends $B$.
Since column $k$ lists all bipartitions, the optimal solution to the input matrix consisting of the first $k$ columns would be given by the minimum in that column.
To compute entries in column $C(k+1,\cdot)$, we add up local costs incurred in column $k+1$ and costs from the previous column (see Algorithm~\ref{alg:dp-table}).
To adhere to the semantics of $C(k+1,B)$ described above, only entries in column $k$ whose bipartitions are \emph{compatible} with $B$ are to be considered as possible ``predecessors'' of $C(k+1, B)$.

\begin{definition}[Bipartition compatibility]
For bipartitions $B = (P, Q)$ of $A$ and $B' = (P', Q')$ of $A'$, $B$ and $B'$ are \emph{compatible} if $P \cap A \cap A' = P' \cap A \cap A'$ and $Q \cap A \cap A' = Q' \cap A \cap A'$, denoted by $B \simeq B'$
\end{definition}

For example, consider the second column from~\eqref{eq:bubble_matrix} and~\eqref{eq:weight_matrix1}. Let us compute $C(2,.)$ for different bipartitions using recurrence in Algorithm~\ref{alg:dp-table}:
\[C(2, (\{r_1,r_2,r_3\},\emptyset)) = \min\{9+0, 10+0, 9+0,10+0, 8+0, 8+0\} \]
\[ + \min\{C(1, (\{r_1,r_2,r_3\},\emptyset)\} = 8+9 = 17 \]

To fill DP column $C(2,.)$, we can analogously compute this for the remaining bipartitions $(\{r_1,r_2\},\{r_3\})$,
$(\{r_1,r_3\},\{r_2\})$, $(\emptyset,\{r_1,r_2,r_3\})$, $(\{r_3\},\{r_1,r_2\})$, and $(\{r_2\},\{r_1,r_3\})$.

\begin{algorithm}
    \caption{\label{alg:dp-table}\textsc{DP TABLE}}
    \SetKwInOut{Input}{Input}
    \SetKwInOut{Output}{Output}
    \Input{$C(1,.)$ for all bipartitions of bubble $k$.}
    \Output{$C(k,.)$ for all the columns $k$ up to the last column $|L|$}
    \For{ all columns $k \in \{2 \ldots |L|\}$}{
        \For{ all bipartitions $B \in \mathcal{B}(A(k))$}{
            Compute $\Delta_C(k,B)$ using Equation~\ref{eq:dp-cell}. \\
            Combine it with cost from column $k-1$ to obtain cost for column $k$:
            \[C(k,B)= \Delta_C(k,B) + \min_{\substack{B'\in\mathcal{B}(A(k-1)):B\simeq B'}}C(k-1,B')\]
        }
        where $\mathcal{B}\big(A(k)\big)$ denotes the set of all bipartitions of $A(k)$.
    }
\end{algorithm}

\textit{Backtracing}. We can backtrace from the last column $C(|L|, \cdot)$ to compute an optimal bipartition $B=(R,S)$ of all input reads.
Given this bipartition, we obtain minimum-cost haplotypes as follows:
Let $B_k=(R_k,S_k)$ with $R_k=R\cap A(k)$ and $S_k=S\cap A(k)$ be the induced bipartition in column $k$.
We then set
\begin{equation}\label{eqn:haplo_c1}
h_0(k)= a_i \quad \text{with } i:=\argmin_{i'\in \{0,1, \ldots |l_k|\}} W_{k,{R_k}}^{i'}, \nonumber
\end{equation}
\begin{equation}\label{eqn:haplo_c2}
h_1(k)= a_j \quad \text{with } j:=\argmin_{j'\in \{0,1, \ldots |l_k|\}} W_{k,{S_k}}^{j'}, \nonumber
\end{equation}
where $a_i$ and $a_j$ refer to the corresponding allele paths of bubble $k$ (see Definition~\ref{def:allele-path}).
% The full length haplotypes $(h_0, h_1)$ are then obtained by concatenating the bubble paths $\big(h_0(k), h_1(k)\big)$ over all bubbles.

\textit{Time complexity}. 
% As in the previous WhatsHap algorithm \citep{Patterson2015}, we performed algorithm engineering to save the phase information from two consecutive bubbles.
Computing one DP column takes $\mathcal{O}( {m \choose 2} \cdot 2^{|A(k)|})$ time, and the total running time is $\mathcal{O}( {m \choose 2} \cdot 2^{|A(k)|} \cdot |L|)$ for $|L|$ bubbles, where $m$ is the maximum number of alleles in any bubble from $L$. 
Running time is independent of read-length and, therefore, the algorithm is suitable for the increased read lengths available from upcoming sequencing technologies.

\subsection{Generation of final assemblies}
To generate final assemblies, for every connected component in the base sequence graph $G_s$, we traverse along the haplotype paths $(h_0,h_1)$ running through that component. For the nodes in each path, we concatenate together the nodes' sequences from the base sequence graph $G_s$ (in either in their forward or reverse-complement orientations, as specified by the path) in order to generate the final haplotig sequences.

\section{Datasets and experimental setup}
To evaluate the performance of our method, we consider the real data available from two haploid yeast strains SK1 and Y12 \citep{yue2017contrasting}, which we combine to generate a pseudo-diploid yeast.
Both the SK1 and Y12 yeast strains are deeply sequenced using Illumina and PacBio sequencing.
The Illumina dataset is sequenced to an average coverage of 469$\times$ with 151\,bp paired end reads. We randomly downsample the dataset to a lower average coverage of 50$\times$.
The PacBio data is sequenced to an average coverage of 334$\times$ with an average read length of 4510\,bp. 
For coverage analysis, we randomly downsample the PacBio reads to obtain datasets of different coverages $10\times$, $20\times$ and $30\times$ with their average read-lengths of 4482, 4501 and 4516\,bp respectively.

\subsection{Pipeline implementation}
\textit{Sequence graph}.
The first step in our pipeline is to perform error correction on the Illumina data by using BFC \citep{li2015bfc}, which, in our experience, retains heterozygosities well for diploid genomes.
BFC is used with default parameters and provided with a genome size of 12.16\,Mbp.
The second step is to generate a sequence graph that includes heterozygosity information.
To construct such a graph, we first construct the assembly graph by using a modified version of SPAdes v3.10.1 \citep{bankevich2012spades}.
We modify the original SPAdes to skip the bubble removal step and retain the heterozygosity information in the graph, and run it with default parameters plus the \texttt{{-}{-}only-assembler} option.
It uses the short Illumina reads to generate a De~Bruijn-based assembly graph without any error correction.
We then convert the assembly graph to a bluntified sequence graph using VG \citep{garrison2017sequence}.
After graph simplification, the resulting sequence graph has 158,567 nodes and 190,767 edges.

\textit{Bubble detection}. In the next stage, we use VG's snarl decomposition algorithm \citep{paten2017superbubbles} to detect the regions of heterozygosity, or \textit{snarls}, in the sequence graph. This results in 29,071 bubbles.
% and 58,183 total alleles of bubbles. 
% \todo{So there must be bubbles with only one allele since $29095\cdot 2= 58190$. How many multi-allelic bubbles are there?}.

\textit{PacBio Alignments}. After bubble detection, we align different coverage levels (10$\times$, 20$\times$ and 30$\times$) of long read PacBio data to the generated sequence graph using GraphAligner\footnote{\url{https://github.com/maickrau/GraphAligner}}.
% For efficient alignment of PacBio reads, we use a seed-based approach.
This resulted in 21,868, 43,459 and 73,129 PacBio alignments for input coverages of $10\times$, $20\times$ and $30\times$, respectively.

\textit{Bubble ordering}. To obtain an ordering of bubbles, we perform \textit{de novo} assembly using Canu v1.5 \citep{koren2017canu} on each PacBio dataset.
As suggested by \cite{giordano2017novo}, we use Canu v1.5 with the following parameter values: \texttt{corMhapSensitivity=high,} \texttt{corMinCoverage=2,} \texttt{correctedErrorRate=0.10,}\\
\texttt{minOverlapLength=499,} \texttt{corMaxEvidenceErate=0.3}.
Next, we align these Canu contigs to the sequence graph to obtain the bubble ordering, which we define as the sequence of bubbles encountered by each aligned contig.
Note that we use Canu solely for bubble ordering.
In this paper, we restrict ourselves to phasing bubbles only in unique, non-repetitive regions.
We detect repetitive bubbles based on the coverage depth of the PacBio alignments and remove them from downstream analyses.
The coverage depth threshold used is 1.67 times the average coverage.
This results in 148, 80, and 71 bubble chains, and 26,576, 27,556 and 27,741 bubbles, at coverages of $10\times$, $20\times$, and $30\times$ respectively.

% \todo{What is the definition of ``ordered bubble''? How many bubble chains and how many bubbles per chain on average?} 
\begin{figure}[t!]\centering
\includegraphics[width=\columnwidth]{evaluation.pdf}
\caption{For a subgraph of $G_s$, this example shows the true (top) and predicted (bottom) versions of two haplotype alignments (red and blue) through a series of bubbles. When comparing the correspondingly-colored lines between the two versions, we see one switch between SV1 and SV2: the prediction contains one switch error. 
Six bubbles have been phased, for a total of five phase connections between consecutive bubbles. Therefore, the phasing error rate is 1/5.}
\label{fig:evaluation}
\end{figure}
\textit{Graph-based phasing}. For each of the coverage conditions, we take as input the ordered bubbles, the long-read PacBio alignments and the sequence graph, and solve the gMEC problem by assuming constant weights in the weight matrix $\mathcal{W}$.
The optimal bipartition is computed via backtracing and the final haplotigs are generated by concatenating the node labels of the two optimal paths.
These steps have been implemented in our WhatsHap software as a subcommand \texttt{phasegraph}\footnote{Presently this functionality resides in the \texttt{MAV} branch, which will be merged to the \texttt{master} branch in the near future and will be part of future WhatsHap releases.}.
\subsection{Running Falcon Unzip}
The main goal of this study is to measure the performance of phasing using a graph based approach, 
and, in particular, the quality of haplotypes at heterozygous sites achievable by using this method with low coverage PacBio data.
Therefore, we compared our graph-based approach to the state-of-the-art contig based phasing method Falcon Unzip, which also generates diploid assemblies.

The Falcon Unzip \citep{chin2016phased} algorithm first constructs a string graph composed of ``haploid consensus'' contigs, with bubbles representing structural variant sites between homologous loci. 
Sequenced reads are then phased and separated for each haplotype on the basis of heterozygous positions. 
Phased reads are finally used to assemble the backbone sequence (primary contigs) and the alternative haplotype sequences (haplotigs). 
The combination of primary contigs and haplotigs constitutes the final diploid assembly, which includes phasing information dividing single-nucleotide polymorphisms and structural variants between the two haplotypes.

We ran Falcon Unzip using the parameters given in the official parameter guide\footnote{http://pb-falcon.readthedocs.io/en/latest/parameters.html}.
We tried to run Falcon Unzip for lower coverages of 10$\times$ and 20$\times$, but it did not generate output in these cases (and we assume it is not designed for such low coverages).
Therefore, we only ran Falcon Unzip for 20$\times$ PacBio coverage.
Primary contigs and haplotigs were polished using the Quiver algorithm and corrected for SNPs and indels using Illumina data via Pilon, with the parameters ``\texttt{{-}{-}diploid}'' and ``\texttt{{-}{-}fix all}'' \citep{walker2014pilon}.
% To compare the Falcon Unzip haplotigs to the true references, we align the haplotigs produced by Falcon Unzip to our sequence graph and then 
% compared these aligned haplotigs to the true reference alignments through the bubbles.

\subsection{Assembly performance assessment}
To evaluate the accuracy of the predicted haplotypes, we align reference assemblies of the two yeast strains SK1 and Y12 \citep{yue2017contrasting} to the sequence graph.
% In doing so, first, we obtain the true references based on previously generated assemblies of two yeast strains SK1 and Y12 using high coverage PacBio data \citep{yue2017contrasting} and, second, we align these references to the sequence graph.
We emphasize that these reference assemblies are only used for evaluation purposes and are not a part of our assembly pipeline.
We use the following performance measures for the evaluation of diploid assemblies:

\textit{Phasing error rate}. Over the yeast genome, we compare the different diploid assemblies with the ground truth haploid genomes of SK1 and Y12.
As with the reference assemblies, we align the haplotigs produced by Falcon Unzip to our sequence graph.
For each phased bubble chain, the predicted haplotype is expressed as a mosaic of the two true haplotypes, minimizing the number of switches. 
This minimum then gives the number of switch errors.
% Note that the second predicted haplotype is exactly the complement of the first one, due to only considering heterozygous sites, and so does not have to be considered. \todo{But that's only true for bi-allelic loci, right? What happens at mulit-allellic loci?}
The phasing error rate is defined as the number of switch errors divided by the number of phased bubbles.
Figure~\ref{fig:evaluation} illustrates this calculation for a toy example.  The top panel shows the true references aligned to the sequence graph. At the bottom, predicted haplotypes (from Falcon Unzip or our graph-based approach) are aligned to the graph.
Comparing the true and predicted haplotypes, we see one switch between SV1 and SV2, which means that the switch error count is one. 
The number of phase connections between consecutive bubbles is five and the resulting switch error rate for this example is 1/5.

\textit{Average Percent Identity}. We consider the best assignment of each haplotig to either of the two true references, obtained by aligning the haplotig to the references.
For each whole diploid assembly, we compute the average of the best-alignment percent identities over all haplotigs. 

\textit{Assembly contiguity}. We assess the contiguity of the assemblies by computing the N50 of haplotig size.

\textit{Assembly completeness}. We consider two assembly completeness statistics: first, the total length of haplotigs assembled by each method, and second, the total number of unphased contigs.

% \begin{center}
\begin{table}[!ht]
\centering
\begin{tabular}{ |l|c|c|c| } 
 \hline
 Statistics & PacBio & Graph-based  & Falcon Unzip \\ 
 & coverage &  approach &  \\ 
  \hline
  \multicolumn{4}{|c|}{Diploid assemblies Quality}\\
  \hline
 Average Identity[\%] & 10$\times$& 99.50 & \textemdash\\
   & 20$\times$& 99.61  &\textemdash\\
   & 30$\times$& 99.80 &99.4 \\
 Phasing error rate[\%] & 10$\times$& 2.5 & \textemdash\\
   & 20$\times$& 1.5  &\textemdash\\
   & 30$\times$& 0.7 & 3.8 \\
  \hline
  \multicolumn{4}{|c|}{Contiguity}\\
  \hline
   N50 haplotig size [bp]& 10$\times$& 40k &\textemdash\\
   & 20$\times$& 42k &\textemdash\\
   & 30$\times$& 43k &32k\\ 
     \hline
  \multicolumn{4}{|c|}{Completeness}\\
  \hline
  Haplotig size [Mbp] & 10$\times$& 20.7 &\textemdash\\
   & 20$\times$& 21.1 &\textemdash\\
   & 30$\times$& 23.9 &16.6\\
   \# Unphased contigs  & 10$\times$& 2 &\textemdash\\
   & 20$\times$& 2 &\textemdash\\
   & 30$\times$& 2 &77\\
 \hline
\end{tabular}
\\[10pt]
 \caption{Comparison of two phasing methods, Falcon Unzip and our graph-based approach, at different PacBio coverage levels. For computing the ``haplotig N50'', we only consider those portions of a contig for which two haplotypes are available, i.e.\ those regions where Falcon reports both a primary contig and an alternative haplotig.
 For ``haplotig size'', we sum the length of contigs on both haplotypes (``primary contigs'' plus ``haplotigs'' in terms of Falcon's output), so the target size is twice the genome size (24.3Mbp in case of yeast).}
\label{table:graph_unzip}
\end{table}
% \end{center}
\section{Results}
In this section, we present the results of our analysis of the diploid assemblies generated by our method and by Falcon Unzip on the data sets described above.

\textit{Coverage analysis}. To discover a cost-effective method for assembling a diploid genome, we consider PacBio datasets that vary in terms of coverage---specifically, 10$\times$, 20$\times$ and 30$\times$ coverage are considered.
One of the primary aims of our study is to compare two approaches---the graph-based approach we implemented and the contig-based phasing done by Falcon Unzip. In doing so, we quantify the agreement between the diploid assemblies generated by both methods and the true references.
Table \ref{table:graph_unzip} shows the assembly performance statistics for both of these methods.
In order to assess the accuracy of the competing diploid assemblies, we compute the phasing error rate and the average percent identity at different PacBio coverages.
For the graph-based approach, we observe that as we increase the long read coverage from 10$\times$ to 30$\times$, the average identity of haplotigs increases from 99.5\% to 99.8\% and 
the phasing error rate decreases from 2.5\% to 0.7\%. In contrast, Falcon Unzip produces haplotigs with an average identity of 99.4\% and phasing error rate of 3.8\% at 30$\times$ coverage. 
Overall, comparing the agreement between the graph-based approach (at 10$\times$ coverage) and Falcon Unzip (at 30$\times$ coverage) to the true references, our graph-based approach delivers better haplotigs with respect to all measures reported in Table~\ref{table:graph_unzip}.
We believe that one reason for this is that we use an Illumina-based graph as a backbone.
Furthermore, optimally solving the gMEC formulation of the phasing problem most likely contributes to generating accurate haplotigs.
Overall, our analysis supports the conclusion that our approach delivers accurate haplotype sequences even at a long read coverage as low as 10$\times$.

To analyse the effect of different coverages of the Illumina short-read datasets on the quality of our haplotigs, we went back to the original, high coverage Illumina dataset (which we had been using downsampled to 50$\times$ coverage) and downsampled it to 100$\times$ coverage, i.e.\ twice the amount of reads used above.
We observed that increasing the coverage did not have a drastic effect on the quality of haplotigs.
The average phasing identity rose to 99.81\% and the total haplotig size was 23.9\,Mbp, which is virtually identical to the results for 50$\times$ as reported in Table~\ref{table:graph_unzip}.

With an increase in average PacBio coverage from 10$\times$ to 30$\times$, the haplotype contiguity achievable by using our approach improves from 40\,kbp to 43\,kbp.
By way of comparison, Falcon Unzip delivers haplotigs with a N50 length of 32\,kbp at the same coverage level. This highlights the fact that our approach generates more contiguous haplotypes compared to Falcon Unzip.
In terms of haplotype completeness, our approach yields diploid assemblies of length 20.7\,Mbp, 21.1\,Mbp and 23.9\,Mbp at average PacBio coverages of 10$\times$, 20$\times$ and 30$\times$ respectively.
At coverage 30$\times$, Falcon Unzip delivers a total assembly size of 16.6\,Mbp, while the total length of both haplotypes of the pseudo-diploid yeast genome is 24.3\,Mbp.
Our approach therefore delivers more complete haplotypes at a long-read coverage of 10$\times$ compared to Falcon Unzip at a coverage of 30$\times$.
There are 2 haplotigs that are not phased by our approach; this is due to the lack of heterozygosity over those regions.
In comparison, there are 77 (out of 123) contigs that are not phased by Falcon Unzip.
% We believe that the haplotig completeness and contiguity using the graph-based approach is mainly derived from aligning PacBio reads to the sequence graph, and additionally performing bubble calling and phasing them (as illustrated in Figure~\ref{fig:ex_graph_approach}).
In summary, our graph-based approach delivers complete and contiguous haplotype sequences even at a relatively low coverage of 10$\times$.
\begin{figure*}[t!]
\begin{center}
\includegraphics[width=\textwidth]{bubble-breakdown}%
\end{center}
\caption{Structural variation analysis of phased bubbles from our graph-based approach.
a: Joint distribution of allele length and Hamming distance, for pure substitutions.
b. Distribution of size difference between the two alleles, for mixed bubbles and indels. 
Pure substitutions always have a size difference of 0, and are not included in the figure.
c. Joint distribution of the length of the longer allele and the substitution rate, for mixed bubbles.
With a higher substitution rate, the bubble has more substitutions, and with a lower rate more indels.
}
\label{fig:bubble_breakdown}
\end{figure*}

\textit{Bubble characterization}.
We attempted to characterize the nature of the heterozygous genomic variation encoded in the phased bubbles.
There are 25,033 bi-allelic bubbles phased by our approach when using 30$\times$ coverage PacBio data.
Of these bubbles, there are 15,293 for which both allele sequences have a length of at most 1\,bp, out of which 15,258 are single base pair substitutions (SNVs) and 35 are 1\,bp indels.
The remaining 9,740 bubbles either encode two or more small variants or more complex differences.
To differentiate these cases, we computed an alignment between the two allele paths and refer to those bubbles for which the alignment contains only substitutions but no indels as ``pure substitutions''.
Figure~\ref{fig:bubble_breakdown}a shows the joint distribution of length and (Hamming) distance for these pure substitution bubbles.
This analysis reveals, on the one hand, that many longer pure substitions have a low distance and hence encode multiple SNVs and, on the other hand, that there also exists a population of more complex substitutions.
For the 1,489 bubbles not classified as pure substitutions, which we refer to as ``mixed bubbles'', Figure~\ref{fig:bubble_breakdown}b shows the absolute length difference between the two alleles.
While this difference is small for most bubbles, there are 93 bubbles with a length difference of 21\,bp or more.
To further elucidate the nature of the sequence differences, Figure~\ref{fig:bubble_breakdown}c presents the joint distribution of length of the longer allele and substitition rate, which is defined as the fraction of substitutions among all edit operations done to align the two sequences. (That is, a pure insertion or deletion has a substitution rate of 0.)

% the x-axis represents the edit distance between the two allele (haplotype) paths,
% % \todo{I wouldn't call this type and talk of edit distance right away. Confusing otherwise.} 
% and the y-axis represents the maximum length from these two haplotype sequences for each bubble.
% % The structural variant type for each bubble is determined by calculating the edit distance between the two allele (haplotype) paths, and the maximum variant size is calculated by considering the maximum length from these two haplotype sequences.
% The color of each point in the plot indicates the number of corresponding bubbles.
% Using our approach, in Figure~\ref{fig:sv_scatter}, we observe that 17,733 bubbles (out of 29,071) are at a bottom left corner with an edit distance of 1 bp, indicating that the bubbles describe SNVs or 1 bps indels. 
% There are a few points that lie on the diagonal (i.e. with one edit per base in the longest allele), which represent larger indels.
% % \todo{Also below 20bp, right?}.
% The points along the vertical line and those in between vertical and diagonal of the plot represent the bubbles containing SNPs or indels that are close to each other and that end up being contained in a single bubble.
% Furthermore, we see that some points represent larger and more complex variation.
% Overall, the distribution of phased bubbles in the plot indicates that the graph-based approach using both Illumina and PacBio data can phase both small and large structural variants.

% Taken together, the above analysis shows that our graph-based approach delivers better diploid assemblies even at lower PacBio coverages.
% In addition, the graph-based approach helps to detect and phase larger structural variants present in the diploid genomes.

% \begin{figure}[t!]\centering
% \includegraphics[width=\columnwidth]{falcon_bubbles.pdf}
% \caption{Structural variation analysis of phased bubbles from Falcon Unzip. The x axis and y axis show edit distance and maximum size for the two haplotype paths going through each bubble. The color shows the density of bubbles.}
% \label{fig:sv_scatter_falcon}
% \end{figure}

\section{Discussion}
The Falcon Unzip method \citep{chin2016phased} is based purely on PacBio reads, which exhibit a high error rate; it is therefore not suitable for lower coverages.
By using (costly) high coverage PacBio data, Falcon Unzip can generate good quality assemblies with an average haplotig identity of up to 99.99\% \citep{chin2016phased}.
% \todo{I do not see this number in the table above}. %, which is not the cost-effective way to solve the problem.
However, it follows a conservative approach for phasing genomic variants.
As sketched in Figure~\ref{fig:ex_graph_approach}, Falcon Unzip generates long primary contigs, but tends to phase them only partially.

To address the above problems, we have created a novel graph-based approach to diploid genome assembly that combines different sequencing technologies.
By using one technology producing shorter, more accurate reads, and a second technology delivering long reads, we produce accurate, complete and contiguous haplotypes. 
Our method also provides a cost-effective way of generating high quality diploid assemblies.
By performing phasing directly in the space of sequence graphs---without flattening them into contigs in intermediate steps---we can phase large structural variants, which is not possible using linear approaches. 
We have tested our approach using real data, in the form of a pseudo-diploid yeast genome, and we have shown that we deliver accurate and complete haplotigs.
Furthermore, we have shown that we can detect and phase structural variants.

In this study, we restricted ourselves to phasing unique, non-repetitive regions of the genome.
As a next step, we plan to develop techniques for phasing repetitive regions as well.
Resolving repeats and polyploid phasing are closely related problems, as pointed out by \cite{Chaisson2017}.
Therefore, we will aim to solve heterozygous variants and repeats in a joint phasing framework, in order to obtain even more contiguous diploid genome assemblies that include both types of features.
The machinery we plan to develop for this purpose would also remove the need to run an external assembler (Canu) for bubble ordering.
Finally, our framework allows, in principle, for incorporating additional data from other sequencing technologies, such as chromatin conformation capture \citep{burton2013chromosome}, linked read sequencing \citep{weisenfeld2017direct}, and single-cell template strand sequencing (Strand-seq; \citealp{Porubsky2016}).
In previous studies on reference-based haplotyping, we have shown such integrative approaches to be very powerful when infering chromosome-scale haplotypes \citep{porubsky2017dense,chaisson2017multi}; we believe similar results can be obtained for \textit{de novo} diploid genome assemblies.
Moreover, studying the biological implications of the phased structural variation we can now detect also constitute an exciting future research direction.


  \chapter{Conclusions and discussion}
We summarize our algorithmic contributions to the four open problems introduced in Chapter 1 concerning the area of haplotyping.
Furthermore, we provide broader perspectives on how these approaches can be utilized to answer different biological questions and further help in developing new life saving techniques.

\section{Conclusions}
Deriving accurate and complete haplotypes from sequencing datasets helps in genome-engineering, evolutionary studies, and functional and comparative genomics.
In this thesis, we have provided efficient algorithms to perform haplotyping based on different information sources.

There are two ways to perform haplotyping using sequencing data: reference-based phasing and haplotype-aware de novo assembly.
In reference-based phasing, reads are aligned to the reference genome and then the phasing is performed using aligned reads over the variants.
The reference-based phasing for a single individual is formulated as the Minimum Error Correction (MEC) and its weighted version as wMEC (as explained in Chapter 1).
Different types of sequencing datasets generate different types of instances such as MEC, Gapless-MEC, and Binary-MEC.
To study the complexity and approximation status of these instances is very important, and helps in unraveling and understanding the structures.

\subsection{Approximation status for \GMEC}
Third-generation sequencing technologies generate single-ended reads in the order of 10Kb in length. These reads are often aligned to the reference genome for performing phasing.
The alignment of these reads can be mathematically represented in the form of a matrix, which is denoted by \GMEC.
\GMEC is a generalization for \BMEC because \BMEC contains only binary values, whereas \GMEC contains wildcards at the end of each row. 
It is known that \BMEC is in PTAS and \MEC is in APX-hard.
There is a gap about the approximation status for \GMEC, which has been open for 10 years.  
As a part of this thesis, we study the approximation status of \GMEC and conclude that \GMEC is in QPTAS.
Proving \GMEC in QPTAS guarantees that the \GMEC is not in APX-hard.

The main challenge in \GMEC instances is that we would like to sample sufficiently many rows such that we get haplotypes from end-to-end, otherwise missing its parts which may lead to an unbounded approximation.
To handle this problem, we provided a dynamic programming approach that well captures the structure of the considered instances.

We started with simple instances (SWC) such that all the rows start from one column and are ordered in the increasing length of the binary part.
Such instances are simple to solve because we can apply the \BMEC algorithm for the sub-matrices starting from column one and then argue that we generate good solution strings.
As soon as, we are at a sub-instance consisting of wildcards and binary values, we require to be extra careful and, therefore, we sample more rows and consider a weighted mahority to generate haplotypes as explained in Chapter 3.
Furthermore, we show that in expectation, the algorithm works well at each column and is in PTAS. 
We have generalized this algorithm to solve complete SWC instances by using clever DP algorithms.

We further generalized this DP algorithm for SWC instances to solve sub-interval free instances. 
The main insights are that we can determine a sequence of columns. At each column, we get SWC instances at its left and right side.
In order to combine the consecutive columns, we introduced the notion of dominance such that we can solve two consecutive columns simultaneously in the DP, still getting a good approximation.
We proved that this dynamic programming algorithm is in PTAS.

Next, we generalize this dynamic programming algorithm to solve general \GMEC instances. 
The main observation is that we can divide the instances into different length classes and each length class can be solved in quasi-polynomial time.
To combine different length class instances, we use clever techniques in DP and prove that the generalized dynamic program is in QPTAS.

This dynamic programming provides good insights into the structure of instances, which further motivates to derive other types of algorithms that potentially work well in practice.

\subsection{Parameterized algorithm for phasing individual genomes}
There are different sequencing technologies available to sequence the genome, one of them is strand-specific technology and others are long-read technologies.
The strand-specific technology provides global but sparse information about the haplotypes. The main advantage of Strand-Seq technology is that it provides Illumina reads along with the directionality information.
The downside of Strand-Seq data is its sparseness, which creates a necessity to integrate with other available local long-read information sources such as PacBio and ONT to get complete haplotypes.

We have presented an integrative phasing strategy, which is a parameterized algorithm, to generate accurate, contiguous and complete haplotypes. 
The parameterized algorithm has coverage as a parameter, which is usually small enough in practice that makes it feasible to solve these instances faster.
We have provided a dynamic programming to solve these instances. The main idea is to go column wise from left to right. At each column, we store the best allele assignment score for each bipartition.
In the recurrence step, when we compute the bipartition at $k+1$ column, we consider the bipartition from the previous column $k$ 
such that the reads assigned to different haplotypes remain the same over two consecutive columns and we store the best allele assignment for the corresponding bipartition in DP table.
In this way, we recurse until the last column and finally backtrace to generate the haplotypes.

We have demonstrated the effectiveness of this algorithm on the real human genome. 
We have performed a comprehensive coverage analysis on the level of coverage required from each sequencing technology in this integrative framework.

In the situation, when the Strand-Seq data is not available to provide global information about the haplotypes, it is better to utilize the information from pedigrees.
\subsection{Parameterized algorithm for phasing pedigrees}
For pedigree of genomes, we have extended the parameterized algorithm for a single individual to incorporate the pedigree information.
In the model, we want to find two haplotypes for each individual in a pedigree such that we obey the Mendelian laws of inheritance.
We formulated the combination of read-based phasing and genetic haplotyping into an integrative framework known as MEC on pedigrees (PedMec).

% We solve the parameterized algorithm using dynamic programming manner. 
As input, we have SNP matrices and corresponding weight matrices for each individual in a pedigree. 
Additionally, we have the recombination vector that represents the likelihood of recombination between every consecutive variant.
We have provided the dynamic programming algorithm to solve PedMec instances.
The major difference between the pedigree framework to the single individual case is the number of partitions.
In a pedigree framework, we have four partitions, instead of two, because the reads from the mother, father and child each can be partitioned into two sets, 
with additional constraints that the reads from child partitions belong to either mother or father.
The constraint is represented by the transmission value, which means which haplotype from each parent is transmitted to the child.
We compute the best allele assignment score for these four partitions at each column and store them in a DP table.
The recurrence step proceeds similarly to the single individual case, but with additional step of trying all possibilities of recombination in mother or father.
We continue this process until the last column and recurse to find haplotypes for each individual in a pedigree.

The running time of pedigree algorithm increases by a factor of $2^t$, where $t$ is the number of trio-relationships, but it still remains linear in the number of variants.
Therefore, the algorithm is suitable for long-read technologies.

We demonstrate the effectiveness of our algorithm on both real and simulated data. We considered the AJ trio from Genome in a Bottle Consortium (GIAB).
We showed that we provided long-range chromosomal-length accurate haplotypes by incorporating trio information.
The main conclusion is we require low coverages of 2$\times$ for each individual with family relationship information as opposed to a single individual at 15$\times$ coverage.

In all these approaches, we performed phasing based on the reads aligned to the reference genome. Therefore, there is a reference bias attached to it.
Additionally, we are also interested in phasing sequences or variants that are unique to the genome.

\subsection{Haplotype-aware de novo assembly}
To overcome the problem of reference bias, we proposed a new approach to reconstructing the genome sequences of diploid organisms directly from the reads.
The process of obtaining haplotype sequences from reads is called haplotype-aware denovo assembly.
Constructing these sequences is very important to understanding the true characteristics of diploid organisms.

Current assemblers collapse the heterozygosity information represented by bubbles in assembly graphs and therefore generate a haploid consensus sequence.
A popular diploid assembler --- Falcon Unzip, which is purely PacBio based --- has the ability to generate diploid assemblies.
The main disadvantages of Falcon Unzip method are that it fails for low coverage datasets and for genomes with a low heterozygosity rate.

In order to handle these problems, we have provided a novel graph-based diploid assembly pipeline, which is basically a hybrid of short accurate and long error-prone read data.
The pipeline involves constructing the base assembly graph using Illumina data such that it acts as a backbone for subsequent steps.
Using accurate data, it is possible to accurately detect SNVs represented by bubbles in the assembly graph.
Furthermore, we aligned the long reads to this graph to span over the bubbles and detect the bubble chains.
By using long alignments that span the bubble chains, we perform phasing using a generalized wMEC model for a single individual.
The generalized wMEC model is extended to MEC on bubble chains (gMEC), which is based on the observation that phasing bubble chains is equivalent to phasing multi-allelic variants.
The running time increases by a factor of $a \choose 2$, where $a$ is the maximum number of alleles in any bubble.

We have demonstrated the effectiveness of our algorithm by combining two yeast strains to form a pseudo-diploid yeast genome.
We have showed that our graph-based approach has the ability to generate more accurate, contiguous and complete assemblies even at a low coverage of 10$\times$.
Additionally, we have pointed out that we can detect and phase large structural variations.

\section{Discussion}
A major revolution in approaches to the haplotyping problem occurred due to the third generation sequencing technologies.
Using long reads delivered by these technologies, it has become possible to generate accurate, contiguous and complete haplotypes.
Indeed, there are some complex regions over the diploid genomes that are not yet solved by current available approaches, either reference-based or de novo approaches.
We believe that both reference-based and de novo approaches can be lifted up to solve the complex and repetitive regions over the genome based on available long-read datasets.
% Currently available datasets provide a good signal to generate near complete and accurate diploid genomes.
In a de novo context, different graph parameters such as tree-width, maximum branching factor, copy number or other parameters can be explored to solve the complex regions of genome.
In a reference-based context, ILP, a greedy heuristic or other parameterized approaches have the power to solve complex instances efficiently.

The techniques used in this thesis on longer reads can also help to solve higher ploidy genomes. For higher ploidy genomes, the current WhatsHap algorithm directly leads to running time of $k^c$, where $k$ is the ploidy and $c$ is the coverage.
More efficient ways could be explored to further improve the running time by pruning the search space.
Another extension of WhatsHap algorithm is to separate species based on long read data, thus solving the popular meta-genome assembly problem.
Using some efficient computation techniques or greedy heuristic approaches, the algorithm can be implemented in practice. 
Another possible direction is producing of transcriptome
assembly of RNA sequences. In all these problem, the main challenge is scalability, and therefore, efficient data structure and algorithms need to be developed to address these challenges.

Detecting large structural variants in complex regions was not possible earlier.
These variants include single nucleotide or larger genomic ranges by substitution, insertion, deletion, copy number. 
Haplotypes can help in detecting structural variants in complex regions and perform SV analyses properly.
Therefore, algorithms for haplotyping can further help in SV detection and analysis.

In a general view, it is expected that in coming years the reads get longer and error rates will be reduced.
Constant algorithmic efforts are required to handle the datasets that will arise from new sequencing technologies.
The methods proposed in this thesis could be refined, new applications or integration of other kinds of information could be incorporated.

Besides algorithmic contributions to haplotyping, this thesis also offers some haplotyping tools, which enable haplotype-resolved studies of genetic variation and genome instability.
Accurate and complete haplotypes help in investigating mechanisms behind complex phenotypes in humans, consisting of ageing and common diseases including cancer. 
The deep insights into biological mechanisms of disease can be achieved, which further help in improving drug response.
Also, the haplotype knowledge is useful in population genetics.
The question on how evolution has shaped the genomic architecture can be answered by haplotype comparisons and various genome features can be investigated.

In summary, haplotype information is essential to the complete description of individual genomes. The haplotype resolved approaches enable us to discover novel genome features, that were previously hidden or not recognized.
Haplotype-aware methods are a promising next step to address a lot of open biological questions and improve our understanding of functioning of genome.
The impact of haplotype-resolved approaches to genomics in biomedical research and clinical medicine is everlasting.


% \paragraph{Clinical applications}
% What are the applications for which genome-wide haplotype resolution is desirable or necessary? 
% The first application is the accurate interpretation of personal genomes, particularly in the context of medical genetics. 
% As humans are diploid organisms, haplotype information is essential to each personal genome, for instance, to assess the phase of potentially disease-causing recessive mutations (that is, compound heterozygosity).
% In pharmacogenetics, the phasing of metabolically relevant variants onto haplotypes helps to predict the drug response profiles of patients, improve dosing and reduce the extent of adverse reactions.
% 
% Second, haplotype knowledge is useful in population genetics and human disease studies. 
% For example, the inference of Neanderthal ancestry in non-Africans exploited the availability of a human reference genome that was derived from local segments of African and European ancestry.
% More generally, haplotype inference and variant imputation are increasingly important parts of human disease studies involving large cohorts --- that is, rare variant-common disease and common variant-common disease study designs. 
% 
% Last, haplotype information can also facilitate non-invasive fetal genome sequencing. 
% Accurate early inference of allelic inheritance genome-wide has the ability to simultaneously determine the risk of the thousands of individually rare, but collectively common, Mendelian disorders in a single test.
% 
% % http://www.annualreviews.org/doi/pdf/10.1146/annurev.med.56.082103.104540
% Human leuokocyte antigen (HLA) matching is a clear example of how haplotypes
% can be used in the clinic to improve outcome. In this scenario, transplant recipients
% and donors are genotyped at several markers along the major histocompatibility
% complex. The HLA haplotypes are then determined by ordering the alleles along
% the chromosomes. Patients who match the donor haplotypes closely are predicted
% to have a better transplant outcome than patients who do not. The development
% of HLA haplotype matching has proved to be crucial in making transplantation
% between unrelated patients and donors a success.
% 
% Finally, a new understanding of the biology of common disease must be achieved
% to meaningfully link individual genotypes to complex phenotypes.
% 
% 
% 
% 

 {
%\bibliographystyle{unsrtnat}
   \bibliographystyle{apalike}
   \bibliography{main}
 }
\appendix
\appendixpage
\noappendicestocpagenum
\addappheadtotoc

\chapter{Additional Details}

\section{Proof of Lemma~\ref{lem:SWC}}\label{app:SWC}
\begin{proof}
    We show the claim by using a randomized argument. 
    To this end, we assume that for each $i$, the rows from $R_i$ and $S_i$ are selected uniformly at random from $R_i \cap \tau(M)$ and $S_i \cap \tau(M)$ and
    the rows from $R'_i$ and $S'_i$ are selected uniformly at random from $R'_i \cap \tau'(M)$ and $S'_i \cap \tau'(M)$.
    We argue that for each column, the expected number of errors is at most a factor $(1 + O(\epsilon))$ larger than in an optimal solution.
    Then the claim follows from linearity of expectation and the fact that there is a selection with at most the expected number of errors. 

    We consider the $j$th column of $M$.
    Let $c := \tau(M)_{*,j}$, but without rows that have an entry ``$-$'' in column $j$.
    Let $p := |\{ i : c_i = 0\}|/|c|$ be the fraction of zeros in $c$.
    By swapping the zeros and ones we can assume \WLOG that $p \ge 1-p$, \ie, $p \ge 1/2$.
    Our assumption implies $\tau_j = 0$ and the optimal solution has $(1-p)|c|$ errors within $c$.

    The general idea of the proof is as follows.
    Suppose we would select exactly one row from $\tau(M)$ uniformly at random.
    Then with probability $p$, the algorithm has $(1-p)|c|$ errors in $c$ and with probability $(1-p)$ the number of errors is $p|c|$.
    Therefore the expected number of errors is $(p(1-p) + (1-p)p) |c| = 2 p (1-p) |c|$.
    We obtain the approximation ratio $2 p (1-p) |c| / ((1-p)|c|) = 2 p$.

    We will see that the approximation ratio improves with choosing several rows instead of a single one.
    Additionally, we have to handle the circumstance that we only sample from $U \cup L$ and ignore $X$.

    There is a further issue regarding $U$.
    Let $s$ be the smallest index such that $R_s$ and $c$ intersect, \ie, $R_s$ is the first set with binary entries in column $j$.
    Then rows sampled for $R_s$ may be located outside of $c$ at positions with wildcards in column $j$.
    We avoid the complications caused by the wildcards by only considering classes $R_{i}$ for $i > s$.

    To summarize, $c$ has at least $\epsilon r$ selected entries and we ignore at most $2 \epsilon^2 r$ of these due to $X$ and $R_s$.
    For each $i > s$, we sample $1/\epsilon^3$ rows from $R_i$. 
    Let $c'$ be $c$ restricted to $\bigcup_{i > s} R_{i}$ and let $c''$ be $c$ restricted to  $L$.
    Let $\hat{c}$ be $c$ without $R_s$ and $X$ and let $\bar{c}$ be the part of $c$ in $X \cup R_s$.
    For each $i$, let $c'_i$ be the fraction of zeros of $c'$ in $R_i$ and $c''_i$ the fraction of zeros of $c''$ in $S_i$.

    For each $i \le \ell$, we define $p'_i$ to be the fraction of zeros $c'_i$ and $p''_i$ the fraction of zeros $c''_i$.

    We define a random variables $Y'_{i,k}$ for each $s < i \le 1/\epsilon^2$ and $Y''_{i,k}$ for each $1 \le i \le 1/\epsilon^2$. 
    In both cases, $1 \le k \le 1/\epsilon^3$.
    For each $i,k$, we pick an entry from $c'_{i}$ ($c''_i$) uniformly at random. 
    Then $Y'_{i,k}$ ($Y''_{i,k}$) is the value of the picked entry.
    For all $i,k$, $E[Y'_{i,k}] = 1 - p'_{i}$ and $E[Y''_{i,k}] = 1 - p''_i$. 
    Observe that the $Y_{i,k}$ are independent Poisson trials.
    Let $Y' := \sum_{s<i ,1 \le k \le 1/\epsilon^3} Y'_{i,k}$ 
    and $Y'':= \sum_{i, 1 \le k \le 1/\epsilon^3} Y''_{i,k}$.
    We want to use Chernoff bounds to control the probability to take the wrong decision.
    It is sufficient to consider $Y'$ with $s = 1/\epsilon^2-1$, since in all other cases the probabilities are amplified more.
    Let $\mu' := E[Y']$.
    We analyze the ranges of $\mu'$ separately.
    \subparagraph{Case 1:} Let us assume that $\mu' \in [0,1/(2\euler\epsilon^3)]$.
    We define $\delta' := 1/(2\mu'\epsilon^3) - 1$.
    Using a multiplicative Chernoff bound (cf.~\cite{MU05_probability}), we obtain
    \begin{align}
        \Pr(Y' \ge 1/(2\epsilon^3)) &< \Bigl(\frac{\euler^{\delta'}}{(1+\delta')^{(1+\delta')}}\Bigr)^{\mu'} = \Bigl(\frac{1}{1 + \delta'}\Bigr)^{\mu'} \Bigl(\frac{\euler}{{1 + \delta'}}\Bigr)^{\mu'\delta'}\\
        &= (2 \mu' \epsilon^3)^{\mu'}(\euler \cdot 2\mu'\epsilon^3)^{(1/(2\epsilon^3)  - \mu')}\label{eq:rhs}
    \end{align}
    Note that both terms of \eqref{eq:rhs} are numbers between zero and one.
    If $\mu' < 1/\epsilon$, the right term is smaller than $\epsilon^4 \mu'$. 
    Otherwise the left term is smaller than $\epsilon^4 \mu'$

    The range of $\mu'$ implies that the majority of entries in $\hat{c}'$ is zero.
    Recall that $\hat{c}'$ has an $\epsilon^3 \mu'$ fraction of zeros.
    The expected number of errors done by the algorithm is therefore at most
    $(1 - \epsilon^4 \cdot \mu') \cdot (\epsilon^3 \mu') + \epsilon^4 \cdot \mu' \cdot (1 - \epsilon^3 \mu') = (1+\epsilon) \epsilon^3 \mu'$.

    \subparagraph{Case 2:} Let us assume that $\mu' \in (1/(2\euler\epsilon^3), 1/(2\epsilon^3) - 1/\epsilon^2]$.
    We use Hoeffding's inequality \cite{Hoe63_probability} to analyze the range.
    To this end, we scale $Y'$ and obtain $\bar{Y}':= \epsilon^3 Y'$, which has values between zero and one.
    Then 
    \[
        \Pr(\bar{Y}' - E[\bar{Y}'] \ge \epsilon) \le \euler^{-2\epsilon^2/\epsilon^3} = \euler^{-2/\epsilon}\,.
    \]

    Since for sufficiently small $\epsilon$, $\euler^{-2/\epsilon} < \epsilon/(2\euler) \le \epsilon^4 \mu'$,
    again we obtain a $(1+\epsilon)$-approximation in expectation.

    All other ranges now follow immediately:
    For $\mu' \in (1/(2\epsilon^3) - 1/\epsilon^2,1/(2\epsilon^3)]$ every solution is a $(1+O(\epsilon))$-approximation and for larger $\mu'$ the majority of entries in $\hat{c}'$ is one.
    The analysis is analogous.

    In order to combine $Y'$ and $Y''$, we introduce a bias for $Y'$ such that we count rows $i$ for $s < i \le \ell$ with a factor $(1-\epsilon)/(\epsilon - \epsilon^2)$.
    Then
    \[
        \bar{Y} := \frac{\bar{Y}' \cdot (\ell-s)(1-\epsilon)/(\epsilon - \epsilon^2) + \bar{Y}'' \cdot \ell}{(\ell-s)(1-\epsilon)/(\epsilon - \epsilon^2) + \ell}\,.
    \]
    Then, using the union bound, setting $\sigma_j = 0$ for $\bar{Y} < 1/2$ and $\sigma_j = 1$ otherwise gives an expected $1+O(\epsilon)$ approximation within $\hat{c}$.
    Errors in $\bar{c}$ are either also errors in an optimal solution, or they contribute at most a factor $O(\epsilon)$ to the total number of errors.
    Thus overall we obtain an approximation ratio $1+O(\epsilon)$ within $c$.
    The algorithm $\SWC$ has at most the same approximation ratio, since the only difference is that we do not fix the $Y_{i,k}$ to be zero or one.
    Thus the random process used by the algorithm can only have a lower variance.

    This finishes our analysis for $\tau(M)_{*,j}$.
    For $\tau'(M)_{*,j}$, the proof is analogous.

    % I DON'T UNDERSTAND THE FOLLOWING ANYMORE, THEREFORE COMMENTED.
    %To finish the proof we still have to argue that we may assign the $r$ rows with maximal values $d_i$ to $\sigma$ and the remaining $r'$ rows to $\sigma'$.
    %The reason is that at all times we compare against the assignment of $(\tau,\tau')$, which is a stronger result then the mere approximation ratio.
    %Since $|\tau(M)|=r$ and the choice of $d_i$ gives the assignment with fewest errors, the claim follows.
\end{proof}

We introduced a small but easy to handle imprecision due to the assumption that we can choose exactly the same number of strings from each range.



%%% Local Variables:
%%% mode: latex
%%% TeX-master: "main"
%%% End:

\listoffigures
\listoftables
\listofalgorithms
\end{document}
